\documentclass[12pt]{article}

% report, book
%  Русский язык

\usepackage{hyperref,bookmark}
\usepackage[warn]{mathtext} %русский язык в формулах
\usepackage[T2A]{fontenc}			% кодировка
\usepackage[utf8]{inputenc}			% кодировка исходного текста
\usepackage[english,russian]{babel}	% локализация и переносы
\usepackage[title,toc,page,header]{appendix}
\usepackage{amsfonts}
\usepackage{bm}


% Математика
\usepackage{amsmath,amsfonts,amssymb,amsthm,mathtools} 
%%% Дополнительная работа с математикой
%\usepackage{amsmath,amsfonts,amssymb,amsthm,mathtools} % AMS
%\usepackage{icomma} % "Умная" запятая: $0,2$ --- число, $0, 2$ --- перечисление

\usepackage{cancel}%зачёркивание
\usepackage{braket}
%% Шрифты
\usepackage{euscript}	 % Шрифт Евклид
\usepackage{mathrsfs} % Красивый матшрифт


\usepackage[left=2cm,right=2cm,top=1cm,bottom=2cm,bindingoffset=0cm]{geometry}
\usepackage{wasysym}

%размеры
\renewcommand{\appendixtocname}{Приложения}
\renewcommand{\appendixpagename}{Приложения}
\renewcommand{\appendixname}{Приложение}
\makeatletter
\let\oriAlph\Alph
\let\orialph\alph
\renewcommand{\@resets@pp}{\par
  \@ppsavesec
  \stepcounter{@pps}
  \setcounter{subsection}{0}%
  \if@chapter@pp
    \setcounter{chapter}{0}%
    \renewcommand\@chapapp{\appendixname}%
    \renewcommand\thechapter{\@Alph\c@chapter}%
  \else
    \setcounter{subsubsection}{0}%
    \renewcommand\thesubsection{\@Alph\c@subsection}%
  \fi
  \if@pphyper
    \if@chapter@pp
      \renewcommand{\theHchapter}{\theH@pps.\oriAlph{chapter}}%
    \else
      \renewcommand{\theHsubsection}{\theH@pps.\oriAlph{subsection}}%
    \fi
    \def\Hy@chapapp{appendix}%
  \fi
  \restoreapp
}
\makeatother
\newtheorem{resh}{Решение}
\newtheorem{theorem}{Теорема}
\newtheorem{predl}[theorem]{Предложение}
\newtheorem{sled}[theorem]{Следствие}

\theoremstyle{definition}
\newtheorem{zad}{Задача}[section]
\newtheorem{upr}[zad]{Упражнение}
\newtheorem{defin}[theorem]{Определение}

\title{Решение заданий\\ ОП "Квантовая теория поля, теория струн и математическая физика"\\[2cm]
Семинары по квантовой механике -- I\\ (И.В. Побойко, Д.С. Антоненко, Н.А. Степанов)}
\author{Коцевич Андрей Витальевич, группа Б02-920}
\date{5 семестр, 2021}

\begin{document}
\maketitle
\newpage
\tableofcontents{}
\newpage
\section{Основы квантовой механики.}
\subsection*{Упражнения (20 баллов)}
\textbf{Упражнение 1. Унитарные матрицы (5 баллов).}\\
Покажите, что унитарные матрицы, как и эрмитовы, диагонализуемы. \textit{Указание}: покажите, что эрмитова и анти-эрмитова часть унитарного оператора диагонализуемы совместно.\\
\textbf{Решение.}\\
Разложим унитарный оператор на эрмитову и анти-эрмитову части:
\begin{equation}
    \hat{U}=\frac{\hat{U}+\hat{U}^\dagger}{2}+i\frac{\hat{U}-\hat{U}^\dagger}{2i}
\end{equation}
Вычислим их коммутатор, учитывая унитарность оператора $\hat{U}$ ($\hat{U}^\dagger\hat{U}=\hat{U}\hat{U}^\dagger=\mathbb{I}$):
\begin{equation}
    [\hat{U}+\hat{U}^\dagger,\hat{U}-\hat{U}^\dagger]=\hat{U}\hat{U}+\hat{U}^\dagger\hat{U}-\hat{U}\hat{U}^\dagger-\hat{U}^\dagger\hat{U}^\dagger-\hat{U}\hat{U}+\hat{U}^\dagger\hat{U}-\hat{U}\hat{U}^\dagger+\hat{U}^\dagger\hat{U}^\dagger=2(\hat{U}^\dagger\hat{U}-\hat{U}\hat{U}^\dagger)=0
\end{equation}
Часть $\frac{\hat{U}+\hat{U}^\dagger}{2}$ эрмитова, $i\frac{\hat{U}-\hat{U}^\dagger}{2i}$ антиэрмитова. Докажем следующее предложение:
\begin{predl}
Если операторы коммутируют, то их можно одновременно диагонализировать.
\end{predl}
\begin{proof}
Пусть $\vec{a}$ -- собственный вектор оператора $\hat{A}$, соответствующий некратному собственному значению $\lambda$ ($\hat{A}\vec{a}=\lambda\vec{a}$). Вектор $\hat{B}\vec{a}$ тоже будет принадлежать собственному значению $\lambda$ матрицы $A$:
\begin{equation}
    \hat{A}\hat{B}\vec{a}=\hat{B}\hat{A}\vec{a}=\lambda\hat{B}\vec{a}
\end{equation}
Следовательно, вектор $B\vec{a}$ пропорционален $\vec{a}$, т.е. $\vec{a}$ является собственным вектором матрицы $\hat{B}$.\\
В случае кратного собственного значения $\lambda$ имеем несколько собственных векторов $x_i$, принадлежащим этому собственному значению. Пусть $Bx_i=b_{ij}x^j$, где $b_{ij}$ -- некоторые числа. Матрицу $b_{ij}$ можно диагонализовать, выбрав в качестве принадлежащих $\lambda$ собственных векторов матрицы $A$ другие векторы $y_i=c_{ij}x^j:Ay_i=\lambda y_i$ и $By_i=\Lambda_iy_i$ (в последнем равенстве суммы по $i$ нет).\\
Т.е. у операторов общая система собственных векторов. Можно перейти к базису, состоящему из собственных векторов, и в этом базисе обе матрицы будут диагональны.
\end{proof}
Таким образом, эрмитова и анти-эрмитовы части диагонализуемы совместно. Следовательно, и их сумма -- унитарный оператор $\hat{U}$ диагонализуем.\\\\
\textbf{Упражнение 2. Замена базиса (5 баллов).}\\
В квантовой механике замена базиса реализуется унитарными преобразованиями $\ket{\psi'}=\hat{U}\ket{\psi}$.
\begin{enumerate}
    \item Покажите, что гамильтониан при этом заменяется на $\hat{H}'=\hat{U}\hat{H}\hat{U}^\dagger$.
    \item Последнее утверждение необходимо модифицировать, если унитарное преобразование зависит явно от времени $\hat{U}=\hat{U}(t)$. Покажите, что в таком случае гамильтониан необходимо заменить на $\hat{H}'=\hat{U}\hat{H}\hat{U}^\dagger-i\hbar\hat{U}\partial_t\hat{U}^\dagger$.
\end{enumerate}
\textbf{Решение.}\\
    Запишем нестационарное уравнение Шрёдингера:
    \begin{equation}
        i\hbar\frac{\partial\ket{\psi}}{\partial t}=\hat{H}\ket{\psi}
    \end{equation}
    Поскольку преобразование $\hat{U}$ унитарное, то
    \begin{equation}
        \hat{U}\hat{U}^\dagger=\hat{\mathbb{I}}\rightarrow \hat{U}^{-1}=\hat{U}^\dagger\rightarrow\ket{\psi}=\hat{U}^\dagger\ket{\psi'}
    \end{equation}
    \begin{enumerate}
    \item \begin{equation}
        i\hbar\frac{\partial(\hat{U}^\dagger\ket{\psi'})}{\partial t}=\hat{H}\hat{U}^\dagger\ket{\psi'}\rightarrow i\hbar\frac{\partial\ket{\psi'}}{\partial t}=\hat{U}\hat{H}\hat{U}^\dagger\ket{\psi'}
    \end{equation}
    Таким образом, 
    \begin{equation}
        \boxed{\hat{H}'=\hat{U}\hat{H}\hat{U}^\dagger}
    \end{equation}
    \item \begin{equation}
        i\hbar\frac{\partial(\hat{U}^\dagger(t)\ket{\psi'})}{\partial t}=\hat{H}\hat{U}^\dagger\ket{\psi'}\rightarrow i\hbar\hat{U}^\dagger\frac{\partial\ket{\psi'}}{\partial t}+i\hbar\ket{\psi'}\partial_t\hat{U}^\dagger=\hat{U}\hat{H}\hat{U}^\dagger\ket{\psi'}
    \end{equation}
    \begin{equation}
        i\hbar\frac{\partial\ket{\psi'}}{\partial t}=(\hat{U}\hat{H}\hat{U}^\dagger-i\hbar\hat{U}\partial_t\hat{U}^\dagger)\ket{\psi'}
    \end{equation}
    \begin{equation}
        \boxed{H'=\hat{U}\hat{H}\hat{U}^\dagger-i\hbar\hat{U}\partial_t\hat{U}^\dagger}
    \end{equation}
\end{enumerate}
\textbf{Упражнение 3. Матрицы Паули (10 баллов).}\\
Покажите следующие свойства матриц Паули (по повторяющимся индексам подразумевается суммирование):
\begin{enumerate}
    \item Они, совместно с единичной матрицей $\sigma^0=\hat{\mathbb{I}}_{2\times2}$, представляют собой базис в пространстве эрмитовых матриц $2\times2$.
    \item Они удовлетворяют следующими правилам перемножения: $\hat{\sigma}^\alpha\hat{\sigma}^\beta=\delta_{\alpha\beta}\hat{\mathbb{I}}+i\epsilon_{\alpha\beta\gamma}\hat{\sigma}^\gamma$ ($\alpha,\beta,\gamma\in\{x,y,z\}$, а $\epsilon_{\alpha\beta\gamma}$ -- символ Леви-Чивиты).
    \item Они удобно экспоненциируются: $\exp(ian_\alpha\hat{\sigma}^\alpha) = \cos a+in_\alpha\hat{\sigma}^\alpha\sin a$ (тут $\vec{n}$ -- произвольный единичный вектор). \textit{Указание}: разложите экспоненту в ряд; из-за простого правила произведения матриц Паули, произвольные степени отих линейных комбинаций вычисляются достаточно просто.
\end{enumerate}
\textbf{Решение.}\\
Матрицы Паули:
\begin{equation}
    \hat{\sigma}^x=\begin{pmatrix}
    0 & 1\\
    1 & 0
    \end{pmatrix},\quad \hat{\sigma}^y=\begin{pmatrix}
    0 & -i\\
    i & 0
    \end{pmatrix},\quad \hat{\sigma}^z=\begin{pmatrix}
    1 & 0\\
    0 & -1
    \end{pmatrix}
\end{equation}
\begin{enumerate}
    \item Проверим, что матрицы Паули являются линейно независимой системой от противного: предположим, что существуют $a_0$, $a_1$, $a_2$, $a_3 \in\mathbb{R}$:
    \begin{equation}
        a_0\hat{\sigma}^0+a_1\hat{\sigma}^1+a_2\hat{\sigma}^2+a_3\hat{\sigma}^3=0
    \end{equation}
    \begin{equation}
        \begin{pmatrix}
        a_0+a_3 & a_1-ia_2\\
        a_1+ia_2 & a_0-a_3\\
        \end{pmatrix}=\begin{pmatrix}
        0 & 0\\
        0 & 0\\
        \end{pmatrix}
    \end{equation}
    Единственное решение: $a_0=a_1=a_2=a_3=0$, значит матрицы Паули действительно линейно независимы. Покажем, что любая эрмитова матрица $2\times2$ $B=(b_i)$ лежит в их линейной оболочке:
    \begin{equation}
    \begin{pmatrix}
        a_0+a_3 & a_1-ia_2\\
        a_1+ia_2 & a_0-a_3\\
        \end{pmatrix}=\begin{pmatrix}
        b_1 & b_2\\
        b_3 & b_4\\
    \end{pmatrix}
    \end{equation}
    Заметим, что матрица слева (а значит и справа) является эрмитовой (на главной диагонали величины вещественные, на побочной -- комплексно-сопряжённые). Единственное решение: $a_0=\frac{b_1+b_4}{2}$, $a_1=\frac{b_2+b_3}{2}$, $a_2=\frac{i(b_2-b_3)}{2}$, $a_3=\frac{b_1-b_4}{2}$. Таким образом, матрицы Паули совместно с единичной матрицей образуют базис в пространстве эрмитовых матриц $2\times 2$.
\item Проверим соотношения на матрицы Паули:
\begin{equation}
    \sigma^1\sigma^1=\sigma^2\sigma^2=\sigma^3\sigma^3=\mathbb{I}
\end{equation}
\begin{equation}
    \sigma^1\sigma^2=-\sigma^2\sigma^1=-i\sigma^3,\quad \sigma^1\sigma^3=-\sigma^3\sigma^1=-i\sigma^2,\quad \sigma^2\sigma^3=-\sigma^3\sigma^2=-i\sigma^1
\end{equation}
Таким образом,
\begin{equation}
    \boxed{\hat{\sigma}^\alpha\hat{\sigma}^\beta=\delta_{\alpha\beta}\hat{\mathbb{I}}+i\epsilon_{\alpha\beta\gamma}\hat{\sigma}^\gamma}
\end{equation}
\item 
\begin{equation}
    \exp(ian_\alpha\hat{\sigma}^\alpha) = \sum\limits_{k=0}^\infty\frac{(ian_\alpha\hat{\sigma}^\alpha)^k}{k!}
\end{equation}
Найдём $(n_\alpha\hat{\sigma}^\alpha)^2$. Для этого воспользуемся правилом перемножения матриц Паули из п. 2:
\begin{equation}
    n_\alpha\hat{\sigma}^\alpha n_\beta\hat{\sigma}^\beta=\hat{\mathbb{I}}+i\epsilon^{\alpha\beta\gamma}n_\alpha n_\beta\hat{\sigma}_\gamma=\hat{\mathbb{I}}
\end{equation}
Из этого следует:
\begin{equation}
    (n_\alpha\hat{\sigma}^\alpha)^{2k}=\hat{\mathbb{I}},\quad (n_\alpha\hat{\sigma}^\alpha)^{2k+1}=n_\alpha\hat{\sigma}^\alpha, \quad k\in\{0\}\cup \mathbb{N}
\end{equation}
\begin{equation}
    \exp(ian_\alpha\hat{\sigma}^\alpha) = \sum\limits_{k=0}^\infty\frac{(-1)^ka^{2k}(n_\alpha\hat{\sigma}^\alpha)^{2k}}{(2k)!}+i\sum\limits_{k=0}^\infty\frac{(-1)^ka^{2k+1}(n_\alpha\hat{\sigma}^\alpha)^{2k+1}}{(2k+1)!}
\end{equation}
\begin{equation}
    \exp(ian_\alpha\hat{\sigma}^\alpha)=\mathbb{I}\sum\limits_{k=0}^\infty\frac{(-1)^ka^{2k}}{(2k)!}+in_\alpha\hat{\sigma}^\alpha\sum\limits_{k=0}^\infty\frac{(-1)^ka^{2k+1}}{(2k+1)!}
\end{equation}
\begin{equation}
    \boxed{\exp(ian_\alpha\hat{\sigma}^\alpha)=\mathbb{I}\cos a+in_\alpha\hat{\sigma}^\alpha\sin a}
\end{equation}
\end{enumerate}
\subsection*{Задачи (80 баллов)}
\textbf{Задача 1.$^*$ Осцилляция Раби (50 баллов).}\\
На двухуровневую систему накладывается периодическое поле, которое может вызывать переходы между этой парой уровней:
\begin{equation}
    \hat{H}(t)=
    \begin{pmatrix}
    \varepsilon_1 & Ve^{-i\omega t}\\
    Ve^{i\omega t} & \varepsilon_2
    \end{pmatrix}
\end{equation}
В начальный момент времени система находилась в состоянии $\ket{\psi(t=0)}=\ket{\uparrow}$. Определите вероятность обнаружить её в состоянии $\ket{\downarrow}$ через произвольное время $t$. Что происходит при резонансе, когда отстройка частоты $\delta\equiv\varepsilon_1-\varepsilon_2-\hbar\omega$ обращается в нуль?\\
\textit{Указание}: покажите, что от зависимости гамильтониана от времени можно избавиться <<переходом во вращающуюся систему отсчёта>> (rotating wave approximation) -- унитарным преобразованием (см. упражнения 2, 3) вида $\hat{U}(t) =e^{i\hat{\sigma}_z\omega_0t}$. Чему равна соответствующая частота $\omega_0$?\\
\textbf{Решение.}\\
Воспользуемся п. 3 из упражнения 3:
\begin{equation}
    U(t)=e^{i\hat{\sigma}_z\omega_0t}=\cos\omega_0t+i\sin\omega_0t\begin{pmatrix}
    1 & 0\\
    0 & -1
    \end{pmatrix}=\begin{pmatrix}
    e^{i\omega_0t} & 0\\
    0 & e^{-i\omega_0t}
    \end{pmatrix},\quad U^\dagger(t)=\begin{pmatrix}
    e^{-i\omega_0t} & 0\\
    0 & e^{i\omega_0t}
    \end{pmatrix}
\end{equation}
Воспользуемся п. 2 из упражнения 2:
\begin{equation}
    \hat{H}'=\hat{U}\hat{H}\hat{U}^\dagger-i\hbar\hat{U}\partial_t\hat{U}^\dagger
\end{equation}
\begin{equation}
    \hat{H}'=\begin{pmatrix}
    \varepsilon_1 & Ve^{i(2\omega_0-\omega)t}\\
    Ve^{-i(2\omega_0-\omega)t} & \varepsilon_2
    \end{pmatrix}-\begin{pmatrix}
    \hbar\omega_0 & 0\\
    0 & -\hbar\omega_0
    \end{pmatrix}=\begin{pmatrix}
    \varepsilon_1-\hbar\omega_0 & Ve^{i(2\omega_0-\omega)t}\\
    Ve^{-i(2\omega_0-\omega)t} & \varepsilon_2+\hbar\omega_0
    \end{pmatrix}
\end{equation}
Избавимся от зависимости от времени, подобрав соответствующее $\omega_0$:
\begin{equation}
    \boxed{\omega_0=\frac{\omega}{2}}
\end{equation}
\begin{equation}
    \hat{H}'=\begin{pmatrix}
    \varepsilon_1-\frac{\hbar\omega}{2} & V\\
    V & \varepsilon_2+\frac{\hbar\omega}{2}
    \end{pmatrix}
\end{equation}
При этом переходе собственные векторы гамильтониана меняются на фазу:
\begin{equation}
    \begin{pmatrix}
    \ket{1'}\\
    \ket{2'}
    \end{pmatrix}=\begin{pmatrix}
    e^{i\omega_0t} & 0\\
    0 & e^{-i\omega_0t}
    \end{pmatrix}\begin{pmatrix}
    \ket{1}\\
    \ket{2}
    \end{pmatrix}=\begin{pmatrix}
    e^{i\omega_0t}\ket{1}\\
    e^{-i\omega_0t}\ket{2}
    \end{pmatrix}
\end{equation}
Собственные числа дают спектр возможных значений энергии:
\begin{equation}
    \left(\varepsilon_1-\frac{\hbar\omega}{2}-E\right)\left(\varepsilon_2+\frac{\hbar\omega}{2}-E\right)-V^2=0
\end{equation}
\begin{equation}
    E_{1,2}=\frac{\varepsilon_1+\varepsilon_2\pm\sqrt{(\varepsilon_2-\varepsilon_1+\hbar\omega)^2+4V^2}}{2}
\end{equation}
Соответствующие нормированные собственные состояния системы:
\begin{equation}
    \ket{1'}=\frac{1}{C_1}\begin{pmatrix}
    \frac{\varepsilon_1-\varepsilon_2-\hbar\omega+\sqrt{(\varepsilon_2-\varepsilon_1+\hbar\omega)^2+4V^2}}{2V}\\
    1
    \end{pmatrix},\quad \ket{2'}=
    \frac{1}{C_2}\begin{pmatrix}
    \frac{\varepsilon_1-\varepsilon_2-\hbar\omega-\sqrt{(\varepsilon_2-\varepsilon_1+\hbar\omega)^2+4V^2}}{2V}\\
    1
    \end{pmatrix}
\end{equation}
\begin{equation}
    \ket{1}= e^{-i\omega_0t}\ket{1'}=\frac{e^{-i\omega_0t}}{C_1}\begin{pmatrix}
    \frac{\varepsilon_1-\varepsilon_2-\hbar\omega+\sqrt{(\varepsilon_2-\varepsilon_1+\hbar\omega)^2+4V^2}}{2V}\\
    1
    \end{pmatrix}=\frac{e^{-i\omega_0t}}{C_1}\begin{pmatrix}
    a\\
    1
    \end{pmatrix}=e^{-i\omega_0t}\frac{a\ket{\uparrow}+\ket{\downarrow}}{C_1}
\end{equation}
\begin{equation}
    \ket{2}=e^{i\omega_0t}\ket{2'}=
    \frac{e^{i\omega_0t}}{C_2}\begin{pmatrix}
    \frac{\varepsilon_1-\varepsilon_2-\hbar\omega-\sqrt{(\varepsilon_2-\varepsilon_1+\hbar\omega)^2+4V^2}}{2V}\\
    1
    \end{pmatrix}=\frac{e^{i\omega_0t}}{C_2}\begin{pmatrix}
    b\\
    1
    \end{pmatrix}=e^{i\omega_0t}\frac{b\ket{\uparrow}+\ket{\downarrow}}{C_2}
\end{equation}
Разложим начальное условие по стационарным состояниям:
\begin{equation}
    \ket{\psi(0)}=\ket{\uparrow}=\frac{C_1e^{i\omega_0t}\ket{1}-C_2e^{-i\omega_0t}\ket{2}}{a-b}
\end{equation}
Дальнейшая эволюция:
\begin{equation}
    \ket{\psi(t)}=\frac{C_1e^{i\omega_0t}e^{-\frac{iE_1t}{\hbar}}\ket{1}-C_2e^{-i\omega_0t}e^{-\frac{iE_2t}{\hbar}}\ket{2'}}{a-b}=\frac{e^{-\frac{iE_1t}{\hbar}}(a\ket{\uparrow}+\ket{\downarrow})-e^{-\frac{iE_2t}{\hbar}}(b\ket{\uparrow}+\ket{\downarrow})}{a-b}
\end{equation}
\begin{equation}
    \ket{\psi(t)}=\frac{e^{-\frac{i(\varepsilon_1+\varepsilon_2)t}{2\hbar}}}{a-b}\left(e^{-\frac{i\sqrt{(\varepsilon_2-\varepsilon_1+\hbar\omega)^2+4V^2}t}{2\hbar}}(a\ket{\uparrow}+\ket{\downarrow})-e^{\frac{i\sqrt{(\varepsilon_2-\varepsilon_1+\hbar\omega)^2+4V^2}t}{2\hbar}}(b\ket{\uparrow}+\ket{\downarrow})\right)
\end{equation}
\begin{equation}
    \psi_\downarrow(t)=\braket{\downarrow|\psi(t)}=\frac{e^{-\frac{i(\varepsilon_1+\varepsilon_2)t}{2\hbar}}}{a-b}\left(e^{-\frac{i\sqrt{(\varepsilon_2-\varepsilon_1+\hbar\omega)^2+4V^2}t}{2\hbar}}-e^{\frac{i\sqrt{(\varepsilon_2-\varepsilon_1+\hbar\omega)^2+4V^2}t}{2\hbar}}\right)
\end{equation}
Вероятность обнаружить систему в состоянии $\ket{\downarrow}$ через время $t$:
\begin{equation}
    P_\downarrow(t)=|\psi_\downarrow(t)|^2=\frac{4\sin^2\left(\frac{\sqrt{(\varepsilon_2-\varepsilon_1+\hbar\omega)^2+4V^2}t}{2\hbar}\right)}{(a-b)^2}
\end{equation}
\begin{equation}
    \boxed{P_\downarrow(t)=\frac{4V^2\sin^2\left(\frac{\sqrt{(\varepsilon_2-\varepsilon_1+\hbar\omega)^2+4V^2}t}{2\hbar}\right)}{(\varepsilon_2-\varepsilon_1+\hbar\omega)^2+4V^2}}
\end{equation}
Случай резонанса:
\begin{equation}
    \delta\equiv\varepsilon_1-\varepsilon_2-\hbar\omega=0
\end{equation}
\begin{equation}
    \boxed{P_\downarrow(t)=\sin^2\frac{Vt}{\hbar}}
\end{equation}\\
\textbf{Задача 2. Два спина (30 баллов).}\\
Найдите уровни энергии и собственные состояния для следующего гамильтониана, описывающего систему двух взаимодействующих спинов 1/2:
\begin{equation}
    \hat{H}=-J(\vec{\sigma}_1\otimes\vec{\sigma}_2)=-J(\hat{\sigma}_1^x\otimes\hat{\sigma}_2^x+\hat{\sigma}_1^y\otimes\hat{\sigma}_2^y+\hat{\sigma}_1^z\otimes\hat{\sigma}_2^z)
\end{equation}
\textbf{Решение.}\\
\begin{equation}
    \hat{H}=-J\left(\begin{pmatrix}
    0 & 0 & 0 & 1\\
    0 & 0 & 1 & 0\\
    0 & 1 & 0 & 0\\
    1 & 0 & 0 & 0
    \end{pmatrix}+\begin{pmatrix}
    0 & 0 & 0 & -1\\
    0 & 0 & 1 & 0\\
    0 & 1 & 0 & 0\\
    -1 & 0 & 0 & 0
    \end{pmatrix}+\begin{pmatrix}
    1 & 0 & 0 & 0\\
    0 & -1 & 0 & 0\\
    0 & 0 & -1 & 0\\
    0 & 0 & 0 & 1
    \end{pmatrix}\right)
\end{equation}
\begin{equation}
    \hat{H}=J\begin{pmatrix}
    1 & 0 & 0 & 0\\
    0 & -1 & 2 & 0\\
    0 & 2 & -1 & 0\\
    0 & 0 & 0 & 1
    \end{pmatrix}
\end{equation}
Собственные числа дают спектр возможных значений энергии:
\begin{equation}
    \boxed{E_1=3J,\quad E_{2,3,4}=-J}
\end{equation}
Соответствующие нормированные собственные состояния системы:
\begin{equation}
    \boxed{\ket{1}=\frac{1}{\sqrt{2}}\begin{pmatrix}
    0\\
    -1\\
    1\\
    0
    \end{pmatrix}=\frac{1}{\sqrt{2}}(\ket{\downarrow\uparrow}-\ket{\uparrow\downarrow}),\quad \ket{2}=\begin{pmatrix}
    0\\
    0\\
    0\\
    1
    \end{pmatrix}=\ket{\downarrow\downarrow}}
\end{equation}
\begin{equation}
    \boxed{\ket{3}=\frac{1}{\sqrt{2}}\begin{pmatrix}
    0\\
    1\\
    1\\
    0
    \end{pmatrix}=\frac{1}{\sqrt{2}}(\ket{\downarrow\uparrow}+\ket{\uparrow\downarrow}),\quad \ket{4}=\begin{pmatrix}
    1\\
    0\\
    0\\
    0
    \end{pmatrix}=\ket{\uparrow\uparrow}}
\end{equation}
\section{Матрица плотности}
\subsection*{Упражнения (20 баллов)}
\textbf{Упражнение 1. (10 баллов).}\\
Вычислите среднее значение спина $\braket{\vec{S}}$ и его дисперсию $\braket{(\vec{S}-\braket{\vec{S}})^2}$ для чистого $\ket{\chi}=\frac{\ket{\uparrow}+\ket{\downarrow}}{\sqrt{2}}$ и смешанного $\hat{\rho}=\frac{\hat{\mathbb{P}}_\uparrow+\hat{\mathbb{P}}_\downarrow}{2}$ состояний спина 1/2. \textit{Комментарий}: первая величина -- это вектор, а вторая -- это скаляр, длина вектора. \textit{Указание}: Для частицы со спином 1/2 (например, электрон) \textit{оператор спина} (собственного момента) равен $\hat{\vec{S}} = \frac{\hbar}{2}\hat{\vec{\sigma}}$ (то есть $\hat{S}_x = (\hbar/2)\hat{\sigma}_x$ и т. д.).\\
\textbf{Решение.}\\
Решим упражнение для чистого состояния:
\begin{equation}
    \ket{\chi}\bra{\chi}=\frac{1}{2}(\ket{\uparrow}\bra{\uparrow}+\ket{\uparrow}\bra{\downarrow}+\ket{\downarrow}\bra{\uparrow}+\ket{\downarrow}\bra{\downarrow})=\frac{1}{2}\begin{pmatrix}
    1 & 1\\
    1 & 1\\
    \end{pmatrix}
\end{equation}
\begin{equation}
    \braket{\hat{S}_x}=\text{Tr}(\ket{\chi}\bra{\chi}\hat{S}_x)=\frac{\hbar}{4}\text{Tr}\left(\begin{pmatrix}
    1 & 1\\
    1 & 1\\
    \end{pmatrix}\begin{pmatrix}
    0 & 1\\
    1 & 0\\
    \end{pmatrix}\right)=\frac{\hbar}{2}
\end{equation}
\begin{equation}
    \braket{\hat{S}_y}=\text{Tr}(\ket{\chi}\bra{\chi}\hat{S}_y)=\frac{\hbar}{4}\text{Tr}\left(\begin{pmatrix}
    1 & 1\\
    1 & 1\\
    \end{pmatrix}\begin{pmatrix}
    0 & -i\\
    i & 0\\
    \end{pmatrix}\right)=0
\end{equation}
\begin{equation}
    \braket{\hat{S}_z}=\text{Tr}(\ket{\chi}\bra{\chi}\hat{S}_z)=\frac{\hbar}{4}\text{Tr}\left(\begin{pmatrix}
    1 & 1\\
    1 & 1\\
    \end{pmatrix}\begin{pmatrix}
    1 & 0\\
    0 & -1\\
    \end{pmatrix}\right)=0
\end{equation}
Среднее значение спина:
\begin{equation}
    \boxed{\braket{\vec{S}}=\frac{\hbar}{2}\begin{pmatrix}
    1\\
    0\\
    0\\
    \end{pmatrix}}
\end{equation}
\begin{equation}
    \vec{S}^2=\frac{\hbar^2}{4}(\sigma^2_x+\sigma^2_y+\sigma^2_z)=\frac{3\hbar^2}{4}
\end{equation}
\begin{equation}
    \braket{\vec{S}}^2=\frac{3\hbar^2}{4}
\end{equation}
Дисперсия:
\begin{equation}
    \boxed{\braket{\vec{S}^2}-\braket{\vec{S}}^2=\frac{\hbar^2}{2}}
\end{equation}
Решим упражнение для смешанного состояния:
\begin{equation}
    \hat{\rho}=\frac{\hat{\mathbb{P}}_\uparrow+\hat{\mathbb{P}}_\downarrow}{2}=\frac{1}{2}(\ket{\uparrow}]\bra{\uparrow}+\ket{\downarrow}]\bra{\downarrow})=\frac{1}{2}\begin{pmatrix}
    1 & 0\\
    0 & 1\\
    \end{pmatrix}
\end{equation}
\begin{equation}
    \braket{\hat{S}_x}=\text{Tr}(\ket{\chi}\bra{\chi}\hat{S}_x)=\frac{\hbar}{4}\text{Tr}\left(\begin{pmatrix}
    1 & 0\\
    0 & 1\\
    \end{pmatrix}\begin{pmatrix}
    0 & 1\\
    1 & 0\\
    \end{pmatrix}\right)=0
\end{equation}
\begin{equation}
    \braket{\hat{S}_y}=\text{Tr}(\ket{\chi}\bra{\chi}\hat{S}_y)=\frac{\hbar}{4}\text{Tr}\left(\begin{pmatrix}
    1 & 0\\
    0 & 1\\
    \end{pmatrix}\begin{pmatrix}
    0 & -i\\
    i & 0\\
    \end{pmatrix}\right)=0
\end{equation}
\begin{equation}
    \braket{\hat{S}_z}=\text{Tr}(\ket{\chi}\bra{\chi}\hat{S}_z)=\frac{\hbar}{4}\text{Tr}\left(\begin{pmatrix}
    1 & 0\\
    0 & 1\\
    \end{pmatrix}\begin{pmatrix}
    1 & 0\\
    0 & -1\\
    \end{pmatrix}\right)=0
\end{equation}
Среднее значение спина:
\begin{equation}
    \boxed{\braket{\vec{S}}=0}
\end{equation}
\begin{equation}
    \vec{S}^2=\frac{\hbar^2}{4}(\sigma^2_x+\sigma^2_y+\sigma^2_z)=\frac{3\hbar^2}{4}
\end{equation}
\begin{equation}
    \braket{\vec{S}}^2=0
\end{equation}
Дисперсия:
\begin{equation}
    \boxed{\braket{\vec{S}^2}-\braket{\vec{S}}^2=\frac{3\hbar^2}{4}}
\end{equation}
\textbf{Упражнение 2. (10 баллов).}\\
Предположим, наблюдатель C хочет придумать эксперимент, который сможет отличить состояние «ЭПР» от классически запутанного («Носки») (из семинара). Какую наблюдаемую ему нужно придумать? Выразите её через матрицы Паули (и их тензорные произведения)?\\
\textbf{Решение.}\\
Наблюдаемая -- гамильтониан из задачи 2 семинара 1:
\begin{equation}
    \hat{H}=-J(\hat{\sigma}_1^x\otimes\hat{\sigma}_2^x+\hat{\sigma}_1^y\otimes\hat{\sigma}_2^y+\hat{\sigma}_1^z\otimes\hat{\sigma}_2^z)
\end{equation}
Для <<ЭПР>>:
\begin{equation}
    \text{tr}(\hat{\rho}_{AB}\hat{H})=\text{tr}\left(\frac{J}{2}\begin{pmatrix}
    0 & 0 & 0 & 0\\
    0 & 1 & 1 & 0\\
    0 & 1 & 1 & 0\\
    0 & 0 & 0 & 0\\
    \end{pmatrix}\begin{pmatrix}
    1 & 0 & 0 & 0\\
    0 & -1 & 2 & 0\\
    0 & 2 & -1 & 0\\
    0 & 0 & 0 & 1\\
    \end{pmatrix}\right)=\text{tr}\left(\frac{1}{2}\begin{pmatrix}
    0 & 0 & 0 & 0\\
    0 & 1 & 1 & 0\\
    0 & 1 & 1 & 0\\
    0 & 0 & 0 & 0\\
    \end{pmatrix}\right)=J
\end{equation}
Для <<Носок>>:
\begin{equation}
    \text{tr}(\hat{\rho}_{AB}\hat{H})=\text{tr}\left(\frac{J}{2}\begin{pmatrix}
    0 & 0 & 0 & 0\\
    0 & 1 & 0 & 0\\
    0 & 0 & 1 & 0\\
    0 & 0 & 0 & 0\\
    \end{pmatrix}\begin{pmatrix}
    1 & 0 & 0 & 0\\
    0 & -1 & 2 & 0\\
    0 & 2 & -1 & 0\\
    0 & 0 & 0 & 1\\
    \end{pmatrix}\right)=\text{tr}\left(\frac{J}{2}\begin{pmatrix}
    0 & 0 & 0 & 0\\
    0 & -1 & 2 & 0\\
    0 & 2 & -1 & 0\\
    0 & 0 & 0 & 0\\
    \end{pmatrix}\right)=-J
\end{equation}
Как видно, следы не совпадают. Искомая наблюдаемая:
\begin{equation}
    \boxed{\hat{H}=-J(\hat{\sigma}_1^x\otimes\hat{\sigma}_2^x+\hat{\sigma}_1^y\otimes\hat{\sigma}_2^y+\hat{\sigma}_1^z\otimes\hat{\sigma}_2^z)}
\end{equation}
\subsection*{Задачи (80 баллов)}
\textbf{Задача 1. Блоховское представление двухуровневой системы (10 баллов).}
\begin{enumerate}
    \item Покажите, что матрицу плотности произвольной двухуровневой системы самого общего вида можно разложить по матрицам Паули в следующем виде:
    \begin{equation}
        \hat{\rho}=\frac{1}{2}(\hat{\mathbb{I}}+\hat{\bm{\sigma}}\cdot\bm{n})
    \end{equation}
    где $\bm{n}$ -- какой-то вектор. При каком условии на $\bm{n}$ это корректная матрица плотности?
    \item При каком условии на $\bm{n}$, эта матрица плотности описывает чистое состояние?
    \item Вычислите средние значения $\braket{\hat{\sigma}_{x,y,z}}$ по состоянию, описываемому такой матрицей плотности.
\end{enumerate}
\textbf{Решение.}\\
Пусть с вероятностью $p_\alpha$ система может быть приготовлена в состоянии $\ket{\alpha}$. Тогда для такого усреднения мы можем записать:
    \begin{equation}
        \braket{\hat{O}}=\sum\limits_\alpha p_\alpha\braket{\alpha|\hat{O}|\alpha}=\text{Tr}(\ket{\alpha}p_\alpha\bra{\alpha}\hat{O})\equiv\text{Tr}(\hat{\rho}\hat{O})
    \end{equation}
    Матрица плотности:
    \begin{equation}
        \hat{\rho}=\sum\limits_\alpha p_\alpha\ket{\alpha}\bra{\alpha}
    \end{equation}
    Матрица плотности -- эрмитов оператор $\hat{\rho}=\hat{\rho}^\dagger$ с нормировкой $\text{Tr}\hat{\rho}=\sum\limits_\alpha p_\alpha=1$.\\
    Удобно диагонализовать матрицу плотности (любая эрмитова матрица диагонализуема с вещественными собственными числами $\lambda_n$):
    \begin{equation}
        \hat{\rho}\ket{n}=\lambda_n\ket{n},\quad
    \end{equation}
    \begin{equation}
        \text{Tr}\hat{\rho}=\sum\limits_n \lambda_n=1
    \end{equation}
    Матрица плотности неотрицательно определена:
    \begin{equation}
        \braket{\psi|\hat{\rho}|\psi}=\sum\limits_\alpha p_\alpha\braket{\psi|\alpha}\braket{\alpha|\psi}=\sum\limits_\alpha p_\alpha\braket{\psi|\alpha}\braket{\alpha|\psi}=\sum\limits_\alpha p_\alpha|\braket{\alpha|\psi}|^2\geq0
    \end{equation}
    Следовательно, все $\lambda_n\geq0$. А поскольку $\sum\limits_n\lambda_n=1$, то все $\lambda_n\in[0,1]$. Собственные числа $\lambda_n$ являются вероятностями оказаться в состоянии $\ket{n}$.
\begin{enumerate}
    \item Матрицы Паули совместно с единичной матрицей $\sigma^0=\hat{\mathbb{I}}_{2\times2}$ представляют собой базис в пространстве эрмитовых матриц $2\times2$ (см. упр. 3 недели 1). Поскольку $\text{Tr}\hat{\rho}=1$ и $\text{Tr}\sigma_i=0$, то коэффициент перед $\mathbb{I}$ равен $\frac{1}{2}$:
    \begin{equation}
        \hat{\rho}=\frac{1}{2}(\hat{\mathbb{I}}+\hat{\vec{\sigma}}\cdot\vec{n})=\frac{1}{2}\begin{pmatrix}
            1+n_z & n_x-in_y\\
            n_x+in_y & 1-n_z
        \end{pmatrix}
    \end{equation}
    \begin{equation}
        \det\hat{\rho}=\frac{1}{2}(1-n_z^2-n_x^2-n_y^2)=\frac{1}{2}(1-\bm{n}^2)
    \end{equation}
    \begin{equation}
        \det\hat{\rho}=\prod\limits_n\lambda_n\geq0\rightarrow\boxed{|\bm{n}|\leq1}
    \end{equation}
    Таким образом, конец вектора $\bm{n}$ лежит в шаре радиуса 1.
    \item Квантовая система, описываемая какой-то волновой функцией $\ket{\psi}$ (с вероятностью $p = 1$) можно описать на языке матрицы плотности
    \begin{equation}
        \hat{\rho}=\ket{\psi}\bra{\psi}
    \end{equation}
    Состояние, описываемое такой матрицей плотности, носит название \textit{чистого}. В противном случае состояние называется \textit{смешанным}.\\
    Матрица плотности чистого состояния является проектором:
    \begin{equation}
        \hat{\rho}^2=\ket{\psi}\braket{\psi|\psi}\bra{\psi}=\ket{\psi}\bra{\psi}=\hat{\rho}
    \end{equation}
    \begin{equation}
        \text{Tr}\hat{\rho}^2=\text{Tr}\hat{\rho}=1
    \end{equation}
    В общем случае
    \begin{equation}
        \text{Tr}\hat{\rho}^2=\sum\limits_\alpha p^2_\alpha\leq\sum\limits_\alpha p_\alpha=\text{Tr}\hat{\rho}=1
    \end{equation}
    В случае чистого состояния одно из собственных значений равно 1, остальные равны 0.
    \begin{equation}
        \det\hat{\rho}=\frac{1}{2}(1-|\bm{n}|^2)=0\rightarrow\boxed{|\bm{n}|=1}
    \end{equation}
    Таким образом, конец вектора $\bm{n}$ лежит на сфере радиуса 1.
    \item Среднее значение по состоянию:
    \begin{equation}
        \braket{\sigma_i}=\text{Tr}(\rho\sigma_i)=\frac{1}{2}\text{Tr}(\hat{\sigma}_i+\hat{\sigma_i}^2n_i)=n_i
    \end{equation}
    \begin{equation}
        \boxed{\braket{\sigma_{x,y,z}}=n_{x,y,z}}
    \end{equation}
\end{enumerate}
\textbf{Задача 2. Термодинамика двухуровневой системы (10 баллов).}\\
Двухуровневая система описывается гамильтонианом $\hat{H}= -\bm{h}\cdot\hat{\bm{\sigma}}$, где $\bm{h} = (h_x, h_y, h_z)$, и находится при температуре $T$. Вычислите средние значения $\hat{\sigma}_{x,y,z}$. \textit{Указание}: воспользуйтесь результатом упражнений из первого семинара.\\
\textbf{Решение.}\\
Гиббсовский (канонический) ансамбль:
\begin{equation}
    \hat{\rho}=\frac{1}{Z}e^{-\beta\hat{H}},\quad Z=\text{Tr}(e^{-\beta\hat{H}})=\text{Tr}(e^{\beta\vec{h}\cdot\hat{\vec{\sigma}}})
\end{equation}
Воспользуемся упр. 3, п. 3 семинара 1 (там был случай $h=1$, но он очевидным образом обобщается на случай $h\neq1$):
\begin{equation}
    e^{\beta\bm{h}\cdot\hat{\bm{\sigma}}}=\mathbb{I}\cosh{\beta h}+\frac{\sinh{\beta h}}{h}\bm{h}\cdot\hat{\bm{\sigma}}
\end{equation}
\begin{equation}
    Z=\text{Tr}\left(\mathbb{I}\cosh{\beta h}+\frac{\sinh{\beta h}}{h}\bm{h}\cdot\hat{\bm{\sigma}}\right)=2\cosh\beta h
\end{equation}
\begin{equation}
    \hat{\rho}=\frac{\mathbb{I}\cosh{\beta h}+\frac{\sinh{\beta h}}{h}\bm{h}\cdot\hat{\bm{\sigma}}}{2\cosh\beta h}=\frac{1}{2}\mathbb{I}+\frac{1}{2h}\bm{h}\cdot\hat{\bm{\sigma}}\tanh\beta h=\frac{1}{2}
    \begin{pmatrix}
    1+\frac{h_z}{h}\tanh\beta h & \frac{h_x-ih_y}{h}\tanh\beta h\\
    \frac{h_x+ih_y}{h}\tanh\beta h & 1+\frac{h_z}{h}\tanh\beta h\\
    \end{pmatrix}
\end{equation}
\begin{equation}
    \braket{\hat{\sigma}_\alpha}=\text{Tr}(\hat{\rho}\sigma_\alpha)=\frac{1}{2}\text{Tr}\left(\hat{\sigma}_\alpha+\frac{h_\alpha\tanh\beta h}{h}\mathbb{I}\right)=\frac{h_\alpha}{h}\tanh\beta h
\end{equation}
\begin{equation}
    \boxed{\braket{\hat{\sigma}_x}=\frac{h_x}{h}\tanh\beta h,\quad\braket{\hat{\sigma}_y}=\frac{h_y}{h}\tanh\beta h,\quad\braket{\hat{\sigma}_z}=\frac{h_z}{h}\tanh\beta h}
\end{equation}
\textbf{Задача 3.$^*$ Квантовый парадокс Зенона (30 баллов).}\\
Рассмотрите двухуровневую систему, описываемую следующим гамильтонианом:
\begin{equation}
    \hat{H}=\begin{pmatrix}
    E_0 & -\Delta\\
    -\Delta & E_0
    \end{pmatrix}
\end{equation}
В начальный момент система приготовлена в состоянии $\ket{\psi(0)} = \ket{\uparrow}$. Если бы мы позволили системе эволюционировать самой по себе, то она бы совершала осцилляции Раби; в частности, через время $T = \frac{\pi\hbar}{2\Delta}$ мы бы обнаружили её в состоянии $\ket{\downarrow}$ с вероятностью $P_\downarrow(T) = 1$. Однако теперь вместо этого через каждый промежуток времени $\tau\ll T$ мы проводим измерение наблюдаемой $\hat{\sigma}^z$. Определите вероятность $P_\downarrow(T)$ в таком случае.\\
\textbf{Решение.}\\
Волновая функция в начальный момент:
\begin{equation}
    \ket{\psi(0)}=\ket{\uparrow}
\end{equation}
Матрица плотности в начальный момент:
\begin{equation}
    \hat{\rho}(0)=\ket{\psi(0)}\bra{\psi(0)}=\begin{pmatrix}
    1 & 0\\
    0 & 0\\
    \end{pmatrix}
\end{equation}
Оператор эволюции:
\begin{equation}
    \hat{U}(t)=\exp\left(-\frac{i\hat{H}t}{\hbar}\right)=\exp\left(\frac{it}{\hbar}(\Delta\hat{\sigma}_x-E_0\hat{\mathbb{I}})\right)=\exp\left(\frac{it}{\hbar}\Delta\hat{\sigma}_x\right)\exp\left(-\frac{it}{\hbar}E_0\right)
\end{equation}
Воспользуемся упражнением 3 семинара 1:
\begin{equation}
    \exp\left(\frac{it}{\hbar}\Delta\hat{\sigma}_x\right)=\hat{\mathbb{I}}\cos\frac{\Delta t}{\hbar}+i\hat{\sigma}_x\sin\frac{\Delta t}{\hbar}
\end{equation}
\begin{equation}
    \hat{U}(t)=\begin{pmatrix}
    \cos\frac{\Delta t}{\hbar} & i\sin\frac{\Delta t}{\hbar}\\
    i\sin\frac{\Delta t}{\hbar} & \cos\frac{\Delta t}{\hbar}
    \end{pmatrix}\exp\left(-\frac{it}{\hbar}E_0\right)
\end{equation}
Действие оператора эволюции:
\begin{equation}
    \ket{\psi(t)}=\hat{U}(t)\ket{\psi(0)},\quad \bra{\psi(t)}=\bra{\psi(0)}\hat{U}^\dagger(t)
\end{equation}
\begin{multline}
    \hat{\rho}(t)=\hat{U}(t)\ket{\psi(0)}\bra{\psi(0)}\hat{U}^\dagger(t)=\hat{U}(t)\hat{\rho}(0)\hat{U}^\dagger(t)=\\=\begin{pmatrix}
    \cos\frac{\Delta t}{\hbar} & i\sin\frac{\Delta t}{\hbar}\\
    i\sin\frac{\Delta t}{\hbar} & \cos\frac{\Delta t}{\hbar}
    \end{pmatrix}\begin{pmatrix}
    1 & 0\\
    0 & 0\\
    \end{pmatrix}\begin{pmatrix}
    \cos\frac{\Delta t}{\hbar} & -i\sin\frac{\Delta t}{\hbar}\\
    -i\sin\frac{\Delta t}{\hbar} & \cos\frac{\Delta t}{\hbar}
    \end{pmatrix}=\begin{pmatrix}
    \cos^2\frac{\Delta t}{\hbar} & -\frac{i}{2}\sin\frac{2\Delta t}{\hbar}\\
    \frac{i}{2}\sin\frac{2\Delta t}{\hbar} & \sin^2\frac{\Delta t}{\hbar}
    \end{pmatrix}
\end{multline}
После измерения происходит редукция фон-Неймана, приводящая к дефазировке:
\begin{equation}
    \hat{\rho}(\tau)=\begin{pmatrix}
    \cos^2\frac{\Delta\tau}{\hbar} & 0\\
    0 & \sin^2\frac{\Delta\tau}{\hbar}
    \end{pmatrix}
\end{equation}
Вероятность направлений спина после 1 измерения:
\begin{equation}
    P_\uparrow(\tau)=\rho_{11}=\cos^2\frac{\Delta\tau}{\hbar},\quad P_\downarrow(\tau)=\rho_{22}=\sin^2\frac{\Delta\tau}{\hbar}
\end{equation}
Те же вероятности были получены в разделе 1.4 без матрицы плотности. Найдём марицу плотности $\hat{\rho}(n\tau)$ через $\hat{\rho}((n-1)\tau)$.
\begin{multline*}
    \hat{U}(\tau)\hat{\rho}((n-1)\tau)\hat{U}^\dagger(\tau)=\begin{pmatrix}
    \cos\frac{\Delta t}{\hbar} & i\sin\frac{\Delta t}{\hbar}\\
    i\sin\frac{\Delta t}{\hbar} & \cos\frac{\Delta t}{\hbar}
    \end{pmatrix}\begin{pmatrix}
    P_\uparrow((n-1)\tau) & 0\\
    0 & P_\downarrow((n-1)\tau)\\
    \end{pmatrix}\begin{pmatrix}
    \cos\frac{\Delta t}{\hbar} & -i\sin\frac{\Delta t}{\hbar}\\
    -i\sin\frac{\Delta t}{\hbar} & \cos\frac{\Delta t}{\hbar}
    \end{pmatrix}=\\=\begin{pmatrix}
    P_\uparrow((n-1)\tau)\cos^2\frac{\Delta t}{\hbar}+P_\downarrow((n-1)\tau)\sin^2\frac{\Delta t}{\hbar} & -\frac{i}{2}(P_\uparrow((n-1)\tau)-P_\downarrow((n-1)\tau))\sin\frac{2\Delta t}{\hbar}\\
    \frac{i}{2}(P_\uparrow((n-1)\tau)-P_\downarrow((n-1)\tau))\sin\frac{2\Delta t}{\hbar} & P_\uparrow((n-1)\tau)\sin^2\frac{\Delta t}{\hbar}+P_\downarrow((n-1)\tau)\cos^2\frac{\Delta t}{\hbar}\\
    \end{pmatrix}
\end{multline*}
После дефазировки:
\begin{equation*}
    \hat{\rho}(n\tau)=\begin{pmatrix}
    P_\uparrow((n-1)\tau)\cos^2\frac{\Delta t}{\hbar}+P_\downarrow((n-1)\tau)\sin^2\frac{\Delta t}{\hbar} & 0\\
    0 & P_\uparrow((n-1)\tau)\sin^2\frac{\Delta t}{\hbar}+P_\downarrow((n-1)\tau)\cos^2\frac{\Delta t}{\hbar}\\
    \end{pmatrix}
\end{equation*}
Вероятность направлений спина после $n$ измерений:
\begin{equation}
    P_\uparrow(n\tau)=P_\downarrow((n-1)\tau)\sin^2\frac{\Delta\tau}{\hbar}+P_\uparrow((n-1)\tau)\cos^2\frac{\Delta\tau}{\hbar}
\end{equation}
\begin{equation}
    P_\downarrow(n\tau)=P_\downarrow((n-1)\tau)\cos^2\frac{\Delta\tau}{\hbar}+P_\uparrow((n-1)\tau)\sin^2\frac{\Delta\tau}{\hbar}
\end{equation}
Таким образом, нужно решить векторное рекуррентное уравнение (или, что эквивалентно, систему уравнений):
\begin{equation}
    \begin{pmatrix}
    P_\uparrow(n\tau)\\
    P_\downarrow(n\tau)
    \end{pmatrix}=
    \begin{pmatrix}
        \cos^2\frac{\Delta\tau}{\hbar} & \sin^2\frac{\Delta\tau}{\hbar}\\
        \sin^2\frac{\Delta\tau}{\hbar} & \cos^2\frac{\Delta\tau}{\hbar}\\
    \end{pmatrix}
    \begin{pmatrix}
    P_\uparrow((n-1)\tau)\\
    P_\downarrow((n-1)\tau)
    \end{pmatrix},\quad\begin{pmatrix}
    P_\uparrow(0)\\
    P_\downarrow(0)
    \end{pmatrix}=\begin{pmatrix}
    1\\
    0
\end{pmatrix}
\end{equation}
Решение этого уравнения:
\begin{equation}
    \begin{pmatrix}
    P_\uparrow(n\tau)\\
    P_\downarrow(n\tau)
    \end{pmatrix}=\begin{pmatrix}
        \cos^2\frac{\Delta\tau}{\hbar} & \sin^2\frac{\Delta\tau}{\hbar}\\
        \sin^2\frac{\Delta\tau}{\hbar} & \cos^2\frac{\Delta\tau}{\hbar}\\
    \end{pmatrix}^n
    \begin{pmatrix}
    1\\
    0
    \end{pmatrix}
\end{equation}
Диагонализуем матрицу:
\begin{equation}
    \begin{pmatrix}
        \cos^2\frac{\Delta\tau}{\hbar} & \sin^2\frac{\Delta\tau}{\hbar}\\
        \sin^2\frac{\Delta\tau}{\hbar} & \cos^2\frac{\Delta\tau}{\hbar}\\
    \end{pmatrix}=\begin{pmatrix}
        1 & -1\\
        1 & 1\\
    \end{pmatrix}\begin{pmatrix}
        1 & 0\\
        0 & \cos\frac{2\Delta\tau}{\hbar}\\
    \end{pmatrix}\frac{1}{2}\begin{pmatrix}
        1 & 1\\
        -1 & 1\\
    \end{pmatrix}
\end{equation}
\begin{equation*}
    \begin{pmatrix}
        \cos^2\frac{\Delta\tau}{\hbar} & \sin^2\frac{\Delta\tau}{\hbar}\\
        \sin^2\frac{\Delta\tau}{\hbar} & \cos^2\frac{\Delta\tau}{\hbar}\\
    \end{pmatrix}^n=\begin{pmatrix}
        1 & -1\\
        1 & 1\\
    \end{pmatrix}\begin{pmatrix}
        1 & 0\\
        0 & \cos^n\frac{2\Delta\tau}{\hbar}\\
    \end{pmatrix}\frac{1}{2}\begin{pmatrix}
        1 & 1\\
        -1 & 1\\
    \end{pmatrix}=\frac{1}{2}\begin{pmatrix}
        1+\cos^n\frac{2\Delta\tau}{\hbar} & 1-\cos^n\frac{2\Delta\tau}{\hbar}\\
        1-\cos^n\frac{2\Delta\tau}{\hbar} & 1+\cos^n\frac{2\Delta\tau}{\hbar}\\
    \end{pmatrix}
    \end{equation*}
    \begin{equation}
    \begin{pmatrix}
    P_\uparrow(n\tau)\\
    P_\downarrow(n\tau)
    \end{pmatrix}=\frac{1}{2}\begin{pmatrix}
        1+\cos^n\frac{2\Delta\tau}{\hbar} & 1-\cos^n\frac{2\Delta\tau}{\hbar}\\
        1-\cos^n\frac{2\Delta\tau}{\hbar} & 1+\cos^n\frac{2\Delta\tau}{\hbar}\\
    \end{pmatrix}
    \begin{pmatrix}
    1\\
    0
    \end{pmatrix}=\frac{1}{2}\begin{pmatrix}
        1+\cos^n\frac{2\Delta\tau}{\hbar}\\ 1-\cos^n\frac{2\Delta\tau}{\hbar}\\
    \end{pmatrix}
\end{equation}
Учитывая, что $T=n\tau$, получим
\begin{equation}
    P_\downarrow(T)=\frac{1}{2}\left(1-\cos^\frac{T}{\tau}\frac{2\Delta\tau}{\hbar}\right)
\end{equation}
По формуле Тейлора до 2 порядка:
\begin{equation}
    \boxed{P_\downarrow(T)=\frac{\Delta^2}{\hbar^2}T\tau}
\end{equation}
\textbf{Задача 4.$^*$ Дефазировка (30 баллов).}\\
Спин-$1/2$ находится в магнитном поле, направленном вдоль оси $z$:
\begin{equation}
    \hat{H}=-B\hat{\sigma}^z
\end{equation}
В результате взаимодействия с окружающей средой, магнитное поле испытывает случайные флуктуации: $B \equiv B_0 +\delta B(t)$, которые мы предполагаем гауссовыми с нулевым средним и коррелятором $\braket{\delta B(t)\delta B(t')} = \Gamma \delta(t-t')$. Пусть в начальный момент времени, матрица плотности была самого общего вида:
\begin{equation}
    \hat{\rho}(0)=\begin{pmatrix}
    \rho_{11} & \rho_{12}\\
    \rho_{21} & \rho_{22}\\
    \end{pmatrix}
\end{equation}
Определите зависимость усреднённой по таким флуктуациям матрицы плотности от времени $\braket{\hat{\rho}(t)}$.\\
\textit{Указание}: в процессе решения вам будет необходимо вычислять средние типа $\left<\exp\left(i\int\limits_{t_1}^{t_2}\delta B(t)dt\right)\right>$. Это предлагается сделать двумя способами:
\begin{enumerate}
    \item Воспользоваться теоремой Вика, из которой следует, что для любой гауссовой величины с нулевым средним выполнено $\braket{e^A}=e^{\frac{1}{2}\braket{A^2}}$.
    \item Дискретизовать время, разбив на маленькие участки $t \in (n\delta t,(n + 1)\delta t)$ и предположив $\delta B_n = \text{const}$; тогда такой локальный по времени коррелятор эквивалентен $\braket{\delta B_n\delta B_m} = \delta_{nm}\frac{\Gamma}{\delta t}$. В результате вы получите дискретное Гауссово распределение, с которым вы уже умеете работать.
\end{enumerate}
\textbf{Решение.}\\
Уравнение Гейзенберга для эволюции матрицы плотности $\hat{\rho}(t)$:
\begin{equation}
    \frac{\partial\hat{\rho}(t)}{\partial t}=\frac{i}{\hbar}[\hat{\rho}(t),\hat{H}(t)]=-\frac{iB(t)}{\hbar}[\hat{\rho}(t),\hat{\sigma}^z]
\end{equation}
\begin{equation}
    [\hat{\rho}(t),\hat{\sigma}^z]=\left[\begin{pmatrix}
    \rho_{11}(t) & \rho_{12}(t)\\
    \rho_{21}(t) & \rho_{22}(t)\\
    \end{pmatrix},\begin{pmatrix}
    1 & 0\\
    0 & -1\\
    \end{pmatrix}\right]=\begin{pmatrix}
    \rho_{11}(t) & -\rho_{12}(t)\\
    \rho_{21}(t) & -\rho_{22}(t)\\
    \end{pmatrix}-\begin{pmatrix}
    \rho_{11}(t) & \rho_{12}(t)\\
    -\rho_{21}(t) & -\rho_{22}(t)\\
    \end{pmatrix}
\end{equation}
\begin{equation}
    \frac{\partial\hat{\rho}(t)}{\partial t}=\frac{iB(t)}{\hbar}\begin{pmatrix}
    0 & 2\rho_{12}(t)\\
    -2\rho_{21}(t) & 0\\
    \end{pmatrix}
\end{equation}
\begin{equation}
    \frac{\partial\hat{\rho}_{11}(t)}{\partial t}=\frac{\partial\hat{\rho}_{22}(t)}{\partial t}=0,\quad \frac{\partial\hat{\rho}_{12}(t)}{\partial t}=\frac{2iB}{\hbar}\hat{\rho}_{12}(t),\quad\frac{\partial\hat{\rho}_{21}(t)}{\partial t}=-\frac{2iB}{\hbar}\hat{\rho}_{21}(t)
\end{equation}
\begin{equation}
    \hat{\rho}(t)=\begin{pmatrix}
    \hat{\rho}_{11}(t_0) & \hat{\rho}_{12}(t_0)\exp\left(\frac{2i}{\hbar}\int\limits_{t_0}^tB(t')dt'\right)\\
    \hat{\rho}_{21}(t_0)\exp\left(-\frac{2i}{\hbar}\int\limits_{t_0}^tB(t')dt'\right) & \hat{\rho}_{22}(t_0)
    \end{pmatrix}
\end{equation}
\begin{equation*}
    \hat{\rho}(t)=\begin{pmatrix}
    \hat{\rho}_{11}(t_0) & \hat{\rho}_{12}(t_0)\exp\left(\frac{2i}{\hbar}B_0(t-t_0)\right)\exp\left(\frac{2i}{\hbar}\int\limits_{t_0}^t\delta B(t')dt'\right)\\
    \hat{\rho}_{21}(t_0)\exp\left(-\frac{2i}{\hbar}B_0(t-t_0)\right)\exp\left(-\frac{2i}{\hbar}\int\limits_{t_0}^t\delta B(t')dt'\right) & \hat{\rho}_{22}(t_0)
    \end{pmatrix}
\end{equation*}
Усредним матрицу плотности $\hat{\rho}$ по случайным флуктуациям двумя способами:
\begin{enumerate}
    \item Воспользуемся теоремой Вика:
    \begin{multline}
        \left<\exp\left(\frac{2i}{\hbar}\int\limits_{t_0}^t\delta B(t')dt'\right)\right>=\exp\left(-\frac{2}{\hbar^2}\left<\left(\int\limits_{t_0}^t\delta B(t')dt'\right)^2\right>\right)=\\=\exp\left(-\frac{2}{\hbar^2}\left<\int\limits_{t_0}^t\delta B(t')dt'\int\limits_{t_0}^t\delta B(t'')dt''\right>\right)=\exp\left(-\frac{2}{\hbar^2}\int\limits_{t_0}^t\int\limits_{t_0}^t\braket{\delta B(t')\delta B(t'')}dt'dt''\right)=\\=\exp\left(-\frac{2}{\hbar^2}\int\limits_{t_0}^t\int\limits_{t_0}^t\Gamma\delta(t'-t'')dt'dt''\right)=\exp\left(-\frac{2\Gamma}{\hbar^2}\int\limits_{t_0}^t dt''\right)=\exp\left(-\frac{2\Gamma}{\hbar^2}(t-t_0)\right)
    \end{multline}
    \begin{equation}
        \boxed{\braket{\hat{\rho}(t)}=\begin{pmatrix}
        \hat{\rho}_{11}(t_0) &     \hat{\rho}_{12}(t_0)\exp\left(\frac{2(iB_0\hbar-\Gamma)}{\hbar^2}(t-t_0)\right)\\
        \hat{\rho}_{21}(t_0)\exp\left(-\frac{2(iB_0\hbar-\Gamma)}{\hbar^2}(t-t_0)\right) & \hat{\rho}_{22}(t_0)
        \end{pmatrix}}
    \end{equation}
    \item Дискретизуем время:
    \begin{equation}
        \left<\exp\left(\frac{2i}{\hbar}\int\limits_{t_0}^t\delta B(t')dt'\right)\right>=\left<\exp\left(\frac{2i}{\hbar}\sum\limits_{n=1}^N\delta B_n\delta t\right)\right>=\left<\prod\limits_{n=1}^N\exp\left(\frac{2i\delta t}{\hbar}\delta B_n\right)\right>
    \end{equation}
    где $N=\frac{t-t_0}{\delta t}$. Разложим каждую экспоненту по формуле Тейлора до 1 порядка:
    \begin{equation}
        \exp\left(\frac{2i\delta t}{\hbar}\delta B_n\right)=1+\frac{2i\delta t}{\hbar}\delta B_n-\frac{2(\delta t)^2}{\hbar^2}(\delta B_n)^2
    \end{equation}
    Поскольку $\delta B_n$ в каждой экспоненте независимы, то
    \begin{multline}
        \left<\prod\limits_{n=1}^N\exp\left(\frac{2i\delta t}{\hbar}\delta B_n\right)\right>=\prod\limits_{n=1}^N\left<\exp\left(\frac{2i\delta t}{\hbar}\delta B_n\right)\right>=\prod\limits_{n=1}^N\left<1+\frac{2i\delta t}{\hbar}\delta B_n-\frac{2(\delta t)^2}{\hbar^2}(\delta B_n)^2\right>=\\=\prod\limits_{n=1}^N\left(1-\frac{2\delta t}{\hbar^2}\Gamma\right)=\prod\limits_{n=1}^N\left(1-\frac{2\Gamma(t-t_0)}{\hbar^2N}\right)^N
    \end{multline}
    Устремим число слагаемых в разбиении $N\rightarrow\infty$ и воспользуемся 2 замечательным пределом:
    \begin{equation}
        \left<\prod\limits_{n=1}^N\exp\left(\frac{2i\delta t}{\hbar}\delta B_n\right)\right>=\exp\left(-\frac{2\Gamma}{\hbar^2}(t-t_0)\right)
    \end{equation}
    Таким образом, получаем тот же ответ:
    \begin{equation}
        \boxed{\braket{\hat{\rho}(t)}=\begin{pmatrix}
        \hat{\rho}_{11}(t_0) &     \hat{\rho}_{12}(t_0)\exp\left(\frac{2(iB_0\hbar-\Gamma)}{\hbar^2}(t-t_0)\right)\\
        \hat{\rho}_{21}(t_0)\exp\left(-\frac{2(iB_0\hbar-\Gamma)}{\hbar^2}(t-t_0)\right) & \hat{\rho}_{22}(t_0)
        \end{pmatrix}}
    \end{equation}
\end{enumerate}
\section{Связанные состояния. Мелкая яма}
\subsection*{Упражнения (15 баллов)}
\textbf{Упражнение 1. Измерение (5 баллов).}\\
Состояние трёхмерной частицы описывается нормированной волновой функцией $\psi(x, y, z)$. Какова вероятность того, что частица находится в интервале $z_1 < z < z_2$, а её импульс при этом — в интервале $p_1 < p_y < p_2$?\\
\textbf{Решение.}\\
Преобразование Фурье волновой функции $\psi(x,y,z)$ по $y$:
\begin{equation}
    \Psi(x,p_y,z)=\int\limits_{-\infty}^\infty dye^{-\frac{ip_yy}{\hbar}}\psi(x,y,z)
\end{equation}
Плотность вероятности:
\begin{equation}
    \rho=|\Psi(x,p_y,z)|^2=\left|\int\limits_{-\infty}^\infty dye^{-\frac{ip_yy}{\hbar}}\psi(x,y,z)\right|^2
\end{equation}
Искомая вероятность:
\begin{equation}
    P=\int\limits_{-\infty}^\infty\int\limits_{p_1}^{p_2}\int\limits_{z_1}^{z_2}dx\frac{dp_y}{2\pi\hbar}dz\rho
\end{equation}
\begin{equation}
    \boxed{P=\int\limits_{-\infty}^\infty\int\limits_{p_1}^{p_2}\int\limits_{z_1}^{z_2}dx\frac{dp_y}{2\pi\hbar}dz\left|\int\limits_{-\infty}^\infty dye^{-\frac{ip_yy}{\hbar}}\psi(x,y,z)\right|^2}
\end{equation}\\
\textbf{Упражнение 2. Прямоугольная яма (10 баллов).}\\
В стандартном курсе квантовой механики вы наверняка сталкивались с задачей о частице в прямоугольной яме:
\begin{equation}
    U(x)=\begin{cases}
-U_0,\quad -\frac{a}{2}\leq x\leq\frac{a}{2} \\
0,\quad \text{otherwise}\\
\end{cases}
\end{equation}
Продемонстрируйте, что в случае, когда эта яма — мелкая, точное решение совпадает с приближённой формулой для мелкой ямы.\\
\textbf{Решение.}\\
Запишем стационарное уравнение Шрёдингера:
\begin{equation}
    -\frac{\hbar^2}{2m}\frac{d^2\psi}{dx^2}=(E-U(x))\psi(x)
\end{equation}
Подставим $U(x)$:
\begin{equation}
    -\frac{\hbar^2}{2m}\frac{d^2\psi}{dx^2}=
    \begin{cases}
    E\psi(x),\quad x\in(-\infty,-\frac{a}{2})\cup(\frac{a}{2},\infty)\\\
    (E+U_0)\psi(x),\quad -\frac{a}{2}\leq x\leq\frac{a}{2}
    \end{cases}
\end{equation}
Условие связанного состояния:
\begin{equation}
    -U_0<E<0
\end{equation}
Введём обозначения:
\begin{equation}
    \kappa^2=-\frac{2mE}{\hbar^2},\quad k^2=\frac{2m(E+U_0)}{\hbar^2}
\end{equation}
Поскольку потенциал $U(x)$ -- чётная функция, то и гамильтониан $H(x)=\frac{p^2}{2m}+U(x)$ чётный и коммутирует с оператором инверсии. Собственные функции оператора инверсии -- чётные и нечётные функции. Будем искать чётные $\psi(x)$. Из существования нормировки $\int\limits_{-\infty}^\infty|\psi(x)|^2dx=1$ следует, что в решении нужно оставить только затухающие экспоненты:
\begin{equation}
    \psi(x)=\begin{cases}
    \psi_I(x)=Ae^{\kappa x},\quad x<-\frac{a}{2}\\
    \psi_{II}(x)=B\cos kx,\quad -\frac{a}{2}<x<\frac{a}{2}\\
    \psi_{III}(x)=Ae^{-\kappa x},\quad x>\frac{a}{2}\\
    \end{cases}
\end{equation}
Запишем граничные условия (из чётности достаточно их записать в одной точке $x=\frac{a}{2}$):
\begin{equation}
    \psi_{II}\left(\frac{a}{2}\right)=\psi_{III}\left(\frac{a}{2}\right),\quad\psi'_{II}\left(\frac{a}{2}\right)=\psi'_{III}\left(\frac{a}{2}\right)
\end{equation}
\begin{equation}
    Ae^{\frac{-\kappa a}{2}}=B\cos \frac{ka}{2},\quad -\kappa Ae^{\frac{-\kappa a}{2}}=-\kappa B\sin \frac{ka}{2}
\end{equation}
\begin{equation}
    \tan\frac{ka}{2}=\frac{\kappa}{k}>0
\end{equation}
Вернёмся к начальным величинам:
\begin{equation}
    \tan\sqrt{\frac{m(E+U_0)}{2}}\frac{a}{\hbar}=\sqrt{-\frac{E}{E+U_0}}
\end{equation}
В случае мелкой ямы: 
\begin{equation}
    U_0\ll\frac{\hbar^2}{ma^2},\quad |E|<U_0\ll\frac{\hbar^2}{ma^2}
\end{equation}
В случае нечётных функций $\tan\frac{ka}{2}$ получился бы отрицательным и условие мелкой ямы выполниться бы не могло.
Поскольку $\sqrt{\frac{m(E+U_0)}{2}}\frac{a}{\hbar}$ мало, то и $\sqrt{-\frac{E}{E+U_0}}$ тоже, значит $E\ll U_0$
\begin{equation}
    \sqrt{\frac{mU_0}{2}}\frac{a}{\hbar}=\sqrt{-\frac{E}{U_0}}
\end{equation}
\begin{equation}
    \boxed{E=-\frac{ma^2U_0^2}{2\hbar^2}}
\end{equation}
Сравниваем с формулой мелкой ямы:
\begin{equation}
    \boxed{|E|=\frac{m}{2\hbar^2}\left(\int\limits_{-\infty}^\infty U(x)dx\right)^2=\frac{ma^2U_0^2}{2\hbar^2}}
\end{equation}
Как видно, формулы совпадают.
\subsection*{Задачи (85 баллов)}
\textbf{Задача 1. Каноническое квантование (10 баллов).}\\
Один из способов построения квантовой механики заключается в постулировании коммутационных соотношений, свя-занных с классической скобкой Пуассона: $[\hat{A},\hat{B}]\rightarrow i\hbar\{\hat{A},\hat{B}\}$; для канонически сопряжённых операторов координаты иимпульса это даёт $[\hat{x},\hat{p}] =i\hbar$.\\
Пусть известно, что у оператора $\hat{x}$ имеется непрерывный спектр собственных значений $\mathbb{R}$, а также известны коммутационные соотношения $[\hat{x},\hat{p}] =i\hbar$. Продемонстрируйте, что только этих знаний достаточно, чтобы вывести явный видоператора импульса в координатном представлении $\hat{p}=-i\hbar\frac{\partial}{\partial x}$.\\
\textit{Дополнительно}: покажите, что не существует конечномерных представлений этой алгебры: операторы $\hat{x}$ и $\hat{p}$, определённые таким образом, не могут действовать в гильбертовом пространстве конечной размерности.\\
\textit{Указания}:
\begin{enumerate}
    \item Вычислите по индукции $[\hat{x},\hat{p}^n]$.
    \item Оператор трансляции определим стандартным образом как $\hat{T}_a=e^{i\hat{p}a/\hbar}$. Используя предыдущий пункт, вычислите коммутатор $[\hat{x},\hat{T}_a]$.
    \item Пусть $\ket{x}$ -- базис собственных состояний оператора $\hat{x}$, так что $\hat{x}\ket{x}=x\ket{x}$. Покажите, что построенный оператор $\hat{T}_a$ действительно является оператором трансляции: $\hat{T}_a\ket{x}=\ket{x-a}$.
    \item Используя матричный элемент $\braket{x|\hat{T}_a|\psi}$ для инфинитезимальной трансляции $a\rightarrow 0$, найдите матричный элемент $\braket{x|\hat{p}|\psi}\equiv\hat{p}\psi(x)$.
\end{enumerate}
\textbf{Решение.}\\
Воспользуемся указаниями:
\begin{enumerate}
    \item Вычислим $[\hat{x},\hat{p}^n]$ по индукции:
    \begin{equation}
        [\hat{x},\hat{p}]=i\hbar
    \end{equation}
    \begin{multline}
        [\hat{x},\hat{p}^n]\psi(x)\equiv\hat{x}\hat{p}^n\psi(x)-\hat{p}^n\hat{x}\psi(x)=\hat{x}\hat{p}^{n-1}\hat{p}\psi(x)-\hat{p}^{n-1}\hat{p}\hat{x}\psi(x)=\hat{x}\hat{p}^{n-1}\hat{p}\psi(x)-\\-\hat{p}^{n-1}(\hat{x}\hat{p}-i\hbar)\psi(x)=[\hat{x},\hat{p}^{n-1}]\hat{p}\psi(x)+i\hbar\hat{p}^{n-1}\psi(x)
    \end{multline}
    \begin{equation}
        \boxed{[\hat{x},\hat{p}^n]=i\hbar n\hat{p}^{n-1}}
    \end{equation}
    \item Распишем оператор трансляции в ряд Тейлора:
    \begin{equation}
        \hat{T}_a=e^{i\hat{p}a/\hbar}=\sum\limits_{n=0}^\infty\frac{1}{n!}\left(\frac{i\hat{p}a}{\hbar}\right)^n
    \end{equation}
    \begin{equation}
        [\hat{x},\hat{T}_a]=\left[\hat{x},\sum\limits_{n=0}^\infty\frac{1}{n!}\left(\frac{i\hat{p}a}{\hbar}\right)^n\right]=\sum\limits_{n=1}^\infty\frac{1}{n!}\left(\frac{ia}{\hbar}\right)^n[\hat{x},\hat{p}^n]
    \end{equation}
    Воспользуемся предыдущим пунктом:
    \begin{equation}
        [\hat{x},\hat{T}_a]=\sum\limits_{n=1}^\infty\frac{1}{n!}\left(\frac{ia}{\hbar}\right)^ni\hbar n\hat{p}^{n-1}=-a\sum\limits_{n=0}^\infty \frac{1}{n!}\left(\frac{ia\hat{p}}{\hbar}\right)^n
    \end{equation}
    \begin{equation}
        \boxed{[\hat{x},\hat{T}_a]=-a\hat{T}_a}
    \end{equation}
    \item Воспользуемся предыдущим пунктом:
    \begin{equation}
        [\hat{x},\hat{T}_a]\ket{x}=-a\hat{T}_a\ket{x}
    \end{equation}
    \begin{equation}
        \hat{x}\hat{T}_a\ket{x}-\hat{T}_a\hat{x}\ket{x}=-a\hat{T}_a\ket{x}
    \end{equation}
    \begin{equation}
        \hat{x}\hat{T}_a\ket{x}=\hat{T}_ax\ket{x}-a\hat{T}_a\ket{x}=(x-a)\hat{T}_a\ket{x}
    \end{equation}
    Как видно, действие оператора $\hat{x}$ на $\ket{x-a}$
    \begin{equation}
        \hat{x}\ket{x-a}=(x-a)\ket{x-a}
    \end{equation}
    совпадает с действием на $\hat{T}_a\ket{x}$, следовательно
    \begin{equation}
        \boxed{\hat{T}_a\ket{x}=\ket{x-a}}
    \end{equation}
    \item Для инфинитезимальной трансляции $a\rightarrow0$:
    \begin{equation}
        \braket{x|\hat{T}_a|\psi}=\braket{x|\mathbb{I}+i\hat{p}a/\hbar|\psi}=\psi(x)+\frac{ia}{\hbar}\hat{p}\psi(x)
    \end{equation}
    \begin{equation}
        \braket{x|\hat{T}_a|\psi}=\braket{\hat{T}_a^\dagger x|\psi}=\braket{x+a|\psi}=\psi(x+a)=\psi(x)+a\frac{\partial\psi(x)}{\partial x}
    \end{equation}
    Сравнивая предущие равенства, получаем
    \begin{equation}
        \boxed{\hat{p}=-i\hbar\frac{\partial}{\partial x}}
    \end{equation}
\end{enumerate}
Покажем, что не существует конечномерных представлений этой алгебры. От противного, предположим существуют представления $\hat{x}$ и $\hat{p}$ в виде матриц конечного размера $n\times n$.
\begin{equation}
    \text{tr}([\hat{x},\hat{p}]_{n\times n})=\text{tr}(i\hbar\mathbb{I}_{n\times n})=i\hbar n
\end{equation}
\begin{equation}
    \text{tr}([\hat{x},\hat{p}]_{n\times n})=\text{tr}(\hat{x}\hat{p}-\hat{p}\hat{x})=0
\end{equation}
$i\hbar n\neq 0$, противоречие.\\\\
\textbf{Задача 2. Преобразование Галлилея (10 баллов).}\\
Пусть частица находится в потенциале, который движется со скоростью $v$:
\begin{equation}
    \hat{H}(t)=\frac{\hat{p}^2}{2m}+V(\hat{x}-vt)
\end{equation}
Придумайте унитарное преобразование $\hat{U}(t)$ (<<преобразование Галлилея>>), которое приведёт Гамильтониан к аналогичному, но независящему от времени виду:
\begin{equation}
    \hat{H}'\equiv\hat{U}(t)\hat{H}(t)\hat{U}^\dagger(t)-i\hbar\hat{U}(t)\partial_t\hat{U}^\dagger(t)=\frac{\hat{p}^2}{2m}+V(\hat{x})
\end{equation}
Запишите явно его действие на произвольную волновую функцию $\braket{x|\hat{U}(t)|\psi(t)}$.\\
\textbf{Решение.}\\
Избавимся от зависимости от времени в качестве преобразования при помощи оператора трансляции $\hat{T}_{vt}$:
\begin{equation}
    \hat{T}_{vt}=\exp\left(vt\frac{\partial}{\partial x}\right)=\exp\left(\frac{ivt}{\hbar}\hat{p}\right),\quad \hat{T}^\dagger_{vt}=\hat{T}^{-1}_{vt}=\exp\left(-\frac{ivt}{\hbar}\hat{p}\right)
\end{equation}
\begin{equation}
    \hat{T}_{vt}(t)\hat{H}(t)\hat{T}_{vt}^\dagger(t)=\hat{T}_{vt}(t)\left(\frac{\hat{p}^2}{2m}+V(\hat{x}-vt)\right)\hat{T}_{vt}^\dagger(t)-i\hbar\hat{T}_{vt}(t)\partial_t\hat{T}^\dagger_{vt}(t)
\end{equation}
Поскольку $\left[\hat{T}_{vt}(t),\frac{\hat{p}^2}{2m}\right]=0$, то
\begin{equation}
     \hat{T}_{vt}(t)\hat{H}(t)\hat{T}_{vt}^\dagger(t)=\frac{\hat{p}^2}{2m}+V(\hat{x})-i\hbar\exp\left(\frac{ivt}{\hbar}\hat{p}\right)\left(-\frac{iv\hat{p}}{\hbar}\right)\exp\left(-\frac{ivt}{\hbar}\hat{p}\right)=\frac{\hat{p}^2}{2m}+V(\hat{x})-v\hat{p}
\end{equation}
Избавимся от слагаемого $-v\hat{p}$ при помощи оператора трансляции по импульсу:
\begin{equation}
    \exp\left(a\frac{\partial}{\partial p}\right)=\exp\left(-\frac{ia\hat{x}}{\hbar}\right)
\end{equation}
Поскольку $[\exp\left(-\frac{ia\hat{x}}{\hbar}\right),V(x)]=0$, то
\begin{multline}
    \exp\left(-\frac{ia\hat{x}}{\hbar}\right)\left(\frac{\hat{p}^2}{2m}+V(\hat{x})-v\hat{p}\right)\exp\left(\frac{ia\hat{x}}{\hbar}\right)=\frac{(\hat{p}+a)^2}{2m}-v(\hat{p}+a)+V(\hat{x})=\\=\frac{\hat{p}^2}{2m}+\frac{a\hat{p}}{m}+\frac{a^2}{2m}-v\hat{p}-va+V(\hat{x})
\end{multline}
Выберем $a=mv$:
\begin{equation}
    \exp\left(-\frac{ia\hat{x}}{\hbar}\right)\left(\frac{\hat{p}^2}{2m}+V(\hat{x})-v\hat{p}\right)\exp\left(\frac{ia\hat{x}}{\hbar}\right)=\frac{\hat{p}^2}{2m}-\frac{mv^2}{2}+V(\hat{x})
\end{equation}
Избавимся от слагаемого $-\frac{mv^2}{2}$ при помощи оператора $e^{ibt}$, который коммутирует и с $\hat{x}$, и с $\hat{p}$:
\begin{multline}
    \exp(ibt)\left(\frac{\hat{p}^2}{2m}-\frac{mv^2}{2}+V(\hat{x})\right)\exp(-ibt)-i\hbar\exp(ibt)(-ib)\exp(-ibt)=\frac{\hat{p}^2}{2m}-\frac{mv^2}{2}+\\+V(\hat{x})-\hbar b
\end{multline}
Выберем $b=-\frac{mv^2}{2\hbar}$. Таким образом, искомое унитарное преобразование:
\begin{equation}
    \boxed{U(t)=\exp\left(-\frac{imv^2t}{2\hbar}\right)\exp\left(-\frac{imv\hat{x}}{\hbar}\right)\exp\left(\frac{ivt\hat{p}}{\hbar}\right)}
\end{equation}
Подействуем оператором $\hat{U}(t)$ на функцию $\psi$:
\begin{equation}
    \braket{x|\hat{U}(t)|\psi(t)}=\braket{\hat{U}^\dagger(t)x|\psi(t)}=\braket{x+vt|\psi(t)}\exp\left(-\frac{imv(x+vt)}{\hbar}\right)\exp\left(-\frac{imv^2t}{2\hbar}\right)
\end{equation}
\begin{equation}
    \boxed{\braket{x|\hat{U}(t)|\psi(t)}=\psi(t,x+vt)\exp\left(-\frac{imv(x+vt)}{\hbar}\right)\exp\left(-\frac{imv^2t}{2\hbar}\right)}
\end{equation}\\
\textbf{Задача 3. Туннельное расщепление (10 баллов).}\\
Рассмотрите две мелкие ямы, моделируемые следующим потенциалом:
\begin{equation}
    U(x)=-\frac{\hbar^2\kappa}{m}(\delta(x+L/2)+\delta(x-L/2))
\end{equation}
Такая задача включена в стандартный курс квантовой механики; предлагается провести её исследование.
\begin{enumerate}
    \item Нарисуйте (схематично, но со всеми ключевыми особенностями) зависимость уровней энергии связанных состояний от расстояния между ямами $L$.
    \item Пусть расстояние между ямами много больше характерного масштаба волновых функций для каждой из отдельных ям (туннельный режим), $L\gg \kappa^{-1}$. Определите расщепление между связанными состояниями.
    \item Эта задача может быть рассмотрена как модель ковалентной связи. Считая теперь $L$ классической динамической переменной, определите силу (в туннельном режиме) и характер взаимодействия между ямами, если частица находится в основном состоянии.
\end{enumerate}
\textbf{Решение.}
\begin{enumerate}
    \item Запишем стационарное уравнение Шрёдингера:
\begin{equation}
    -\frac{\hbar^2}{2m}\frac{d^2\psi}{dx^2}=(E-U(x))\psi(x)=E\psi(x)
\end{equation}
Условие связанного состояния:
\begin{equation}
    E<0
\end{equation}
Обозначим $k^2=-\frac{2mE}{\hbar^2}$.\\
Поскольку потенциал $U(x)$ -- чётная функция, то и гамильтониан $H(x)=\frac{p^2}{2m}+U(x)$ чётный и коммутирует с оператором инверсии. Собственные функции оператора инверсии -- чётные и нечётные функции. Из существования нормировки $\int\limits_{-\infty}^\infty|\psi(x)|^2dx=1$ следует, что в решении нужно оставить только затухающие на бесконечности экспоненты.
\begin{enumerate}
    \item Будем искать чётные $\psi(x)$.
    \begin{equation}
        \psi(x)=\begin{cases}
        \psi_I(x)=Ae^{kx},\quad x<-\frac{L}{2}\\
        \psi_{II}(x)=B\cosh k x,\quad -\frac{L}{2}<x<\frac{L}{2}\\
        \psi_{III}(x)=Ae^{-k x},\quad x>\frac{L}{2}\\
        \end{cases}
    \end{equation}
    Запишем граничные условия (из чётности достаточно их записать в одной точке $x=\frac{L}{2}$):
    \begin{equation}
        \psi_{II}\left(\frac{L}{2}\right)=\psi_{III}\left(\frac{L}{2}\right),\quad\psi'_{III}\left(\frac{L}{2}\right)-\psi'_{II}\left(\frac{L}{2}\right)=-2k_0\psi_{III}\left(\frac{L}{2}\right)
    \end{equation}
    \begin{equation}
        Ae^{-\frac{k L}{2}}=B\cosh \frac{k L}{2},\quad A(2\kappa-k)e^{-\frac{k L}{2}}=k B\sinh\frac{k L}{2}
    \end{equation}
    \begin{equation}\label{eq1}
        1+e^{-y}=\frac{y}{\kappa L},\quad y=k L
    \end{equation}
    Получилось трансцендентное уравнение. Решение $y=0$ не походит, поскольку $E<0$. Энергии связанных состояний для чётной функции $\psi$:
    \begin{equation}
        E_1=-\frac{\hbar^2k^2}{2m}=-\frac{\hbar^2y^2}{2mL^2}
    \end{equation}
    \begin{equation}
        \boxed{E_1=-\frac{\hbar^2y^2}{2mL^2},\quad 1+e^{-y}=\frac{y}{\kappa L}}
    \end{equation}
    \item Будем искать нечётные $\psi(x)$.
    \begin{equation}
        \psi(x)=\begin{cases}
        \psi_I(x)=Ae^{k x},\quad x<-\frac{L}{2}\\
        \psi_{II}(x)=B\sinh k x,\quad -\frac{L}{2}<x<\frac{L}{2}\\
        \psi_{III}(x)=-Ae^{-k x},\quad x>\frac{L}{2}\\
        \end{cases}
    \end{equation}
    Запишем граничные условия (из чётности достаточно их записать в одной точке $x=\frac{L}{2}$):
    \begin{equation}
        \psi_{II}\left(\frac{L}{2}\right)=\psi_{III}\left(\frac{L}{2}\right),\quad\psi'_{III}\left(\frac{L}{2}\right)-\psi'_{II}\left(\frac{L}{2}\right)=-2\kappa\psi_{III}\left(\frac{L}{2}\right)
    \end{equation}
    \begin{equation}
        Ae^{-\frac{k L}{2}}=B\sinh \frac{k L}{2},\quad A(2\kappa-k)e^{-\frac{k L}{2}}=k B\cosh\frac{k L}{2}
    \end{equation}
    \begin{equation}\label{eq2}
        1-e^{-y}=\frac{y}{\kappa L},\quad y=k L
    \end{equation}
    Получилось трансцендентное уравнение. Решение $y=0$ не походит, поскольку $E<0$. Второе решение будет только при $\kappa L\geq1$ (при $\kappa L=1$ функции в левой и правой части касаются при $y=0$). Энергии связанных состояний для нечётной функции $\psi$:
    \begin{equation}
        E_2=-\frac{\hbar^2k^2}{2m}=-\frac{\hbar^2y^2}{2mL^2}
    \end{equation}
    \begin{equation}
        \boxed{E_2=-\frac{\hbar^2y^2}{2mL^2},\quad 1-e^{-y}=\frac{y}{\kappa L}}
    \end{equation}
\end{enumerate}
Графики зависимостей $E_1$ и $E_2$ от $\kappa L$ представлены на рис. 1.
\begin{figure}
    \centering
    \includegraphics[scale=0.35]{gr1.png}
    \caption{Графики зависимостей $E_1(\kappa L)$ (синий график) и $E_2(\kappa L)$ (жёлтый график)}
    \label{fig:my_label}
\end{figure}
\item В туннельном режиме (при $\kappa L\gg1$) трансцендентные уравнения (\ref{eq1}) и (\ref{eq2}) удобно решить методом последовательных приближений. Нулевое приближение:
\begin{equation}
    y_1^{(0)}=y_2^{(0)}=\kappa L
\end{equation}
Первое приближение:
\begin{equation}
    y_1^{(1)}=\kappa L(1+e^{-\kappa L}),\quad y_2^{(1)}=\kappa L(1-e^{-\kappa L})
\end{equation}
\begin{equation}\label{eq3}
    E_1=-\frac{\hbar^2\kappa^2}{2m}(1+2e^{-\kappa L})+\mathcal{O}(e^{-2\kappa L}),\quad E_2=-\frac{\hbar^2\kappa^2}{2m}(1-2e^{-\kappa L})+\mathcal{O}(e^{-2\kappa L})
\end{equation}
\begin{equation}
    \Delta E=E_2-E_1=\frac{2\hbar^2\kappa^2e^{-\kappa L}}{m}+\mathcal{O}(e^{-2\kappa L})
\end{equation}
\begin{equation}
    \boxed{\Delta E=\frac{2\hbar^2\kappa^2_0e^{-\kappa L}}{m}+\mathcal{O}(e^{-2\kappa L})}
\end{equation}
\item Основное состояние -- состояние с минимально допустимой энергией. В туннельном режиме (при $\kappa L\gg1$) минимальная энергия
\begin{equation}
    E_1(L)=-\frac{\hbar^2\kappa ^2}{2m}(1+2e^{-\kappa L})+\mathcal{O}(e^{-2\kappa L})
\end{equation}
\begin{equation}
    \boxed{F=-\frac{\partial E_1}{\partial L}=-\frac{\hbar^2\kappa^3e^{-\kappa L}}{m}+\mathcal{O}(e^{-2\kappa L})}
\end{equation}
Поскольку $F<0$, то ямы притягиваются.
\end{enumerate}
\textbf{Задача 3.1. Модель сильной связи (15 баллов).}\\
В туннельном режиме данная задача может также служить иллюстрацией для модели сильной связи, которая описывает пару нижних уровней энергии. Все вычисления в этой задаче необходимо проводить в ведущем приближении по параметру $L\gg \kappa^{-1}$.
\begin{enumerate}
    \item Спроецируйте гамильтониан на линейное подпространство, натянутое на собственные функции каждой из ям по отдельности: $\{\ket{\psi_1}, \ket{\psi_2}\}=\{\psi_0(x+L/2),\psi_0(x-L/2)\}$. Выпишите соответствующую ему матрицу $2 \times 2$ в этом базисе.
    \item Обратите внимание, что базисные вектора не являются ортогональными. Вычислите матрицу Грама для этого базиса (матрица скалярных произведений $G_{ij} = \braket{\psi_i|\psi_j}$).
    \item Характеристическое уравнение, определяющее собственные числа для неортонормированного базиса, имеет вид $\det(\hat{H}-E\cdot\hat{G}) = 0$. Найдите собственные уровни энергии в заданном приближении.
\end{enumerate}
\textit{Подсказка}: обратите внимание: гамильтониан задачи можно записать в виде $\hat{H} = \hat{T}+\hat{U}_1+\hat{U}_2$, и состояния $\ket{\psi_{1,2}}$ являются собственными для $\hat{H}_{1,2} = \hat{T} + \hat{U}_{1,2}$ соответственно. Это обстоятельство может значительно упростить вычисления. Кроме того, предлагается заметить, что часть членов (какие?) имеет экспоненциальную зависимость $\propto\exp(-\kappa L)$, а часть -- зависимость $\propto\exp(-2\kappa L)$, в связи с чем их в ведущем порядке можно положить нулём.\\
\textbf{Решение.}\\
Воспользуемся подсказкой: запишем гамильтониан задачи в виде
\begin{equation}
    \hat{H}=\hat{T}+\hat{U}_1+\hat{U_2},\quad \hat{U}_1=-\frac{\hbar^2\kappa}{m}\delta(x+L/2),\quad \hat{U}_2=-\frac{\hbar^2\kappa}{m}\delta(x-L/2)
\end{equation}
\begin{equation}
    \hat{H}_1=\frac{\hat{p}^2}{2m}-\frac{\hbar^2\kappa}{m}\delta(x+L/2),\quad \hat{H}_2=\frac{\hat{p}^2}{2m}-\frac{\hbar^2\kappa}{m}\delta(x-L/2)
\end{equation}
Запишем стационарное уравнение Шрёдингера:
\begin{equation}
    \hat{H}_i\ket{\psi_i}=E\ket{\psi_i}
\end{equation}
Обозначим $k^2=-\frac{2mE}{\hbar^2}$.
\begin{itemize}
    \item Рассмотрим $\hat{H}_1$.
\begin{equation}
    \left(-\frac{\hbar^2}{2m}\frac{\partial^2}{\partial x^2}-\frac{\hbar^2\kappa}{m}\delta(x+L/2)\right)\psi=E\psi
\end{equation}
\begin{equation}
    \psi(x)=\begin{cases}
    \psi_I(x)=Ae^{k x},\quad x<-\frac{L}{2}\\
    \psi_{II}(x)=Be^{-k x},\quad x>-\frac{L}{2}\\
    \end{cases}
\end{equation}
Запишем граничные условия:
\begin{equation}
    \psi_{I}\left(-\frac{L}{2}\right)=\psi_{II}\left(-\frac{L}{2}\right),\quad\psi'_{II}\left(-\frac{L}{2}\right)-\psi'_{I}\left(-\frac{L}{2}\right)=-2\kappa\psi_{II}\left(-\frac{L}{2}\right)
\end{equation}
\begin{equation}
    Ae^{-\frac{kL}{2}}=Be^{\frac{kL}{2}},\quad -k Be^{\frac{kL}{2}}-k Ae^{-\frac{kL}{2}}=-2\kappa Be^{\frac{kL}{2}}
\end{equation}
\begin{equation}
    B=Ae^{-kL},\quad k=\kappa
\end{equation}
\begin{equation}
    \psi_1(x)=C_1e^{-\kappa|x+L/2|}
\end{equation}
Найдём $C_1$ из нормировки:
\begin{equation}
    \braket{\psi_1|\psi_1}=\int\limits_{-\infty}^\infty C^2_1e^{-\kappa|2x+L|}dx=2C^2_1\int\limits_0^\infty e^{-2kx}dx=\frac{C_1^2}{k}=1\rightarrow C_1=\sqrt{k}=\sqrt{\kappa}
\end{equation}
\begin{equation}
    \psi_1(x)=\sqrt{\kappa}e^{-\kappa|x+L/2|}
\end{equation}
\begin{equation}
    E=-\frac{\hbar^2\kappa^2}{2m}
\end{equation}
\item Рассмотрим $\hat{H}_2$.
\begin{equation}
    \left(-\frac{\hbar^2}{2m}\frac{\partial^2}{\partial x^2}-\frac{\hbar^2\kappa}{m}\delta(x-L/2)\right)\psi=E\psi
\end{equation}
\begin{equation}
    \psi(x)=\begin{cases}
    \psi_I(x)=Ae^{k x},\quad x<\frac{L}{2}\\
    \psi_{II}(x)=Be^{-k x},\quad x>\frac{L}{2}\\
    \end{cases}
\end{equation}
Запишем граничные условия:
\begin{equation}
    \psi_{I}\left(\frac{L}{2}\right)=\psi_{II}\left(\frac{L}{2}\right),\quad\psi'_{II}\left(\frac{L}{2}\right)-\psi'_{I}\left(\frac{L}{2}\right)=-2\kappa\psi_{II}\left(\frac{L}{2}\right)
\end{equation}
\begin{equation}
    Ae^{\frac{kL}{2}}=Be^{-\frac{kL}{2}},\quad -k Be^{-\frac{kL}{2}}-k Ae^{\frac{kL}{2}}=-2\kappa Be^{-\frac{kL}{2}}
\end{equation}
\begin{equation}
    B=Ae^{kL},\quad k=\kappa
\end{equation}
\begin{equation}
    \psi_2(x)=C_2e^{-\kappa|x-L/2|}
\end{equation}
Найдём $C_2$ из нормировки:
\begin{equation}
    \braket{\psi_2|\psi_2}=\int\limits_{-\infty}^\infty C^2_2e^{-\kappa|2x-L|}dx=2C^2_2\int\limits_0^\infty e^{-2kx}dx=\frac{C_2^2}{k}=1\rightarrow C_2=\sqrt{k}=\sqrt{\kappa}
\end{equation}
\begin{equation}
    \psi_2(x)=\sqrt{\kappa}e^{-\kappa|x-L/2|}
\end{equation}
\begin{equation}
    E=-\frac{\hbar^2\kappa^2}{2m}
\end{equation}
\end{itemize}
\begin{enumerate}
    \item Проекция гамильтониана $\hat{H}$ на линейное подпространство $\{\ket{\psi_1}, \ket{\psi_2}\}$:
    \begin{multline}
        \braket{\psi_1|\hat{H}|\psi_1}=\braket{\psi_1|\hat{H}_1-\frac{\hbar^2\kappa}{m}\delta(x-L/2)|\psi_1}=E-\frac{\hbar^2\kappa}{m}\braket{\psi_1|\delta(x-L/2)|\psi_1}=\\=-\frac{\hbar^2\kappa^2}{2m}(1+2e^{-2\kappa L})= -\frac{\hbar^2\kappa^2}{2m}(1+\mathcal{O}(e^{-2\kappa L}))
    \end{multline}
    \begin{multline}
        \braket{\psi_2|\hat{H}|\psi_2}=\braket{\psi_2|\hat{H}_2-\frac{\hbar^2\kappa}{m}\delta(x+L/2)|\psi_2}=E-\frac{\hbar^2\kappa}{m}\braket{\psi_2|\delta(x+L/2)|\psi_2}=\\=-\frac{\hbar^2\kappa^2}{2m}(1+2e^{-2\kappa L})= -\frac{\hbar^2\kappa^2}{2m}(1+\mathcal{O}(e^{-2\kappa L}))
    \end{multline}
    \begin{multline}
        \braket{\psi_1|\hat{H}|\psi_2}=\braket{\psi_1|\hat{H}_2-\frac{\hbar^2\kappa}{m}\delta(x+L/2)|\psi_2}=E\braket{\psi_1|\psi_2}-\frac{\hbar^2\kappa}{m}\braket{\psi_1|\delta(x-L/2)|\psi_2}=\\=-\frac{\hbar^2\kappa^3}{2m}\int\limits_{-\infty}^\infty e^{-\kappa(|x+L/2|+|x-L/2|)}dx-\frac{\hbar^2\kappa^2}{m}e^{-\kappa L}=-\frac{\hbar^2\kappa^3}{2m}\left(\int\limits_{-\infty}^{-L/2} e^{2\kappa x}dx+\int\limits_{-L/2}^{L/2} e^{-\kappa L}dx+\int\limits_{L/2}^\infty e^{-2\kappa x}dx\right)-\\-\frac{\hbar^2\kappa^2}{m}e^{-\kappa L}=-\frac{3\hbar^2\kappa^2}{2m}e^{-\kappa L}-\frac{\hbar^2\kappa^3L}{2m}e^{-kL}=-\frac{\hbar^2\kappa^2}{2m}e^{-\kappa L}(3+\kappa L)
    \end{multline}
    \begin{equation}
        \braket{\psi_2|\hat{H}|\psi_1}=\braket{\hat{H}^\dagger\psi_2|\psi_1}=\braket{\hat{H}\psi_2|\psi_1}=\braket{\psi_1|\hat{H}\psi_2}^\dagger=\braket{\psi_1|\hat{H}\psi_2}=-\frac{\hbar^2\kappa^2}{2m}e^{-\kappa L}(3+\kappa L)
    \end{equation}
    Соответствующая $\hat{H}$ матрица:
    \begin{equation}
    \boxed{
        \hat{H}=\begin{pmatrix}
        -\frac{\hbar^2\kappa^2}{2m}(1+\mathcal{O}(e^{-2\kappa L})) & -\frac{\hbar^2\kappa^2}{2m}e^{-\kappa L}(3+\kappa L)\\
        -\frac{\hbar^2\kappa^2}{2m}e^{-\kappa L}(3+\kappa L) & -\frac{\hbar^2\kappa^2}{2m}(1+\mathcal{O}(e^{-2\kappa L}))
        \end{pmatrix}}
    \end{equation}
    \item Вычислим скалярные произведения $\psi_i,\psi_j$:
    \begin{equation}
        G_{ii}=\braket{\psi_i,\psi_i}=1
    \end{equation}
    \begin{equation}
        G_{12}=G_{21}=\braket{\psi_1,\psi_2}=\int\limits_{-\infty}^\infty \psi_1(x)\psi_2(x)dx=e^{-\kappa L}(1+\kappa L)
    \end{equation}
    Матрица Грама:
    \begin{equation}
     \boxed{G=\begin{pmatrix}
     1 & e^{-\kappa L}(\kappa L+1)\\
     e^{-\kappa L}(\kappa L+1) & 1\\
     \end{pmatrix}}
    \end{equation}
    \item \begin{equation}
        \hat{H}-E\hat{G}=\begin{pmatrix}
     -\frac{\hbar^2\kappa^2}{2m}-E & -e^{-\kappa L}(\frac{\hbar^2\kappa^2}{2m}(3+\kappa L)+E(\kappa L+1))\\
     -e^{-\kappa L}(\frac{\hbar^2\kappa^2}{2m}(3+\kappa L)+E(\kappa L+1)) & -\frac{\hbar^2\kappa^2}{2m}-E\\
     \end{pmatrix}
    \end{equation}
    Характеристическое уравнение:
    \begin{equation}
        \det(\hat{H}-E\hat{G})=0
    \end{equation}
    \begin{equation}
        \left(\frac{\hbar^2\kappa^2}{2m}+E\right)^2-e^{-2\kappa L}\left(\frac{\hbar^2\kappa^2}{2m}(3+\kappa L)+E(\kappa L+1)\right)^2=0
    \end{equation}
    \begin{equation}
         \frac{\hbar^2\kappa^2}{2m}+E=\pm e^{-\kappa L}\left(\frac{\hbar^2\kappa^2}{2m}(3+\kappa L)+E(\kappa L+1)\right)
    \end{equation}
    \begin{equation}
        E(1\mp e^{-\kappa L}(\kappa L+1))=\frac{\hbar^2\kappa^2}{2m}(\pm e^{-\kappa L}(3+\kappa L)-1)
    \end{equation}
    \begin{equation}\label{eq4}
        \boxed{E=-\frac{\hbar^2\kappa^2}{2m}(1\pm2e^{-\kappa L})+\mathcal{O}(e^{-2\kappa L})}
    \end{equation}
    Как видно, формулы (\ref{eq3}) и (\ref{eq4}) совпадают.
    \end{enumerate}
\textbf{Задача 4. Глубокая мелкая яма (40 баллов).}\\
Найдите энергию основного состояния в регуляризованном одномерном Кулоновском потенциале $U(x) = -\frac{e^2}{\sqrt{x^2+a^2}}$, считая что регуляризация происходит на масштабах меньше Боровского радиуса $a\ll a_B=\frac{\hbar^2}{me^2}$.\\
\textbf{Решение.}\\
Найдём фурье-образ потенциала:
\begin{equation}
    \bar{U}(p)=\int\limits_{-\infty}^\infty U(x)\exp\left(-\frac{ipx}{\hbar}\right)dx=-\int\limits_{-\infty}^\infty \frac{e^2\exp(-\frac{ipx}{\hbar})}{\sqrt{x^2+a^2}}dx
\end{equation}
В данном случае нельзя выносить $\bar{U}(p)$ за интеграл, поскольку $\bar{U}(p)$ меняется быстро ($U(x)$ меняется медленно). Поэтому уравнение самосогласования
\begin{equation}
    -\int\limits_{-\infty}^\infty\frac{\bar{U}(p)(dp)}{|E|+\frac{p^2}{2m}}=1
\end{equation}
\begin{equation}
    \int\limits_{-\infty}^\infty\left(\int\limits_{-\infty}^\infty \frac{e^2\exp(-\frac{ipx}{\hbar})}{2\pi\hbar\sqrt{x^2+a^2}(|E|+\frac{p^2}{2m})}dx\right)dp=1
\end{equation}
Поменяем пределы интегрирования:
\begin{equation}
    \int\limits_{-\infty}^\infty\left(\int\limits_{-\infty}^\infty \frac{e^2\exp(-\frac{ipx}{\hbar})}{2\pi\hbar\sqrt{x^2+a^2}(|E|+\frac{p^2}{2m})}dp\right)dx=1
\end{equation}
\begin{equation}
    \int\limits_{-\infty}^\infty\sqrt{\frac{m}{2|E|}}\frac{e^2\exp(-\frac{\sqrt{2m|E|}|x|}{\hbar})}{\hbar\sqrt{x^2+a^2}}dx=1
\end{equation}
Пусть $k=a\frac{\sqrt{2m|E|}}{\hbar}=\frac{a}{e}\sqrt{\frac{2|E|}{a_B}}$ и $u=\frac{|x|}{a}$, тогда
\begin{equation}
    2\int\limits_0^\infty\left(\frac{\exp(-ku)}{\sqrt{1+u^2}}\right)du=\frac{ka_B}{a}
\end{equation}
В пределе малых $k$ экспоненту можно разложить до 1 члена в ряде Тейлора и взять верхний предел $\frac{1}{k}$:
\begin{equation}
    \int\limits_0^\infty\left(\frac{\exp(-ku)}{\sqrt{1+u^2}}\right)du\approx \int\limits_0^{1/k}\frac{du}{\sqrt{1+u^2}}=\ln|u+\sqrt{1+u^2}|_0^{1/k}=\ln\left(\frac{2}{k}\right)
\end{equation}
\begin{equation}
    2\ln\left(\frac{2}{k}\right)=\frac{ka_B}{a}\rightarrow k\exp\left(\frac{ka_B}{2a}\right)=2
\end{equation}
Обозначим $z=\frac{ka_B}{2a}$.
\begin{equation}
    z\exp(z)=\frac{a_B}{a}
\end{equation}
Получилось трансцендентное уравение. Решим его методом последовательных приближений. Первое приближение (оно плохое, но от него ничего не зависит):
\begin{equation}
    z^{(0)}=1
\end{equation}
\begin{equation}
    z^{(1)}=\ln\left(\frac{a_B}{a}\right)
\end{equation}
Следующее приближение будет иметь поправку $\mathcal{O}(\ln\ln(a_B/a))$. Таким образом,
\begin{equation}
    z=\ln(a_B/a)+\mathcal{O}(\ln\ln(a_B/a))\rightarrow k=\frac{2a\ln(a_B/a)}{a_B}
\end{equation}
Энергия основного состояния
\begin{equation}
    \boxed{E=-2\frac{e^2}{a_B}\ln^2\frac{a_B}{a}}
\end{equation}
\section{Непрерывный спектр. Задача рассеяния}
\subsection*{Упражнения (25 баллов)}
\textbf{Упражнение 1 (10 баллов).}\\
Отнормируйте на дельта-функцию от энергии состояния непрерывного спектра, инфинитные в обе стороны для случая потенциала, имеющего различные асимптотики на бесконечностях $U(-\infty) = 0$, $U(+\infty) = U_0$. Для простоты, считайте $E > U_0 > 0$.\\
\textbf{Решение.}\\
Запишем стационарное уравнение Шрёдингера:
\begin{equation}
    -\frac{\hbar^2}{2m}\psi''+U(x)\psi=E\psi\rightarrow \psi''+\frac{2m}{\hbar^2}(E-U)\psi=0
\end{equation}
При $x\rightarrow-\infty$:
\begin{equation}
    \psi''+\frac{2mE}{\hbar^2}\psi=0
\end{equation}
\begin{equation}
    \psi(x)=a_Le^{ik_Lx}+b_Le^{-ik_Lx},\quad k_L=\frac{\sqrt{2mE}}{\hbar}
\end{equation}
При $x\rightarrow\infty$:
\begin{equation}
    \psi''+\frac{2m}{\hbar^2}(E-U_0)\psi=0
\end{equation}
\begin{equation}
    \psi(x)=a_Le^{ik_Rx}+b_Le^{-ik_Rx},\quad k_R=\frac{\sqrt{2m(E-U_0)}}{\hbar}
\end{equation}
Состояния стационарной задачи рассеяния:
\begin{equation}
    \psi_k^{(+)}(x)=\begin{cases}
    e^{ik_Lx}+re^{-ik_Lx},\; x\rightarrow -\infty\\
    te^{ik_Rx},\quad\quad\quad\quad\; x\rightarrow+\infty
    \end{cases},\quad
    \psi_k^{(-)}(x)=\begin{cases}
    t'e^{ik_Lx},\quad\quad\quad\quad\; x\rightarrow-\infty\\
    e^{ik_Rx}+r'e^{-ik_Rx},\; x\rightarrow+\infty
    \end{cases}
\end{equation}
Вероятности отражения $R$ и прохождения $T$:
\begin{equation}
    R=|r|^2,\quad T=\frac{k_R}{k_L}|t|^2,\quad R+T=1
\end{equation}
\begin{equation}
    \braket{\psi^{(+)}_{k_L}|\psi^{(+)}_{k'_L}}=\int\limits_{-\infty}^\infty dx\psi^{(+)*}_{k_L}\psi^{(+)}_{k'_L}=\int\limits_{-\infty}^\Lambda dx\psi^{(+)*}_{k_L}\psi^{(+)}_{k'_L}+\int\limits_\Lambda^\infty dx\psi^{(+)*}_{k_L}\psi^{(+)}_{k'_L}+\int\limits_{-\Lambda}^\Lambda dx\psi^{(+)*}_{k_L}\psi^{(+)}_{k'_L}
\end{equation}
При произвольном значении $\Lambda$, последний член конечен при $k = k'$ (а значит и при $E=E'$) и предствляет собой регулярную функцию от $k$ и $k'$. Поэтому дельта-функциональный вклад возникнуть может только с бесконечности, с первых двух членов. Проводя эти рассуждения для достаточно больших $\Lambda$, понимаем, что для нормировки можно пользоваться только асимптотическим представлением волновых функций.
\begin{equation}
    \braket{\psi^{(+)}_{k_L}|\psi^{(+)}_{k'_L}}=\text{reg.}+\int\limits_{-\infty}^\Lambda dx(e^{-ik_Lx}+r_k^*e^{ik_Lx})(e^{ik'_Lx}+r_{k'}e^{-ik'_Lx})+\int\limits_\Lambda^\infty dxt^*_ke^{-ik_R x}t_{k'}e^{-ik'_R x}
\end{equation}
Уже в этом выражении заменяя пределы интегрирования на ноль, добавим регулярный вклад, и характер сингулярности не изменится. Поскольку амплитуды являются <<медленными огибающими>>, то в них мы можем положить $k = k'$.
\begin{equation}
    \braket{\psi^{(+)}_{k_L}|\psi^{(+)}_{k'_L}}=\text{reg.}+\int\limits_{-\infty}^0 dx(e^{-i(k_L-k'_L)x}+|r_k|^2e^{i(k_L-k'_L)x})+\int\limits_0^\infty dx|t_k|^2e^{-i(k_R-k'_R) x}
\end{equation}
\begin{equation}
    k^2_L-k'^2_L=\frac{2m(E-E')}{\hbar^2}=k^2_R-k'^2_R
\end{equation}
\begin{equation}
    k_R-k'_R=\frac{k^2_R-k'^2_R}{k_R+k'_R}=\frac{k^2_L-k'^2_L}{k_R+k'_R}
\end{equation}
\begin{equation*}
    \int\limits_0^\infty dx|t_k|^2e^{-i(k_R-k'_R) x}=\int\limits_0^\infty du\frac{k_R+k'_R}{k_L+k'_L}|t_k|^2e^{-i(k_L-k'_L)u}=\int\limits_0^\infty du\frac{k_R}{k_L}|t_k|^2e^{-i(k_L-k'_L)u}=\int\limits_0^\infty dxTe^{-i(k_L-k'_L)x}
\end{equation*}
\begin{equation*}
    \braket{\psi^{(+)}_{k_L}|\psi^{(+)}_{k'_L}}=\text{reg.}+\int\limits_{-\infty}^0 dxe^{-i(k_L-k'_L)x}+\int\limits_{0}^\infty Re^{-i(k_L-k'_L)x}+\int\limits_0^\infty dxTe^{-i(k_L-k'_L)x}=\int\limits_{-\infty}^\infty dxe^{-i(k_L-k'_L)x}
\end{equation*}
\begin{equation}
    \braket{\psi^{(+)}_{k_L}|\psi^{(+)}_{k'_L}}=\text{reg.}+2\pi\delta(k_L-k'_L)=\text{reg.}+\frac{2\pi\hbar}{\sqrt{2m}}\delta(\sqrt{E}-\sqrt{E'})=\text{reg.}+\frac{2\pi\hbar}{\sqrt{2m}}\delta\left(\frac{E-E'}{\sqrt{E}+\sqrt{E'}}\right)
\end{equation}
\begin{equation}
    \boxed{\braket{\psi^{(+)}_E|\psi^{(+)}_{E'}}=\text{reg.}+2\pi\hbar\sqrt{\frac{2E}{m}}\delta(E-E')}
\end{equation}
\textbf{Упражнение 2 (15 баллов).}\\
Рассмотрите движение одномерной частицы в поле мелкой ямы, $U(x) = -\frac{\kappa}{m}\delta(x)$. Покажите непосредственным вычислением полноту базиса собственных состояний гамильтониана. Рассмотрите как случай $\kappa > 0$ (когда в яме имеется связанное состояние), так и $\kappa < 0$.\\
\textit{Указание:} при проверке условия полноты, $\delta(x-x') =\sum\limits_n\psi^*_n(x)\psi_n(x')$, для простоты можете рассмотреть только случай $x, x' > 0$.\\
\textbf{Решение.}\\
Состояния стационарной задачи рассеяния:
\begin{equation}
    \psi_k^{(+)}(x)=\begin{cases}
    e^{ikx}+re^{-ikx},\; x\rightarrow -\infty\\
    te^{ikx},\quad\quad\quad\;\;\; x\rightarrow+\infty
    \end{cases},\quad
    \psi_k^{(-)}(x)=\begin{cases}
    t'e^{ikx},\quad\quad\quad\;\;\; x\rightarrow-\infty\\
    e^{ikx}+r'e^{-ikx},\; x\rightarrow+\infty
    \end{cases}
\end{equation}
Запишем граничные условия для $\psi_k^{(+)}(0)$:
\begin{equation}
    1+r=t,\quad ikt-(ik-ikr)=-2\kappa t \rightarrow t=\frac{ik}{\kappa+ik},\quad r=\frac{\kappa}{\kappa+ik}
\end{equation}
\begin{equation}
    t'=t,\quad r'=-\frac{r^*}{t^*}t=\frac{\kappa}{ik}\frac{-ik}{\kappa+ik}=-\frac{\kappa}{\kappa+ik}=r
\end{equation}
\begin{multline*}
    f(x,x')=\frac{1}{2\pi}\int\limits_0^\infty dk(e^{ik(x-x')}+|r|^2e^{-ik(x-x')}+|t|^2e^{-ik(x-x')}+\\+re^{ik(x+x')}+r^*e^{-ik(x+x')})=\frac{1}{2\pi}\int\limits_0^\infty dk(e^{ik(x-x')}+e^{-ik(x-x')})+\frac{1}{2\pi}\int\limits_0^\infty dk\left( \frac{\kappa}{\kappa+ik}e^{ik(x+x')}+\frac{1}{2\pi}\frac{\kappa}{\kappa-ik}e^{-ik(x+x')}\right)
\end{multline*}
\begin{equation}
    \int\limits_0^\infty dk(e^{ik(x-x')}+e^{-ik(x-x')})=\int\limits_{-\infty}^\infty dke^{ik(x-x')}=2\pi\delta(x-x')
\end{equation}
\begin{equation*}
    \int\limits_0^\infty dk\left( \frac{\kappa}{\kappa+ik}e^{ik(x+x')}+\frac{\kappa}{\kappa-ik}e^{-ik(x+x')}\right)= \int\limits_{-\infty}^\infty dk \frac{\kappa}{\kappa+ik}e^{ik(x+x')}=
    \begin{cases}
        2\pi\exp(-\kappa(x+x')),\quad \kappa>0\\
        0,\quad\quad\quad\quad\quad\quad\quad\quad\quad \kappa<0
    \end{cases}
\end{equation*}
Последний интеграл взят при помощи вычетов.
Рассмотрим случаи:
\begin{enumerate}
    \item $\kappa<0$:
    \begin{equation}
        f(x,x')=\delta(x-x')
    \end{equation}
    \item $\kappa>0$:
    Добавим к $f(x,x')$ связанное состояние $\psi_0(x)$: $\psi_0^*(x)\psi_0(x')=\kappa e^{-\kappa(|x|+|x'|)}$. Тогда и в этом случае
    \begin{equation}
        f(x,x')=\delta(x-x')
    \end{equation}
\end{enumerate}
Таким образом, соотношение полноты доказано в обоих случаях.
\subsection*{Задачи (75 баллов)}
\textbf{Задача 1. Эволюция волновой функции (25 баллов)}
\begin{enumerate}
    \item Частица находится в основном состоянии гармонического осциллятора $U(x) = \frac{m\omega^2x^2}{2}$. Потенциал выключают на время $T$, спустя которое его снова включают. Определите вероятность того, что частица окажется в \textit{основном} состоянии.
    \item Определите такую вероятность для потенциала $-\frac{\hbar^2\kappa}{m}\delta(x)$, но для случаев малых и больших времён $T$.
\end{enumerate}
\textit{Подсказка:} для асимптотической оценки интеграла в пункте 2 для случая малых $T$ вам может пригодиться приём -- преобразование Хаббарда-Стратановича (гауссов интеграл, заменяющий квадратичное по $A$ выражение в экспоненте на линейное):
\begin{equation}
    e^{-iA^2}=\frac{e^{\frac{i\pi}{4}}}{\sqrt{\pi}}\int_{-\infty}^\infty dze^{iz^2-2izA}
\end{equation}
\textbf{Решение.}\\
Решим задачу для произвольного потенциала $U(x)$, а затем подставим конкретные потеницалы. Cтационарное уравнение Шрёдингера:
\begin{equation}
    -\frac{\hbar^2}{2m}\psi''+U(x)\psi=E\psi\rightarrow \psi''+\frac{2m}{\hbar^2}\left(E-U(x)\right)\psi=0
    \end{equation}
Далее определяем основное состояние как решение этого уравнения. Перейдём в импульсное представление:
\begin{equation}
     \Psi(p)=\int\limits_{-\infty}^\infty\psi(x)\exp\left(-\frac{ipx}{\hbar}\right)dx
\end{equation}
Рассмотрим эволюциию частицы при выключенном потенциале $U=0$. Нестационарное уравнение Шрёдингера в импульсном представлении:
\begin{equation}
    i\hbar\frac{\partial\Psi(p,t)}{\partial t}=\bra{p}\hat{H}(t)\ket{\psi(t)}=\frac{p^2}{2m}\Psi(p,t),\quad \Psi(p,0)=\Psi(p)
\end{equation}
\begin{equation}
    \Psi(p,T)=\Psi(p)\exp\left(-\frac{ip^2}{2m\hbar}T\right)
\end{equation}
Вероятность того, что частица окажется в основном состоянии:
\begin{equation}
    P=\left|\int\limits_{-\infty}^\infty(dp)\Psi^*(p)\Psi(p,T)\right|^2=\left|\int\limits_{-\infty}^\infty(dp)\Psi^2(p)\exp\left(-\frac{ip^2}{2m\hbar}T\right)\right|^2
\end{equation}
\begin{enumerate}
    \item Гармонический осциллятор
    \begin{equation}
        U(x)=\frac{m\omega^2x^2}{2}
    \end{equation}
    Основное состояние:
    \begin{equation}
        \psi(x)=\left(\frac{m\omega}{\pi\hbar}\right)^\frac{1}{4}\exp\left(-\frac{m\omega x^2}{2\hbar}\right)
    \end{equation}
    Перейдём в импульсное представление:
    \begin{equation}
        \Psi(p)=\int\limits_{-\infty}^\infty\psi(x)\exp\left(-\frac{ipx}{\hbar}\right)dx=\sqrt{2}\left(\frac{\pi\hbar}{m\omega^3}\right)^\frac{1}{4}\exp\left(-\frac{p^2}{2m\omega^2\hbar}\right)
    \end{equation}
    Эволюция частицы при выключенном потенциале $U=0$:
    \begin{equation}
        \Psi(p,T)=\Psi(p)\exp\left(-\frac{ip^2}{2m\hbar}T\right)
    \end{equation}
    Вероятность того, что частица окажется в основном состоянии:
    \begin{multline}
        P(T)=\left|\int\limits_{-\infty}^\infty(dp)\Psi^2(p)\exp\left(-\frac{ip^2}{2m\hbar}T\right)\right|^2=\\=\left|\int\limits_{-\infty}^\infty(dp)\left(\sqrt{2}\left(\frac{\pi\hbar}{m\omega^3}\right)^\frac{1}{4}\exp\left(-\frac{p^2}{2m\omega^2\hbar}\right)\right)^2\exp\left(-\frac{ip^2}{2m\hbar}T\right)\right|^2
    \end{multline}
    \begin{equation}
        \boxed{P(T)=\frac{1}{\sqrt{1+\frac{\omega^2T^2}{4}}}}
    \end{equation}
    \item 
    \begin{equation}
        U(x)=-\frac{\hbar^2\kappa}{m}\delta(x)
    \end{equation}
    Собственные функции для гамильтониана с таким потенциалом были найдены в задаче 3.1:
    \begin{equation}
        \psi(x)=\sqrt{\kappa}\exp(-\kappa|x|)
    \end{equation}
    Перейдём в импульсное представление:
    \begin{equation}
        \Psi(p)=\int\limits_{-\infty}^\infty\psi(x)\exp\left(-\frac{ipx}{\hbar}\right)dx=\frac{2\kappa^\frac{3}{2}\hbar^2}{p^2+\kappa^2\hbar^2}
    \end{equation}
    Вероятность того, что частица окажется в основном состоянии:
    \begin{multline}
        P(T)=\left|\int\limits_{-\infty}^\infty(dp)\Psi^2(p)\exp\left(-\frac{ip^2}{2m\hbar}T\right)\right|^2=\left|\int\limits_{-\infty}^\infty(dp)\left(\frac{2\kappa^\frac{3}{2}\hbar^2}{p^2+\kappa^2\hbar^2}\right)^2\exp\left(-\frac{ip^2}{2m\hbar}T\right)\right|^2
    \end{multline}
    Обозначим $A=p\sqrt{\frac{T}{2m\hbar}}$. Преобразование Хаббарда-Стратановича:
    \begin{equation}
        e^{-iA^2}=\frac{e^{\frac{i\pi}{4}}}{\sqrt{\pi}}\int_{-\infty}^\infty dze^{iz^2-2izA}=\frac{e^{\frac{i\pi}{4}}}{\sqrt{\pi}}\int_{-\infty}^\infty dze^{iz^2-2izp\sqrt{\frac{T}{2m\hbar}}}
    \end{equation}
    \begin{equation}
        P(T)=\left|\int\limits_{-\infty}^\infty(dp)\left(\frac{2\kappa^\frac{3}{2}\hbar^2}{p^2+\kappa^2\hbar^2}\right)^2\frac{e^{\frac{i\pi}{4}}}{\sqrt{\pi}}\int_{-\infty}^\infty dze^{iz^2-2izp\sqrt{\frac{T}{2m\hbar}}}\right|^2
    \end{equation}
    Проинтегрируем по $p$ при помощи вычетов (полюсы в $p=\pm a$ 2 порядка):
    \begin{equation}
        P(T)=\frac{16\kappa^6\hbar^8}{4\pi^3\hbar^2}\frac{\pi^2}{4\hbar^6\kappa^6}\left|2\int\limits_0^\infty dze^{-2\kappa\hbar\sqrt{\frac{T}{2m\hbar}}z}\left(1+2\kappa\hbar\sqrt{\frac{T}{2m\hbar}}z\right)e^{iz^2}\right|^2
    \end{equation}
    \begin{equation}
         P(T)=\frac{4}{\pi}\left|\int\limits_0^\infty dze^{-\kappa\sqrt{\frac{2\hbar T}{m}}z}\left(1+\kappa\sqrt{\frac{2\hbar T}{m}}z\right)e^{iz^2}\right|^2
    \end{equation}
    \begin{equation*}
        P(T)=\frac{4}{\pi}\left|\frac{1}{8}\left(4ia+(1+i)(a^2+2i)e^{i\frac{a^2}{4}}\sqrt{2}\pi(-i-(1-i)\int_0^{a/\sqrt{2}\pi}dt\cos^2t+(1+i)\int_0^{a/\sqrt{2}\pi}dt\sin^2t\right)\right|^2
    \end{equation*}
    \begin{equation}
        \int_0^{a/\sqrt{2}\pi}dt\cos^2t\sim x,\quad \int_0^{a/\sqrt{2}\pi}dt\sin^2t\sim\frac{x^3}{3}
    \end{equation}
    Пусть $a=\sqrt{\frac{2\hbar T}{m}}\kappa\ll1$, тогда
    \begin{equation}
        P(T)=\frac{4}{\pi}\left|\frac{1+i}{2}\sqrt{\frac{\pi}{2}}+\frac{\sqrt{\pi}(1-i)}{8\sqrt{2}}a^2+\mathcal{O}(a^3)\right|^2=\frac{4}{\pi}\left(\frac{\pi}{4}-\frac{\sqrt{\pi}}{6\sqrt{2}}a^3\right)=1-\frac{4\sqrt{\pi}}{6\sqrt{2}\pi}a^3+\mathcal{O}(a^3)
    \end{equation}
    При малых временах $T$:
    \begin{equation}
        \boxed{P(T)=1-\frac{4\kappa^3}{3\sqrt{\pi}}\left(\frac{\hbar T}{m}\right)^\frac{3}{2}+\mathcal{O}(T^3)}
    \end{equation}
    При больших временах $T$:
    \begin{equation}
        \int_0^{a/\sqrt{2}\pi}dt\cos^2t\sim\sqrt{\frac{\pi}{8}}-\frac{\cos x^2}{2x},\quad \int_0^{a/\sqrt{2}\pi}dt\sin^2t\sim\sqrt{\frac{\pi}{8}}+\frac{\sin x^2}{2x}
    \end{equation}
    \begin{equation}
        P(T)=\frac{4}{\pi}\left|\frac{2}{a}+\mathcal{O}\left(\frac{1}{a^3}\right)\right|=\frac{16}{\pi a^2}+\mathcal{O}\left(\frac{1}{a^3}\right)
    \end{equation}
    \begin{equation}
        \boxed{P(T)=\frac{8m}{\pi\hbar\kappa^2T}+\mathcal{O}\left(\frac{1}{T^3}\right)}
    \end{equation}
\end{enumerate}
\textbf{Задача 2. Квазистационарные состояния (50 баллов)}\\
В этой задаче мы будем исследовать уравнение Шрёдингера в потенциале следующего вида:
\begin{equation}
    U(x)=\begin{cases}
        \infty, \quad\quad\quad x<0\\
        \frac{\kappa}{m}\delta(x-a),\; x>0
    \end{cases}
\end{equation}
Рассеивающий потенциал (величину $\kappa$) мы будем предполагать сильным. Если заменить этот потенциал на бесконечную стенку, то в этой задаче имеется множество стационарных состояний вида $\psi_n(x) = \sin knx$, $k_n =\frac{\pi n}{a}$ и $E_n^{(0)}=\frac{k_n^2}{2m}$. Возможность туннелирования превращает эти состояния в \textit{квазистационарные} -- хоть и не стационарные, но долгоживущие, и на достаточно небольших промежутках времени их можно считать стационарными. Ниже будут изложены три различных точки зрения на такие состояния и показана их эквивалентность.\\
\textit{Указание:} хоть дальше и идёт речь об энергетическом представлении -- во всех промежуточных вычислениях удобнее
пользоваться квантовым числом — импульсом $k =\sqrt{2mE}$.\\
\textbf{Задача 2.1. Распад квазистационарного состояния (25 баллов)}\\
В начальный момент времени частица была «приготовлена» в одном из квазистационарных состояний, $\psi(x, t = 0) =\psi_0(x)\sqrt{\frac{2}{a}}\sin k_nx$.
\begin{enumerate}
    \item Определите непрерывный спектр исходной задачи, нормированный на дельта-функцию от разницы энергий.
    \item Рассмотрите разложение исходной волновой функции по этому непрерывному спектру. Обратите внимание, что $|\psi(E)|^2$ вблизи $E_n^{(0)}$ имеет максимум; покажите, что в окрестности этого максимума вероятность устроена как лоренциан:
    \begin{equation}
        |\psi(E)|^2\approx c\frac{\Gamma_n}{(E-E_n)^2+\frac{\Gamma_n^2}{4}}
    \end{equation}
    и при этом $E_n$ близка к $E_n^{(0)}$, а $\Gamma_n \ll E_n$. Найдите величины $c$, $E_n$, $\Gamma_n$.
    \item Найдите амплитуду того, что через достаточно большое время t частица останется в в этом квазистационарном состоянии $c(t)=\int\psi^*_0(x)\psi(x,t)dx$. Покажите, что на достаточно больших временах соответствующая вероятность $P(t) = |c(t)|^2$ затухает экспоненциально $P(t) \propto \exp (-t/\tau)$. Определите соответствующее «время жизни» $\tau_n$.
\end{enumerate}
Найденное <<время жизни>> окажется достаточно большим. Это означает, что на временах $t \ll \tau_n$, состояние $\ket{n}$ практически стационарно -- волновая функция не меняется, за исключением тривиальной динамической фазы $e^{-iE_nt}$.\\
\textbf{Решение.}
\begin{enumerate}
    \item 
    \item
    \begin{equation}
        |\psi(E)|^2=\frac{1}{8\sqrt{2mE}\pi k_n}\frac{\frac{k_n^3}{2m\kappa^2a}}{(E-E_n^{(0)}(1-\frac{1}{2\kappa a}))^2+\frac{1}{4}(\frac{k_n^3}{2m\kappa^2a})^2}    
    \end{equation}
    \begin{equation}
        \boxed{c=\frac{1}{8\sqrt{2mE}\pi k_n},\quad E_n=E_n^{(0)}\left(1-\frac{1}{2\kappa a}\right),\quad\Gamma_n=\frac{k_n^3}{2m\kappa^2a}}
    \end{equation}
    Как видно, $E_n$ близко к $E_n^{(0)}$ и $\Gamma_n\ll E_n$.
    \item
    \begin{equation}
        \psi(E)=\sqrt{c}\frac{\sqrt{\Gamma_n}}{(E-E_n)+\frac{i\Gamma_n}{2}}    
    \end{equation}
\end{enumerate}
\textbf{Задача 2.2. Задержка волнового пакета (20 баллов)}\\
Другой способ <<смотреть>> на квазистационарные состояния -- это исследовать задачу рассеяния.
\begin{enumerate}
    \item Определите волновые функции исходной задачи, которые соответствуют задаче рассеяния, то есть имеющие асимптотическое поведение $\psi(x\rightarrow\infty) = e^{-ikx} + re^{ikx}$. Величина коэффициента отражения $R = |r|^2$ тривиальным образом равна единице, поскольку движение инфинитно только в одну сторону; но вот амплитуда отражения $r(E) \equiv e^{i\delta(E)}$ имеет нетривиальные свойства.
    \item Покажите, что амплитуда $r$, как функция \textit{комплексной} энергии $\varepsilon$ имеет серию полюсов, близких к вещественной оси $\varepsilon_n = E_n - \frac{i\Gamma_n}{2}$. Покажите, что на вещественной оси в окрестности $E\approx E_n$ амплитуда устроена следующим образом:
    \begin{equation}
        r(E)\approx e^{i\phi}\frac{E-E_n-\frac{i\Gamma_n}{2}}{E-E_n+\frac{i\Gamma_n}{2}}
    \end{equation}
    (с такими же величинами $E_n$ и $\Gamma_n$; такой вид связан с условием $|r|^2 = 1$). Как при этом устроена фаза рассеяния $\delta(E)$?
    \item Рассмотрите нестационарную задачу рассеяния: пусть на такую систему налетает волновой пакет, локализованный по энергии вблизи одного из таких метастабильных состояний, $E\approx E_n$. Определите время задержки такого волнового пакета $\tau = \hbar\frac{\partial\delta}{\partial E}$; и покажите, что непосредственно при резонансе $E = E_n$, время задержки связано с временем жизни метастабильного состояния. Обратите внимание -- существенно, что полюса находятся в нижней комплексной полуплоскости; в противном случае мы бы получили $\tau(E_n) < 0$, что очевидным образом противоречит причинности.
\end{enumerate}
С этим эффектом связана иная физическая интерпретация квазистационарных состояний: при рассеянии на резонансе частица «попадает» в метастабильное состояние, где она «застревает» на достаточно продолжительное время, и лишь затем вылетает.\\
\textbf{Решение.}
\begin{enumerate}
    \item 
    \begin{equation}
        \psi(x\rightarrow\infty) = e^{-ikx} + re^{ikx}
    \end{equation}
    \item Амплитуда отражения:
    \begin{multline}
        r=\frac{1-\frac{i\kappa}{k}(1-e^{-2ika})}{-1-\frac{i\kappa}{k}(1-e^{2ika})}=-e^{-2ika}\frac{ke^{ika}-i\kappa e^{ika}+i\kappa e ^{-ika}}{ke^{-ika}+i\kappa e^{-ika}-i\kappa e^{ika}}=\\=e^{-2i(k-k_n)a}\frac{ik_n(k-k_n)a+k_n+2\kappa(k-k_n)a}{ik_n(k-k_n)a-k_n-2\kappa(k-k_n)a}
    \end{multline}
    \begin{equation}
        k_n(k-k_n)=\frac{m(E-E_n^{(0)})}{\hbar^2}
    \end{equation}
    \begin{equation}
        r=e^{-2i(k-k_n)a}\frac{ia\frac{m(E-E^{(0)}_n)}{\hbar^2}+k_n+2\kappa a\frac{m(E-E_n^{(0)})}{k_n\hbar^2}}{ia\frac{m(E-E^{(0)}_n)}{\hbar^2}-k_n-2\kappa a\frac{m(E-E_n^{(0)})}{k_n\hbar^2}}
    \end{equation}
    Полюсы:
    \begin{equation}
        i(E-E^{(0)}_n)-\frac{\hbar^2k_n}{ma}-\frac{2\kappa}{k_n}(E-E_n^{(0)})=0
    \end{equation}
    \begin{equation}
        (E-E_n^{(0)})\left(i-\frac{2\kappa}{k_n}\right)=\frac{\hbar^2k_n}{ma}
    \end{equation}
    \begin{equation}
        E-E_n^{(0)}=\frac{\hbar^2k_n^2}{2\kappa ma}\frac{1}{\frac{ik_n}{2\kappa}-1}\approx-\frac{\hbar^2k_n^2}{2\kappa ma}\left(1+\frac{ik_n}{2\kappa}\right)=-\frac{\hbar^2E^{(0)}_n}{\kappa a}-\frac{i\hbar^2k_n^3}{4\kappa^2ma}
    \end{equation}
    \begin{equation}
        E_n=E^{(0)}_n\left(1-\frac{\hbar^2}{\kappa a}\right)-\frac{i\hbar^2k_n^3}{4\kappa^2ma}
    \end{equation}
    \begin{equation}
        r=\frac{E-E_n-i\frac{\Gamma_n}{2}}{E-E_n+i\frac{\Gamma_n}{2}}=\frac{(E-E_n-i\frac{\Gamma_n}{2})^2}{(E-E_n)^2+\frac{\Gamma^2_n}{4}}=\frac{(E-E_n)^2-\frac{\Gamma_n^2}{4}-i\Gamma_n(E-E_n)}{(E-E_n)^2+\frac{\Gamma^2_n}{4}}
    \end{equation}
    \begin{equation}
        \boxed{\delta(E)=-\arctan\frac{\Gamma_n(E-E_n)}{(E-E_n)^2-\frac{\Gamma^2_n}{4}}}
    \end{equation}
    \item Время задержки:
    \begin{equation}
        \tau(E)=\hbar\frac{\partial\delta}{\partial E}=\frac{\Gamma_n\hbar}{(E-E_n)^2+\frac{\Gamma_n^2}{4}}
    \end{equation}
    \begin{equation}
        \boxed{\tau(E_n)=\frac{4\hbar}{\Gamma_n}}
    \end{equation}
\end{enumerate}
\section{Точно решаемые потенциалы. Часть 1}
\subsection*{Упражнения (40 баллов)}
\textbf{Упражнение 1 (20 баллов)}\\
Исследуйте прямоугольную двумерную яму $U(r)=-U_0\theta(a-|r|)$ в обратном пределе глубокой ямы $U_0\gg\frac{\hbar^2}{Ma^2}$. Оцените $N_m$ -- количество связанных состояний в яме для фиксированного орбитального квантового числа $m \in \mathbb{Z}$, а также полное число связанных состояний $N =\sum\limits_m N_m$.\\
Определите промежуточную асимптотику уровней энергии $E_{n,m}$ (интересны высоколежащие уровни, но не близкие к <<верху ямы>>).\\
\textbf{Решение.}\\
Стационарное уравнение Шрёдингера:
\begin{equation}
    \hat{H}\psi=E\psi
\end{equation}
\begin{equation}
    \hat{H}=\frac{\hat{\vec{p}}^{\;2}}{2m}+U(\vec{r})=-\frac{\hbar^2}{2m}\Delta-U_0\theta(a-|r|)
\end{equation}
Лапласиан в полярных координатах:
\begin{equation}
    \Delta=\frac{1}{r}\frac{\partial}{\partial r}\left(r\frac{\partial}{\partial r}\right)+\frac{1}{r^2}\frac{\partial^2}{\partial\varphi^2}=\frac{\partial^2}{\partial r^2}+\frac{1}{r}\frac{\partial}{\partial r}+\frac{1}{r^2}\frac{\partial^2}{\partial\varphi^2}
\end{equation}
Разделим переменные на радиальную и угловую части:
\begin{equation}
    \psi(r,\varphi)=\psi(r)e^{im\varphi}
\end{equation}
Уравнение на радиальную часть:
\begin{equation}
    -\frac{\hbar^2}{2M}\left(\psi''(r)+\frac{1}{r}\psi'(r)-\frac{m^2}{r^2}\psi(r)\right)-U_0\theta(a-|r|)\psi(r)=E\psi(r)
\end{equation}
\begin{enumerate}
    \item Вне ямы $r>a$:
    \begin{equation}
        -\frac{\hbar^2}{2M}\left(\psi''(r)+\frac{1}{r}\psi'(r)-\frac{m^2}{r^2}\psi(r)\right)=E\psi(r)
    \end{equation}
    Пусть $\chi^2=-\frac{2ME}{\hbar^2}$ и $z=\chi r$, тогда
    \begin{equation}
        \psi'(r)=\frac{d\psi}{dz}\frac{dz}{dr}=\chi\psi'(z),\quad \psi''(r)=\frac{d\psi'}{dz}\frac{dz}{dr}=\chi^2\psi''(z)
    \end{equation}
    \begin{equation}
        \psi''(z)+\frac{1}{z}\psi'(z)-\left(1+\frac{m^2}{z^2}\right)\psi(z)=0
    \end{equation}
    Решение уравнения:
    \begin{equation}
        \psi(z)=A_1I_m(z)+A_2K_m(z)
    \end{equation}
    Асимптотика при больших $z\gg1$:
    \begin{equation}
        I_m(z)\approx\frac{e^z}{\sqrt{2\pi z}},\quad K_m(z)\approx\sqrt{\frac{\pi}{2z}}e^{-z}
    \end{equation}
    Поскольку волновая функция должна затухать на бесконечности, $A_1=0$, $A_2=A$:
    \begin{equation}
        \psi(z)=AK_m(z)
    \end{equation}
    \item Внутри ямы $r<a$:
    \begin{equation}
        -\frac{\hbar^2}{2M}\left(\psi''(r)+\frac{1}{r}\psi'(r)-\frac{m^2}{r^2}\psi(r)\right)-U_0\psi(r)=E\psi(r)
    \end{equation}
    Пусть $k^2=\frac{2M(U_0+E)}{\hbar^2}$ и $z=kr$, тогда
    \begin{equation}
        \psi'(r)=\frac{d\psi}{dz}\frac{dz}{dr}=k\psi'(z),\quad \psi''(r)=\frac{d\psi'}{dz}\frac{dz}{dr}=k^2\psi''(z)
    \end{equation}
    \begin{equation}
        \psi''(z)+\frac{1}{z}\psi'(z)+\left(1-\frac{m^2}{z^2}\right)\psi(z)=0
    \end{equation}
    Решение уравнения:
    \begin{equation}
        \psi(z)=B_1J_m(z)+B_2Y_m(z)
    \end{equation}
    Асимптотика при малых $z\ll1$:
    \begin{equation}
        J_m(z)\approx\frac{1}{m!}\left(\frac{z}{2}\right)^m,\quad Y_m(z)\approx
        \begin{cases}
            -\frac{(m-1)!}{\pi}\left(\frac{2}{z}\right)^m,\quad m\neq0\\
            -\frac{2}{\pi}(\ln\frac{2}{z}-\gamma),\quad m=0
        \end{cases}
    \end{equation}
    Поскольку волновая функция должна быть регулярной в 0, $B_1=B$, $B_2=0$:
    \begin{equation}
        \psi(z)=BJ_m(z)
    \end{equation}
\end{enumerate}
Сшивка логарифмической производной:
\begin{equation}
    \frac{d\ln\psi(a-0)}{dz}=\frac{d\ln\psi(a+0)}{dz}\rightarrow\frac{kaJ'_m(ka)}{J_m(ka)}=\frac{\chi aK'_m(\chi a)}{K_m(\chi a)}
\end{equation}
Рассматривается задача о глубокой яме $U_0\gg\frac{\hbar^2}{Ma^2}$. поэтому $z\gg 1$. Соответствующие асимптотики при $z\gg 1$:
\begin{equation}
    J_m(z)\approx\sqrt{\frac{2}{\pi z}}\cos\left(z-\frac{m\pi}{2}-\frac{\pi}{4}\right),\quad K_m(z)\approx\sqrt{\frac{\pi}{2z}}e^{-z} 
\end{equation}
\begin{equation*}
    J'_m(z)=-\sqrt{\frac{\pi}{2z^3}}\left(z\sin\left(z-\frac{\pi m}{2}-\frac{\pi}{4}\right)+\frac{1}{2}\cos\left(z-\frac{\pi m}{2}-\frac{\pi}{4}\right)\right),\quad K'_m(z)=-\sqrt{\frac{\pi}{2z^3}}e^{-z}\left(\frac{1}{2}+z\right)
\end{equation*}
Рассмотрим случай $\chi a\gg ka\gg 1$.
    \begin{equation}
        -ka\frac{1}{ka}\left(ka\tan\left(ka-\frac{\pi m}{2}-\frac{\pi}{4}\right)+\frac{1}{2}\right)=-\chi a\frac{1}{\chi a}\left(\frac{1}{2}+\chi a\right)
    \end{equation}
    \begin{equation}
        \tan\left(ka-\frac{\pi m}{2}-\frac{\pi}{4}\right)=\frac{\chi}{k}\gg1
    \end{equation}
    \begin{equation}
        ka-\frac{\pi m}{2}-\frac{\pi}{4}=\frac{\pi}{2}+\pi n\rightarrow ka=\frac{3\pi}{4}+\frac{\pi m}{2}+\pi n
    \end{equation}
    \begin{equation}
        k^2a^2=\pi^2\left(\frac{3}{4}+\frac{m}{2}+n\right)^2=\frac{2Ma^2(U_0-|E_{n.m}|)}{\hbar^2}
    \end{equation}
    \begin{equation}
        \boxed{E_{n,m}=U_0-\frac{\pi^2\hbar^2}{2Ma^2}\left(\frac{3}{4}+\frac{m}{2}+n\right)^2}
    \end{equation}
    Число решений:
    \begin{equation}
        \boxed{N_m=\frac{\sqrt{2Ma^2U_0}}{\pi\hbar}-\frac{m}{2}-\frac{3}{4}}
    \end{equation}
    \begin{equation}
        N_{m_{\text{max}}}=0\rightarrow m_{\text{max}}=2\frac{\sqrt{2Ma^2U_0}}{\pi\hbar}-\frac{3}{2}
    \end{equation}
    \begin{equation}
        N=\sum_{m=0}^{m_\text{max}} N_m=\sum_{m=0}^{m_\text{max}}\left(\frac{\sqrt{2Ma^2U_0}}{\pi\hbar}-\frac{m}{2}-\frac{3}{4}\right)\approx\frac{m_\text{max}(m_\text{max}+1)}{2}-\frac{m_\text{max}(m_\text{max}-1)}{2}=\frac{m_\text{max}^2}{4}
    \end{equation}
    \begin{equation}
        \boxed{N=\frac{2Ma^2U_0}{\pi^2\hbar^2}}
    \end{equation}
\textbf{Упражнение 2 (20 баллов)}\\
Определите волновую функцию для движения в трёхмерном Кулоновском потенциале $U(\vec{r}) = -\alpha/r$ частицы с орбитальным квантовым числом $L$ на строго нулевой энергии $E = 0$.\\
\textit{Указание:} соответствующее уравнение сводится к уравнению Бесселя.\\
\textbf{Решение.}\\
Стационарное уравнение Шрёдингера:
\begin{equation}
    \hat{H}\Psi=E\Psi=0
\end{equation}
\begin{equation}
    \hat{H}=\frac{\hat{\vec{p}}^{\;2}}{2M}+U(\vec{r})=-\frac{\hbar^2}{2M}\Delta-\frac{\alpha}{r}
\end{equation}
Лапласиан в сферических координатах:
\begin{equation}
    \Delta=\frac{1}{r^2}\frac{\partial}{\partial r}\left(r^2\frac{\partial}{\partial r}\right)+\frac{1}{r^2\sin\theta}\frac{\partial}{\partial\theta}\left(\sin\theta\frac{\partial}{\partial\theta}\right)+\frac{1}{r^2\sin\theta}\frac{\partial}{\partial\varphi^2}
\end{equation}
Разделим переменные на радиальную и угловую части:
\begin{equation}
    \Psi(r,\theta,\varphi)=\psi(r)Y(\theta,\varphi)
\end{equation}
Уравнение на радиальную часть:
\begin{equation}
    -\frac{\hbar^2}{2mr^2}(r^2\psi'(r))'+\left(\frac{\hbar^2L(L+1)}{2mr^2}-\frac{\alpha}{r}\right)\psi(r)=0
\end{equation}
\begin{equation}
    \psi''(r)+\frac{2}{r}\psi'(r)-\left(\frac{L(L+1)}{r^2}-\frac{2m\alpha}{\hbar r}\right)\psi(r)=0
\end{equation}
Сделаем замену $z=2\sqrt{\frac{2m\alpha r}{\hbar}}\rightarrow r=\frac{z^2\hbar}{8m\alpha}$, $u(z)=\psi(r)$:
\begin{equation}
    \psi'(r)=\frac{d\psi}{dr}=\frac{dz}{dr}\frac{du}{dz}=\sqrt{\frac{2m\alpha}{\hbar r}}u'(z)=\frac{4m\alpha}{\hbar z}u'(z)
\end{equation}
\begin{equation}
    \psi''(r)=\frac{d^2\psi}{dr^2}=\frac{4m\alpha}{\hbar z}\frac{d}{dz}\left(\frac{4m\alpha}{\hbar z}u'(z)\right)=\frac{16m^2\alpha^2}{\hbar^2z^3}u'(z)+\frac{16m^2\alpha^2}{\hbar^2z^2}u''(z)
\end{equation}
\begin{equation}
    \frac{16m^2\alpha^2}{\hbar^2z^2}u''(z)+\frac{48m^2\alpha^2}{\hbar^2z^3}u'(z)+\frac{16m^2\alpha^2}{\hbar^2z^2}\left(1-\frac{4L(L+1)}{z^2}\right)u(z)=0
\end{equation}
\begin{equation}
    z^2u''(z)+3zu'(z)+(z^2-4L(L+1))u(z)=0
\end{equation}
Сделаем замену $v(z)=zu(z)\rightarrow u(z)=\frac{v(z)}{z}$:
\begin{equation}
    u'(z)=\frac{v'(z)}{z}-\frac{v(z)}{z^2},\quad u''(z)=\frac{v''(z)}{z}-\frac{2v'(z)}{z^2}+\frac{2v(z)}{z^3}
\end{equation}
\begin{equation}
    zv''(z)-2v'(z)+\frac{2v(z)}{z}+3v'(z)-\frac{3v(z)}{z}+zv(z)-\frac{4L(L+1)}{z}v(z)=0
\end{equation}
\begin{equation}
    v''(z)+\frac 1zv'(z)+\left(1-\frac{(2L+1)^2}{z^2}\right)v(z)=0
\end{equation}
Получилось уравнение Бесселя с $m=2L+1$. Его решение:
\begin{equation}
    v(z)=A_1J_{2L+1}(z)+A_2Y_{2L+1}(z)
\end{equation}
Вернёмся к замене:
\begin{equation}
    \boxed{\psi(r)=\frac{1}{2}\sqrt{\frac{\hbar}{2m\alpha r}}\left(A_1J_{2L+1}\left(2\sqrt{\frac{2m\alpha r}{\hbar}}\right)+A_2Y_{2L+1}\left(2\sqrt{\frac{2m\alpha r}{\hbar}}\right)\right)}
\end{equation}
\subsection*{Задачи (60 баллов)}
\textbf{Задача 1 (20 баллов)}\\
Рассмотрите трёхмерную частицу массы $m$, движущуюся в трёхмерной шаровой полости радиуса $a$ с бесконечными стенками:
\begin{equation}
    U(\vec{r})=\begin{cases}
        0,\quad\;\;|\vec{r}|<a\\
        \infty,\quad|\vec{r}|>a\\
    \end{cases}
\end{equation}
Каким уравнением определяются уровни энергии с фиксированным орбитальным моментом $L$? Решите уравнение точно для $L = 0$, а также приближённо для высоколежащих уровней энергии при $L\geq 1$.\\
\textit{Указание:} свободное движение в трёхмерном пространстве также сводится к функциям Бесселя, но с полуцелым индексом $J_{n+1/2}(z)$ (которые, в действительности, специальными функциями не являются, а имеют явное представление через обычные тригонометрические функции).\\
\textbf{Решение.}\\
При $r>a$:
\begin{equation}
    \psi(r)=0
\end{equation}
Стационарное уравнение Шрёдингера при $r<a$ для радиальной части:
\begin{equation}
    \psi''(r)+\frac{2}{r}\psi'(r)+\left(\frac{2mE}{\hbar^2}-\frac{L(L+1)}{r^2}\right)\psi(r)=0
\end{equation}
Введём замену $\psi(r)=\frac{u(r)}{\sqrt{r}}$.
\begin{equation}
    \psi'(r)=\frac{u'(r)}{\sqrt{r}}-\frac{u(r)}{2r^\frac{3}{2}},\quad \psi''(r)=\frac{u''(r)}{\sqrt{r}}-\frac{u'(r)}{r^\frac{3}{2}}+\frac{3u(r)}{4r^\frac{5}{2}}
\end{equation}
\begin{equation}
    \frac{u''(r)}{\sqrt{r}}-\frac{u'(r)}{r^\frac{3}{2}}+\frac{3u(r)}{4r^\frac{5}{2}}+\frac{2u'(r)}{r^\frac{3}{2}}-\frac{u(r)}{r^\frac{5}{2}}+\left(\frac{2mE}{\hbar^2}-\frac{L(L+1)}{r^2}\right)\frac{u(r)}{\sqrt{r}}=0
\end{equation}
\begin{equation}
    u''(r)+\frac{u'(r)}{r}+\left(\frac{2mE}{\hbar^2}-\frac{L(L+1)}{r^2}-\frac{1}{4}\right)u(r)=0
\end{equation}
\begin{equation}
    u''(r)+\frac{u'(r)}{r}+\left(\frac{2mE}{\hbar^2}-\frac{(L+\frac12)^2}{r^2}\right)u(r)=0
\end{equation}
\begin{equation}
    u(r)=B_1J_{L+\frac{1}{2}}\left(\frac{\sqrt{2mE}}{\hbar}r\right)+B_2Y_{L+\frac{1}{2}}\left(\frac{\sqrt{2mE}}{\hbar}\right)
\end{equation}
Вернёмся к замене:
\begin{equation}
    \psi(r)=\frac{B_1}{\sqrt{r}}J_{L+\frac{1}{2}}\left(\frac{\sqrt{2mE}}{\hbar}r\right)+\frac{B_2}{\sqrt{r}}Y_{L+\frac{1}{2}}\left(\frac{\sqrt{2mE}}{\hbar}r\right)
\end{equation}
Поскольку функция Неймана сингулярна в 0, то $B_2=0$. Сшивка при $r=a$:
\begin{equation}
    J_{L+\frac{1}{2}}\left(\frac{\sqrt{2mE}a}{\hbar}\right)=0
\end{equation}
\begin{enumerate}
    \item Случай $L=0$.
    \begin{equation}
        \sqrt{\frac{2\hbar}{\pi a\sqrt{2mE}}}\sin\left(\frac{\sqrt{2mE}a}{\hbar}\right)=0\rightarrow \frac{\sqrt{2mE}a}{\hbar}=\pi n
    \end{equation}
    \begin{equation}
        \boxed{E_n=\frac{\pi^2\hbar^2n^2}{2ma^2}}
    \end{equation}
    \item Случай $L\geq1$. Для высоколежащих уровней $\frac{\sqrt{2mE}a}{\hbar}\gg1$. Асимптотика функции Бесселя при $z\gg1$:
    \begin{equation}
        J_m(z)\approx\sqrt{\frac{2}{\pi z}}\cos\left(z-\frac{m\pi}{2}-\frac{\pi}{4}\right)
    \end{equation}
    \begin{equation}
        \frac{\sqrt{2\hbar}}{\sqrt{\pi a\sqrt{2mE}}}\cos\left(\frac{\sqrt{2mE}a}{\hbar}-\frac{(L+\frac{1}{2})\pi}{2}-\frac{\pi}{4}\right)=0
    \end{equation}
    \begin{equation}
        \sin\left(\frac{\sqrt{2mE}a}{\hbar}-\frac{\pi L}{2}\right)=0
    \end{equation}
    \begin{equation}
        \frac{\sqrt{2mE}a}{\hbar}-\frac{\pi L}{2}=\pi n
    \end{equation}
    \begin{equation}
        \boxed{E_n=\frac{\pi^2\hbar^2(2n+L)^2}{8ma^2}}
    \end{equation}
\end{enumerate}
\textbf{Задача 2. Ступенька (40 баллов)}\\
Определите коэффициенты прохождения и отражения от потенциала $U(x) = \frac{1}{2}U_0\tanh\frac{x}{a}$. Указание: при помощи замены $\xi =-e^{-2x/a}$ задача приводится к гипергеометрической функции $_2F_1$.\\
\textbf{Решение.}\\
Стационарное уравнение Шрёдингера:
\begin{equation}
    -\frac{\hbar^2}{2m}\psi''(x)+\left(\frac{1}{2}U_0\tanh\frac{x}{a}-E\right)\psi(x)=0
\end{equation}
Пусть $\xi=-e^{-\frac{2x}{a}}$, тогда $x=-\frac{a}{2}\ln(-\xi)$.
\begin{equation}
    \th\frac{x}{a}=\frac{1+\xi}{1-\xi}
\end{equation}
\begin{equation}
    \psi'(x)=-\frac{2}{a}\xi\psi'(\xi),\quad \psi''(x)=\frac{4\xi}{a^2}\psi'(\xi)+\frac{4\xi^2}{a^2}\psi''(\xi)
\end{equation}
Пусть $k^2=\frac{ma^2}{4\hbar^2}U_0$, $\kappa^2=\frac{ma^2}{2\hbar^2}E$.
\begin{equation}
    \xi\frac{\partial\psi}{\partial\xi}+\xi^2\frac{\partial^2\psi}{\partial\xi^2}+k^2\frac{\xi+1}{\xi-1}\psi+\kappa^2\psi=0
\end{equation}
\begin{equation}
    \psi(\xi)=\Psi(\xi)\xi^{\sqrt{k^2-\chi^2}}
\end{equation}
\begin{equation}
    \xi(1-\xi)\Psi''(\xi)-(2\sqrt{k^2-\kappa^2}+1)(\xi-1)\Psi'(\xi)-2k^2\Psi(\xi)=0
\end{equation}
\begin{multline}
    \Psi(\xi)=C_1\xi^{\sqrt{k^2-\kappa^2}}{}_2F_1(\sqrt{-k^2-\kappa^2}+\sqrt{k^2-\kappa^2},-\sqrt{-k^2-\kappa^2}+\sqrt{k^2-\kappa^2},1+2\sqrt{k^2-\chi^2},\xi)+\\+C_2\xi^{-\sqrt{k^2-\kappa^2}}{}_2F_1(-\sqrt{k^2-\kappa^2}-\sqrt{-k^2-\kappa^2},-\sqrt{k^2-\kappa^2}+\sqrt{-k^2-\kappa^2},1-2\sqrt{k^2-\chi^2},\xi)
\end{multline}
Рассмотрим 2 базисное решение:
\begin{equation}
    \psi(\xi)=\xi^{-i\sqrt{\kappa^2-k^2}}{}_2F_1(-\sqrt{k^2-\kappa^2}-\sqrt{-k^2-\kappa^2},-\sqrt{k^2-\kappa^2}+\sqrt{-k^2-\kappa^2},1-2\sqrt{k^2-\chi^2},\xi)
\end{equation}
При $x\rightarrow\infty$
\begin{equation}
    {}_2F_1(...,e^{-\frac{2x}{a}})\sim 1
\end{equation}
\begin{equation}
    \psi(x)=(-1)^{i\sqrt{k^2+\kappa^2}}e^{-i\frac{2x}{a}\sqrt{k^2+\kappa^2}}=e^{\pi\sqrt{\kappa^2-k^2}}e^{i\frac{2x}{a}\sqrt{\kappa^2-k^2}}
\end{equation}
При $x\rightarrow-\infty$
\begin{multline}
    \frac{\sin(\pi(b-a))}{\pi}{}_2F_1(a,b,c,\xi)=\frac{(-\xi)^a}{\Gamma(b)\Gamma(c-a)}{}_2F_1(a,a-c+1,a-b+1,\frac{1}{\xi})-\\-\frac{(-\xi)^{-b}}{\Gamma(a)\Gamma(c-b)}{}_2F_1(b,b-c+1,b-a+1,\frac{1}{\xi})
\end{multline}
\begin{equation}
    {}_2F_1(a,b,c,\frac{1}{\xi})\sim 1
\end{equation}
\begin{multline}
    \Psi(\xi)=-\frac{\pi}{i\sinh(\pi\sqrt{k^2+\kappa^2})}e^{\pi\sqrt{\kappa^2-k^2}}\left(\frac{e^{-i\frac{2x}{a}\sqrt{k^2+\kappa^2}}}{(i\sqrt{\kappa^2-k^2}-i\sqrt{\kappa^2+k^2})\Gamma^2(i\sqrt{\kappa^2-k^2}-i\sqrt{\kappa^2+k^2})}\right.-\\-\left.\frac{e^{i\frac{2x}{a}\sqrt{k^2+\kappa^2}}}{(i\sqrt{\kappa^2-k^2}+i\sqrt{\kappa^2+k^2})\Gamma^2(i\sqrt{\kappa^2-k^2}+i\sqrt{\kappa^2+k^2})}\right)
\end{multline}
\begin{equation}
    r=\frac{\pi i}{\sinh\pi\sqrt{\kappa^2+k^2}}(i\sqrt{\kappa^2-k^2}+i\sqrt{\kappa^2+k^2})\Gamma^2(i\sqrt{\kappa^2-k^2}+i\sqrt{\kappa^2+k^2})
\end{equation}
\begin{equation}
    t=\frac{\pi i}{\sinh\pi\sqrt{\kappa^2+k^2}}(i\sqrt{\kappa^2-k^2}-i\sqrt{\kappa^2+k^2})\Gamma^2(i\sqrt{\kappa^2-k^2}-i\sqrt{\kappa^2+k^2})
\end{equation}
Определим коэффициенты отражения и прохождения. Упростим $r$ и $t$ и получим:
\begin{equation}
    \boxed{R=|r|^2=\frac{\sinh^2(\pi(\sqrt{\kappa^2-k^2}-\sqrt{\kappa^2+k^2}))}{\sinh^2(\pi(\sqrt{\kappa^2-k^2}+\sqrt{\kappa^2+k^2}))}}
\end{equation}
\begin{equation}
    \boxed{T=\frac{\sqrt{\kappa^2+k^2}}{\sqrt{\kappa^2-k^2}}|t|^2=\frac{\sinh(2\pi(\sqrt{\kappa^2-k^2}))\sinh(2\pi\sqrt{\kappa^2+k^2}))}{\sinh^2(\pi(\sqrt{\kappa^2-k^2}+\sqrt{\kappa^2+k^2}))}}
\end{equation}
\section{Точно решаемые потенциалы. Часть 2}
\subsection*{Упражнения (35 баллов).}
\textbf{Упражнение 1. Функция Эйри (10 баллов)}\\
Используя асимптотики функций Эйри для инфинитного движения (без бесконечной стенки), отнормируйте их на дельта-функцию от энергии, а также убедитесь в соотношении полноты. А именно, выпишите волновые функции, удовлетворяющие следующему условию:
\begin{equation}
    \int dx\psi^*_E(x)\psi_{E'}(x)=\delta(E-E')
\end{equation}
и убедитесь, что для них выполняется соотношение
\begin{equation}
    \int dE\psi^*_E(x')\psi_E(x)=\delta(x-x')
\end{equation}
\textbf{Решение.}\\
Стационарное уравнение Шрёдингера:
\begin{equation}
    -\frac{\hbar^2}{2m}\psi''(x)+Fx\psi(x)=E\psi(x)
\end{equation}
Сдвинем начало координат:
\begin{equation}
    z=\frac{x-E/F}{(\hbar^2/2mF)^\frac{1}{3}}
\end{equation}
Такая подстановка приводит к уравнению Эйри:
\begin{equation}
    \psi''(z)-z\psi(z)=0
\end{equation}
Общее решение -- произвольная линейная комбинация функций Эйри первого и второго рода:
\begin{equation}
    \psi(z)=C_1\text{Ai}(z)+C_2\text{Bi}(z)
\end{equation}
Асимптотики функции Эйри:
\begin{equation}
    \text{Ai}(z)\approx\begin{cases}
        \frac{1}{2\sqrt{\pi}z^\frac{1}{4}}\exp\left(-\frac{2}{3}z^\frac{3}{2}\right),\quad\quad\quad\quad z\rightarrow\infty\\
        \frac{1}{\sqrt{\pi}(-z)^\frac{1}{4}}\sin\left(\frac{2}{3}(-z)^\frac{3}{2}+\frac{\pi}{4}\right),\quad z\rightarrow-\infty
    \end{cases}
\end{equation}
\begin{equation}
    \frac{1}{\sqrt{\pi}(-z)^\frac{1}{4}}\sin\left(\frac{2}{3}(-z)^\frac{3}{2}+\frac{\pi}{4}\right)=\frac{1}{2\sqrt{\pi}(-z)^\frac{1}{4}}\left(e^{i\left(\frac{2}{3}(-z)^\frac{3}{2}-\frac{\pi}{4}\right)}+e^{-i\left(\frac{2}{3}(-z)^\frac{3}{2}+\frac{3\pi}{4}\right)}\right)
\end{equation}
\begin{equation}
    \psi_E(x)=C\text{Ai}(z)
\end{equation}
\begin{multline}
    \int dx\psi^*_E(x)\psi_{E'}(x)=\\=\frac{C^2}{4\pi(zz')^\frac{1}{4}}\int\limits_{-\infty}^0dx\left(e^{-i\left(\frac{2}{3}(-z)^\frac{3}{2}-\frac{\pi}{4}\right)}+e^{i\left(\frac{2}{3}(-z)^\frac{3}{2}+\frac{3\pi}{4}\right)}\right)\left(e^{i\left(\frac{2}{3}(-z')^\frac{3}{2}-\frac{\pi}{4}\right)}+e^{-i\left(\frac{2}{3}(-z')^\frac{3}{2}+\frac{3\pi}{4}\right)}\right)+\\+\frac{C^2}{4\pi(zz')^\frac{1}{4}}\int\limits_0^\infty dx\left(e^{-i\left(\frac{2}{3}(-z)^\frac{3}{2}-\frac{\pi}{4}\right)}+e^{i\left(\frac{2}{3}(-z)^\frac{3}{2}+\frac{3\pi}{4}\right)}\right)\left(e^{i\left(\frac{2}{3}(-z')^\frac{3}{2}-\frac{\pi}{4}\right)}+e^{-i\left(\frac{2}{3}(-z')^\frac{3}{2}+\frac{3\pi}{4}\right)}\right)+\text{reg.}=\\=\frac{C^2}{4\pi(zz')^\frac{1}{4}}\int\limits_{-\infty}^0dx\left(e^{\frac{2i}{3}\left((-z')^\frac{3}{2}-(-z)^\frac{3}{2}\right)}+e^{-\frac{2i}{3}\left((-z')^\frac{3}{2}-(-z)^\frac{3}{2}\right)}\right)+\text{reg.}
\end{multline}
\begin{equation}
    \Delta z=z-z'=\frac{\Delta E}{F}\left(\frac{2mF}{\hbar^2}\right)^\frac{1}{3}
\end{equation}
\begin{multline}
    \int dx\psi^*_E(x)\psi_{E'}(x)=\frac{C^2}{4\pi z^\frac{1}{2}}\left(\frac{\hbar^2}{2mF}\right)^\frac{1}{3}\int\limits_{-\infty}^0dz(e^{-\frac{2}{3}i(-z)^\frac{3}{2}\frac{3\Delta z}{2z}}+e^{\frac{2}{3}i(-z)^\frac{3}{2}\frac{3\Delta z}{2z}})=\\=\left(\frac{\hbar^2}{2mF}\right)^\frac{1}{3}\frac{C^2}{2\pi}\int\limits_{-\infty}^\infty d(\sqrt{z})e^{-i\Delta z\sqrt{z}}=C^2\left(\frac{\hbar^2\sqrt{F}}{2m}\right)^\frac{2}{3}\delta(E'-E)=\delta(E'-E)
\end{multline}
\begin{equation}
    C=\left(\frac{2m}{\hbar^2\sqrt{F}}\right)^\frac{1}{3}
\end{equation}
\begin{equation}
    \boxed{\psi_E(x)=\left(\frac{2m}{\hbar^2\sqrt{F}}\right)^\frac{1}{3}\text{Ai}\left(\frac{x-E/F}{(\hbar^2/2mF)^\frac{1}{3}}\right)}
\end{equation}
\begin{multline}
    \int dE\psi^*_E(x')\psi_E(x)=\frac{C^2}{4\pi}\int\limits_{-\infty}^0dE\left(e^{\frac{2i}{3}\left((-z')^\frac{3}{2}-(-z)^\frac{3}{2}\right)}+e^{-\frac{2i}{3}\left((-z')^\frac{3}{2}-(-z)^\frac{3}{2}\right)}\right)+\text{reg.}=\\=-\frac{FC^2}{4\pi}\left(\frac{\hbar^2}{2mF}\right)^\frac{1}{3}\int\limits_{-\infty}^0dz\frac{e^{iz^\frac{1}{2}\Delta z}+e^{-iz^\frac{1}{2}\Delta z}}{z^\frac{1}{2}}=FC^2\left(\frac{\hbar^2}{2mF}\right)^\frac{2}{3}\delta(x-x')=\delta(x-x')
\end{multline}
\begin{equation}
    \boxed{\int dE\psi^*_E(x')\psi_E(x)=\delta(x-x')}
\end{equation}
\textbf{Упражнение 2. Квантовый гармонический осциллятор (10 баллов)}
\begin{enumerate}
    \item Вычислите $\braket{\hat{x}^4}$ по произвольному собственному состоянию гармонического осциллятора $\ket{n}$.
    \item Частица находилась в основном состоянии гармонического осциллятора с частотой $\omega$. Пусть в какой-то момент времени характерная частота осциллятора мгновенно меняется и становится равной $\omega'$. Вычислите вероятность остаться в основном состоянии.
\end{enumerate}
\textbf{Решение.}\\
\begin{enumerate}
\item Действие операторов повышения и понижения:
\begin{equation}
    \hat{a}\ket{n}=\sqrt{n}\ket{n},\quad \hat{a}^\dagger\ket{n}=\sqrt{n+1}\ket{n}
\end{equation}
\begin{equation}
    \hat{x}=\sqrt{\frac{\hbar}{2m\omega}}(\hat{a}+\hat{a}^\dagger)
\end{equation}
\begin{multline}
    \braket{x^4}=\bra{n}\hat{x}^4\ket{n}=\frac{\hbar^2}{4m^2\omega^2}\bra{n}(\hat{a}^2+\hat{a}\hat{a}^\dagger+\hat{a}^\dagger\hat{a}+(\hat{a}^\dagger)^2)^2\ket{n}=\frac{\hbar^2}{4m^2\omega^2}\bra{n}\hat{a}^4+\hat{a}^3\hat{a}^\dagger+\hat{a}^2\hat{a}^\dagger\hat{a}+\\+\hat{a}^2(\hat{a}^\dagger)^2+\hat{a}\hat{a}^\dagger\hat{a}^2+\hat{a}\hat{a}^\dagger\hat{a}\hat{a}^\dagger+\hat{a}(\hat{a}^\dagger)^2\hat{a}+\hat{a}(\hat{a}^\dagger)^3+\hat{a}^\dagger\hat{a}^3+\hat{a}^\dagger\hat{a}^2\hat{a}^\dagger+\hat{a}^\dagger\hat{a}\hat{a}^\dagger\hat{a}+\hat{a}^\dagger\hat{a}(\hat{a}^\dagger)^2+(\hat{a}^\dagger)^2\hat{a}^2+\\+(\hat{a}^\dagger)^2\hat{a}\hat{a}^\dagger+(\hat{a}^\dagger)^3\hat{a}+(\hat{a}^\dagger)^4\ket{n}=\frac{\hbar^2}{4m^2\omega^2}((n+1)(n+2)+(n+1)^2+n(n+1)+n(n+1)+n^2+n(n-1))
\end{multline}
\begin{equation}
    \boxed{\braket{x^4}=\frac{3\hbar^2}{4m^2\omega^2}(2n^2+2n+1)}
\end{equation}
\item Волновая функция основного состояния:
\begin{equation}
    \psi(x)=\sqrt[4]{\frac{m\omega}{\pi\hbar}}\exp\left(-\frac{m\omega x^2}{2\hbar}\right)
\end{equation}
\begin{equation}
    P=\left|\int\limits_{-\infty}^\infty\psi^*(x)\psi'(x)dx\right|^2=\left|\sqrt[4]{\frac{m^2\omega\omega'}{\pi^2\hbar^2}}\int\limits_{-\infty}^\infty\exp\left(-\frac{m(\omega+\omega')x^2}{2\hbar}\right)dx\right|^2
\end{equation}
\begin{equation}
    \boxed{P=\frac{2\sqrt{\omega\omega'}}{\omega+\omega'}}
\end{equation}
\end{enumerate}
\textbf{Упражнение 3. Когерентные состояния (15 баллов)}\\
Когерентные состояния гармонического осциллятора определяются как собственные состояния для понижающего оператора $\hat{a}$ с собственным комплексным числом $\alpha \in \mathbb{C}: \hat{a}\ket{\alpha} = \alpha\ket{\alpha}$.
\begin{enumerate}
    \item Найдите координатное представление когерентного состояния $\psi_\alpha(x) \equiv \braket{x|\alpha}$. Для гамильтониана квантового гармонического осциллятора $\hat{H} = \hbar\omega(\hat{a}^\dagger\hat{a}+\frac12)$, найдите также $\psi_\alpha(x, t)\equiv\braket{x|\alpha(t)}$.
    \item Выразите когерентное состояние $\alpha$ явно через собственные состояния осциллятора, нормировав его условием $\braket{0|\alpha} = 1$.
    \item Представьте их в виде $\ket{\alpha} = \hat{C}(\alpha)\ket{0}$; найдите явный вид оператора $\hat{C}(\alpha)$.
    \item Вычислите перекрытие когерентных состояний $\braket{\alpha|\alpha'}$.
    \item Когерентные состояния образуют \textit{переполненный базис}. Докажите следующую формулу для <<разложения единицы>>:
    \begin{equation}
        \hat{\mathbb{I}}=\int\frac{d\alpha d\alpha^*}{\pi}e^{-|\alpha|^2}\ket{\alpha}\bra{\alpha}
    \end{equation}
    (мы определили $d\alpha d\alpha^* \equiv d(\text{Re}\alpha)d(\text{Im}\alpha)$).
\end{enumerate}
\textbf{Решение.}
\begin{enumerate}
    \item Оператор уничтожения:
    \begin{equation}
        \hat{a}=\sqrt{\frac{m\omega}{2\hbar}}\hat{x}+\frac{i}{\sqrt{2\hbar m\omega}}\hat{p}
    \end{equation}
    \begin{equation}
        \braket{x|\hat{a}|\alpha}=\alpha\braket{x|\alpha}=\alpha\psi_\alpha(x)
    \end{equation}
    \begin{equation}
        \left(\sqrt{\frac{m\omega}{2\hbar}}\hat{x}+\frac{i}{\sqrt{2\hbar m\omega}}\hat{p}\right)\psi_\alpha(x)=\alpha\psi_\alpha(x)
    \end{equation}
    \begin{equation}
        \left(\sqrt{\frac{m\omega}{2\hbar}}x+\sqrt{\frac{\hbar}{2m\omega}}\frac{\partial}{\partial x}\right)\psi_\alpha(x)=\alpha\psi_\alpha(x)
    \end{equation}
    \begin{equation}
        \sqrt{\frac{\hbar}{2m\omega}}\frac{\partial\psi_\alpha(x)}{\partial x}=\left(\alpha-\sqrt{\frac{m\omega}{2\hbar}}x\right)\psi_\alpha(x)
    \end{equation}
    \begin{equation}
        \boxed{\psi_\alpha(x)=e^{\sqrt{\frac{2m\omega}{\hbar}}\alpha x-\frac{m\omega x^2}{2\hbar}}}
    \end{equation}
    Действие оператора эволюции (воспользуемся п. 2 этой задачи):
    \begin{equation}
        \ket{\alpha,t}=\hat{U}(t)\ket{\alpha}=\sum\limits_n\frac{\alpha^n}{\sqrt{n!}}e^{-\frac{iE_nt}{\hbar}}\ket{n}
    \end{equation}
    \begin{equation}
        E_n=\hbar\omega\left(\frac{1}{2}+n\right)
    \end{equation}
    \begin{equation}
        \ket{\alpha,t}=e^{-\frac{i\omega t}{2}}\sum\limits_n\frac{\alpha^n}{\sqrt{n!}}e^{-i\omega nt}\ket{n}=e^{-\frac{i\omega t}{2}}\sum\limits_n\frac{(\alpha e^{-i\omega t})^n}{\sqrt{n!}}\ket{n}
    \end{equation}
    \begin{equation}
        \psi_\alpha(x, t)\equiv\braket{x|\alpha(t)}=e^{-\frac{i\omega t}{2}}\sum\limits_n\frac{(\alpha e^{-i\omega t})^n}{\sqrt{n!}}\braket{x|n}
    \end{equation}
    \begin{equation}
        \boxed{\psi_\alpha(x, t)=e^{-\frac{i\omega t}{2}}\sum\limits_n\frac{(\alpha e^{-i\omega t})^n}{\sqrt{2^n}n!}\sqrt[4]{\frac{m\omega}{\pi\hbar}}H_n\left(x\sqrt{\frac{m\omega}{\hbar}}\right)}
    \end{equation}
    \item Разложим когеренстное состояние по базису:
    \begin{equation}
        \ket{\alpha}=\sum\limits_nC_n(\alpha)\ket{n},\quad C_n(\alpha)=\braket{n|\alpha}=\frac{1}{\alpha}\braket{n|\hat{a}\alpha}=\frac{1}{\alpha}\braket{\hat{a}^\dagger n|\alpha}
    \end{equation}
    \begin{equation}
        \hat{a}^\dagger\ket{n}=\sqrt{n+1}\ket{n+1}
    \end{equation}
    Получаем рекуррентное соотношение:
    \begin{equation}
        \alpha C_n(\alpha)=\sqrt{n+1}C_{n+1}(\alpha)\rightarrow C_n(\alpha)=\frac{\alpha^n}{\sqrt{n!}}C_0(\alpha)
    \end{equation}
    Условие нормировки:
    \begin{equation}
        C_0(\alpha)=\braket{0|\alpha}=1
    \end{equation}
    \begin{equation}
        \boxed{\ket{\alpha}=\sum\limits_n\frac{\alpha^n}{\sqrt{n!}}\ket{n}}
    \end{equation}
    \item
    \begin{equation}
        \ket{n}=\frac{(\hat{a}^\dagger)^n}{\sqrt{n!}}\ket{0}
    \end{equation}
    \begin{equation}
        \ket{\alpha}=\sum\limits_n\frac{(\alpha a^\dagger)^n}{n!}\ket{0}=e^{\alpha\hat{a}^\dagger}\ket{0}=C(\alpha)\ket{0}
    \end{equation}
    \begin{equation}
        \boxed{C(\alpha)=e^{\alpha\hat{a}^\dagger}}
    \end{equation}
    \item
    \begin{multline}
        \braket{\alpha|\alpha'}=\braket{0|\hat{C}^\dagger(\alpha)\hat{C}(\alpha')|0}=\braket{0|e^{\alpha^*\hat{a}}e^{\alpha'\hat{a}^\dagger}|0}=\left<0\left|\sum\limits_{k=0}^\infty\frac{(\alpha^*\hat{a})^k}{k!}\sum\limits_{n=0}^\infty\frac{(\alpha'\hat{a}^\dagger)^n}{n!}\right|0\right>=\\=\left<k\left|\sum\limits_{k=0}^\infty\frac{\hat{a}^k}{\sqrt{k!}}\sum\limits_{n=0}^\infty\frac{\alpha'^n}{\sqrt{n!}}\right|n\right>=\sum\limits_{k=0}^\infty\frac{(\alpha^*)^k}{\sqrt{k!}}\sum\limits_{n=0}^\infty\frac{(\alpha')^n}{\sqrt{n!}}\delta_{kn}=\sum\limits_{n=0}^\infty\frac{(\alpha^*\alpha')^n}{n!}=e^{\alpha^*\alpha'}
    \end{multline}
    \begin{equation}
        \boxed{\braket{\alpha|\alpha'}=e^{\alpha^*\alpha'}}
    \end{equation}
    \item 
    \begin{equation}
        \braket{m|\hat{\mathbb{I}}|n}=\delta_{mn}
    \end{equation}
    \begin{multline}
        \left<m\left|\int\frac{d\alpha d\alpha^*}{\pi}e^{-|\alpha|^2}\ket{\alpha}\bra{\alpha}\right|n\right>=\int\frac{d\alpha d\alpha^*}{\pi}e^{-|\alpha|^2}\braket{m|\alpha}\braket{\alpha|n}=\\=\int\frac{d\alpha d\alpha^*}{\pi}e^{-|\alpha|^2}\frac{(\alpha^*)^m}{\sqrt{m!}}\frac{\alpha^n}{\sqrt{n!}}
    \end{multline}
    \begin{equation}
        \alpha=re^{i\varphi},\quad d(\text{Re}\alpha)d(\text{Im}\alpha)=rdrd\varphi
    \end{equation}
    \begin{multline}
        \int\limits_0^\infty\int\limits_0^{2\pi}\frac{drd\varphi}{\pi}re^{-r^2}\frac{r^{m+n}e^{i(n-m)\varphi}}{\sqrt{m!n!}}=\frac{1}{\sqrt{m!n!}}\int\limits_0^\infty drr^{m+n+1}e^{-r^2}\int\limits_0^{2\pi}\frac{d\varphi}{\pi}e^{i(n-m)\varphi}=\\=\frac{2}{\sqrt{m!n!}}\int\limits_0^\infty drr^{m+n+1}e^{-r^2}\delta_{nm}=\frac{1}{m!}\int\limits_0^\infty d(r^2)r^{2m}e^{-r^2}\delta_{mn}=\frac{\Gamma(m+1)}{m!}\delta_{mn}=\delta_{mn}
    \end{multline}
    Таким образом,
    \begin{equation}
        \boxed{\hat{\mathbb{I}}=\int\frac{d\alpha d\alpha^*}{\pi}e^{-|\alpha|^2}\ket{\alpha}\bra{\alpha}}
    \end{equation}
\end{enumerate}
\subsection*{Задачи (65 баллов)}
\textbf{Задача 1. Теорема Вика (20 баллов)}\\
Квантовый гармонический осциллятор находится при температуре $T$, т.е. описывается матрицей плотности $\hat{\rho}=\frac{1}{Z}e^{-\beta\omega \hat{a}^\dagger\hat{a}}$. Вычислите среднее значение по этому состоянию от оператора $\braket{e^{\alpha_1\hat{a}+\alpha_2\hat{a}^\dagger}}$ и докажите следующее соотношение:
\begin{equation}
    \braket{e^{\alpha_1\hat{a}+\alpha_2\hat{a}^\dagger}}=e^{\frac{1}{2}\braket{(\alpha_1\hat{a}+\alpha_2\hat{a}^\dagger)^2}}
\end{equation}
\textit{Указание:} вам может пригодиться базис когерентных состояний, а также формула Бейкера-Кэмбелла-Хаусдорфа:
\begin{equation}
    e^{\hat{A}+\hat{B}}=e^{\hat{A}}e^{\hat{B}}e^{-\frac{1}{2}[\hat{A},\hat{B}]},\quad если\; [\hat{A},[\hat{A},\hat{B}]]=[\hat{B},[\hat{A},\hat{B}]]=0
\end{equation}
\textbf{Решение.}
Рассмотрим $\hat{O}=e^{\alpha_1\hat{a}+\alpha_2\hat{a}^\dagger}$.
\begin{equation}
    \braket{\hat{O}}=\text{Tr}(\hat{\rho}\hat{O})=\text{Tr}\left(\frac{1}{Z}e^{-\beta\omega \hat{a}^\dagger\hat{a}}e^{\alpha_1\hat{a}+\alpha_2\hat{a}^\dagger}\right)
\end{equation}
\begin{equation}
    \braket{\hat{O}}=\text{Tr}\left(\frac{1}{Z}e^{-\beta\omega \hat{a}^\dagger\hat{a}}e^{\alpha_1\hat{a}+\alpha_2\hat{a}^\dagger}\right)=\frac{1}{Z}\int\frac{d\alpha d\alpha^*}{\pi}e^{-|\alpha|^2}\braket{\alpha|e^{\alpha_1\hat{a}+\alpha_2\hat{a}^\dagger}e^{-\beta\omega\hat{a}^\dagger\hat{a}}|\alpha}
\end{equation}
\begin{equation}
    e^{\alpha_1\hat{a}+\alpha_2\hat{a}^\dagger}=e^{\alpha_1\hat{a}}e^{\alpha_2\hat{a}^\dagger}e^{-\frac{\alpha_1\alpha_2}{2}[\hat{a}^\dagger,\hat{a}]}=e^{\alpha_1\hat{a}}e^{\alpha_2\hat{a}^\dagger}e^{\frac{\alpha_1\alpha_2}{2}}
\end{equation}
\begin{equation}
    \braket{\hat{O}}=\frac{1}{Z}\int\frac{d\alpha d\alpha^*}{\pi}e^{-|\alpha|^2}e^{\alpha_2\alpha^*}e^{\frac{\alpha_1\alpha_2}{2}}\braket{\alpha|e^{\alpha_1\hat{a}}e^{-\beta\omega\hat{a}^\dagger\hat{a}}|\alpha}
\end{equation}
\begin{equation}
    e^{-\beta\omega\hat{a}^\dagger\hat{a}}\ket{\alpha}=e^{-\beta\omega\hat{a}^\dagger\hat{a}}\sum\limits_{n=0}^\infty\frac{\alpha^n}{\sqrt{n!}}\ket{n}=\sum\limits_{n=0}^\infty\frac{\alpha^n}{\sqrt{n!}}e^{-\beta\omega n}\ket{n}=e^{-\beta\omega}\ket{n}
\end{equation}
\begin{multline}
    \braket{\hat{O}}=\frac{e^{\frac{\alpha_1\alpha_2}{2}}}{Z}\int\frac{d\alpha d\alpha^*}{\pi}e^{-|\alpha|^2}e^{\alpha_2\alpha^*}e^{\alpha_1\alpha e^{-\beta\omega}}\braket{\alpha|e^{-\beta\omega}|\alpha}=\\=\frac{e^{\frac{\alpha_1\alpha_2}{2}}}{Z}\int\frac{d\alpha d\alpha^*}{\pi}e^{-|\alpha|^2(1-e^{-\beta\omega})}e^{\alpha_2\alpha^*}e^{\alpha_1\alpha e^{-\beta\omega}}
\end{multline}
\begin{equation}
    \braket{\hat{O}}=\frac{e^{\frac{\alpha_1\alpha_2}{2}}}{Z}(1-e^{-\beta\omega})\int\limits_{-\infty}^\infty\frac{du}{\sqrt{\pi}}e^{-(u^2-\frac{\alpha_2+\alpha_1e^{-\beta\omega}}{\sqrt{1-e^{-\beta\omega}}}u)}\int\limits_{-\infty}^\infty\frac{du}{\sqrt{\pi}}e^{-(u^2+i\frac{\alpha_2-\alpha_1e^{-\beta\omega}}{\sqrt{1-e^{-\beta\omega}}}u)}
\end{equation}
\begin{equation}
    Z=\text{Tr}(e^{-\beta\omega \hat{a}^\dagger\hat{a}})=\sum\limits_{n=0}^\infty\braket{n|e^{-\beta\omega \hat{a}^\dagger\hat{a}}|n}=\sum\limits_{n=0}^\infty\braket{n|e^{-\beta\omega n}|n}=\sum\limits_{n=0}^\infty e^{-\beta\omega n}=\frac{1}{1-e^{-\beta\omega}}
\end{equation}
\begin{equation}
    \braket{\hat{O}}=e^{\frac{\alpha_1\alpha_2}{2}}e^{\frac{(\alpha_2+\alpha_1e^{-\beta\omega})^2}{4(1-e^{-\beta\omega})}}e^{-\frac{(\alpha_2-\alpha_1e^{-\beta\omega})^2}{4(1-e^{-\beta\omega})}}=e^{\frac{\alpha_1\alpha_2}{2}\left(\frac{1+e^{-\beta\omega}}{1-e^{-\beta\omega}}\right)}
\end{equation}
\begin{equation}
    \boxed{\braket{e^{\alpha_1\hat{a}+\alpha_2\hat{a}^\dagger}}=e^{\frac{\alpha_1\alpha_2}{2}\coth\frac{\beta\omega}{2}}}
\end{equation}
Рассмотрим $\hat{O}=(\alpha_1\hat{a}+\alpha_2\hat{a}^\dagger)^2$.
\begin{multline}
    \braket{\hat{O}}=\text{Tr}(\hat{\rho}\hat{O})=\text{Tr}\left(\frac{1}{Z}e^{-\beta\omega \hat{a}^\dagger\hat{a}}(\alpha_1\hat{a}+\alpha_2\hat{a}^\dagger)^2\right)=\sum\limits_{n=0}^\infty\left<n\left|\frac{1}{Z}e^{-\beta\omega \hat{a}^\dagger\hat{a}}(\alpha_1\hat{a}+\alpha_2\hat{a}^\dagger)^2\right|n\right>=\\=\frac{1}{Z}\sum\limits_{n=0}^\infty\left<n\left|\frac{1}{Z}e^{-\beta\omega \hat{a}^\dagger\hat{a}}(\alpha_1^2\hat{a}^2+\alpha_1\alpha_2\hat{a}\hat{a}^\dagger+\alpha_1\alpha_2\hat{a}^\dagger\hat{a}+\alpha_2^2\hat{a}^{\dagger2})\right|n\right>=\frac{\alpha_1\alpha_2}{Z}\sum\limits_{n=0}^\infty\braket{n|e^{-\beta\omega\hat{a}^\dagger\hat{a}}(2n+1)|n}=\\=\frac{\alpha_1\alpha_2}{Z}\sum\limits_{n=0}^\infty e^{-\beta\omega n}(2n+1)=\frac{\alpha_1\alpha_2}{Z}\frac{1+e^{-\beta\omega}}{(1-e^{-\beta\omega})^2}=\alpha_1\alpha_2\frac{1+e^{-\beta\omega}}{1-e^{-\beta\omega}}=\alpha_1\alpha_2\coth\frac{\beta\omega}{2}
\end{multline}
\begin{equation}
    \boxed{e^{\frac{1}{2}\braket{(\alpha_1\hat{a}+\alpha_2\hat{a}^\dagger)^2}}=e^{\frac{\alpha_1\alpha_2}{2}\coth\frac{\beta\omega}{2}}=\braket{e^{\alpha_1\hat{a}+\alpha_2\hat{a}^\dagger}}}
\end{equation}
Таким образом, теорема Вика доказана.\\\\
\textbf{Задача 2. QHO in a box (20 баллов)}\\
Квантовый гармонический осциллятор помещён в большую <<коробку>> размера $2L$ ровно посередине, так что потенциальная энергия имеет следующий вид:
\begin{equation}
    U(x)=\begin{cases}
        \frac{m\omega^2x^2}{2},\quad|x|<L\\
        \infty,\quad|x|>L
    \end{cases}
\end{equation}
Определите \textit{силу}, с которой осциллятор, находясь в одном из низколежащих уровней энергии $n\ll\frac{m\omega L^2}{\hbar}$, действует на эти стенки.\\
\textbf{Решение.}\\
Общее решение в области $|x|<L$:
\begin{equation}
    \psi(y)=C_1{}_1F_1\left(-\frac{\epsilon}{2},\frac{1}{2},y^2\right)e^{-\frac{y^2}{2}}+C_2y{}_1F_1\left(-\frac{\epsilon-1}{2},\frac{3}{2},y^2\right)e^{-\frac{y^2}{2}}=C_1\psi_1(y)+C_2\psi_2(y)
\end{equation}
\begin{equation}
    x=\pm L\rightarrow y=\pm L\sqrt{\frac{m\omega}{\hbar}}
\end{equation}
\begin{equation}
    \psi\left(\pm L\sqrt{\frac{m\omega}{\hbar}}\right)=0
\end{equation}
\begin{equation}
    {}_1F_1\left(-\frac{\epsilon}{2},\frac{1}{2},y^2\right)e^{-\frac{m\omega L^2}{\hbar}}=0
\end{equation}
Поскольку $\frac{m\omega L^2}{\hbar}\gg1$, то воспользуемся асимптотической формулой
\begin{equation}
    {}_1F_1(a,b,z)=\frac{\Gamma(b)}{\Gamma(a)}e^zz^{a-b}+\frac{\Gamma(b)}{\Gamma(b-a)}(-z)^{-a}
\end{equation}
\begin{equation}
    \frac{\Gamma\left(\frac{1}{2}\right)}{\Gamma\left(-\frac{\epsilon}{2}\right)}e^zz^{-\frac{\epsilon+1}{2}}+\frac{\Gamma\left(\frac{1}{2}\right)}{\Gamma\left(\frac{\varepsilon+1}{2}\right)}(-z)^\frac{\epsilon}{2}=0
\end{equation}
\begin{equation}
    e^z+\frac{\Gamma(-\frac{\epsilon}{2})}{\Gamma(\frac{\varepsilon+1}{2})}(-1)^\frac{\epsilon}{2}z^{\epsilon+\frac{1}{2}}=0
\end{equation}
\begin{equation}
    \Gamma\left(-\frac{\varepsilon}{2}\right)\Gamma\left(1+\frac{\varepsilon}{2}\right)=\frac{\pi}{\sin\left(-\frac{\pi\varepsilon}{2}\right)}
\end{equation}
\begin{equation}
    \frac{\Gamma\left(-\frac{\varepsilon}{2}\right)}{\Gamma\left(\frac{\varepsilon+1}{2}\right)}=-\frac{\pi}{\sin\frac{\pi\varepsilon}{2}\Gamma\left(1+\frac{\varepsilon}{2}\right)\Gamma\left(\frac{1+\varepsilon}{2}\right)}=-\frac{2^\varepsilon\sqrt{\pi}}{\sin\frac{\pi\varepsilon}{2}\Gamma(1+\varepsilon)}=-\frac{2^\epsilon\sqrt{\pi}}{\sin\frac{\pi\varepsilon}{2}\varepsilon\Gamma(\varepsilon)}
\end{equation}
\begin{equation}
    \Gamma(\varepsilon)=\Gamma(n+\delta n)=(n-1+\delta n)...(1+\delta n)\Gamma(1+\delta n)\approx\Gamma(n)
\end{equation}
\begin{equation}
    \frac{\Gamma\left(-\frac{\varepsilon}{2}\right)}{\Gamma\left(\frac{\varepsilon+1}{2}\right)}=-\frac{2^n\sqrt{\pi}}{\sin(\frac{\pi n}{2}+\frac{\pi\delta n}{2})n\Gamma(n)\Gamma(1+\delta n)}\approx
    \begin{cases}
        -\frac{(-1)^k2^n\sqrt{\pi}}{n\Gamma(n)},\quad n=2k+1\\
        -\frac{(-1)^k2^{n+1}}{\sqrt{\pi}n\delta n\Gamma(n)},\quad n=2k
    \end{cases}
\end{equation}
\begin{equation}
    e^z-\frac{2^{n+1}\sqrt{\pi}}{\pi\delta nn\Gamma(n)}z^{n+\frac{1}{2}}=0\quad(n=2k)
\end{equation}
\begin{equation}
    \delta n=\frac{e^{-z}2^{n+1}}{\sqrt{\pi}n\Gamma(n)}z^{n+\frac{1}{2}}
\end{equation}
\begin{equation}
    {}_1F_1\left(-\frac{\varepsilon-1}{2},\frac{3}{2},\frac{m\omega L^2}{\hbar}\right)e^{-\frac{m\omega L^2}{2\hbar}}=0
\end{equation}
\begin{equation}
    \frac{\Gamma\left(\frac{3}{2}\right)}{\Gamma\left(-\frac{\varepsilon-1}{2}\right)}e^zz^{-\frac{\varepsilon}{2}-1}+\frac{\Gamma\left(\frac{3}{2}\right)}{\Gamma\left(\frac{\varepsilon+1}{2}\right)}(-z)^\frac{\epsilon-1}{2}=0
\end{equation}
\begin{equation}
    e^z+\frac{\Gamma\left(-\frac{\varepsilon-1}{2}\right)}{\Gamma\left(1+\frac{\epsilon}{2}\right)}(-1)^\frac{\epsilon+1}{2}z^{\varepsilon+\frac{1}{2}}=0
\end{equation}
\begin{equation}
    \Gamma\left(-\frac{\varepsilon-1}{2}\right)\Gamma\left(\frac{\epsilon+1}{2}\right)=\frac{\pi}{\sin\left(-\frac{\pi(\varepsilon-1)}{2}\right)}
\end{equation}
\begin{multline}
    \frac{\Gamma\left(-\frac{\varepsilon-1}{2}\right)}{\Gamma\left(1+\frac{\varepsilon}{2}\right)}=-\frac{\pi}{\sin\left(\frac{\pi(\varepsilon-1)}{2}\right)\Gamma\left(\frac{\varepsilon}{2}+1\right)\Gamma\left(\frac{1+\varepsilon}{2}\right)}=-\frac{\pi}{\sqrt{\pi}2^{-\varepsilon}\varepsilon\Gamma(\varepsilon)\sin\left(\frac{\pi(\varepsilon-1)}{2}\right)}=\\=-\frac{2^\varepsilon\sqrt{\pi}}{\sin\left(\frac{\pi}{2}(n-1)+\frac{\pi}{2}\delta n\right)n\Gamma(n)}\approx
    \begin{cases}
        \frac{(-1)^k2^\varepsilon\sqrt{\pi}}{n\Gamma(n)},\quad n=2k+1\\
        \frac{(-1)^k2^{\varepsilon+1}}{\sqrt{\pi}n\delta n\Gamma(n)},\quad n=2k
    \end{cases}
\end{multline}
\begin{equation}
    e^z-\frac{2^{n+1}\sqrt{\pi}}{\pi\delta nn\Gamma(n)}z^{n+\frac{1}{2}}=0\quad(n=2k+1)
\end{equation}
\begin{equation}
    \delta n=\frac{e^{-z}2^{n+1}}{\sqrt{\pi}n\Gamma(n)}z^{n+\frac{1}{2}}
\end{equation}
Таким образом, $\forall n\in\mathbb{Z}\hookrightarrow\delta n=e^{-\frac{m\omega L^2}{\hbar}}\left(\frac{m\omega L^2}{\hbar}\right)^{n+\frac{1}{2}}\frac{2^{n+1}}{\sqrt{\pi}\Gamma(n+1)}$.
\begin{equation}
    E_n=\hbar\omega\left(\frac{1}{2}+n+\frac{2^{n+1}e^{-\frac{m\omega L^2}{\hbar}}\left(\frac{m\omega L^2}{\hbar}\right)^{n+\frac{1}{2}}}{\sqrt{\pi}\Gamma(n+1)}\right)
\end{equation}
\begin{equation}
    \boxed{F_n=-\frac{\partial E_n}{\partial L}\approx \frac{2^{n+2}m\omega^2Le^{-\frac{m\omega L^2}{\hbar}}\left(\frac{m\omega L^2}{\hbar}\right)^{n+\frac{1}{2}}}{\sqrt{\pi}\Gamma(n+1)}}
\end{equation}
\textbf{Задача 3. Hydrogen atom in 2D (25 баллов)}\\
Определите уровни энергии и кратности их вырождения, а также стационарные волновые функции для двумерной частицы, движущейся в притягивающем потенциале $U(\vec{r}) = -\frac{e^2}{r}$. Указание: задача приводится к вырожденной гипергеометрической функции ${}_1F_1$.\\
\textbf{Решение.}\\
Стационарное уравнение Шрёдингера:
\begin{equation}
    -\frac{\hbar^2}{2M}\Delta\psi-\frac{e^2}{r}\psi=E\psi
\end{equation}
\begin{equation}
    -\frac{\hbar^2}{2M}\left(\frac{\partial^2\psi}{\partial r^2}+\frac{1}{r}\frac{\partial\psi}{\partial r}+\frac{1}{r^2}\frac{\partial^2\psi}{\partial\varphi^2}\right)-\frac{e^2}{r}\psi=E\psi
\end{equation}
Разделим переменные:
\begin{equation}
    \psi(r,\varphi)=R(r)e^{im\varphi}
\end{equation}
Введём замену: $\kappa^2=-\frac{2mE}{\hbar^2}$.
\begin{equation}
    \frac{\partial^2R}{\partial r^2}+\frac{1}{r}\frac{\partial R}{\partial r}-\left(\frac{2Me^2}{\hbar^2r}+\kappa^2+\frac{m^2}{r^2}\right)R=0
\end{equation}
Пусть $x=\kappa r$, тогда
\begin{equation}
    R''+\frac{R'}{x}-\left(1+\frac{2Me^2}{\hbar^2\kappa x}+\frac{m^2}{x^2}\right)R=0
\end{equation}
Введём замену $R=x^m f$:
\begin{equation}
    R'=mx^{m-1}f+x^mf',\quad R''=m(m-1)x^{m-2}f+2mx^{m-1}f'+x^mf''
\end{equation}
\begin{equation}
    xf''+(2m+1)f'-\left(x+\frac{2Me^2}{\hbar^2\kappa}\right)f=0
\end{equation}
\begin{equation}
    x\rightarrow\infty:f''-f=0\rightarrow f\sim e^{-x}
\end{equation}
Введём замену $f(x)=e^{-x}\chi(x)$.
\begin{equation}
    f'(x)=-e^{-x}\chi+e^{-x}\chi',\quad f''(x)=e^{-x}\chi-2e^{-x}\chi'+e^{-x}\chi''
\end{equation}
\begin{equation}
    x\chi''+(2m+1-2x)\chi'-\left(2m+1+\frac{2Me^2}{\hbar^2\kappa}\right)\chi=0
\end{equation}
\begin{equation}
    \chi(x)=C{}_1F_1\left(m+\frac{1}{2}-\frac{Me^2}{\hbar^2\kappa},2m+1,2x\right)
\end{equation}
Второе линейно независимое решение сингулярно в 0.
\begin{equation}
    m+\frac{1}{2}-\frac{Me^2}{\hbar^2\kappa}=-n
\end{equation}
\begin{equation}
    \boxed{E_{n,m}=-\frac{\hbar^2}{2M}\chi^2=-\frac{Me^2}{2\hbar^2\left(|m|+n+\frac{1}{2}\right)^2}}
\end{equation}
Пусть $N=|m|+n$, тогда кратность вырождения:
\begin{equation}
    \boxed{K=2N-1}
\end{equation}
\begin{equation}
    \boxed{\psi(r,\varphi)=Ce^{im\varphi}\left(\frac{\sqrt{-2mE}}{\hbar} r\right)^{|m|}e^{-\frac{\sqrt{-2mE}}{\hbar} r}{}_1F_1\left(-n,2m+1,2\frac{\sqrt{-2mE}}{\hbar}r\right)}
\end{equation}
\section{Стационарная теория возмущений}
\subsection*{Задачи (100 баллов)}
\textbf{Задача 1. Поляризуемость частицы на кольце (25 баллов)}\\
Частица с массой $M$ и зарядом $e$ движется на кольце радиуса $R$. К системе прикладывается слабое электрическое поле $\mathcal{E}$ параллельно плоскости кольца. Определите поляризуемости $\alpha_n = -\frac{\partial^2E_n}{\partial\mathcal{E}^2}$ всех уровней энергии.\\
Проведите вычисления, используя два различных базиса:
\begin{enumerate}
    \item Базис состояний с фиксированным угловым моментом $m$ («волны де-Бройля»)
    \item Базис состояний с фиксированной чётностью
\end{enumerate}
\textit{Указание:} обратите внимание, что поскольку в задаче с приложенным электрическим полем чётность по-прежнему сохраняется, то для вычисления второго пункта применять вырожденную теорию возмущений нет необходимости.\\
\textbf{Решение.}
\begin{enumerate}
    \item Базис состояний с фиксированным угловым моментом:
    \begin{equation}
        \hat{p}^2=-\hbar^2\Delta=-\frac{\hbar^2}{r^2}\frac{\partial^2}{\partial\varphi^2}
    \end{equation}
    Невозмущённая задача:
    \begin{equation}
        \hat{H_0}\psi=E^{(0)}_m\psi
    \end{equation}
    \begin{equation}
        -\frac{\hbar^2}{2M}\Delta\psi=E^{(0)}_m\psi\rightarrow\psi_m(\varphi)=\frac{e^{im\varphi}}{\sqrt{2\pi}}
    \end{equation}
    Константа получена из условия нормировки:
    \begin{equation}
        \int\limits_0^{2\pi}\psi_m^*\psi_{m'}d\varphi=\delta(m-m')
    \end{equation}
    Таким образом, $\psi_m(\varphi)=\frac{e^{im\varphi}}{\sqrt{2\pi}},m\in\mathbb{Z}$ -- базис волновых функций де Бройля. Энергия основного состояния:
    \begin{equation}
        E^{(0)}_m=\frac{m^2}{2MR^2}
    \end{equation}
    Как видно, кратность вырождения энергии равна 2 (состояниям $m$ и $-m$ соответствует энергия $E_m$).\\
    Возмущённый гамильтониан:
    \begin{equation}
        \hat{H}=\hat{H}_0+\hat{V}=-\frac{\hbar^2}{2MR^2}\frac{\partial^2}{\partial\varphi^2}-e\mathcal{E}R\cos\varphi
    \end{equation}
    Матричные элементы возмущения:
    \begin{equation}
        V_{\alpha\beta}\equiv\braket{n_\alpha^{(0)}|\hat{V}|n_\beta^{(0)}}=-e\mathcal{E}R\int\limits_0^{2\pi}\cos\varphi e^{i(\beta-\alpha)\varphi}d\varphi=-\frac{1}{2}e\mathcal{E}R(\delta_{\beta-\alpha,1}+\delta_{\beta-\alpha,-1})
    \end{equation}
    Как видно, $V_{\alpha\alpha}=0$, а также вырождение не снято, поэтому нужно рассматривать 2 порядок. Эффективный гамильтониан:
    \begin{equation}
        \hat{H}^{\text{(eff)}}_{\alpha\beta}=\sum\limits_{m\neq n}\frac{V_{\alpha m}V_{m\beta}}{\omega_{mn}},\quad \omega_{mn}=E^{(0)}_m-E^{(0)}_n=\frac{m^2-n^2}{2MR^2}
    \end{equation}
    Сужение гамильтониана:
    \begin{equation}
        \hat{H}_{\alpha\beta}^{(\text{eff})}=e^{2}\mathcal{E}^{2}MR^{4}
        \left(\begin{array}{cc}
        \frac{1}{4n^{2}-1} & \frac{\delta_{n,1}}{2}\\
        \frac{\delta_{n,1}}{2} & \frac{1}{4n^{2}-1}
        \end{array}\right)
    \end{equation}
    Секулярное уравнение $\text{det}(\hat{H}_{\alpha\beta}^{(\text{eff})}-E_n^{(2)})=0$ даёт решения
    \begin{equation}
        E_1^{(2)}=\frac{e^2\mathcal{E}^2MR^4}{6},\quad E_1^{(2)}=-\frac{5e^2\mathcal{E}^2MR^4}{6},\quad E_n^{(2)}=-\frac{e^2\mathcal{E}^2MR^4}{4n^2-1}
    \end{equation}
    \begin{equation}
        \boxed{\alpha_1\in\{-\frac{e^2MR^4}{3},\frac{5e^2MR^4}{3}\},\quad \alpha_n=\frac{2e^2MR^4}{4n^2-1}}
    \end{equation}
    Рассмотрим случай $n=0$, в котором вырождения нет.
    \begin{equation}
        V_{00}=0
    \end{equation}
    Второй порядок теории возмущений:
    \begin{equation}
        E_{0}^{(2)}=\sum_{n\neq0}\frac{|V_{0n}|^{2}}{\omega_{0n}}
    \end{equation}
    \begin{equation}
        E_{0}^{(2)}=-e^{2}\mathcal{E}^{2}MR^{4}
    \end{equation}
    \begin{equation}
        \boxed{\alpha_0=2e^{2}MR^{4}}
    \end{equation}
    \item Базис состояний с фиксированной чётностью.\\
    Рассмотрим базис из функций:
    \begin{equation}
        \psi_{m}^{(+)}=\frac{1}{\sqrt{\pi}}\cos(m\varphi),\quad\psi_{m}^{(-)}=\frac{1}{\sqrt{\pi}}\sin(m\varphi)
    \end{equation}
    Матричные элементы возмущения:
    \begin{equation}
        V_{\alpha+,\beta+}=\frac{e\mathcal{E}R}{\varphi}\int_{0}^{2\pi}\cos(\varphi \alpha)\cos(\varphi\beta)\cos\varphi d\varphi=\frac{e\mathcal{E}R}{2}\left(\delta_{\alpha-\beta,1}+\delta_{\beta-\alpha,1}+\delta_{n+m,1}\right)
    \end{equation}
    \begin{equation}
        V_{\alpha-,\beta-}=\frac{e\mathcal{E}R}{\pi}\int_{0}^{2\pi}\sin(\varphi\alpha)\sin(\varphi\cdot\beta)\cos\varphi d\phi=\frac{e\mathcal{E}R}{2}\left(\delta_{\alpha-\beta,1}+\delta_{\alpha-\beta,1}-\delta_{\alpha+\beta,1}\right)
    \end{equation}
    \begin{equation}
        V_{\alpha-,\beta+}=\frac{e\mathcal{E}R}{\pi}\int_{0}^{2\pi}\sin(\varphi\alpha)\cos(\varphi\beta)\cos\varphi d\phi=0=V_{\alpha+,\beta-}
    \end{equation}
    Сужение эффективного гамильтониана:
    \begin{equation}
        \hat{H}_{\alpha\beta}^{(\text{eff})}=\frac{e^{2}\mathcal{E}^{2}MR^{4}}{\hbar^{2}}\left(\begin{array}{cc}
        \frac{\delta_{n,1}}{2}+\frac{1}{4n^{2}-1} & 0\\
        0 & -\frac{\delta_{n,1}}{2}+\frac{1}{4n^{2}-1}
    \end{array}\right)
    \end{equation}
    Решая секулярное уравнение, получаем тот же ответ.\\
    В случае $n=0$:
    \begin{equation}
         V_{0\beta,+}=\frac{e\mathcal{E}R}{\sqrt{2}}\delta_{n,1},V_{0\beta,-}\equiv0
    \end{equation}
    Поправка к энергии получается такой же.
\end{enumerate}
\textbf{Задача 2. Двуслойный графен (15 баллов)}\\
В силу решёточной структуры графена, волновая функция электронов представляет собой двухкомпонентный псевдоспинор $\psi(x)=(\psi_A(x), \psi_B(x))^T$, где различные компоненты отвечают различным подрешёткам. Возбуждения вблизи так называемой $K$-долины зоны Бриллюэна описываются эффективным гамильтонианом размера 2×2, который записывается следующим образом:
\begin{equation}
    \hat{H}=v_F\hat{\vec{\sigma}}\cdot\hat{\vec{p}}=v_F(\hat{\sigma}_x\hat{p}_x+\hat{\sigma}_y\hat{p}_y)=v_F
    \begin{pmatrix}
    0 & \hat{p}_x-i\hat{p}_y\\
    \hat{p}_x+i\hat{p}_y & 0
    \end{pmatrix}
\end{equation}
Для двуслойного графена, соответственно, волновая функция уже образует четырёхкомпонентный спинор
\begin{equation}
    \psi(x)=(\psi^{(1)}_A(x), \psi^{(1)}_B(x), \psi^{(2)}_A(x), \psi^{(2)}_B(x))
\end{equation}
Если бы между не было никакого взаимодействия, то гамильтониан системы имеет просто блочно-диагональный вид $\hat{H}=v_F\begin{pmatrix}
\hat{\vec{\sigma}}\cdot\hat{\vec{p}} & 0\\
0 & \hat{\vec{\sigma}}\cdot\hat{\vec{p}}
\end{pmatrix}$; однако, если два листа графена положить друг на друга определённым образом, то возникает возможность туннелирования электронов между слоями. Гамильтониан такой системы имеет вид:
\begin{equation}
    \hat{H}=\begin{pmatrix}
    0 & v_F(\hat{p}_x-i\hat{p}_y) & 0 & 0\\
    v_F(\hat{p}_x+i\hat{p}_y) & 0 & \Delta & 0\\
    0 & \Delta & 0 & v_F(\hat{p}_x-i\hat{p}_y)\\
    0 & 0 & v_F(\hat{p}_x+i\hat{p}_y) & 0
    \end{pmatrix}
\end{equation}
При нулевом импульсе $p$, матрица имеет двукратно вырожденное собственное число 0 и однократно вырожденные числа $\pm\Delta$. Если мы исследуем свойства возбуждений при маленькой энергии $|E| \ll \Delta$ и, соответственно, малом импульсе $p \ll \frac{\Delta}{v_F}$, то высокоэнергетические состояния $\pm\Delta$ несущественны. Используя теорию возмущений, выведите эффективный
гамильтониан, описывающий низкоэнергетические возбуждения, и определите спектр соответствующих двух зон.\\
\textbf{Решение.}\\
Пусть $a=v_F(\hat{p}_x-i\hat{p}_y)$.
\begin{equation}
    w_{11}=w_{14}=0,\quad w_{12}=\Delta,\quad w_{13}=-\Delta
\end{equation}
1 порядок не убирает вырождение по энергии.
\begin{equation}
    \hat{V}=v_F\begin{pmatrix}
    \hat{\vec{\sigma}}\cdot\hat{\vec{p}} & 0\\
    0 & \hat{\vec{\sigma}}\cdot\hat{\vec{p}}
    \end{pmatrix}
\end{equation}
\begin{equation}
    H^{(1)}_{11}=\frac{V_{12}V_{21}}{\omega_{12}}+\frac{V_{13}V_{31}}{\omega_{13}}=-\frac{aa^\dagger}{\Delta},\quad H^{(1)}_{14}=\frac{V_{12}V_{24}}{\omega_{12}}+\frac{V_{13}V_{34}}{\omega_{13}}=0
\end{equation}
\begin{equation}
    H^{(1)}_{41}=\frac{V_{42}V_{21}}{\omega_{12}}+\frac{V_{13}V_{31}}{\omega_{13}}=0,\quad H^{(1)}_{44}=\frac{V_{42}V_{24}}{\omega_{12}}+\frac{V_{43}V_{34}}{\omega_{13}}=\frac{aa^\dagger}{\Delta}
\end{equation}
\begin{equation}
    \boxed{\hat{H}^{\text{eff}}=\frac{v_F^2p^2}{\Delta}\begin{pmatrix}
    -1 & 0\\
    0 & 1\\
    \end{pmatrix}}
\end{equation}
\begin{equation}
    \text{det}(\hat{H}^{\text{eff}}-E_0^{(2)})=0\rightarrow \boxed{E_0^{(2)}=\pm\frac{p^2}{\Delta}v_F^2}
\end{equation}
\textbf{Задача 3. Поляризуемость атома водорода (25 баллов)}\\
Атом водорода находится в основном состоянии, описываемом волновой функцией $\psi_{100}(r) = \frac{e^{-r/a}}{\sqrt{\pi a^3}}$, где $a = \frac{\hbar^2}{me^2}$ -- Боровский радиус. Он помещается в постоянное электрическое поле, описываемое гамильтонианом $\hat{V} = -e\mathcal{E}\hat{z}$; требуется определить сдвиг уровня энергии и поляризуемость основного состояния.\\
Для расчёта сдвига энергии во втором порядке теории возмущений, вообще говоря, необходимо вычислять все матричные элементы $\braket{1, 0, 0|\hat{z}|n, l, m}$, а кроме них -- ещё и матричные элементы $\braket{1, 0, 0|\hat{z}|k, l, m}$ ($k$ определяет энергию $\frac{\hbar^2k^2}{2m} = E$) для состояний непрерывного спектра, а затем вычислять соответствующую сумму и интеграл. Это сложная задача; но, к счастью, тут можно поступить иначе. Для этого первую поправку по теории возмущений к волновой функции предлагается найти \textit{точно}, а не искать её в виде разложения.
\begin{enumerate}
    \item Напишите точное уравнение для поправки первого порядка $\psi^{(1)}(r)$ по теории возмущений. Покажите, что в нём можно разделить переменные, сделав анзац $\psi^{(1)}(r,\theta) = f(r)e^{-r/a}\cos\theta$; напишите уравнение на функцию $f(r)$.\\
    \textit{Hint:} обезразмерьте это уравнение!
    \item Покажите, что это уравнение имеет простое полиномиальное решение $f(r) = A + Br + Cr^2$; найдите $A$, $B$, $C$.
    \item Найдя волновую функцию, вычислите поправку $E^{(2)}_{100}$ и поляризуемость $\alpha$.
\end{enumerate}
\textit{Замечание}: альтернативный способ решения задачи об атоме водорода в постоянном электрическом поле — это использование параболических координат. Эти координаты замечательны в частности тем, что в них разделяются переменные и в электрическом поле; а кроме этого, лишь небольшое количество матричных элементов оператора $\hat{z}$ отлично от нуля. Именно благодаря этому обстоятельству поправка первого порядка и имеет такой простой вид.\\
\textbf{Решение.}
\begin{enumerate}
    \item Гамильтониан невозмущённой системы:
    \begin{equation}
        \hat{H}_0=\frac{\hat{p}^2}{2m}-\frac{e^2}{r}
    \end{equation}
    Волновая функция и энергия основного состояния:
    \begin{equation}
        \psi^{(0)}(r)=\frac{e^{-r/a}}{\sqrt{\pi a^3}},\quad E^{(0)}=-\frac{e^2}{2a}
    \end{equation}
    Возмущение:
    \begin{equation}
        \hat{V}=-e\mathcal{E}\hat{z}
    \end{equation}
    \begin{equation}
        \psi=\psi^{(0)}(r)+\psi^{(1)}(r,\theta)
    \end{equation}
    \begin{equation}
        (\hat{H}_0+\hat{V})(\psi^{(0)}(r)+\psi^{(1)}(r,\theta))=(E^{(0)}+E^{(1)})(\psi^{(0)}(r)+\psi^{(1)}(r,\theta))
    \end{equation}
    \begin{equation}
        z=r\cos\theta
    \end{equation}
    \begin{equation}
        \hat{H}_0\psi^{(1)}+\hat{V}\psi^{(0)}=E^{(0)}\psi^{(1)}
    \end{equation}
    \begin{multline}
        -\frac{ae^2}{2}\left(\frac{\partial^2}{\partial r^2}+\frac{2}{r}\frac{\partial}{\partial r}+\frac{1}{r^2}\frac{\partial^2}{\partial\theta^2}+\frac{1}{r^2}\ctg\theta\frac{\partial}{\partial\theta}\right)\psi^{(1)}(r,\theta)-\frac{e^2}{r}\psi^{(1)}(r,\theta)-\frac{e\mathcal{E}e^{-\frac{r}{a}}}{\sqrt{\pi a^3}}r\cos\theta=\\=-\frac{e^2}{2a}\psi^{(1)}(r,\theta)
    \end{multline}
    \begin{equation}
        \psi^{(1)}(r,\theta) = f(r)e^{-r/a}\cos\theta
    \end{equation}
    \begin{equation}
        \frac{af''(r)}{2}+\frac{(a-r)f'(r)}{r}+\frac{r^2-2ar-2a^2}{2ar^2}f(r)+\frac{f(r)}{r}+\frac{\mathcal{E}r}{e\sqrt{\pi a^3}}=\frac{f(r)}{2a}
    \end{equation}
    \begin{equation}
        \frac{af''(r)}{2}+\frac{(a-r)f'(r)}{r}-\frac{af(r)}{r^2}+\frac{\mathcal{E}r}{e\sqrt{\pi a^3}}=0
    \end{equation}
    Пусть $x=\frac{r}{a}$, тогда
    \begin{equation}
        \boxed{f''(x)+\frac{2(1-x)f'(x)}{x}-\frac{2f(x)}{x^2}+2\sqrt{\frac{a}{\pi}}\frac{\mathcal{E}x}{e}=0}
    \end{equation}
    \item Данное уравнение имеет простое полиноминальное решение:
    \begin{equation}
        f(x) = A + Bx + Cx^2
    \end{equation}
    \begin{equation}
        C+\frac{(1-x)(B+2Cx)}{x}-\frac{A+Bx+Cx^2}{x^2}+\sqrt{\frac{a}{\pi}}\frac{\mathcal{E}x}{e}=0
    \end{equation}
    \begin{equation}
        2C-B-2Cx-\frac{A}{x^2}+\sqrt{\frac{a}{\pi}}\frac{\mathcal{E}x}{e}=0
    \end{equation}
    Приравняем коэффициенты при каждой степени:
    \begin{equation}
    \begin{cases}
        A=0,\\
        2C-B=0.\\
        C=\sqrt{\frac{a}{\pi}}\frac{\mathcal{E}}{2e}
    \end{cases}
    \end{equation}
    \begin{equation}
        \boxed{A=0,\quad B=\sqrt{\frac{a}{\pi}}\frac{\mathcal{E}}{e},\quad C=\sqrt{\frac{a}{\pi}}\frac{\mathcal{E}}{2e}}
    \end{equation}
\item\begin{equation}
        f(x) = A + Bx + Cx^2=\sqrt{\frac{a}{\pi}}\frac{\mathcal{E}x}{e}\left(1+\frac{x}{2}\right)
    \end{equation}
    \begin{equation}
        \psi^{(1)}(r,\theta) = f(r)e^{-r/a}\cos\theta
    \end{equation}
    \begin{equation}
        \boxed{\psi^{(1)}(r,\theta)=\frac{\mathcal{E}r}{e\sqrt{\pi a}}\left(1+\frac{r}{2a}\right)e^{-r/a}\cos\theta}
    \end{equation}
    2 порядок теории возмущений:
    \begin{equation}
        (\hat{H}_0+\hat{V})(\psi^{(0)}(r)+\psi^{(1)}(r,\theta)+\psi^{(2)}(r,\theta))=(E^{(0)}+E^{(2)})(\psi^{(0)}(r)+\psi^{(1)}(r,\theta)+\psi^{(2)}(r,\theta))
    \end{equation}
    \begin{equation}
        \hat{H}_0\psi^{(2)}(r,\theta)+\hat{V}\psi^{(1)}(r,\theta)=E^{(0)}\psi^{(2)}(r)+E^{(2)}\psi^{(0)}(r)
    \end{equation}
    \begin{multline}
        E^{(2)}=\braket{\psi^{(0)}(r)|\hat{V}|\psi^{(1)}(r,\theta)}=\int\limits_{-\pi}^\pi d\varphi\int\limits_1^{-1} d\cos\theta\int\limits_0^\infty dr r^2\frac{e^{-r/a}}{\sqrt{\pi a^3}}(-e\mathcal{E}r\cos\theta)\frac{\mathcal{E}r}{e\sqrt{\pi a}}*\\*\left(1+\frac{r}{2a}\right)e^{-r/a}\cos\theta=-\frac{9}{4}a^3\mathcal{E}^2
    \end{multline}
    \begin{equation}
        \boxed{E^{(2)}=-\frac{9}{4}a^3\mathcal{E}^2}
    \end{equation}
    \begin{equation}
        \boxed{\chi=-\frac{\partial^2E^{(2)}}{\partial\mathcal{E}^2}=\frac{9a^3}{2}}
    \end{equation}
\end{enumerate}
\textbf{Задача 4. Суперсимметричная квантовая механика (45 баллов)}\\
Рассмотрите для простоты квантовый гармонический осциллятор для частицы массы $m = 1$ с частотой $\omega = 1$, гамильтониан которого имеет вид $\hat{H} =\frac{1}{2}(\hat{p}^2+\hat{x}^2-1)$, так что основное состояние имеет нулевую энергию $E_0 = 0$. Помимо этого, осциллятор содержит ангармонизм \textit{очень} специального вида $V(x) = gx(1-x^2) + \frac{1}{2}g^2x^4$; безразмерный параметр $g \ll 1$. Требуется исследовать сдвиг энергии основного состояния с точки зрения теории возмущений.
\begin{enumerate}
    \item \textbf{(10 баллов)} Найдите поправку $O(g^2)$ к энергии основного состояния осциллятора.
    \item \textbf{(10 баллов)} Полученный ответ должен вас натолкнуть на мысль, что нужно считать старшие порядки ТВ (если нет, то перепроверьте). Удобнее всего это делать в координатном базисе, выписывая явно дифференциальные уравнения на поправки к волновым функциям. Выпишите такое уравнение на поправки $O(g)$ и $O(g^2)$, и решите их явно.
    \item \textbf{(15 баллов)} Проделайте это в \textit{произвольном} порядке теории возмущений, выпишите явно поправку произвольного порядка к волновой функции $\psi_0^{(n)}(x)$ и $E_0^{(n)}$. Просуммируйте весь ряд теории возмущений, и убедитесь, что вы, в действительности, нашли \textit{точное решение} соответствующего уравнения Шрёдингера. Расстройтесь, потому что это решение никак не позволяет ответить на вопрос, каков же в действительности сдвиг энергии основного состояния $\delta E_0$.
    \item \textbf{(5 баллов)} Очевидно, что поправка к энергии основного состояния всё-таки имеется -- но она \textit{непертурбативна}, для неё невозможно построить асимптотический ряд по степеням малого параметра $g$. Не смотря на это, теорией возмущений всё-таки можно пользоваться для приближения основного состояния, просто использовать для этого \textit{полный ряд} теории возмущений незаконно. Используя критерии применимости теории возмущений, определите максимальный порядок $n^*$, до которого результату ещё можно верить.
    \item \textbf{(5 баллов)} Погрешность асимптотического ряда для волновой функции -- то есть отличие истинной волновой функции от вычисленной при помощи ряда теории возмущений -- может быть оценена как член ряда теории возмущений с порядковым номером $n^*$. Оцените с экспоненциальной точностью квадрат нормы $\braket{\delta\psi|\delta\psi}$ этой погрешности; это даст также и оценку отличия энергии основного состояния от нуля.
\end{enumerate}
\textit{Указание:} хотя возмущение и содержит линейный член по $x$, с точки зрения вычислений оказывается удобно не сдвигать центр осциллятора.\\
\textbf{Решение.}
\begin{enumerate}
    \item Обозначим возмущения $V_1=gx(1-x^2)$, $V_2(x)=\frac{g^2x^4}{2}$. Основное состояние осциллятора: $\psi^{(0)}(x)=\frac{e^{-\frac{x^2}{2}}}{\sqrt[4]{\pi}}$.\\
    Рассмотрим 1 порядок теории возмущений:
    \begin{equation}
        (\hat{H}+\hat{V}_1)(\ket{\psi^{(0)}}+\ket{\psi^{(1)}})=(E^{(0)}+E^{(1)})(\ket{\psi^{(0)}}+\ket{\psi^{(1)}})
    \end{equation}
    Поскольку $E^{(0)}=E_0=0$, то $\hat{H}\ket{\psi^{(0)}}=0$ и
    \begin{equation}
        \hat{V}_1\ket{\psi^{(0)}}+\hat{H}\ket{\psi^{(1)}}=E^{(1)}\ket{\psi^{(0)}}
    \end{equation}
    Домножив скалярно на $\bra{\psi^{(0)}}$, получим
    \begin{equation}
        \bra{\psi^{(0)}}\hat{V}_1\ket{\psi^{(0)}}+\bra{\psi^{(0)}}\hat{H}\ket{\psi^{(1)}}=\bra{\psi^{(0)}}E^{(1)}\ket{\psi^{(0)}}
    \end{equation}
    \begin{equation}
        E^{(1)}=\bra{\psi^{(0)}}\hat{V}_1\ket{\psi^{(0)}}
    \end{equation}
    Из нечётности $\hat{V}_1$ получим
    \begin{equation}
        \boxed{E^{(1)}=0}
    \end{equation}
    Полученный ответ наталкивает на мысль перейти ко 2 пункту.
    \item 1 порядок теории возмущений:
    \begin{equation}
        \hat{H}\psi^{(1)}+\hat{V}^{(1)}\psi^{(0)}=0
    \end{equation}
    \begin{equation}
        \frac{1}{2}\left(-\frac{\partial^2}{\partial x^2}+x^2-1\right)\psi^{(1)}(x)+gx(1-x^2)\frac{e^{-\frac{x^2}{2}}}{\sqrt[4]{\pi}}=0
    \end{equation}
    \begin{equation}
        \psi^{(1)}(x)=e^{-\frac{x^2}{2}}f(x)
    \end{equation}
    \begin{equation}
        \left(fe^{-\frac{x^2}{2}}\right)''=\left(-xfe^{-\frac{x^2}{2}}+f'e^{-\frac{x^2}{2}}\right)'=e^{-\frac{x^2}{2}}((x^2-1)f(x)-2xf'(x)+f''(x))
    \end{equation}
    \begin{equation}
        2xf'(x)-f''(x)+\frac{2gx(1-x^2)}{\sqrt[4]{\pi}}=0
    \end{equation}
    Ищем $f(x)$ в виде:
    \begin{equation}
        f(x)=Bx+Cx^2+Dx^3
    \end{equation}
    \begin{equation}
        2x(B+2Cx+3Dx^2)-2C-6Dx+\frac{2gx(1-x^2)}{\sqrt[4]{\pi}}=0
    \end{equation}
    \begin{equation}
        \begin{cases}
            C=0,\\
            2B-6D+\frac{2g}{\sqrt[4]{\pi}}=0,\\
            6D-\frac{2g}{\sqrt[4]{\pi}}=0.
        \end{cases}
    \end{equation}
    \begin{equation}
        B=C=0,\quad D=\frac{g}{3\sqrt[4]{\pi}}
    \end{equation}
    \begin{equation}
        f(x)=\frac{gx^3}{3\sqrt[4]{\pi}}
    \end{equation}
    \begin{equation}
        \boxed{\psi^{(1)}(x)=\frac{gx^3e^{-\frac{x^2}{2}}}{3\sqrt[4]{\pi}}}
    \end{equation}
    Рассмотрим 2 порядок теории возмущений:
    \begin{equation}
        (\hat{H}+\hat{V}_1+\hat{V}_2)(\ket{\psi^{(0)}}+\ket{\psi^{(1)}}+\ket{\psi^{(2)}})=(E^{(0)}+E^{(1)}+E^{(2)})(\ket{\psi^{(0)}}+\ket{\psi^{(1)}}+\ket{\psi^{(2)}})
    \end{equation}
    \begin{equation}
        \hat{H}\ket{\psi^{(2)}}+\hat{V}_1\ket{\psi^{(1)}}+\hat{V}_2\ket{\psi^{(0)}}=E^{(0)}\ket{\psi^{(2)}}+E^{(1)}\ket{\psi^{(1)}}+E^{(2)}\ket{\psi^{(0)}}
    \end{equation}
    Домножим скалярно на $\bra{\psi^{(0)}}$ и воспользуемся тем, что $E^{(0)}=0$:
    \begin{equation}
        \bra{\psi^{(0)}}\hat{V}_1\ket{\psi^{(1)}}+\bra{\psi^{(0)}}\hat{V}_2\ket{\psi^{(0)}}=E^{(2)}
    \end{equation}
    \begin{equation}
        E^{(2)}=\int\limits_{-\infty}^\infty\frac{e^{-\frac{x^2}{2}}}{\sqrt[4]{\pi}}\frac{gx^3e^{-\frac{x^2}{2}}}{3\sqrt[4]{\pi}}gx(1-x^2)dx+\int\limits_{-\infty}^\infty\frac{e^{-x^2}}{\sqrt{\pi}}\frac{g^2x^4}{2}dx=-\frac{3g^2}{8}+\frac{3g^2}{8}=0
    \end{equation}
    \begin{equation}
        \boxed{E^{(2)}=0}
    \end{equation}
    \begin{equation}
        \hat{H}^{(0)}\psi^{(2)}+\hat{V}^{(1)}\psi^{(1)}+\hat{V}^{(2)}\psi^{(0)}=0
    \end{equation}
    \begin{equation}
        \frac{1}{2}\left(-\frac{\partial^2}{\partial x^2}+x^2\right)\psi^{(2)}+gx(1-x^2)\frac{gx^3}{3\sqrt[4]{\pi}}e^{-\frac{x^2}{2}}+\frac{g^2x^4e^{-\frac{x^2}{2}}}{2\sqrt[4]{\pi}}=0
    \end{equation}
    Решая ДУ аналогично предыдущему, получим
    \begin{equation}
        \boxed{\psi^{(2)}=\frac{g^2x^6}{18\sqrt[4]{\pi}}e^{-\frac{x^2}{2}}}
    \end{equation}
    \item Рассмотрим произвольный $n$-й порядок теории возмущений:
    \begin{equation}
        \hat{H}\ket{\psi^{(n)}}+\hat{V}_1\ket{\psi^{(n-1)}}+\hat{V}_2\ket{\psi^{(n-2)}}=E^{(0)}\ket{\psi^{(n)}}+E^{(1)}\ket{\psi^{(n-1)}}+...+E^{(n)}\ket{\psi^{(0)}}
    \end{equation}
    Домножим скалярно на $\bra{\psi^{(0)}}$ и воспользуемся тем, что $E^{(0)}=0$:
    \begin{equation}
        E^{(n)}=\bra{\psi^{(0)}}\hat{V}_1\ket{\psi^{(n-1)}}+\bra{\psi^{(0)}}\hat{V}_2\ket{\psi^{(n-2)}}
    \end{equation}
    Докажем по индукции одновременно 2 утверждения: $E^{(n)}=0$ и $\psi^{(n)}=\frac{g^nx^{3n}}{\sqrt[4]{\pi}3^nn!}e^{-\frac{x^2}{2}}$. Пусть это верно $\forall k\leq n-1$. Тогда
    \begin{multline}
        E^{(n)}=\int\limits_{-\infty}^\infty\frac{e^{-\frac{x^2}{2}}}{\sqrt[4]{\pi}}\frac{g^nx^{3(n-1)}}{\sqrt[4]{\pi}3^{n-1}(n-1)!}e^{-\frac{x^2}{2}}gx(1-x^2)dx+\\+\int\limits_{-\infty}^\infty\frac{e^{-\frac{x^2}{2}}}{\sqrt[4]{\pi}}\frac{g^{n-2}x^{3(n-2)}}{\sqrt[4]{\pi}3^{n-2}(n-2)!}e^{-\frac{x^2}{2}}\frac{g^2x^4}{2}dx=-\frac{3^{2-n}(1+(-1)^{3n})g^n\Gamma(\frac{3n}{2}-\frac{1}{2})}{4\sqrt{\pi}(n-2)!}+\\+\frac{3^{2-n}(1+(-1)^{3n})g^n\Gamma(\frac{3n}{2}-\frac{1}{2})}{4\sqrt{\pi}(n-2)!}=0
    \end{multline}
    \begin{equation}
        \boxed{E^{(n)}=0}
    \end{equation}
    Теперь проверим, что будет выполняться:
    \begin{equation}
        \hat{H}\psi^{(n)}+\hat{V}^{(1)}\psi^{(n-1)}+\hat{V}^{(2)}\psi^{(n-2)}=0
    \end{equation}
    \begin{multline}
        \frac{1}{2}\left(-\frac{\partial^2}{\partial x^2}+x^2\right)\frac{g^nx^{3n}}{\sqrt[4]{\pi}3^nn!}e^{-\frac{x^2}{2}}+gx(1-x^2)\frac{g^{n-1}x^{3(n-1)}}{\sqrt[4]{\pi}3^{n-1}(n-1)!}e^{-\frac{x^2}{2}}+\\+\frac{g^2x^4}{2}\frac{g^{n-2}x^{3(n-2)}}{\sqrt[4]{\pi}3^{n-2}(n-2)!}e^{-\frac{x^2}{2}}=0
    \end{multline}
    Таким образом, поправка произвольного порядка к волновой функции:
    \begin{equation}
        \boxed{\psi^{(n)}=\frac{g^nx^{3n}}{\sqrt[4]{\pi}3^nn!}e^{-\frac{x^2}{2}}}
    \end{equation}
    Просуммируем ряд возмущений:
    \begin{equation}
        \psi(x)=\sum\limits_{n=0}^\infty\frac{g^nx^{3n}}{\sqrt[4]{\pi}3^nn!}e^{-\frac{x^2}{2}}=\frac{e^{-\frac{x^2}{2}}}{\sqrt[4]{\pi}} \sum\limits_{n=0}^\infty\frac{1}{n!}\left(\frac{gx^3}{3}\right)^n=\frac{\exp\left(g\frac{x^3}{3}-\frac{x^2}{2}\right)}{\sqrt[4]{\pi}}
    \end{equation}
    Проверим, что это точное решение уравнения Шрёдингера:
    \begin{equation}
        \left(\frac{1}{2}\left(-\frac{\partial^2}{\partial x^2}+x^2-1\right)+gx(1-x^2)+\frac{g^2x^4}{2}\right)\psi(x)=E\psi(x)
    \end{equation}
    \begin{equation}
        \frac{1}{2}\left(-\frac{\partial^2}{\partial x^2}+x^2-1\right)\psi(x)=-\frac{gx(2-2x^2+gx^3)}{2\sqrt[4]{\pi}}\exp\left(g\frac{x^3}{3}-\frac{x^2}{2}\right)
    \end{equation}
    \begin{equation}
        \left(\frac{1}{2}\left(-\frac{\partial^2}{\partial x^2}+x^2-1\right)+gx(1-x^2)+\frac{g^2x^4}{2}\right)\psi(x)=0
    \end{equation}
    Таким образом, точное решение уравнения Шрёдингера:
    \begin{equation}
        \boxed{\psi(x)=\frac{\exp\left(g\frac{x^3}{3}-\frac{x^2}{2}\right)}{\sqrt[4]{\pi}}}
    \end{equation}
    Расстраиваемся, поскольку это решение никак не позволяет ответить на вопрос, какой сдвиг энергии $\delta E_0=\sum\limits_{n=0}^\infty E^{(n)}=0$.
    \item Критерий применимости теории возмущений:
    \begin{equation}
        \braket{\psi^{(n)}|\psi^{(n)}}\gg\braket{\psi^{(n+1)}|\psi^{(n+1)}}
    \end{equation}
    \begin{equation}
        \braket{\psi^{(n)}|\psi^{(n)}}=\int\limits_{-\infty}^\infty\frac{g^{2n}x^{6n}}{\sqrt{\pi}3^{2n}(n!)^2}e^{-x^2}dx=\frac{g^{2n}\Gamma(3n+\frac{1}{2})}{\sqrt{\pi}3^{2n}\Gamma(n+1)^2}=\frac{g^{2n}(6n)!}{2^{6n}3^{2n}(n!)^2(3n)!}
    \end{equation}
    При больших $n$ воспользуемся формулой Стирлинга:
    \begin{equation}
        n!\approx\sqrt{2\pi n}\left(\frac{n}{e}\right)^n
    \end{equation}
    \begin{equation}
        \braket{\psi^{(n)}|\psi^{(n)}}\sim\frac{g^{2n}\sqrt{12\pi n}(6n)^{6n}}{2^{6n}3^{2n}2\pi n^{2n+1}\sqrt{6\pi n}(3n)^{3n}}=\frac{3^ng^{2n}n^{n-1}}{\sqrt{2}\pi}
    \end{equation}
    \begin{equation}
        \frac{\braket{\psi^{(n+1)}|\psi^{(n+1)}}}{\braket{\psi^{(n)}|\psi^{(n)}}}=\frac{3g^2(n+1)^n}{n^{n-1}}\approx3g^2n\ll1
    \end{equation}
    \begin{equation}
        n^*\gg\frac{1}{3g^2}
    \end{equation}
    Например, можно взять другую степень $g$:
    \begin{equation}
        \boxed{n^*=\frac{1}{3g}}
    \end{equation}
    \item Подставим $n^*$:
    \begin{equation}
        \boxed{\braket{\delta\psi|\delta\psi}=\braket{\psi^{(n^*)}|\psi^{(n^*)}}=\frac{3^{n^*}g^{2n^*}(n^{*})^{n^*-1}}{\sqrt{2}\pi}}
    \end{equation}
\end{enumerate}
\section{Нестационарная теория возмущений}
\textbf{Задача 1. Осцилляция Раби (20 баллов)}\\
К атому водорода, находящемся в основном состоянии, прикладывается переменное электрическое поле $\hat{V}=-e\mathcal{E}\hat{z}cos(\omega t)$ с частотой $\omega =\frac34\text{Ry}+\varepsilon$, где <<отстройка частоты>> и характерная величина электрического поля $e\mathcal{E}a$, $\varepsilon \ll \text{Ry}$ (тут $\text{Ry}=\frac{me^4}{2\hbar^2}$ -- постоянная Ридберга, а $a = \frac{1}{me^2}$ -- Боровский радиус); в то же время, соотношение поля $\mathcal{E}$ и отстройки частоты $\varepsilon$, вообще говоря, произвольно. Определите вероятность обнаружить частицу в основном и первом возбуждённом состояниях $n = 1$ и $n = 2$ через большое время $T$.\\
\textbf{Решение.}\\
Нестационарное уравнение Шрёдингера:
\begin{equation}
    i\hbar\frac{\partial}{\partial t}\ket{\psi(t)}=(\hat{H}_0+\hat{V}(t))\ket{\psi(t)}
\end{equation}
\begin{equation}
    \hat{H}_0=\frac{\hat{p}^2}{2m}-\frac{e^2}{r},\quad \hat{V}(t)=-e\mathcal{E}\hat{z}cos(\omega t),\quad\hat{z}=r\cos\theta
\end{equation}
Эволюция основного и возбуждённого состояний, если бы не было возмущения:
\begin{equation}
    \ket{\psi_1(t)}=e^{-\frac{\text{iRyt}}{\hbar}}\ket{\psi_1},\quad \ket{\psi_2(t)}=e^{-\frac{\text{iRyt}}{4\hbar}}\ket{\psi_2}
\end{equation}
Нестационарная теория возмущений:
\begin{equation}
    i\hbar\frac{\partial\psi_2}{\partial t}=V_{21}\psi_1(t),\quad V_{21}=\bra{\psi_2(t)}\hat{V}(t)\ket{\psi_1(t)}
\end{equation}
\begin{equation}
    i\hbar\frac{\partial\psi_2}{\partial t}=-e\mathcal{E}\cos(\omega t)e^{\frac{3i\text{Ry}t}{4\hbar}}\bra{\psi_2}r\cos\theta\ket{\psi_1}\psi_1(t)
\end{equation}
\begin{equation}
    \bra{\psi_2}r\cos\theta\ket{\psi_1}=\iint d\Omega r^3\cos\theta R_{2,1}Y_{1,0}R_{1,0}Y_{0,0}=\frac{2^6\sqrt{2}a}{3^4}
\end{equation}
Остальные усреднения с $\psi_2$ при $n=2$: $l=0, m=0$, $l=1,m=-1$, $l=1,m=1$ дают 0 ввиду нечётности по углу $\theta$.
\begin{equation}
    i\hbar\frac{\partial\psi_2}{\partial t}=-e\mathcal{E}(e^{i\omega t}+e^{-i\omega t})e^\frac{3i\text{Ry}t}{4\hbar}\frac{2^5\sqrt{2}a}{3^4}\psi_1(t)=-e\mathcal{E}e^{-\frac{i\epsilon t}{\hbar}}\frac{2^5\sqrt{2}a}{3^4}\psi_1(t)
\end{equation}
где в последнем слагаемом пренебрегли быстроосциллирующим слагаемым. По аналогии,
\begin{equation}
    i\hbar\frac{\partial\psi_1}{\partial t}=-e\mathcal{E}e^{\frac{i\epsilon t}{\hbar}}\frac{2^5\sqrt{2}a}{3^4}\psi_2(t)
\end{equation}
\begin{equation}
\begin{cases}
    \frac{\partial\psi_1}{\partial t}=i\frac{e\mathcal{E}a}{\hbar}e^{\frac{i\epsilon t}{\hbar}}\frac{2^5\sqrt{2}}{3^4}\psi_2(t),\\
    \frac{\partial\psi_2}{\partial t}=i\frac{e\mathcal{E}a}{\hbar}e^{-\frac{i\epsilon t}{\hbar}}\frac{2^5\sqrt{2}}{3^4}\psi_1(t).
\end{cases}
\end{equation}
Решая полученную систему, получим
\begin{equation}
    \psi_2(t)=-ie^{-i\frac{\epsilon}{2}t}\frac{C}{\sqrt{\epsilon^2+C^2}}\sin\left(\frac{\sqrt{\epsilon^2+C^2}}{2\hbar}t\right)
\end{equation}
где $C=\frac{2^6\sqrt{2}}{3^4}e\mathcal{E}a$.\\
Вероятность оказаться в возбуждённом состоянии:
\begin{equation}
    \boxed{P_{n=2}(T)=\frac{C^2}{\epsilon^2+C^2}\sin^2\left(\frac{\sqrt{\epsilon^2+C^2}}{2\hbar}T\right)}
\end{equation}
Вероятность остаться в основном состоянии:
\begin{equation}
    \boxed{P_{n=1}(T)=1-\frac{C^2}{\epsilon^2+C^2}\sin^2\left(\frac{\sqrt{\epsilon^2+C^2}}{2\hbar}T\right)}
\end{equation}
\textbf{Задача 2. Эффект Ландау-Зенера (15 баллов)}\\
Рассмотрим произвольную двухуровневую систему, описываемую волновой функцией $\ket{\psi(t)}=\begin{pmatrix}
\psi_1(t)\\
\psi_2(t)
\end{pmatrix}$ и следующим
гамильтонианом, зависящим от времени:
\begin{equation}
    \hat{H}(t)=
    \begin{pmatrix}
    \alpha t\quad \gamma\\
    \gamma\quad -\alpha t
    \end{pmatrix}
    \end{equation}
В начальный момент времени $t \rightarrow-\infty$, система находилась в первом состоянии (так что $|\psi_1(t\rightarrow-\infty)|^2 = 1$). Определите вероятность перехода системы во второе состояние после такой эволюции $|\psi_2(t\rightarrow\infty)|^2$, считая «скорость движения уровней» большой $\alpha \gg \gamma^2$.\\
\textbf{Решение.}\\
Выделим в гамильтониане невозмущённую и возмущённую части:
\begin{equation}
    \hat{H}(t)=\hat{H}_0(t)+\hat{V}(t),\quad \hat{H}_0(t)=\begin{pmatrix}
    \alpha t\quad 0\\
    0\quad -\alpha t
    \end{pmatrix},\hat{V}=\begin{pmatrix}
    0\quad \gamma\\
    \gamma\quad 0
    \end{pmatrix}
\end{equation}
Эволюция состояний, если бы не было возмущения ($\gamma=0$):
\begin{equation}
    i\hbar\frac{\partial\ket{\psi_1(t)}}{\partial t}=\alpha t\ket{\psi_1},\quad i\hbar\frac{\partial\ket{\psi_2(t)}}{\partial t}=-\alpha t\ket{\psi_2}
\end{equation}
\begin{equation}
    \ket{\psi_1(t)}=\exp\left(-i\frac{\alpha t^2}{2\hbar}\right)\ket{\psi_1},\quad \ket{\psi_1(t)}=\exp\left(i\frac{\alpha t^2}{2\hbar}\right)\ket{\psi_1}
\end{equation}
Нестационарная теория возмущений:
\begin{equation}
    i\hbar\frac{\partial\psi_2(t)}{\partial t}=V_{21}\psi_1(t),\quad V_{21}=\bra{\psi_2(t)}\hat{V}\ket{\psi_1(t)}
\end{equation}
\begin{equation}
    V_{21}=\gamma\exp\left(-i\alpha t^2\right)
\end{equation}
\begin{equation}
    i\hbar\frac{\partial\psi_2(t)}{\partial t}=\gamma\exp\left(-i\alpha t^2\right)\psi_1(t)\approx\gamma\exp\left(-i\alpha t^2\right)
\end{equation}
\begin{equation}
    \psi_2(t\rightarrow\infty)=-\frac{i\gamma}{\hbar}\int\limits_{-\infty}^\infty\exp\left(-i\alpha t^2\right)dt=\frac{\sqrt{\pi i}\gamma}{\sqrt{\alpha}\hbar}
\end{equation}
Вероятность перехода системы во второе состояние:
\begin{equation}
    \boxed{P=\frac{\pi\gamma^2}{\alpha\hbar^2}}
\end{equation}
\textbf{Задача 3. Осциллятор (15 баллов)}\\
Рассмотрите электрон, движущийся в потенциале гармонического осциллятора. К системе приложено меняющееся со временем электрическое поле, так что Гамильтониан имеет вид:
\begin{equation}
    \hat{H}(t)=\frac{\hat{p}^2}{2m}+\frac{m\omega^2\hat{x}^2}{2}+e\mathcal{E}(t)\hat{x},\quad \mathcal{E}(t)=\mathcal{E}_0e^{-\frac{t^2}{\tau^2}}
\end{equation}
При $t \rightarrow -\infty$, электрон находится в основном состоянии системы. Определите вероятность обнаружить его в произвольном $n$-том возбуждённом состоянии при $t \rightarrow +\infty$ в ведущем порядке теории возмущений (считая поле слабым $e^2\mathcal{E}^2 \ll m\omega^3$).\\
\textbf{Решение.}\\
Эволюция состояний, если бы не было возмущения:
\begin{equation}
    \ket{n(t)}=e^{-i\omega(n+\frac{1}{2})t}
\end{equation}
Определим, какие матричные элементы возмущения вообще отличны от нуля. Используя лестничные операторы, видно, что отличны от нуля только переходы между соседними уровнями:
\begin{equation}
    V_{nm}(t)=-e\mathcal{E}_0e^{-\frac{t^2}{\tau^2}}\bra{n(t)}x\ket{m(t)}=-\frac{e\mathcal{E}_0e^{-\frac{t^2}{\tau^2}}}{\sqrt{2m\omega}}
    \begin{cases}
        e^{i\omega t}\sqrt{n},\quad\quad\quad\; m=n-1\\
        e^{-i\omega t}\sqrt{n+1},\quad m=n+1
    \end{cases}
\end{equation}
$n$-ый порядок теории возмущений:
\begin{equation}
    i\hbar\frac{\partial\psi_n(t)}{\partial t}=V_{nn-1}\psi_{n-1}(t)=-\frac{e\mathcal{E}_0e^{-\frac{t^2}{\tau^2}}e^{i\omega t}}{\sqrt{2m\omega}}\sqrt{n}\psi_{n-1}(t)
\end{equation}
\begin{equation}
    \psi_n(t\rightarrow\infty)=\left(\frac{ie\mathcal{E}_0}{\hbar\sqrt{2m\omega}}\right)^n\sqrt{n!}\int\limits_{-\infty}^\infty d\tau_1e^{-\frac{\tau_1^2}{\tau^2}+i\omega\tau_1}\int\limits_{-\infty}^{\tau_1} d\tau_2e^{-\frac{\tau_2^2}{\tau^2}+i\omega\tau_2}...\int\limits_{-\infty}^{\tau_{n-1}} d\tau_ne^{-\frac{\tau_n^2}{\tau^2}+i\omega\tau_n}
\end{equation}
Эти интегралы можно превратить в интеграл по всему $\mathbb{R}^n$, разделив на $2^{n-1}$:
\begin{equation}
    \psi_n(t\rightarrow\infty)=2^{1-n}\left(\frac{ie\mathcal{E}_0}{\hbar\sqrt{2m\omega}}\right)^n\sqrt{n!}\int\limits_{\mathbb{R}^n}dt^ne^{-\frac{t^2}{\tau^2}}e^{i\omega\sum\limits_{i=1}^n t_i}=\frac{2}{\sqrt{n!}}\left(\frac{ie\mathcal{E}_0\sqrt{\pi}\tau e^{-\frac{\omega^2\tau^2}{4}}}{2\hbar\sqrt{2m\omega}}\right)^n
\end{equation}
Вероятность оказаться в $n$-том состоянии:
\begin{equation}
    \boxed{P_n=\frac{4}{n!}\left(\frac{e^2\mathcal{E}^2_0\pi\tau^2 e^{-\frac{\omega^2\tau^2}{2}}}{8m\hbar^2\omega}\right)^n}
\end{equation}
\section{Адиабатическое приближение в нестационарных задачах. Фаза Берри}
\textbf{Упражнения (10 баллов)}\\
\textbf{Упражнение 1. Ионизация (5 баллов)}\\
Имеется двухъямный потенциал $U(x) = -\frac{\kappa}{m}\left[\delta(x+\frac{L(t)}{2})+\delta(x-\frac{L(t)}{2})\right]$. В начальный момент времени ямы разнесены друг от друга бесконечно далеко, $L\rightarrow\infty$, и электрон находится в одной из ям. Расстояние между ямами затем меленно уменьшается до нуля, так что в какой-то момент времени две ямы сливаются в одну: $U(x) = -\frac{2\kappa}{m}\delta(x)$. Определите, с какой вероятностью электрон останется в яме в результате такого процесса.\\
\textbf{Решение.}\\
Воспользуемся решением 3 задачи 3 недели. При $L\rightarrow\infty$ реализуется туннельный режим. Волновая функция электрона в начальный момент времени:
\begin{equation}
    \ket{\psi}=\frac{1}{\sqrt{2}}(\psi_\text{чёт}+\psi_\text{нечет})
\end{equation}
где $\psi_\text{чёт}$ -- чётная волновая функция, $\psi_\text{нечет}$ -- нечётная. Нечётное состояние существует только при $\kappa L\geq1$. Поэтому при $L\rightarrow0$ выживет только чётное связанное состояние, нужно узнать вероятность нахождения электрона в нём. По адиабатической теореме в процессе, состояние частицы в произвольный момент времени электрон будет оставаться в состоянии той чётности, в котором он оказался в начальный момент.
\begin{equation}
    \psi_\text{чёт}(t)\propto \frac{1}{\sqrt{2}}\psi_\text{чёт}(0)
\end{equation}
Вероятность остаться в яме:
\begin{equation}
    \boxed{P=\frac{1}{2}}
\end{equation}
\textbf{Упражнение 2. Дышите глубже (5 баллов)}\\
Частица массы $m$ движется в бесконечно глубокой потенциальной яме ширины $a(t)$, которая меняется во времени согласно следующему закону:
\begin{equation}
    a(t) = a(1-\alpha\sin^2\omega t),\quad \alpha < 1
\end{equation}
При каких условиях в такой задаче следует ожидать применимость адиабатического приближения?\\
\textbf{Решение.}\\
Уровни энергии в бесконечно глубокой яме:
\begin{equation}
    E_n=\frac{\pi^2\hbar^2n^2}{2ma^2}
\end{equation}
\begin{equation}
    \Delta E=E_{n+1}-E_n=\frac{\pi^2\hbar^2(2n+1)}{2ma^2}\geq\frac{\pi^2\hbar^2}{2ma^2}
\end{equation}
Критерий адиабатичности:
\begin{equation}
    \omega\ll\Delta E
\end{equation}
Таким образом, условия применимости адиабатичности:
\begin{equation}
    \boxed{\frac{\pi^2\hbar^2}{2ma^2}\gg\omega}
\end{equation}
\textbf{Задачи (90 баллов)}\\
\textbf{Задача 1. Переворот спина (20 баллов)}\\
Спин-1/2 находится в магнитном поле $\vec{B}$, которое медленно вращается в плоскости $yz$:
\begin{equation}
    \hat{H}=-\mu\vec{B}\cdot\hat{\vec{\sigma}},\quad \vec{B}(t)=B\left(0,\frac{1}{\cosh\omega t},-\tanh\omega t\right)
\end{equation}
и $\omega \ll \mu B$. В начальный момент времени $t \rightarrow-\infty$ спин и магнитное поле направлены по оси $z$. Найдите в первом неисчезающем порядке вероятность того, что спин останется направленным по оси $z$ после завершения вращения при $t \rightarrow\infty$.\\
\textbf{Решение.}\\
\begin{equation}
    \hat{H}=-\mu\vec{B}\cdot\hat{\vec{\sigma}}=-\mu
    \begin{pmatrix}
    B_z & B_x-iB_y\\
    B_x+iB_y & -B_z
    \end{pmatrix}=\mu B\begin{pmatrix}
    \tanh\omega t & \frac{i}{\cosh\omega t}\\
    -\frac{i}{\cosh\omega t} & -\tanh\omega t
    \end{pmatrix}
\end{equation}
Собственные значения и векторы этой матрицы:
\begin{equation}
    E_-=-\mu B,\quad E_+=\mu B
\end{equation}
\begin{equation}
    \ket{\psi_-}=\frac{1}{\sqrt{1+e^{-2\omega t}}}\begin{pmatrix}
    -ie^{-\omega t}\\
    1
    \end{pmatrix},\quad\ket{\psi_+}=\frac{1}{\sqrt{1+e^{2\omega t}}}\begin{pmatrix}
    ie^{\omega t}\\
    1
    \end{pmatrix}
\end{equation}
В начальный момент $c_-=1$, $c_+=0$. Нестационарная адиабатика:
\begin{equation}
    c_+(T\rightarrow\infty)=\int\limits_{-\infty}^\infty\exp\left(i\int\limits_0^t\omega_{+-}(\tau)d\tau\right)\frac{\bra{\psi_+}\partial_t\hat{H}\ket{\psi_-}}{\omega_{+-}(t)}dt,\quad\omega_{+-}=E_+-E_-=2\mu B
\end{equation}
\begin{equation}
    \partial_t\hat{H}=\frac{\mu B\omega}{\cosh^2\omega t}\begin{pmatrix}
    1 & -i\sinh\omega t\\
    i\sinh\omega t & -1
    \end{pmatrix}
\end{equation}
\begin{equation}
    \bra{\psi_+}\partial_t\hat{H}\ket{\psi_-}=-\frac{\mu B\omega}{\cosh\omega t}
\end{equation}
\begin{equation}
    c_-(T\rightarrow\infty)=-\int\limits_{-\infty}^\infty\frac{\exp\left(i2\mu Bt\right)}{2\cosh\omega t}\omega dt=-\int\limits_{-\infty}^\infty\frac{\exp\left(i\frac{2\mu Bx}{\omega}\right)}{2\cosh x}dx
    %=-\frac{\pi}{2\cosh{\frac{\pi\mu B}{\omega}}}
\end{equation}
Вычислим интеграл при помощи вычетов. Нули знаменателя: $z_k=\pi i\left(\frac{1}{2}+k\right),k\in\mathbb{Z}$. Берём только полюс $z_0=\frac{i\pi}{2}$.
\begin{equation}
    c_-(T\rightarrow\infty)=-2\pi i\underset{z=\frac{i\pi}{2}}{\text{res}}\frac{\exp\left(i\frac{2\mu Bz}{\omega}\right)}{2\cosh z}=-\pi\exp\left(-\frac{\pi\mu B}{\omega}\right)
\end{equation}
Вероятность того, что спин будет направлен по оси $z$:
\begin{equation}
    \boxed{P=\pi^2\exp\left(-\frac{2\pi\mu B}{\omega}\right)}
\end{equation}
\textbf{Задача 2. Эффект Ааронова-Бома (35 баллов)}\\
Электрон может двигаться по кольцу радиуса $R$, и находится в связанном состоянии потенциальной ямы $U(\varphi) = -U_0\delta(\varphi-\varphi_0)$. Через кольцо пропускают магнитный поток $\Phi$, так что гамильтониан системы имеет вид:
\begin{equation}
    \hat{H}=\frac{(\hat{p}_\varphi-\frac{e}{c}A_\varphi)^2}{2m}-U_0\delta(\varphi-\varphi_0),\quad \hat{p}_\varphi=-\frac{i}{R}\frac{\partial}{\partial\varphi},\quad A_\varphi=\frac{\Phi}{2\pi R}
\end{equation}
Яму адиабатически медленно обводят вокруг кольца по часовой стрелке — величина $\varphi_0$ меняется от $\varphi_0(0) = 0$ до $\varphi_0(T) = 2\pi$. Определите зависимость фазы, которую получит волновая функция электрона в результате такой эволюции, от энергии в режиме, когда размер связанного состояния много меньше радиуса кольца.\\
\textbf{Решение.}\\
Стационарное уравнение Шрёдингера:
\begin{equation}
    \hat{H}\psi(\varphi)=E\psi(\varphi)
\end{equation}
Распишем гамильтониан:
\begin{equation}
    \hat{H}=\frac{(\frac{i}{R}\frac{\partial}{\partial\varphi}+\frac{e\Phi}{2\pi Rc})^2}{2m}-U_0\delta(\varphi-\varphi_0)
\end{equation}
%\begin{equation}
    %\hat{H}=-\frac{1}{2mR^2}\partial^2_\varphi+\frac{ie\Phi}{2\pi R^2mc}\partial_\varphi+\frac{e^2\Phi^2}{8\pi^2R^2mc^2}-U_0\delta(\varphi-\varphi_0)
%\end{equation}
При $\varphi\neq\varphi_0$:
\begin{equation}
    \left(\frac{i}{R}\frac{\partial}{\partial\varphi}+\frac{e\Phi}{2\pi Rc}\right)^2\psi(\varphi)=2mE\psi(\varphi)
\end{equation}
Ищем решение $\psi(\varphi)$ в виде экспоненты:
\begin{equation}
    \left(\frac{i}{R}\lambda+\frac{e\Phi}{2\pi Rc}\right)^2=2mE
\end{equation}
\begin{equation}
    \lambda_{1,2}=i\left(\frac{e\Phi}{2\pi c}\pm \sqrt{2mE}R\right)
\end{equation}
Поскольку состояние связанное $E<0$, то 
\begin{equation}
    \lambda_{1,2}=\left(\frac{ie\Phi}{2\pi c}\mp \sqrt{2m|E|}R\right)
\end{equation}
Пусть $A=\frac{e\Phi}{2\pi c}$, $B=\sqrt{2m|E|}R$, тогда
\begin{equation}
    \psi(\varphi)=C_1e^{(iA-B)\varphi}+C_2e^{(iA+B)\varphi}
\end{equation}
Положим $\varphi_0=0$, а после сшивки сдивнем $\varphi$ обратно. Условия сшивки:
\begin{equation}
    \begin{cases}
        \psi(0+0)=\psi(2\pi-0),\\
        \psi'(0)-\psi'(2\pi-0)=-2mU_0R^2;
    \end{cases}
\end{equation}
\begin{equation*}
    \begin{cases}
        C_1+C_2=C_1e^{(iA-B)2\pi}+C_2e^{(iA+B)2\pi},\\
        (iA-B)C_1+(iA+B)C_2-(iA-B)C_1e^{(iA-B)2\pi}-(iA+B)C_2e^{(iA+B)2\pi}=-2mU_0R^2(C_1+C_2);
    \end{cases}
\end{equation*}
Выразим $C_2$ через $C_1$:
\begin{equation}
    C_2=-C_1e^{-2\pi B}\frac{\sin(\pi(A+iB))}{\sin(\pi(A-iB))}
\end{equation}
Характерный размер волновой функции $\ll R$, поэтому $e^{-B\varphi}\ll1$ и $B\gg1$. Условие нормировки:
\begin{equation}
    \int\limits_0^{2\pi}|\psi(\varphi)|^2d\varphi=1
\end{equation}
Пользуясь тем, что $B\gg1$, получим
\begin{equation}
    C_2=\sqrt{B}e^{-2\pi B},\quad C_1=\sqrt{B}e^{2\pi iA}
\end{equation}
\begin{equation}
    \psi(\varphi)=\sqrt{B}e^{2\pi iA}e^{(iA-B)\varphi}+\sqrt{B}e^{-2\pi B}e^{(iA+B)\varphi}
\end{equation}
Сдвигая на угол $\varphi_0$, получим
\begin{equation}
    \psi(\varphi)=\sqrt{B}e^{2\pi iA}e^{(iA-B)(\varphi-\varphi_0)}+\sqrt{B}e^{-2\pi B}e^{(iA+B)(\varphi-\varphi_0)}
\end{equation}
Фаза Берри:
\begin{equation}
    \gamma=\int\limits_0^{2\pi}d\varphi_0\bra{\psi(\varphi_0)}i\partial_{\varphi_0}\ket{\psi(\varphi_0)}
\end{equation}
\begin{equation}
    \partial_{\varphi_0}\psi(\varphi_0)=-\sqrt{B}e^{2\pi iA}(iA-B)e^{(iA-B)(\varphi-\varphi_0)}-\sqrt{B}(iA+B)e^{-2\pi B}e^{(iA+B)(\varphi-\varphi_0)}
\end{equation}
\begin{multline}
    \bra{\psi(\varphi_0)}i\partial_{\varphi_0}\ket{\psi(\varphi_0)}=-\int\limits_0^{2\pi}d\varphi(\sqrt{B}e^{-2\pi iA}e^{(-iA-B)(\varphi-\varphi_0)}+\sqrt{B}e^{-2\pi B}e^{(-iA+B)(\varphi-\varphi_0)})i\\(\sqrt{B}e^{2\pi iA}(iA-B)e^{(iA-B)(\varphi-\varphi_0)}+\sqrt{B}(iA+B)e^{-2\pi B}e^{(iA+B)(\varphi-\varphi_0)})=A
\end{multline}
\begin{equation}
    \gamma=\int\limits_0^{2\pi}d\varphi_0A=2\pi A
\end{equation}
\begin{equation}
    \boxed{\gamma=\frac{e\Phi}{c}}
\end{equation}
\textbf{Задача 3. Вот это поворот! (35 баллов)}\\
Рассмотрим квадратный двумерный ящик размера $L \times L$ с бесконечными стенками. Произвольное стационарное состояние можно характеризовать двумя квантовыми числами $\ket{n_x, n_y}$, которые нумеруют количество узлов $x$ и $y$ компонент волновой функции; в частности, состояния $\ket{1, 0}$ и $\ket{0, 1}$ имеют одинаковую энергию. Пусть в начальный момент времени волновая функция представляла собой их произвольную линейную комбинацию:
\begin{equation}
    \ket{\psi(0)}=\psi_1\ket{1,0}+\psi_2\ket{0,1}
\end{equation}
Затем ящик адиабатически медленно поворачивается в плоскости на угол $2\pi$ по часовой стрелке, так что после поворота его новое положение совпадает с исходным. Определите волновую функцию $\ket{\psi(T)}$ после такого поворота.\\
\textbf{Решение.}\\
Волновая функция двумерного ящика:
\begin{equation}
    \psi_{n,m}=\frac{2}{L}\sin\left(k_n\left(x+\frac{L}{2}\right)\right)\sin\left(k_m\left(\frac{L}{2}+y\right)\right)
\end{equation}
\begin{equation}
    \ket{1,0}=\frac{2}{L}\sin\left(\frac{2\pi}{L}x\right)\cos\left(\frac{\pi}{L}y\right),\quad\ket{0,1}=\frac{2}{L}\sin\left(\frac{2\pi}{L}y\right)\cos\left(\frac{\pi}{L}x\right)
\end{equation}
Связь между неподвижной и повёрнутой координатами:
\begin{equation}
    \begin{cases}
        x_\theta=x\cos\theta-y\sin\theta,\\
        y_\theta=x\sin\theta+y\cos\theta.
    \end{cases}
\end{equation}
\begin{equation*}
    \ket{1,0}=\frac{2}{L}\sin\left(\frac{2\pi}{L}x_\theta\right)\cos\left(\frac{\pi}{L}y_\theta\right)=\frac{2}{L}\sin\left(\frac{2\pi}{L}(x\cos\theta-y\sin\theta)\right)\cos\left(\frac{\pi}{L}(x\sin\theta+y\cos\theta)\right)
\end{equation*}
\begin{equation*}
    \ket{0,1}=\frac{2}{L}\sin\left(\frac{2\pi}{L}y_\theta\right)\cos\left(\frac{\pi}{L}x_\theta\right)=\frac{2}{L}\sin\left(\frac{2\pi}{L}(x\sin\theta+y\cos\theta)\right)\cos\left(\frac{\pi}{L}(x\cos\theta-y\sin\theta)\right)
\end{equation*}
Уравнение для динамической фазы:
\begin{equation}
    i\partial_\theta c_n(\theta)=-i\sum\limits_m\exp\left(i\int\omega_{nm}(\tau)d\tau\right)\bra{n(\theta)}\partial_\theta\ket{m(\theta)}c_m(\theta),\quad \omega_{nm}=E_n-E_m=0
\end{equation}
\begin{equation}
    \partial_\theta c_n(\theta)=-\sum\limits_m\bra{n(\theta)}\partial_\theta\ket{m(\theta)}c_m(\theta)
\end{equation}
\begin{equation}
    \partial_\theta c_1(\theta)=-\bra{1,0}\partial_\theta\ket{1,0}c_1(\theta)-\bra{1,0}\partial_\theta\ket{0,1}c_2(\theta)
\end{equation}
\begin{equation}
    \partial_\theta c_2(\theta)=-\bra{0,1}\partial_\theta\ket{1,0}c_1(\theta)-\bra{0,1}\partial_\theta\ket{0,1}c_2(\theta)
\end{equation}
\begin{equation}
    \bra{1,0}\partial_\theta\ket{1,0}=\bra{0,1}\partial_\theta\ket{0,1}=0,\quad\bra{1,0}\partial_\theta\ket{0,1}=-\bra{0,1}\partial_\theta\ket{1,0}=\frac{256}{27\pi^2}
\end{equation}
\begin{equation}
    \begin{cases}
        \partial_\theta c_1(\theta)=-\frac{256}{27\pi^2}c_2(\theta),\quad c_1(0)=\psi_1\\
        \partial_\theta c_2(\theta)=\frac{256}{27\pi^2}c_1(\theta),\quad c_2(0)=\psi_2
    \end{cases}
\end{equation}
\begin{equation}
    \begin{cases}
        c_1(\theta)=\psi_1\cos\left(\frac{256\theta}{27\pi^2}\right)-\psi_2\sin\left(\frac{256\theta}{27\pi^2}\right)\\
        c_2(\theta)=\psi_1\sin\left(\frac{256\theta}{27\pi^2}\right)+\psi_2\cos\left(\frac{256\theta}{27\pi^2}\right)
    \end{cases}
\end{equation}
\begin{equation}
    \ket{\psi(T)}=c_1(2\pi)\ket{1,0}+c_2(2\pi)\ket{0,1}
\end{equation}
\begin{equation*}
    \boxed{\ket{\psi(T)}=\left(\psi_1\cos\left(\frac{512\theta}{27\pi}\right)-\psi_2\sin\left(\frac{512\theta}{27\pi}\right)\right)\ket{1,0}+\left(\psi_1\sin\left(\frac{512\theta}{27\pi}\right)+\psi_2\cos\left(\frac{512\theta}{27\pi}\right)\right)\ket{0,1}}
\end{equation*}
\section{Стационарное адиабатическое приближение. Быстрые и медленные подсистемы}
\textbf{Задача 1. Жизнь в промежутке (25 баллов)}\\
Частица массы m находится между двумя бесконечными цилиндрами $|R_1 - R_2| \ll R_{1,2}$, которые касаются друг друга внутренним образом. Найдите низколежащие энергетические уровни такой системы.\\
\textbf{Решение.}\\
Адиабатическое разделение переменных:
\begin{equation}
    \hat{H}(r,\varphi)=\hat{H}_{s}(\varphi)+\hat{H}_{f}(r,\varphi)
\end{equation}
\begin{equation}
    \hat{H}_{s}(\varphi)=-\frac{\hbar^{2}}{2mR_{2}^{2}}\left(\frac{\partial}{\partial\varphi}\right)^{2},\quad\hat{H}_{f}(r,\varphi)=-\frac{\hbar^{2}}{2m}\left(\frac{\partial}{\partial r}\right)^{2}
\end{equation}
Зафиксируем положение медленной подсистемы и решим задачу для быстрой. В первом приближении пренебрегаем производной по $\varphi$:
\begin{equation}
    \frac{\hbar^{2}}{2m}\left(\frac{\partial}{\partial r}\right)^{2}\psi(r,\varphi)=\varepsilon^{(f)}(\varphi)\psi(r,\varphi)
\end{equation}
Обозначим $k(\varphi)=\frac{2m\varepsilon^{(f)}(\varphi)}{\hbar^2}$. Тогда решение с учетом граничного условия при $r=R_2$:
\begin{equation}
    \psi(r,\varphi)=A(\varphi)\sin(k(\varphi)(r-R_{2}))
\end{equation}
Пусть $h(\varphi)$ -- длина отрезка луча в полости между цилиндрами, идушего под углом $\varphi$ ($\varphi$ отсчитываем от плоскости симметрии системы), а $\Delta=R_{1}-R_{2}$. Используя $|R_{1}-R_{2}|\ll R_{1,2}$ и теорему косинусов, получим:
\begin{equation}
    h(\varphi)=(R_{1}-R_{2})(1+\cos\varphi)
\end{equation}
Граничное условия на внешнем цилиндре:
\begin{equation}
    \psi(R_{2}+h(\varphi),\varphi)=0\rightarrow k_{N}(\varphi)=\frac{\pi N}{(1+\cos\varphi)\Delta}
\end{equation}
Соответствующие нормированные волновые функции:
\begin{equation}
    \psi_{N}^{(f)}=\sqrt{\frac{2}{\Delta(1+\cos\varphi)}}\sin\left(\frac{\pi N(r-R_{2})}{(1+\cos\varphi)\Delta}\right)
\end{equation}
Уровни энергии быстрой подсистемы:
\begin{equation}
    \varepsilon_N(\varphi)=\frac{\hbar^{2}\pi^{2}N^{2}}{2m\Delta^{2}(1+\cos\varphi)^{2}}
\end{equation}
\begin{equation}
    -\frac{\hbar^{2}}{2mR_{2}^{2}}\frac{d^{2}\psi_{N}(\varphi)}{d\varphi^{2}}+\frac{\hbar^{2}\pi^{2}N^{2}}{2m\Delta^{2}(1+\cos\varphi)^{2}}\psi_{N}(\varphi)=E_{N}^{(s)}\psi_{N}(\varphi)
\end{equation}
Сведём задачу к гармоническому осциллятору. Разложим по малому углу $\varphi$:
\begin{equation}
    -\frac{\hbar^{2}}{2mR_{2}^{2}}\frac{d^{2}\psi_{N}(\varphi)}{d\varphi^{2}}+\frac{\hbar^{2}\pi^{2}N^{2}}{2m\Delta^{2}(1+\cos\varphi)^{2}}\psi_{N}(\varphi)=E_{N}^{(s)}\psi_{N}(\varphi)
\end{equation}
Уровни энергии:
\begin{equation}
    \boxed{E_{N,n}=\frac{\hbar^2\pi^2N^2}{8m(R_1-R_2)^2}+\frac{\hbar^2\pi N}{2\sqrt{2}mR_1(R_1-R_2)}\left(n+\frac{1}{2}\right)}
\end{equation}
\textbf{Задача 3. Три частицы (20 баллов)}\\
В одномерный ящик ширины $L$ помещены три частицы, две из которых имеют массу $m$, а третья находится между ними и имеет массу $M \gg m$. Частицы взаимодействуют точечным образом так, что они оказываются непроницаемы друг для друга. Определите низколежащие уровни энергии такой системы.\\
\textbf{Решение.}\\
Гамильтониан системы:
\begin{equation}
   \hat{H}=-\frac{1}{2m}\left(\frac{\partial}{\partial x_{1}}\right)^{2}-\frac{1}{2m}\left(\frac{\partial}{\partial x_{2}}\right)^{2}-\frac{1}{2M}\left(\frac{\partial}{\partial X}\right)^{2}+U(x_{1},x_{2},X) 
\end{equation}
где $x_{1,2}$ -- координаты частиц $m$, $X$ -- координата частицы $M$. $U(x_{1},x_{2},X)=U_{1}(x_{1},X)+U_{2}(x_{2},X)$ -- потенциальная энергия взаимодействия частиц, обеспечивающая непроницаемость. Частица $M$ -- медленная подсистема, а частицы $m$ в совокупности -- быстрая.\\
Адиабатическое приближение:
\begin{equation}
    \hat{H}_{f}=-\frac{1}{2m}\left(\frac{\partial}{\partial x_{1}}\right)^{2}-\frac{1}{2m}\left(\frac{\partial}{\partial x_{2}}\right)^{2}+U_{1}(x_{1},X)+U_{2}(x_{2},X),\quad\hat{H}_{s}=-\frac{1}{2M}\left(\frac{\partial}{\partial X}\right)^{2}
\end{equation}
Волновые функции шариков $m$, соответствующие $\hat{H}_f$:
\begin{equation}
    \psi_{1,N_{1}}(x_{1},X)=\sqrt{\frac{2}{X}}\sin\left(\frac{\pi N_{1}}{X}x_{1}\right),\quad \psi_{2,N_{2}}(x_{2},X)=\sqrt{\frac{2}{L-X}}\sin\left(\frac{\pi N_{2}}{L-X}(L-x_{2})\right)
\end{equation}
\begin{equation}
    \psi_{N_{1},N_{2}}(x_{1},x_{2},y)=\frac{2}{\sqrt{y(L-y)}}\sin\left(\frac{\pi N_{1}}{y}x_{1}\right)\cdot\sin\left(\frac{\pi N_{2}}{L-y}(L-x_{2})\right)
\end{equation}
Энергия -- сумма энергий каждого движения:
\begin{equation}
   \varepsilon_{N_{1},N_{2}}^{(f)}=\frac{\pi^{2}}{2m}\left(\frac{N_{1}^{2}}{X^{2}}+\frac{N_{2}^{2}}{(L-X)^{2}}\right)
\end{equation}
Эффективная потенциальная энергия для медленной подсистемы:
\begin{equation}
    \hat{H}_s^{(\text{eff})}(X)=\frac{\hat{p}_X^2}{2M}+\frac{\pi^{2}}{2m}\left(\frac{N_{1}^{2}}{X^{2}}+\frac{N_{2}^{2}}{(L-X)^{2}}\right)
\end{equation}
Минимум энергии находится в точке: $X_{0}=\frac{L}{1+(N_{2}/N_{1})^{2/3}}$.
\begin{equation}
    U_\text{eff}''(X_{0})=\frac{3\pi^{2}}{mL^{4}}\left(N_{1}^{2}\left(1+\left(\frac{N_{2}}{N_{1}}\right)^{2/3}\right)^{4}+N_{2}^{2}\left(1+\left(\frac{N_{1}}{N_{2}}\right)^{2/3}\right)^{4}\right)
\end{equation}
\begin{multline}
    E_{N_{1},N_{2},n}=\frac{\pi^{2}}{2mL^{2}}\left(N_{1}^{2}\left(1+\left(\frac{N_{2}}{N_{1}}\right)^{2/3}\right)^{2}+N_{2}^{2}\left(1+\left(\frac{N_{1}}{N_{2}}\right)^{2/3}\right)^{2}\right)+\\+\sqrt{\frac{3\pi^{2}}{MmL^{4}}\left(N_{1}^{2}\left(1+\left(\frac{N_{2}}{N_{1}}\right)^{2/3}\right)^{4}+N_{2}^{2}\left(1+\left(\frac{N_{1}}{N_{2}}\right)^{2/3}\right)^{4}\right)}\left(n+\frac{1}{2}\right)
\end{multline}
\textbf{Задача 4. Фаза Берри (25 баллов)}\\
Частица со спином 1/2 массы $m$ движется по кольцу радиуса $R$. В центр кольца помещён магнитный монополь, который создаёт на кольце большое магнитное поле $B$. Гамильтониан такой системы имеет вид:
\begin{equation}
    \hat{H}=-\frac{1}{2mR^2}\partial^2_\varphi-\mu_BB(\cos\varphi\hat{\sigma}_x+\sin\varphi\hat{\sigma}_y)
\end{equation}
Определите уровни энергии такой системы в пределе $\mu_BB \gg \frac{1}{mR^2}$.\\
\textbf{Решение.}\\
Гамильтониан:
\begin{equation}
    \hat{H}=-\frac{1}{2mR^{2}}\left(\frac{d}{d\varphi}\right)^{2}-\mu_{0}B\left(\begin{array}{cc}
    0 & e^{-i\varphi}\\
    e^{i\varphi} & 0
\end{array}\right)
\end{equation}
Медленная подсистема -- движение по углу, быстрая -- спин. Адиабатика:
\begin{equation}
    \hat{H}=\hat{H}_{s}+\hat{H}_{f},\quad\hat{H}_{s}=-\frac{1}{2mR^{2}}\left(\frac{d}{d\varphi}\right)^{2},\quad\hat{H}_{f}=-\mu_{0}B\left(\begin{array}{cc}
    0 & e^{-i\varphi}\\
    e^{i\varphi} & 0
\end{array}\right)
\end{equation}
Зафиксируем положение медленной подсистемы и решим стационарное уравнение Шрёдингера для быстрой:
\begin{equation}
    \hat{H}_f\ket{\psi^{(f)}}=\varepsilon^{(f)}\ket{\psi^{(f)}}
\end{equation}
Собственные числа и векторы $\hat{H}_f$:
\begin{equation}
    \varepsilon_{-}^{(f)}=-\mu_{0}B_0,\quad \varepsilon_{+}^{(f)}=+\mu_{0}B_0
\end{equation}
\begin{equation}
    \ket{\psi^{(f)}_-}=\frac{1}{\sqrt{2}}\left(\begin{array}{c}
-e^{-i\varphi}\\
1
\end{array}\right),\quad \ket{\psi^{(f)}_+}=\frac{1}{\sqrt{2}}\left(\begin{array}{c}
e^{-i\varphi}\\
1
\end{array}\right)
\end{equation}
Разложим состояние по базису $\{\ket{\psi^{(f)}_-},\ket{\psi^{(f)}_+}\}$:
\begin{equation}
    \ket{\psi(\varphi)}=f_{+}(\varphi)\ket{\psi^{(f)}_+}+f_{-}(\varphi)\ket{\psi^{(f)}_-}
\end{equation}
\begin{multline}
    \frac{d^2\ket{\psi(\varphi)}}{d\varphi^{2}}=f_{+}''(\varphi)\ket{\psi^{(f)}_+(\varphi)}+2f_{+}'(\varphi)\ket{\psi^{(f)}_+(\varphi)}'+f_{+}(\varphi)\ket{\psi^{(f)}_+(\varphi)}''+\\+f_{-}''(\varphi)\ket{\psi^{(f)}_-(\varphi)}+2f_{-}'(\varphi)\ket{\psi^{(f)}_-(\varphi)}'+f_{-}(\varphi)\ket{\psi^{(f)}_-(\varphi)}''
\end{multline}
Подставим в гамильтониан и получим систему уравнений на $f_{\pm}(\varphi)$:
\begin{equation*}
    \begin{cases}
f_{+}''(\varphi)+if_{+}'(\varphi)-\frac{1}{2}f_{+}(\varphi)-if_-'(\varphi)e^{-2i\varphi}-\frac{f_-(\varphi)}{2}e^{-2i\varphi}+2mR^{2}\mu_{0}Bf_{+}(\varphi)=-2mR^{2}Ef_{+}(\varphi)\\
f_{-}''(\varphi)-if_{-}'(\varphi)-\frac{1}{2}f_{-}(\varphi)+if_{-}'(\varphi)e^{2i\varphi}-\frac{f_{+}(\varphi)}{2}e^{2i\varphi}-2mR^{2}\mu_{0}Bf_{-}(\varphi)=-2mR^{2}Ef_{-}(\varphi)
\end{cases}
\end{equation*}
Ищем решение в виде: 
\begin{equation}
    f_{\pm}(\varphi)=\alpha_{\pm}e^{i\omega_{\pm}\varphi}
\end{equation}
Граничные условия:
\begin{equation}
    \begin{cases}
\ket{\psi(0)}=\ket{\psi(2\pi)}\\
\ket{\psi(0)}'=\ket{\psi(2\pi)}'
\end{cases}
\end{equation}
\begin{equation}
    \begin{cases}
f_\pm(0)=f_\pm(2\pi)\\
f_\pm'(0)=f_\pm'(2\pi)
\end{cases}\rightarrow \omega_\pm=n_\pm\in\mathbb{Z}
\end{equation}
\begin{equation}
    \begin{cases}
\alpha_{+}\left(n_{+}^{2}-n_{+}-\frac{1}{2}+2mR^{2}\mu_{0}B+2mR^{2}E\right)+\alpha_{-}\left(n_{-}-\frac{1}{2}\right)=0\\
\alpha_{-}\left(n_{-}^{2}+n_{-}-\frac{1}{2}-2mR^{2}\mu_{0}B+2mR^{2}E\right)-\alpha_{+}\left(n_{+}+\frac{1}{2}\right)=0
\end{cases}
\end{equation}
\begin{multline}
    \left(\left(n+2\right)^{2}-n-\frac{5}{2}+2mR^{2}\mu_{0}B+2mR^{2}E\right)\left(n^{2}+n-\frac{1}{2}-2mR^{2}\mu_{0}B+2mR^{2}E\right)=\\=-\left(n-\frac{1}{2}\right)\left(n+\frac{5}{2}\right)
\end{multline}
Приближение: $2mR^2\mu_{0}B\gg1$, $n\ll\sqrt{2mR^2\mu_0B}$:
\begin{equation}
    \boxed{E_{n}=\pm\mu_{0}B\mp\frac{\left(n-\frac{1}{2}\right)\left(n+\frac{5}{2}\right)}{4m^2R^4\mu_{0}B}}
\end{equation}
\section{Квазиклассическое приближение}
\textbf{Упражнения (45 баллов)}\\
\textbf{Упражнение 1. Другие задачи сшивки (15 баллов)}\\
Найдите квазиклассические уровни энергии для следующих потенциалов, и сравните полученные ответы с уже известными.
\begin{itemize}
    \item \textbf{(5 баллов)} треугольная яма $U(x) =\begin{cases}
    +\infty, x < 0,\\
    Fx, x > 0.
    \end{cases}$
    \item \textbf{(10 баллов)} трёхмерный кулоновский потенциал $U(r) = -\frac{e^2}{r}$ для сферически симметричных состояний. \textit{Указание:} определите граничное условие в нуле, а в правой точке остановки используйте стандартное.
\end{itemize}
\textit{Указание:} обратите внимание, что в этих задачах в одной из точек остановки линейное приближение для потенциала неприменимо, поэтому условие сшивки необходимо модифицировать. Для второй задачи вам может помочь упражнение 5.2.\\
\textbf{Решение.}
\begin{itemize}
    \item Треугольный потенциал.\\
    Волновая функция в квазиклассике:
    \begin{equation}
        \psi(x)=\frac{1}{\sqrt{p(x)}}\cos\left(\int_{b}^{x}p(y)dy-\frac{\pi}{4}\right)
    \end{equation}
    где $b$ -- правая точка остановки. Из непрерывности волновой функции:
    \begin{equation}
        \psi(0)=0
    \end{equation}
    Получаем условие квантования Бора-Зоммерфельда:
    \begin{equation}
        \int_{a}^{b}p(y)dy+\frac{\pi}{4}=\pi n-\frac{\pi}{2},\quad p(x)=\sqrt{2m(E-Fx)}
    \end{equation}
    \begin{equation}
        \boxed{E_{n}=\left(\frac{3\pi F}{2\sqrt{2m}}\left(n-\frac{3}{4}\right)\right)^{2/3}}
    \end{equation}
    \item Кулоновский потенциал.\\
    Поскольку рассматриваются сферически-симметричные состояния, то $l=m=0$.
\end{itemize}
В упражнении 5.2. мы нашли решение для кулоновского потенциала для $E=0$:
\begin{equation}
    \psi(r,\theta,\phi)=J_{2l+1}\left(2\sqrt{\frac{2M\alpha}{\hbar^{2}}r}\right)\left(\sqrt{\frac{2M\alpha}{\hbar^{2}}r}\right)^{-1}\mathcal{Y}_{l,m}(\theta,\phi)
\end{equation}
\begin{equation}
    -\frac{\hbar^{2}}{2m}u''(r)-\frac{e^{2}}{r}u(r)=Eu(r),\quad\psi_{l,m}(r,\theta,\varphi)=R_{E}(r)\mathcal{Y}_{l,m}(\theta,\varphi),\quad R(r)=\frac{u(r)}{r}
\end{equation}
Применим квазиклассическое приближение к этому уравнению. Волновая функция ограничена, а значит $u(r)\rightarrow0$ при $r\rightarrow0$. Т.е. все решения при малых $r$ эквивалентны ассимтотически решению на нулевой энергии, поскольку совпадают стационарные уравнения Шрёдингера:
\begin{equation}
    \frac{\hbar^{2}}{2m}u''(r)-\frac{e^{2}}{r}u(r)=0
\end{equation}
Поскольку в правой точке нужно использовать стандартное граничное условие, то
\begin{equation}
    u(r)=\frac{1}{\sqrt{p(r)}}\cos\left(\int_{b}^{r}p(r)dr-\frac{\pi}{4}\right),\quad p(r)=\sqrt{2m\left(E+\frac{e^{2}}{r}\right)}
\end{equation}
где $b$ - правая точка остановки. Сшиваем решение для нулевой энергии с квазиклассическим решением, где они оба применимы. 
Ассимптотика функции Бесселя:
\begin{equation}
    J_{\nu}(z)\approx\sqrt{\frac{2}{\pi z}}\cos\left(z-\frac{\pi\nu}{2}-\frac{\pi}{2}\right),\quad z\rightarrow\infty
\end{equation}
\begin{equation}
    u(r)\sim r^{1/4}\cos\left(2\sqrt{\frac{2M\alpha}{\hbar^{2}}r}-\frac{\pi}{4}-\frac{\pi}{2}\right)
\end{equation}
\begin{equation}
    u(r)=\frac{1}{\sqrt{p(r)}}\cos\left(\int_{0}^{r}p(r)dr+\phi\right)
\end{equation}
\begin{equation}
    \int_{0}^{r}p(r)dr=\sqrt{2m}\int_{0}^{r}\sqrt{E+\frac{e^{2}}{r}}dr\approx2\sqrt{2me^{2}}\sqrt{r}
\end{equation}
\begin{equation}
    \phi=-\frac{3\pi}{4}
\end{equation}
Условие квантования:
\begin{equation}
    \int\limits_0^bp(r)dr=\pi\hbar n
\end{equation}
\begin{equation}
    \int\limits_0^bp(r)dr=\sqrt{2me^{2}}\int_{0}^{b}\sqrt{\frac{1}{r}-\frac{1}{b}}dr=\sqrt{2me^{2}}\int_{0}^{\infty}\frac{dq}{(b+q^{2})^{2}}\cdot(-2b^{2})=-\frac{\pi}{2}\sqrt{\frac{2me^2}{b}}
\end{equation}
\begin{equation}
    \sqrt{2me^{2}b_{n}}\frac{\pi}{2}=\pi\hbar n
\end{equation}
\begin{equation}
    \boxed{E_{n}=-\frac{me^{4}}{2\hbar^{2}n^{2}}}
\end{equation}
Получился точный ответ для уровней энергии!\\\\
\textbf{Упражнение 2 (15 баллов)}\\
Рассмотрите $N(E)$ -- количество состояний с энергией, меньше заданной $E$ в поле произвольного адекватного трехмерного сферически-симметричного потенциала $U(r)$ в рамках квазиклассического приближения. Покажите, что следующие два способа вычисления приводят к одному и тому же результату а) одномерное правило квантования Бора-Зоммерфельда
для радиальной переменной (отдельно для каждой сферической гармоники) с последующим суммированием по гармоникам и б) использование интеграла по трехмерному фазовому объему: $N(E) =\int d^3r\int \frac{d^3p}{(2\pi\hbar)^3}\theta(E-E_p)$, где $E_p = \frac{p^2}{2m}$.\\
\textbf{Решение.}
\begin{itemize}
    \item[а)] Уравнение Шредингера для радиальной переменной:
    \begin{equation}
        -\frac{\hbar^{2}}{2M}u''(r)+U'(r) u(r)=Eu(r),\quad u(r)=rR_{E}(r),\quad U'(r)=U(r)+\frac{l(l+1)}{2Mr^{2}}
    \end{equation}
    Одномерное правило квантования Бора-Зоммерфельда:
    \begin{equation}
        \int_{r_\text{min}^{l}}^{r_\text{max}^{l}}\sqrt{2M(E_{n,l,m}-U'(r))}dr=\pi\hbar\left(n+\frac{1}{2}\right)
    \end{equation}
    Обозначим через $N_{E,l}$ -- максимальное $n$, для которого $E_{n,l,m}\leq E$.
    \begin{equation}
        \int_{r_\text{min}^{l}}^{r_\text{max}^{l}}\sqrt{2M(E-U'(r))}dr=\pi\hbar\left(N_{E,l}+\frac{1}{2}\right)
    \end{equation}
    \begin{equation}
        N(E)=\sum\limits_m\sum_{l=0}^{l_\text{max}}N_{E,l}=\sum_{l=0}^{l_\text{max}}(2l+1)N_{E,l}
    \end{equation}
    Заменим сумму по $l$ на интеграл:
    \begin{equation}
        N(E)=\int\limits_{0}^{l_\text{max}}dl(2l+1)\frac{\sqrt{2M}}{\pi\hbar}\int\limits_{r_\text{min}^{l}}^{r_\text{max}^{l}}\sqrt{E-U(r)-\frac{l(l+1)}{2Mr^{2}}}dr
    \end{equation}
    Переставим пределы и проинтегрируем по $r$:
    \begin{equation}
        N(E)=-\frac{\sqrt{2M}}{\pi\hbar}\frac{4M}{3}\int\limits_{r_\text{min}}^{r_\text{max}}drr^{2}\left(E-U(r)-\frac{l(l+1)}{2Mr^{2}}\right)^{3/2}\Bigg|_{l=0}^{l=l_\text{max}}
    \end{equation}
    Верхним пределом можно пренебречь:
    \begin{equation}
        \boxed{N(E)=\frac{2(2M)^{3/2}}{3\pi\hbar}\int_{r_\text{min}}^{r_\text{max}}r^{2}(E-U(r))^{3/2}dr}
    \end{equation}
    \item[б)] Интеграл по трёхмерному фазовому объёму:
    \begin{equation}
        N(E)=\int d^3r\int \frac{d^3p}{(2\pi\hbar)^3}\theta(E-E_p)=\int d^3r\int\limits_0^{p_\text{max}}\frac{dp4\pi p^2}{(2\pi\hbar)^3}
    \end{equation}
    \begin{equation}
        p_\text{max}=\sqrt{2m(E-U(\vec{r}))}
    \end{equation}
    \begin{equation}
        N(E)=\frac{4\pi(2m)^{3/2}}{3(2\pi\hbar)^3}\int d^3r(E-U(r))^{3/2}
    \end{equation}
    \begin{equation}
        \boxed{N(E)=\frac{2(2m)^{3/2}}{3\pi\hbar}\int\limits_{r_\text{min}}^{r_\text{max}}drr^2(E-U(r))^{3/2}}
    \end{equation}
\end{itemize}
\textbf{Упражнение 3 (10 баллов)}\\
Найдите вероятность квазиклассического туннелирования под барьером $U(x) = \frac{U_0}{\cosh^2(x/a)}$ для частиц энергии $0 < E < U_0$. Сравните ответ с точным, найденным в 5 семинаре.\\
\textbf{Решение.}\\
Общий вид волновой функции в квазиклассическом приближении:
\begin{equation}
    \psi(x)=c_+\psi_+(x)+c_-\psi_-(x),\quad\psi_\pm(x)=\frac{e^{\pm i\int\limits_{x_0}^xp(x)dx}}{\sqrt{p(x)}}
\end{equation}
Задача рассеяния:
\begin{equation}
    \psi(x)=
    \begin{cases}
    \frac{1}{\sqrt{p(x)}}e^{i\int\limits_{0}^{x}p(z)dz}+\frac{r}{\sqrt{p(x)}}e^{-i\int\limits_0^xp(z)dz}, & x<0\\
    \frac{t}{\sqrt{p(x)}}e^{i\int\limits_{0}^{x}p(z)dz}, & x>0
    \end{cases}
\end{equation}
Это решение переписывается в следующем виде:
\begin{equation}
    \psi(x)=
    \begin{cases}
    \psi_+(x)+r\psi_-(x), & x<0\\
    t\psi_+(x), & x>0
    \end{cases}
\end{equation}
Связь на коэффициенты:
\begin{equation}
    \begin{pmatrix}
    c^+_\text{left}\\
    c^-_\text{left}
    \end{pmatrix}=\hat{A}\hat{S}\hat{A}^\dagger
    \begin{pmatrix}
    c^+_\text{right}\\
    c^-_\text{right}
    \end{pmatrix},\quad\hat{A}=\begin{pmatrix}
    \frac{e^{-\frac{i\pi}{4}}}{2} & e^{\frac{i\pi}{4}}\\
    \frac{e^{\frac{i\pi}{4}}}{2} & e^{-\frac{i\pi}{4}}
    \end{pmatrix},\quad\hat{S}=\begin{pmatrix}
    e^{-S} & 0\\
    0 & e^{S}
    \end{pmatrix}
\end{equation}
\begin{equation}
    \hat{A}\hat{S}\hat{A}^\dagger=\begin{pmatrix}
    \frac{1}{4}e^{-S}+\frac{1}{2}e^S & -\frac{i}{2}e^{-S}+ie^S\\
    \frac{i}{4}e^{-S}-\frac{i}{2}e^S & \frac{1}{2}e^{-S}+e^S
    \end{pmatrix}
\end{equation}
\begin{equation}
    \begin{pmatrix}
    c^+_\text{right}\\
    c^-_\text{right}
    \end{pmatrix}=\begin{pmatrix}
    0\\
    t
    \end{pmatrix}\rightarrow\begin{pmatrix}
    c^+_\text{left}\\
    c^-_\text{left}
    \end{pmatrix}=\begin{pmatrix}
    1\\
    r
    \end{pmatrix}=t\begin{pmatrix}
    -\frac{i}{2}e^{-S}+ie^S\\
    \frac{1}{2}e^{-S}+e^S
    \end{pmatrix}
\end{equation}
\begin{equation}
    t=\frac{i}{\frac{e^{-S}}{2}-e^S},\quad r=i\frac{e^{-S}+2e^S}{e^{-S}-2e^S}
\end{equation}
\begin{equation}
    T=(\frac{e^{-S}}{2}-e^S)^{-2}
\end{equation}
Точки остановки:
\begin{equation}
    \frac{U_0}{\cosh^2(x_{1,2}/a)}=E\rightarrow x_{1,2}=a\;\text{arccosh}\left(\pm\sqrt{\frac{U_0}{E}}\right)
\end{equation}
\begin{equation}
    S=\int\limits_{x_1}^{x_2}\sqrt{2m(U(x)-E)}dx=\sqrt{2m}\pi a(\sqrt{U_0}+\sqrt{E})
\end{equation}
\begin{equation}
    \boxed{T=\left(\frac{e^{-\sqrt{2m}\pi a(\sqrt{U_0}+\sqrt{E})}}{2}-e^{\sqrt{2m}\pi a(\sqrt{U_0}+\sqrt{E})}\right)^{-2}}
\end{equation}
\textbf{Упражнение 4 (5 баллов)}
\begin{itemize}
    \item Отнормируйте волновые функции связанного состояния, которые следуют из правила квантования Бора-Зоммерфельда, на единицу.
    \item Отнормируйте квазиклассические бегущие волны (в отсутствие отражения) на $2\pi\delta(E-E_0)$
\end{itemize}
\textbf{Решение.}
\begin{itemize}
    \item Волновые функции для частицы в потенциальной яме:
    \begin{equation}
        \psi(x)=\frac{C}{\sqrt{p(x)}}\cos\left(\frac{1}{\hbar}\int_{a}^{x}p(y)dy-\frac{\pi}{4}\right)
    \end{equation}
    Условие нормировки:
    \begin{equation}
        \int\limits_a^b|\psi(x)|^{2}dx=\int\limits_a^b\frac{C^{2}}{p(x)}\cos^{2}\left(\frac{1}{\hbar}\int\limits_a^xp(y)dy-\frac{\pi}{4}\right)dx=1
    \end{equation}
    Импульс меняется медленно, а косинус между точками остановки успевает совершить достаточно много колебаний. Поэтому, заменим $\cos^2$ на среднее значение за период. 
    \begin{equation}
        \frac{C^{2}}{2}\int_{a}^{b}\frac{dx}{p(x)}=1
    \end{equation}
    Стоящий здесь интеграл выражается через период колебаний классической частицы в этом же потенциале: 
    \begin{equation}
        T=2\int\limits_a^b\frac{dx}{v(x)}=2m\int\limits_a^b\frac{dx}{p(x)}dx\rightarrow C=\sqrt{\frac{4m}{T}}
    \end{equation}
    \begin{equation}
        \boxed{\psi(x)=\sqrt{\frac{4m}{Tp(x)}}\cos\left(\frac{1}{\hbar}\int_{a}^{x}p(y)dy-\frac{\pi}{4}\right)}
    \end{equation}
    \item Квазиклассические бегущие волны:
    \begin{equation}
        \psi_{\pm}(x)=\frac{C}{\sqrt{p(x)}}\exp\left(\pm\frac{i}{\hbar}\int_{0}^{x}p(y)dy\right)
    \end{equation}
    \begin{equation}
        \braket{\psi_{+}^{E}|\psi_{+}^{E'}}= C^{2}\int_{-\infty}^{\infty}\frac{e^{\frac{i}{\hbar}\int_{0}^{x}(p^{E+\delta E}(y)-p^{E}(y))dy}}{p^{E}(x)}dx
    \end{equation}
    \begin{equation}
        p^{E+\delta E}(y)-p^{E}(y)\approx\frac{\partial p^{E}(y)}{\partial E}\delta E=\frac{m}{p(y)}\delta E
    \end{equation}
    Пусть $\xi=\frac{m}{\hbar}\int\limits_{0}^{x}\frac{dy}{p(y)}$, тогда
    \begin{equation}
        \braket{\psi_{+}^{E}|\psi_{+}^{E'}}=C^{2}\frac{m}{\hbar}\int_{-\infty}^{\infty}e^{i\xi\delta E}d\xi=C^{2}\frac{m}{\hbar}2\pi\delta(E'-E)
    \end{equation}
    Для перекрекстных скалярных произведений:
    \begin{equation}
        \braket{\psi_{+}^{E}|\psi_{-}^{E'}}= C^{2}\frac{m}{\hbar}\int_{-\infty}^{\infty}e^{i\xi(E+E')}d\xi=C^{2}\frac{m}{\hbar}2\pi\delta(E'+E)\approx0
    \end{equation}
    Таким образом, нормированные на дельта-функцию квазиклассические бегущие волны:
    \begin{equation}
        \boxed{\psi_{\pm}(x)=\sqrt{\frac{\hbar}{2\pi mp(x)}}\exp\left(\pm\frac{i}{\hbar}\int_{0}^{x}p(y)dy\right)}
    \end{equation}
\end{itemize}
\textbf{Задача 2. Одномерный кристалл и модель сильной связи (35 баллов)}\\
Определите зоны и их законы дисперсии для периодической задачи, изображённой на рисунке, считая применимым квазиклассическое приближение. Проинтерпретируйте результат в терминах модели сильной связи и следующего гамильтониана (для каждой зоны). Определите $E_n$ и $t_n$.
\begin{equation}
    \hat{H}_n=\begin{pmatrix}
    \ddots & -t_n & 0 & 0 & \hdots\\
    -t_n & E_n & -t_n & 0 & 0\\
    0 & -t_n & E_n & -t_n & 0\\
    0 & 0 & -t_n & E_n & -t_n\\
    \vdots & 0 & 0 & -t_n & \ddots
    \end{pmatrix}
\end{equation}
\textbf{Решение.}\\
Рассмотрим кристалл из очень большого, но все же конечного количества атомов $N\gg1$.
\begin{equation}
    \hat{Q}=\hat{A}^{\dagger}\left(\begin{array}{cc}
    e^{-iS} & 0\\
    0 & e^{iS}
    \end{array}\right)\hat{A}\left(\begin{array}{cc}
    e^{-S_T} & 0\\
    0 & e^{S_T}
    \end{array}\right),\quad\hat{A}=
    \left(\begin{array}{cc}
        \frac{e^{-i\pi/4}}{2} & e^{i\pi/4}\\
        \frac{e^{i\pi/4}}{2} & e^{-i\pi/4}
    \end{array}\right)
\end{equation}
На границах кристалла потенциал имеет бесконечные стенки, то есть, что частица не может вылететь из кристалла. Тогда, если в кристалле $N$ атомов, то столбцы коэффициентов разложения по квазиклассическим функциям слева и справа имеет вид:
\begin{equation}
    \vec{c}_\text{left}=\hat{Q}^{N-1}\hat{S}\hat{A}\vec{c}_\text{right}
\end{equation}
Мы рассматриваем состояния частицы, находящейся в кристалле, поэтому выбираем
\begin{equation}
    \vec{c}_\text{left}=\left(\begin{array}{c}
1\\
0
\end{array}\right),\qquad\vec{c}_\text{right}=\left(\begin{array}{c}
0\\
t
\end{array}\right)
\end{equation}
Характеристический многочлен оператора $\hat{Q}$:
\begin{equation}
    \chi_{\hat{Q}}(z)=z^{2}-\cos S\left(2e^{S_T}+\frac{1}{2}e^{-S_T}\right)z+1
\end{equation}
Произведение собственных чисел $\hat{Q}$ равно 1. Вектор $\vec{c}_\text{right}$ раскладывается по обоим собственным векторам, и поэтому если $|\lambda_{1}|\neq|\lambda_{2}|$, то существует собственное значение по модулю, большее 1. Но тогда при возведении в очень большую степень норма образа вектора $\vec{c}_\text{right}$ будет тожет расти неограниченно. Таким образом,
\begin{equation}
    |\lambda_1|=|\lambda_2|=1\rightarrow\lambda_1=\lambda=e^{i\varphi}, \lambda_2=e^{-i\varphi}
\end{equation}
\begin{equation}
    \hat{Q}^{N-1}=\Lambda\left(\begin{array}{cc}
e^{i\varphi(N-1)} & 0\\
0 & e^{-i\varphi(N-1)}
\end{array}\right)\Lambda^{-1}
\end{equation}
где $\Lambda$ -- матрица перехода к собственным векторам $\hat{Q}$:
\begin{equation}
    \Lambda=\left(\begin{array}{cc}
\sin S & \sin S\\
e^{S_{T}}e^{i\varphi}-\frac{1}{2}\cos S & e^{S_{T}}e^{-i\varphi}-\frac{1}{2}\cos S
\end{array}\right)
\end{equation}
\begin{equation}
    e^{2i(N-1)\varphi}=1\rightarrow\varphi_{k}=\frac{\pi k}{N-1},\qquad k=0,...,2N-3
\end{equation}
По теореме Виета:
\begin{equation*}
    \cos\varphi_{k}=\cos S e^{S_T}\left(1+\frac{1}{4}e^{-2S_T}\right)\rightarrow\cos S_{n,k}=e^{-S_T}\cos\varphi_{k}
\end{equation*}
При помощи метода последовательных приближений получаем:
\begin{equation}
    \boxed{E_{n,k}^{(1)}=E_{n}^{(0)}+(-1)^{n-1}\frac{\hbar\omega}{\pi}\cos\left(\frac{\pi k}{N-1}\right) e^{-\frac{1}{\hbar}\int|p(x)|dx},\quad k=0,...,N-1}
\end{equation}
Рассмотрим модель сильной связи. Система функций:
\begin{equation}
\psi_{n,j}(x)=\begin{cases}
\frac{C_{n,j}}{\sqrt{|p(x)|}}e^{-\frac{1}{\hbar}\int_{x_2}^{x}p(x)dx}, & x>x_2\\
\frac{C_{n,j}}{\sqrt{p(x)}}\cos\left(\frac{1}{\hbar}\int_{x_1}^{x_2}p(x)dx-\frac{\pi}{4}\right), & x_1<x<x_2.\\
\frac{C_{n,j}}{\sqrt{|p(x)|}}e^{\frac{1}{\hbar}\int_{x_1}^{x}p(x)dx} & x>x_1
\end{cases}    
\end{equation}
где $x_1$ -- левая точка остановки, $x_2$ -- правая.\\
Нормировочные константы связанных состояний:
\begin{equation}
    C_{n,j}=\sqrt{\frac{4\pi}{T_{n}}}
\end{equation}
где $T_n$ -- классический период колебаний частицы с данной энергией в рассматриваемом потенциале. Тогда
\begin{equation}
    \boxed{t_n=\braket{\psi_{n,j+1}|\psi_{n,j}}=\frac{4\pi}{T_n}\int_{-z_0}^{z_0}\frac{dz}{|p(z)|}e^{-\int_{-z_0}^{z_0}p(z)dz},\quad U(z_{0})=E_{n}^{(0)}}
\end{equation}
\section{Туннельные эффекты. Надбарьерное отражение}
\textbf{Задача 1. Интерференция в комплексной плоскости (40 баллов)}\\
Найдите с экспоненциальной точностью в рамках квазиклассического приближения вероятность надбарьерного отражения частицы большой энергии $E$ в потенциале $U(x) = -U_0\frac{x^4}{a^4}$.
\begin{enumerate}
    \item \textbf{(10 баллов)} Идентифицируйте квазиклассические точки остановки; нарисуйте (схематично) линии Стокса и анти-Стокса, соответствующие этой задаче.
    \item \textbf{(20 баллов)} Выберите любую из интересующих нас линий анти-Стокса, вдоль которых необходимо исследовать решение. Определите условия сшивки амплитуд квазиклассических решений в окрестности необходимых точек остановки. Определите связь амплитуды прохождения и отражения в соответствующей задаче рассеяния.
    \item \textbf{(10 баллов)} Вычислите необходимые квазиклассические интегралы, найдите ответ.
\end{enumerate}
\textbf{Решение.}
\begin{enumerate}
    \item Произвольное решение раскладывается по паре линейно независимых:
    \begin{equation}
        \psi(z)=C_{+}\psi_{+}(z)+C_{-}\psi_{-}(z)
    \end{equation}
    \begin{equation}
        \psi_{\pm}(z)=\frac{1}{\sqrt{p(z)}}e^{\pm iS(z)},\quad S(z)=\int p(z)dz
    \end{equation}
    Идентифицируем точки остановки:
    \begin{equation}
        p(z_k)=\sqrt{2m\left(E+U_0\frac{z_k^4}{a^4}\right)}=0
    \end{equation}
    \begin{equation}
        \boxed{z_k=a\sqrt[4]{\frac{E}{U_0}}\exp\left(\frac{i\pi}{4}+\frac{i\pi k}{2}\right),\quad k\in\{0,1,2,3\}}
    \end{equation}
    \begin{equation}
        S(z)=\int\limits_{z_k}^z\sqrt{2m\left(E+U_0\frac{z^4}{a^4}\right)}dz
    \end{equation}
    Линии Стокса и анти-Стокса задаются соответственно уравнениями:
    \begin{equation}
    \begin{cases}
        \text{Re}S(z)=0,\\
        \text{Im}S(z)=0.\\
    \end{cases}
    \end{equation}
    Линии изображены на рис. \ref{gr2}.
    \begin{figure}
        \centering
        \includegraphics[scale=0.65]{gr2.png}
        \caption{Линии Стокса (синие) и анти-Стокса (жёлтые)}
        \label{gr2}
    \end{figure}
    \item Задача рассеяния:
    \begin{equation}
        \psi(x)=
        \begin{cases}
        \frac{1}{\sqrt{p(x)}}e^{i\int\limits_{0}^{x}p(z)dz}+\frac{r}{\sqrt{p(x)}}e^{-i\int\limits_{0}^{x}p(z)dz}, & x<0\\
        \frac{t}{\sqrt{p(x)}}e^{i\int\limits_{0}^{x}p(z)dz}, & x>0
        \end{cases}
    \end{equation}
    Это решение переписывается в следующем виде:
    \begin{equation}
        \psi(x)=
        \begin{cases}
        e^{i\int\limits_{0}^{x}p(z)dz}\psi_+(x)+re^{-i\int\limits_{0}^{x}p(z)dz}\psi_-(x), & \text{Re}\;x<0\\
        te^{i\int\limits_{0}^{x}p(z)dz}\psi_+(x), & \text{Re}\;x>0
        \end{cases}
    \end{equation}
    Для нахождения $t$ выберем контур из линий анти-Стокса, приближающихся к действительной оси на бесконечности. При обходе точек остановки дважды пересекается линия Стокса.\\
    Решение между линиями Стокса:
    \begin{equation}
        \psi(z)=\frac{C'_+}{\sqrt{p(z)}}e^{i\int\limits_{z_0}^zp(z')dz'}
    \end{equation}
    Связь между $t$ и $C'_+$:
    \begin{equation}
        C'_+=te^{i\int\limits_0^{z_0}p(z')dz'}
    \end{equation}
    Решение после перехода через 2 линию Стокса:
    \begin{equation}
        \psi(z)=\frac{C''_+}{\sqrt{p(z)}}e^{i\int\limits_{z_1}^zp(z')dz'}
    \end{equation}
    Связь между $t$ и $C''_+$:
    \begin{equation}
        C''_+=C'_+e^{i\int\limits_{z_0}^{z_1}p(z')dz'}=te^{i\int\limits_0^{z_0}p(z')dz'}e^{i\int\limits_{z_0}^{z_1}p(z')dz'}
    \end{equation}
    Условие сшивки:
    \begin{equation}
        C''_+e^{i\int\limits_{z_1}^0p(z')dz'}=1
    \end{equation}
    \begin{equation}
        te^{i\int\limits_0^{z_0}p(z')dz'}e^{i\int\limits_{z_0}^{z_1}p(z')dz'}e^{i\int\limits_{z_1}^0p(z')dz'}=1
    \end{equation}
    \begin{equation}
        \boxed{t=e^{-i\left(\int\limits_0^{z_0}p(z')dz'+\int\limits_{z_0}^{z_1}p(z')dz'+\int\limits_{z_1}^0p(z')dz'\right)}}
    \end{equation}
    Для нахождения $r$ выберем разрез вдоль линии анти-Стокса. Снизу обходим разрез около точки $z_1=e^{i\frac{\pi}{4}}$ ($\text{arg}_{z=z_1}z_1=\frac{13}{12}\pi$) и получим
    \begin{multline}
        C_{-}^{верх}e^{-\frac{i}{4}\frac{13}{12}\pi}e^{-i|\xi|^{3/2}|z_{0}|^{3/2}\frac{\sqrt{2mU_{0}}}{3a^{2}}}+C_{+}^{верх}e^{-\frac{i}{4}\frac{13}{12}\pi}e^{i|\xi|^{3/2}|z_{0}|^{3/2}\frac{\sqrt{2mU_{0}}}{3a^{2}}}=\\=C_{-}^{низ}e^{\frac{i}{4}\frac{11}{12}\pi}e^{i|\xi|^{3/2}|z_{0}|^{3/2}\frac{\sqrt{2mU_{0}}}{3a^{2}}}+C_{+}^{низ}e^{\frac{i}{4}\frac{11}{12}\pi}e^{-i|\xi|^{3/2}|z_{0}|^{3/2}\frac{\sqrt{2mU_{0}}}{3a^{2}}}
    \end{multline}
    Нижний берег разреза:
    \begin{equation}
        \psi=t\psi_{+}\rightarrow C_{-}^{верх}=it,\quad C_{+}^{верх}=0
    \end{equation}
    По верхнему берегу разреза вдоль линии анти-Стокса переходим к $z_1=e^{i\frac{3\pi}{4}}$, и обходим сверху. $C_{+}$ остается 0, при пересечении первой линии Стокса из постоянства $C_{+}$ следует также сохранение $C_{-}$. Вторая линия анти-Стокса с $\psi_-$-доминированием. Пользуемся условие сшивки на действительной оси и получаем
    \begin{equation}
        \boxed{r=ite^{-i\sqrt{2mU_{0}}\left(\int\limits_{z_1}^{0}+\int\limits_{z_0+0}^{z_1+0}+\int\limits_{0}^{z_0}\right)\sqrt{\alpha+\frac{z^{4}}{a^{4}}}dz}}
    \end{equation}
    \item Коэффициент прохождения:
    \begin{equation}
        \boxed{t=1}
    \end{equation}
    \begin{equation}
        \sqrt{\alpha+\frac{z^{4}}{a^{4}}}_{верх}=e^{\frac{i}{2}\cdot2\pi}\sqrt{\alpha+\frac{z^{4}}{a^{4}}}_{низ}
    \end{equation}
    \begin{equation}
        \left(\int_{z_1-0}^{z_0-0}+\int_{z_0+0}^{z_1+0}\right)\sqrt{\alpha+\frac{z^{4}}{a^{4}}}dz=2\int_{z_1-0}^{z_0-0}\sqrt{\alpha+\frac{z^{4}}{a^{4}}}dz
    \end{equation}
    Из голоморфности функций:
    \begin{equation}
        \int_{z_1-0}^{z_0-0}\sqrt{\alpha+\frac{z^{4}}{a^{4}}}dz=\int_{z_1}^{0}\sqrt{\alpha+\frac{z^{4}}{a^{4}}}dz+\int_{0}^{z_0}\sqrt{\alpha+\frac{z^{4}}{a^{4}}}dz
    \end{equation}
    \begin{multline}
        \left(\int_{z_1-0}^{z_0-0}+\int_{z_0+0}^{z_1+0}\right)\sqrt{\alpha+\frac{z^{4}}{a^{4}}}dz=-2i\cos\frac{\pi}{4} a\alpha^{3/4}\int_{0}^{1}\sqrt{1-t^{4}}dt=\\=-2i\cos\frac{\pi}{4} a\alpha^{3/4}\frac{\Gamma(1/4)^{2}}{6\sqrt{2\pi}}
    \end{multline}
    \begin{equation}
        \boxed{r=ie^{-2a\sqrt{mE\sqrt{\frac{E}{U_{0}}}}\frac{\Gamma(1/4)^{2}}{6\sqrt{2\pi}}}}
    \end{equation}
    \end{enumerate}
\end{document}

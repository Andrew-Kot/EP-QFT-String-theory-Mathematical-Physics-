\documentclass[12pt]{article}

% report, book
%  Русский язык

\usepackage{hyperref,bookmark}
\usepackage[warn]{mathtext} %русский язык в формулах
\usepackage[T2A]{fontenc}			% кодировка
\usepackage[utf8]{inputenc}			% кодировка исходного текста
\usepackage[english,russian]{babel}	% локализация и переносы
\usepackage[title,toc,page,header]{appendix}
\usepackage{amsfonts}


% Математика
\usepackage{amsmath,amsfonts,amssymb,amsthm,mathtools} 
%%% Дополнительная работа с математикой
%\usepackage{amsmath,amsfonts,amssymb,amsthm,mathtools} % AMS
%\usepackage{icomma} % "Умная" запятая: $0,2$ --- число, $0, 2$ --- перечисление

\usepackage{cancel}%зачёркивание
\usepackage{braket}
%% Шрифты
\usepackage{euscript}	 % Шрифт Евклид
\usepackage{mathrsfs} % Красивый матшрифт


\usepackage[left=2cm,right=2cm,top=1cm,bottom=2cm,bindingoffset=0cm]{geometry}
\usepackage{wasysym}

%размеры
\renewcommand{\appendixtocname}{Приложения}
\renewcommand{\appendixpagename}{Приложения}
\renewcommand{\appendixname}{Приложение}
\makeatletter
\let\oriAlph\Alph
\let\orialph\alph
\renewcommand{\@resets@pp}{\par
  \@ppsavesec
  \stepcounter{@pps}
  \setcounter{subsection}{0}%
  \if@chapter@pp
    \setcounter{chapter}{0}%
    \renewcommand\@chapapp{\appendixname}%
    \renewcommand\thechapter{\@Alph\c@chapter}%
  \else
    \setcounter{subsubsection}{0}%
    \renewcommand\thesubsection{\@Alph\c@subsection}%
  \fi
  \if@pphyper
    \if@chapter@pp
      \renewcommand{\theHchapter}{\theH@pps.\oriAlph{chapter}}%
    \else
      \renewcommand{\theHsubsection}{\theH@pps.\oriAlph{subsection}}%
    \fi
    \def\Hy@chapapp{appendix}%
  \fi
  \restoreapp
}
\makeatother
\newtheorem{resh}{Решение}
\newtheorem{theorem}{Теорема}
\newtheorem{predl}[theorem]{Предложение}
\newtheorem{sled}[theorem]{Следствие}

\theoremstyle{definition}
\newtheorem{zad}{Задача}[section]
\newtheorem{upr}[zad]{Упражнение}
\newtheorem{defin}[theorem]{Определение}

\title{Решение заданий\\ ОП "Квантовая теория поля, теория струн и математическая физика"\\[2cm]
Семинары по квантовой механике -- II\\ (И.В. Побойко, Н.А. Степанов)}
\author{Коцевич Андрей Витальевич, группа Б02-920}
\date{6 семестр, 2022}

\begin{document}
\maketitle
\newpage
\tableofcontents{}
\newpage
\section{Задача рассеяния, функции Грина и формула Борна.}
\subsection*{Упражнения (30 баллов)}
\textbf{Упражнение 1. Борновское приближение (20 баллов)}\\
В рамках Борновского приближения, рассмотрите рассеяние на следующих потенциалах:
\begin{enumerate}
    \item \textbf{(10 баллов)} $V(\textbf{r}) = V_0\frac{a^n}{r^n+a^n}$, $n > 3$, случай медленных частиц $ka\ll1$. Рассмотрите также предел $n \rightarrow \infty$, когда потенциал превращается в сферическую прямоугольную яму.
    \item \textbf{(10 баллов)} $V(\textbf{r}) = \frac{\alpha}{r}e^{-\kappa r}$ (потенциал Юкавы). Рассмотрите также предельный переход $\kappa\rightarrow 0$ (закон Кулона).
\end{enumerate}
\textbf{Решение.}
Амплитуда рассеяния в борновском приближении:
\begin{equation}
    f(\textbf{n},\textbf{n}')=-\frac{m}{2\pi\hbar^2}\tilde{V}_{\textbf{k}-\textbf{k}'}
\end{equation}
В сферически-симметричном случае:
\begin{equation}
    f(\theta)=-\frac{m}{2\pi\hbar^2}\int d^3\textbf{r}\;V(\textbf{r})e^{-i(\textbf{k}-\textbf{k}')\textbf{r}}=-\frac{m}{\hbar^2}\int\limits_0^\infty drr^2V(r)\int\limits_0^\pi e^{-i|\textbf{k}-\textbf{k}'|r\cos\varphi}\sin\varphi d\varphi
\end{equation}
\begin{equation}
    f(\theta)=-\frac{2m}{\hbar^2|\textbf{k}-\textbf{k}'|}\int\limits_0^\infty drr\sin(|\textbf{k}-\textbf{k}'|r)V(r)=-\frac{2m}{2k\hbar^2\sin\frac{\theta}{2}}\int\limits_0^\infty drr\sin\left(2kr\sin\frac{\theta}{2}\right)V(r)
\end{equation}
\begin{enumerate}
\item $V(\textbf{r}) = V_0\frac{a^n}{r^n+a^n}$.
\begin{equation}
    f(\theta)=-\frac{2ma^nV_0}{2k\hbar^2\sin\frac{\theta}{2}}\int\limits_0^\infty dr\frac{r\sin\left(2kr\sin\frac{\theta}{2}\right)}{r^n+a^n}\approx-\frac{2ma^nV_0}{2k\hbar^2\sin\frac{\theta}{2}}\int\limits_0^\infty dr\frac{2kr^2\sin\frac{\theta}{2}}{r^n+a^n}=-\frac{2\pi ma^3V_0}{\hbar^2n\sin\frac{3\pi}{n}}
\end{equation}
\begin{equation}
    \boxed{\sigma=2\pi\int\limits_0^\pi|f(\theta)|^2\sin\theta d\theta=\frac{16\pi^3m^2a^6V^2_0}{\hbar^4n^2\sin^2\frac{3\pi}{n}}}
\end{equation}
При $n\rightarrow\infty$ (сферическая прямоугольная яма):
\begin{equation}
    f(\theta)=-\frac{2ma^3V_0}{3\hbar^2}\rightarrow\boxed{\sigma=\frac{16\pi m^2a^6V^2_0}{9\hbar^4}}
\end{equation}
\item Потенциал Юкавы: $V(\textbf{r}) = \frac{\alpha}{r}e^{-\kappa r}$.
\begin{equation}
    f(\theta)=-\frac{2m\alpha}{2k\hbar^2\sin\frac{\theta}{2}}\int\limits_0^\infty dr\sin\left(2kr\sin\frac{\theta}{2}\right)e^{-\kappa r}=-\frac{2m\alpha}{\hbar^2(4k^2\sin^2\frac{\theta}{2}+\kappa^2)}
\end{equation}
\begin{equation}
    \boxed{\sigma=2\pi\int\limits_0^\pi|f(\theta)|^2\sin\theta d\theta=\frac{8\pi m^2\alpha^2}{\hbar^4}\int\limits_0^\pi\frac{\sin\theta d\theta}{(4k^2\sin^2\frac{\theta}{2}+\kappa^2)^2}=\frac{16\pi m^2\alpha^2}{\hbar^4\kappa^2(4k^2+\kappa^2)}}
\end{equation}
Кулоновский потенциал: $\kappa\rightarrow0$:
\begin{equation}
    f(\theta)=-\frac{m\alpha}{2\hbar^2k^2\sin^2\frac{\theta}{2}}\rightarrow\boxed{\sigma=\frac{\pi m^2\alpha^2}{2\hbar^4k^4}\int\limits_0^\pi\frac{\sin\theta d\theta}{\sin^4\frac{\theta}{2}}\rightarrow\infty}
\end{equation}
\end{enumerate}
\textbf{Упражнение 2. Соотношение унитарности (10 баллов)}\\
Исходя из соотношения унитарности на языке $T$-матрицы, полученного на семинаре
\begin{equation}
    \hat{T}_E-\hat{T}^\dagger_E = -2\pi i\hat{T}^\dagger_E\delta(E - \hat{H}_0)\hat{T}_E
\end{equation}
получите соотношение унитарности на языке амплитуды рассеяния (результат также был предъявлен на семинаре):
\begin{equation}
    f(\textbf{n}, \textbf{n}') - f^*(\textbf{n}',\textbf{n}) = i\frac{k}{2\pi}\int d\textbf{n}''f(\textbf{n},\textbf{n}'')f^*(\textbf{n}', \textbf{n}'')
\end{equation}
\textbf{Решение.}\\
Связь $T$-матрицы и амплитуды рассеяния:
\begin{equation}
    f(\textbf{n},\textbf{n}')=-\frac{m}{2\pi}\braket{\textbf{k}'|\hat{T}_E|\textbf{k}}
\end{equation}
\begin{multline}
    f(\textbf{n},\textbf{n}')-f^*(\textbf{n}',\textbf{n})=-\frac{m}{2\pi}(\braket{\textbf{k}'|\hat{T}_E|\textbf{k}}-\braket{\textbf{k}'|\hat{T}^\dagger_E|\textbf{k}})=-\frac{m}{2\pi}\braket{\textbf{k}'|(\hat{T}_E-\hat{T}^\dagger_E)|\textbf{k}}=\\=im\braket{\textbf{k}'|\hat{T}^\dagger_E\delta(E-\hat{H}_0)\hat{T}_E|\textbf{k}}
\end{multline}
Вставим единицу:
\begin{equation}
    \mathbb{I}=\int\frac{d^3\textbf{q}}{(2\pi)^3}\ket{\textbf{q}}\bra{\textbf{q}}
\end{equation}
\begin{equation}
    f(\textbf{n},\textbf{n}')-f^*(\textbf{n}',\textbf{n})=im\int\frac{d^3\textbf{q}}{(2\pi)^3}\braket{\textbf{k}'|\hat{T}^\dagger_E|\textbf{q}}\braket{\textbf{q}|\hat{T}_E|\textbf{k}}\delta(E-E_{\textbf{q}})
\end{equation}
\begin{equation}
    E=\frac{k^2}{2m}\rightarrow\delta(E-E_{\textbf{q}})=\frac{m}{k}\delta(k-|\textbf{q}|)
\end{equation}
\begin{equation}
    f(\textbf{n},\textbf{n}')-f^*(\textbf{n}',\textbf{n})=im\frac{4\pi^2}{(2\pi)^3m^2}\frac{m}{k}k^2\int d\textbf{n}''f(\textbf{n},\textbf{n}'')f^*(\textbf{n}',\textbf{n}'')
\end{equation}
\begin{equation}
    \boxed{f(\textbf{n},\textbf{n}')-f^*(\textbf{n}',\textbf{n})=\frac{ik}{2\pi}\int d\textbf{n}''f(\textbf{n},\textbf{n}'')f^*(\textbf{n}',\textbf{n}'')}
\end{equation}
\subsection*{Задачи (70 баллов)}
\textbf{Задача 1. Welcome to Flatland! (40 баллов)}\\
\textbf{Одномерие (15 баллов)}\\
Выведите формулы Борновского приближения для одномерного пространства. Амплитуда рассеяния определяется согласно:
\begin{equation}
    \psi(\textbf{r}) = e^{i\textbf{k}\textbf{r}} + f(\textbf{n}, \textbf{n}')ie^{ik|\textbf{r}|}
\end{equation}
Что из себя представляет величина $f$ и как она связана с амплитудами прохождения и отражения $t$, $r$? Что из себя представляет сечение рассеяния в одномерии?\\
\textbf{Двумерие (25 баллов)}\\
А теперь повторите вывод для двумерного пространства. В двумерии амплитуда определяется согласно:
\begin{equation}
    \psi(\textbf{r})=e^{i\textbf{k}\textbf{r}}+f(\textbf{n},\textbf{n}')e^{\frac{i\pi}{4}}\frac{e^{ikr}}{\sqrt{r}}
\end{equation}
\textit{Указание:} функция Грина свободного движения выражается через какие-то из модифицированных функций Бесселя.\\
\textbf{Решение.}\\
Пусть $\hat{H}_0=\frac{\hat{p}^2}{2m}$ -- оператор Гамильтона для свободной частицы, $\ket{\textbf{k}}$ -- падающая плоская волна, $\ket{\chi}$ -- рассеянная.
\begin{equation}
    \ket{\psi}=\ket{\textbf{k}}+\ket{\chi},\quad \hat{H}_0\ket{\textbf{k}}=E\ket{\textbf{k}}
\end{equation}
\begin{equation}
    (\hat{H}_0+\hat{V})(\ket{\textbf{k}}+\ket{\chi})=E(\ket{\textbf{k}}+\ket{\chi})\rightarrow(E-\hat{H})\ket{\chi}=\hat{V}\ket{\textbf{k}}
\end{equation}
\begin{equation}
    \ket{\chi}=(E-\hat{H})^{-1}\hat{V}\ket{\textbf{k}}
\end{equation}
Резольвента оператора $\hat{H}$:
\begin{equation}
    (E-\hat{H})\hat{G}_E=\hat{\mathbb{I}}
\end{equation}
Представление резольвенты в координатном представлении -- функция Грина $G_E(\textbf{r},\textbf{r}') = \braket{\textbf{r}|\hat{G}_E|\textbf{r}'}$, удовлетворяющая уравнению
\begin{equation}
    (E-\hat{H})G_E(\textbf{r},\textbf{r}')=\delta(\textbf{r}-\textbf{r}')
\end{equation}
Пусть $\hat{G}^{(0)}_E=(E-\hat{H}_0)^{-1}$, тогда
\begin{equation}
    G^{(0)}_E(\textbf{r},\textbf{r}')=\int\frac{d^3\textbf{k}}{(2\pi)^3}\frac{e^{i\textbf{k}(\textbf{r}-\textbf{r}')}}{E-\frac{k^2}{2m}},\quad G^{(0,R)}_E(\textbf{r},\textbf{r}')=\int\frac{d^3\textbf{k}}{(2\pi)^3}\frac{e^{i\textbf{k}(\textbf{r}-\textbf{r}')}}{E+i0-\frac{k^2}{2m}}
\end{equation}
\begin{equation}
    \hat{G}_E=(E-\hat{H})^{-1}=(E-\hat{H}_0-\hat{V})^{-1}=\hat{G}^{(0)}_E+\hat{G}^{(0)}_E\hat{V}\hat{G}^{(0)}_E+...
\end{equation}
\begin{multline}
    \ket{\chi}=(E-\hat{H}+i0)^{-1}\hat{V}\ket{\textbf{k}}=(\hat{G}^{(0,R)}_E\hat{V}+\hat{G}^{(0,R)}_E\hat{V}\hat{G}^{(0,R)}_E\hat{V}+...)\ket{\textbf{k}}=\\=\hat{G}^{(0,R)}_E(\hat{V}+\hat{V}\hat{G}^{(0,R)}_E\hat{V}+...)\ket{\textbf{k}}=\hat{G}^{(0,R)}_E\hat{T}_E\ket{\textbf{k}} 
\end{multline}
$T$-матрица:
\begin{equation}
    \hat{T}_E=\hat{V}+\hat{V}\hat{G}^{(0,R)}_E\hat{V}+...=\hat{V}(1-\hat{G}^{(0,R)}_EV)^{-1}
\end{equation}
Запишем $\ket{\chi}$ в координатном представлении:
\begin{multline}
    \chi_{\textbf{k}}(\textbf{r})=\braket{\textbf{r}|\chi}=\braket{\textbf{r}|\hat{G}^{(0,R)}_E\hat{T}_E|\textbf{k}}=\int d\textbf{r}'\frac{d\textbf{k}'}{(2\pi)^d}\braket{\textbf{r}|\hat{G}^{(0,R)}_E|\textbf{r}'}\braket{\textbf{r}'|\textbf{k}'}\braket{\textbf{k}'|\hat{T}_E|\textbf{k}}=\\=\int d\textbf{r}'\frac{d\textbf{k}'}{(2\pi)^d}\hat{G}^{(0,R)}_E(\textbf{r},\textbf{r}')e^{i\textbf{k}'\textbf{r}'}\hat{T}_E(\textbf{k},\textbf{k}')
\end{multline}
В борновском приближении $\hat{T}_E\approx\hat{V}$.
\begin{equation}
    \chi_{\textbf{k}}(\textbf{r})=\int d\textbf{r}'\frac{d\textbf{k}'}{(2\pi)^d}\hat{G}^{(0,R)}_E(\textbf{r},\textbf{r}')e^{i\textbf{k}'\textbf{r}'}\braket{\textbf{k}'|\hat{V}|\textbf{k}}
\end{equation}
\begin{enumerate}
    \item Рассмотрим одномерный случай. Запаздывающая функция Грина:
    \begin{multline}
    G^{(0,R)}_E(x)=\int\frac{dk}{2\pi}\frac{e^{ikx}}{E+i0-\frac{k^2}{2m}}=\frac{m}{\pi}\int dk\frac{e^{ikx}}{2mE-k^2+i0}=\\=\frac{m}{\pi}2\pi i\text{res}_{k=\pm\sqrt{2mE}}\left(\frac{e^{ikx}}{2mE-k^2}\right)=-i\sqrt{\frac{m}{2E}}e^{i\sqrt{2mE}x}
    \end{multline}
    \begin{equation}
        G^{(0,R)}_E(x,x')=-\frac{im}{k_E}e^{ik_E|x-x'|},\quad k_E=\sqrt{2mE}=k
    \end{equation}
    \begin{equation}
        \chi_k(x)=\int dx'\frac{dk'}{2\pi}\hat{G}^{(0,R)}_E(x,x')e^{ik'x'}\braket{k'|\hat{V}|k}=-\frac{im}{k}\int dx'\frac{dk'}{2\pi}e^{ik|x-x'|}e^{ik'x'}\tilde{V}_{k-k'}
    \end{equation}
    \begin{equation}
        |x-x'|\approx x-n_xx',\quad x\rightarrow\infty
    \end{equation}
    \begin{multline}
        \chi_k(x)=-\frac{im}{k}\int dx'\frac{dk'}{2\pi}e^{ik(x-n_xx')}e^{ik'x'}\tilde{V}_{k-k'}=-\frac{im}{k}\int dx'\delta(x')e^{ik(x-n_xx')}\tilde{V}_{k-k'}=\\=-\frac{im}{k}e^{ikx}\tilde{V}_{k-k'}
    \end{multline}
    \begin{equation}
        \boxed{f(n,n')=-\frac{m}{k}\tilde{V}_{k-k'}=
        \begin{cases}
            r=-\frac{m}{k}\int V(x)dx,\quad\quad \textbf{n}\uparrow\uparrow\textbf{n}'\\
            t=-\frac{m}{k}\int V(x)e^{2kx}dx,\quad \textbf{n}\uparrow\downarrow\textbf{n}'\\
        \end{cases}}
    \end{equation}
    \begin{equation}
        \boxed{\sigma=r^2+t^2}
    \end{equation}
    \item Рассмотрим двумерный случай. Запаздывающая функция Грина:
    \begin{equation}
        G^{(0,R)}_E(\textbf{r})=\int\frac{d^2\textbf{k}}{(2\pi)^2}\frac{e^{i\textbf{k}\textbf{r}}}{E+i0-\frac{k^2}{2m}}=\int\limits_0^\infty\frac{dkk}{4\pi^2}\frac{1}{E+i0-\frac{k^2}{2m}}\int\limits_0^{2\pi} d\varphi e^{ikr\cos\varphi}
    \end{equation}
    Воспользумкся методом стационарной фазы:
    \begin{equation}
        \int\limits_0^{2\pi} d\varphi e^{ikr\cos\varphi}=\sqrt{\frac{2\pi}{kr}}(e^{ikr-i\frac{\pi}{4}}+e^{-ikr+i\frac{\pi}{4}})=2\sqrt{\frac{2\pi}{kr}}\cos\left(kr-\frac{\pi}{4}\right),\quad kr\gg1
    \end{equation}
    \begin{equation}
        G^{(0,R)}_E(\textbf{r})=\frac{\sqrt{2\pi}}{2\pi^2\sqrt{r}}\int\limits_0^\infty \frac{dk\sqrt{k}\cos\left(kr-\frac{\pi}{4}\right)}{E+i0-\frac{k^2}{2m}}
    \end{equation}
    Сделав разрез по положительной части окружности, получаем
    \begin{equation}
        G^{(0,R)}_E(\textbf{r})=\frac{m}{\sqrt{2\pi k_Er}}e^{ik_Er+i\frac{\pi}{4}}\rightarrow G^{(0,R)}_E(\textbf{r},\textbf{r}')=\frac{m}{\sqrt{2\pi k_Er}}e^{ik_E|\textbf{r}-\textbf{r}'|+i\frac{\pi}{4}}
    \end{equation}
    \begin{multline}
        \chi_{\textbf{k}}(\textbf{r})=\int d\textbf{r}'\frac{d\textbf{k}'}{(2\pi)^2}\hat{G}^{(0,R)}_E(\textbf{r},\textbf{r}')e^{i\textbf{k}'\textbf{r}'}\braket{\textbf{k}'|\hat{V}|\textbf{k}}=\\=\frac{m}{\sqrt{2\pi kr}}\int d\textbf{r}'\delta(\textbf{r}')e^{ik|\textbf{r}-\textbf{r}'|+i\frac{\pi}{4}}e^{i\textbf{k}'\textbf{r}'}\tilde{V}_{\textbf{k}-\textbf{k}'}=\frac{m}{\sqrt{2\pi kr}}e^{ikr+i\frac{\pi}{4}}\tilde{V}_{\textbf{k}-\textbf{k}'}
    \end{multline}
    \begin{equation}
        \boxed{f(\textbf{n},\textbf{n}')=\frac{m}{\sqrt{2\pi k}}\tilde{V}_{\textbf{k}-\textbf{k}'}}
    \end{equation}
    Дифференциальное сечение рассеяния:
    \begin{equation}
        \boxed{d\sigma=|f(\textbf{n},\textbf{n}')|^2d\theta}
    \end{equation}
\end{enumerate}
\textbf{Задача 2. Глубокий мелкий рассеиватель? (15 баллов)}\\
Рассмотрите рассеяние медленных частиц на одномерном потенциале $V(x) = -V_0\frac{a}{a+|x|}$. Используя результат предыдущей задачи, вычислите коэффициент прохождения частиц в ведущем Борновском приближении.\\
\textbf{Решение.}\\
Если воспользоваться формулой из предыдущей задачи, то
\begin{equation}
    t=-\frac{m}{k}\int V(x)dx=-\frac{m}{k}V_0\int\frac{adx}{a+|x|}
\end{equation}
Интеграл расходится. Воспользуемся предыдущей формулой, при этом обрезав интеграл по $[-R;R]$, для какого-то $R>0$.
\begin{equation}
    \chi_{k}(x)=-i\frac{m}{k}\int\limits_{-R}^{R}e^{ik|x-x'|}V(x')e^{ikx'}dx', \quad x>R
\end{equation}
\begin{multline}
    \chi_{k}(x)=2iV_0ae^{ikx}\frac{m}{k}\int\limits_0^R\frac{dx'}{x'+a}=2iV_{0}ae^{ikx}\frac{m}{k}\ln\frac{a+R}{a}=2iV_{0}ae^{ikx}\frac{m}{k}(\ln(ka+kR)-\ln ka)
\end{multline}
Частицы медленные, поэтому $ka\ll1$.
\begin{equation}
    R\sim\frac{1}{k}\gg a,\quad \ln(ka+kR)\approx\ln ka
\end{equation}
\begin{equation}
    \chi_{k}(x)=-2iV_{0}ae^{ikx}\frac{m}{k}\ln ka
\end{equation}
\begin{equation}
    \boxed{t=-2V_0a\sqrt{\frac{m}{2E}}\ln ka=-\frac{V_0}{E}ka\ln ka}
\end{equation}
\textbf{Задача 3. Двухатомная молекула (15 баллов)}\\
Смоделируем двухатомную полярную молекулу потенциалом вида $V (\textbf{r}) = V_0(\textbf{r} + \frac{\textbf{R}}{2}) + V_0(\textbf{r}-\frac{\textbf{R}}{2})$ (вектор $\textbf{R}$ соединяет
два атома, а $V_0(\textbf{r})$ представляет собой потенциал рассеяния на отдельном атоме). В рамках Борновского приближения,
свяжите дифференциальное сечение рассеяния на такой молекуле с сечением на отдельном атоме. Считая различные
ориентации молекулы равновероятными, усредните по ним результат. Как связаны полные сечения рассеяния для случая
медленных ($kR \ll 1$) и достаточно быстрых ($ka \sim 1$, но $R\gg a$, где $a$ -- характерный масштаб потенциала $V_0(r)$) частиц.\\
\textbf{Решение.}\\
Амплитуды рассеяния в борновском приближении:
\begin{equation}
    f(\textbf{n},\textbf{n}')=-\frac{m}{2\pi\hbar}\tilde{V}_{\textbf{k}-\textbf{k}'},\quad f_0(\textbf{n},\textbf{n}')=-\frac{m}{2\pi}\tilde{V}_{0\textbf{k}-\textbf{k}'}
\end{equation}
\begin{equation}
    f_0(\textbf{n},\textbf{n}')=-\frac{m}{2\pi}\int d^3\textbf{r}V_0(\textbf{r})e^{-i(\textbf{k}-\textbf{k}')\textbf{r}}
\end{equation}
\begin{multline}
    f(\textbf{n},\textbf{n}')=-\frac{m}{2\pi}\int d^3\textbf{r}V(\textbf{r})e^{-i(\textbf{k}-\textbf{k}')\textbf{r}}=-\frac{m}{2\pi}\int d^3\textbf{r}\left(V_0\left(\textbf{r} + \frac{\textbf{R}}{2}\right) + V_0\left(\textbf{r}-\frac{\textbf{R}}{2}\right)\right)e^{-i(\textbf{k}-\textbf{k}')\textbf{r}}=\\=-\frac{m}{2\pi}\int d^3\textbf{r}\left(V_0(\textbf{r})e^{i(\textbf{k}-\textbf{k}')\frac{\textbf{R}}{2}} + V_0(\textbf{r})e^{-i(\textbf{k}-\textbf{k}')\frac{\textbf{R}}{2}}\right)e^{-i(\textbf{k}-\textbf{k}')\textbf{r}}=2f_0(\textbf{n},\textbf{n}')\cos\left(\frac{(\textbf{k}-\textbf{k}')\textbf{R}}{2}\right)
\end{multline}
Дифференциальное сечение рассеяния:
\begin{equation}
    d\sigma=|f(\textbf{n},\textbf{n}')|^2d\Omega_{\textbf{n}'},\quad d\sigma_0=|f_0(\textbf{n},\textbf{n}')|^2d\Omega_{\textbf{n}'}
\end{equation}
Связь между дифференциальными сечениями:
\begin{equation}
    \boxed{d\sigma=4\cos^2\left(\frac{(\textbf{k}-\textbf{k}')\textbf{R}}{2}\right)d\sigma_0}
\end{equation}
Усредним величину $\frac{d\sigma}{d\sigma_0}$ по различным ориентациям молекулы.
\begin{equation}
    \left<\frac{d\sigma}{d\sigma_0}\right>=\frac{1}{4\pi}\int 4\cos^2\left(\frac{(\textbf{k}-\textbf{k}')\textbf{R}}{2}\right) d\Omega=\frac{4\cdot2\pi}{4\pi}\int\limits_0^\pi\cos^2\left(\frac{|\textbf{k}-\textbf{k}'|R\cos\theta}{2}\right)\sin\theta d\theta
\end{equation}
\begin{equation}
    \left<\frac{d\sigma}{d\sigma_0}\right>=2\left(1+\frac{\sin(|\textbf{k}-\textbf{k}'|R)}{|\textbf{k}-\textbf{k}'|R}\right)
\end{equation}
\begin{equation}
    |\textbf{k}|=|\textbf{k}'|\rightarrow|\textbf{k}-\textbf{k}'|=\sqrt{k^2+k'^2-2kk'\cos\alpha}=\sqrt{2k^2(1-\cos\alpha)}=2k\sin\frac{\alpha}{2}
\end{equation}
\begin{equation}
    \boxed{\left<\frac{d\sigma}{d\sigma_0}\right>=2\left(1+\frac{\sin(2k\sin\frac{\alpha}{2}R)}{2k\sin\frac{\alpha}{2}R}\right)}
\end{equation}
Найдём связь между полными сечениями:
\begin{multline}
    \braket{\sigma}=\int2\left(1+\frac{\sin(2k\sin\frac{\alpha}{2}R)}{2k\sin\frac{\alpha}{2}R}\right)|f_0(\textbf{n},\textbf{n}')|^2\sin\alpha d\alpha d\varphi=\\=2\left(\braket{\sigma_0}+\int\frac{\sin(2k\sin\frac{\alpha}{2}R)}{2k\sin\frac{\alpha}{2}R}|f_0(\textbf{n},\textbf{n}')|^2\sin\alpha d\alpha d\varphi\right)
\end{multline}
\begin{itemize}
    \item Случай медленных частиц ($kR\ll1$).
    \begin{equation}
        \boxed{\braket{\sigma}\approx2\left(\braket{\sigma_0}+\int|f_0(\textbf{n},\textbf{n}')|^2\sin\alpha d\alpha d\varphi\right)=4\braket{\sigma_0}}
    \end{equation}
    \item Случай достаточно быстрых частиц ($ka\sim1$, $R\gg a$). Подынтегральное выражение быстро осциллирует, поэтому
    \begin{equation}
        \boxed{\braket{\sigma}\approx2\braket{\sigma_0}}
    \end{equation}
\end{itemize}
\section{Фазовая теория рассеяния}
\textbf{Задача 1. Flatland, you again? (20 баллов)}\\
Постройте фазовую теорию рассеяния в двумерии для случая осесиметричных потенциалов. А именно, выразите через
фазовые сдвиги $\delta_m$ амплитуду рассеяния $f(\varphi)$, парциальные сечения рассеяния $\sigma_m$ и полное сечение $\sigma$, и выведите оптическую теорему.
Напомним, амплитуда рассеяния в двумерие выражается через асимптотику волновых функций следующим образом:
\begin{equation}
    \psi(\textbf{r})=e^{ikr\cos\varphi}+f(\varphi)e^{i\frac{\pi}{4}}\frac{e^{ikr}}{\sqrt{r}}
\end{equation}
\textbf{Решение.}\\
Двумерное уравнение Шрёдингера:
\begin{equation}
    -\frac{\hbar^2}{2M}\left(\frac{1}{r}\frac{\partial}{\partial r}\left(r\frac{\partial\psi}{\partial r}\right)+\frac{1}{r^2}\frac{\partial^2\psi}{\partial\phi^2}\right)+V(\textbf{r})\psi(\textbf{r})=E\psi(\textbf{r})
\end{equation}
Разложим волновую функцию свободного движения по функциям с определённой проекцией $m$ углового момента на ось $y$, имеющим вид $Q_m(\rho)e^{im\varphi}$ ($p_y=0$).
\begin{equation}
    e^{ikz}\equiv e^{ikr\cos\varphi}=\sum\limits_{m=-\infty}^\infty Q^\text{free}_m(r)e^{im\varphi}
\end{equation}
\begin{equation}
    Q^\text{free}_m(r)=\frac{1}{2\pi}\int\limits_0^{2\pi}\exp(i(kr\cos\varphi-m\varphi))=i^mJ_m(kr)
\end{equation}
где $J_m$ -- функция Бесселя. При $kr\gg1$ для $Q_m$ справедливо асимптотическое выражение:
\begin{equation}
    Q^\text{free}_m(r)\approx i^m\sqrt{\frac{2}{\pi kr}}\sin\left(kr-\frac{\pi}{2}\left(m-\frac{1}{2}\right)\right)
\end{equation}
В случае наличия потенциала нужно добавить фазовый сдвиг:
\begin{equation}
    Q_m(r)\approx i^m\sqrt{\frac{2}{\pi kr}}\sin\left(kr-\frac{\pi}{2}\left(m-\frac{1}{2}\right)+\delta_m\right),\quad \delta_m=\delta_{-m}
\end{equation}
Разложение волновой функции и $f(\varphi)$:
\begin{equation}
    \psi=\sum\limits_{m=-\infty}^\infty C_mQ_m(r)e^{im\varphi},\quad f(\varphi)=\sum\limits_{m=-\infty}^\infty f_m(\varphi)e^{im\varphi}
\end{equation}
\begin{equation}
    C_mQ_m(r)=Q^\text{free}_m(r)+f_m(\varphi)e^{i\frac{\pi}{4}}\frac{e^{ikr}}{r}
\end{equation}
\begin{multline}
    C_m\sqrt{\frac{2}{\pi kr}}\sin\left(kr-\frac{\pi}{2}\left(m-\frac{1}{2}\right)+\delta_m\right)=i^m\sqrt{\frac{2}{\pi kr}}\sin\left(kr-\frac{\pi}{2}\left(m-\frac{1}{2}\right)\right)+\\+f_m(\varphi)e^{i\frac{\pi}{4}}\frac{e^{ikr}}{r}
\end{multline}
\begin{multline}
    \frac{C_m}{i}\sqrt{\frac{1}{2\pi kr}}\left(\exp\left(ikr-\frac{i\pi}{2}\left(m-\frac{1}{2}\right)+i\delta_m\right)-\exp\left(-ikr+\frac{i\pi}{2}\left(m-\frac{1}{2}\right)-i\delta_m\right)\right)=\\=i^{m-1}\sqrt{\frac{1}{2\pi kr}}\left(\exp\left(ikr-\frac{i\pi}{2}\left(m-\frac{1}{2}\right)\right)-\exp\left(-ikr+\frac{i\pi}{2}\left(m-\frac{1}{2}\right)\right)\right)+\\+f_m(\varphi)e^{i\frac{\pi}{4}}\frac{e^{ikr}}{\sqrt{r}}
\end{multline}
Приравнивая коэффиценты при падающих и уходяших волнах, получим
\begin{equation}
    \begin{cases}
        C_m=e^{i\delta_m}i^m,\\
        i^{m-1}\sqrt{\frac{1}{2\pi kr}}\exp\left(-\frac{i\pi}{2}\left(m-\frac{1}{2}\right)\right)(e^{2i\delta_m}-1)=f_m(\varphi)\frac{e^{i\frac{\pi}{4}}}{\sqrt{r}}
    \end{cases}
\end{equation}
\begin{equation}
    f_m(\varphi)=\frac{1}{i\sqrt{2\pi k}}(e^{2i\delta_m}-1)
\end{equation}
\begin{equation}
    \boxed{f(\varphi)=\sum\limits_{m=-\infty}^\infty f_m(\varphi)e^{im\varphi}=\frac{1}{i\sqrt{2\pi k}}\sum\limits_{m=-\infty}^\infty(e^{2i\delta_m}-1)e^{im\varphi}}
\end{equation}
Полное сечение:
\begin{equation}
    \sigma=\int\limits_0^{2\pi}|f(\varphi)|^2d\varphi=\frac{1}{2\pi k}\sum\limits_{m=-\infty}^{\infty}4\sin^2\delta_m2\pi=\frac{4}{k}\sum\limits_{m=-\infty}^{\infty}\sin^2\delta_m
\end{equation}
\begin{equation}
    \boxed{\sigma_m=\frac{4\sin^2\delta_m}{k},\quad \sigma=\frac{4}{k}\sum\limits_{m=-\infty}^{\infty}\sin^2\delta_m}
\end{equation}
\begin{multline}
    \text{Im}f(0)=\text{Im}\frac{1}{i\sqrt{2\pi k}}\sum\limits_{m=-\infty}^\infty(e^{2i\delta_m}-1)=\text{Im}\frac{1}{i\sqrt{2\pi k}}\sum\limits_{m=-\infty}^\infty(\cos(2\delta_m)+i\sin(2\delta_m)-1)=\\=\frac{1}{\sqrt{2\pi k}}\sum\limits_{m=-\infty}^\infty(1-\cos2\delta_m)=\sqrt{\frac{2}{\pi k}}\sum\limits_{m=-\infty}^\infty\sin^2\delta_m
\end{multline}
Оптическая теорема:
\begin{equation}
    \boxed{\text{Im}f(0)=\sqrt{\frac{k}{8\pi}}\sigma}
\end{equation}
\textbf{Задача 2. Мистические сокращения (20 баллов)}\\
Используя результат предыдущей задачи, найдите фазовые сдвиги и сечение рассеяния для двумерного потенциала $V(r) =\frac{\hbar^2\beta}{2mr^2}$. Исследуйте предел <<слабого>> ($\beta\ll 1$, с точностью до членов $O(\beta^2)$, \textbf{10 баллов}) и <<сильного>> ($\beta \gg 1$, \textbf{10 баллов}) потенциала.\\
\textbf{(Для любознательных)} Полученный ответ может вас натолкнуть на некоторую гипотезу, которую, в принципе,
можно потом проверить численно. Попробуйте её доказать.\\
\textbf{Решение.}\\
Гамильтониан:
\begin{equation}
    \hat{H}=-\frac{\hbar^2}{2m}\Delta+\frac{\hbar^2\beta}{2mr^2}=-\frac{\hbar^2}{2m}\left(\frac{1}{r}\frac{\partial}{\partial r}\left(r\frac{\partial}{\partial r}\right)+\frac{1}{r^2}\frac{\partial^2}{\partial\varphi^2}\right)+\frac{\hbar^2\beta}{2mr^2}
\end{equation}
Разложение воновой функции:
\begin{equation}
    \psi(\textbf{r})=\sum\limits_{l=-\infty}^\infty q_l(r)e^{il\varphi}
\end{equation}
Стационарное уравнение Шрёдингера:
\begin{equation}
    \hat{H}\psi(\textbf{r})=E\psi(\textbf{r})
\end{equation}
\begin{equation}
    -\frac{\hbar^2}{2m}\left(q''_l(r)+\frac{q'_l(r)}{r}-\frac{l^2}{r^2}q_l(r)\right)+\frac{\beta\hbar^2 q_l(r)}{2mr^2}=Eq_l(r)
\end{equation}
\begin{equation}
    r^2q''_l(r)+rq'_l(r)+\left(\frac{2mEr^2}{\hbar^2}-l^2-\beta\right)q_l(r)=0
\end{equation}
Введём замену $z=\frac{\sqrt{2mE}}{\hbar}r$.
\begin{equation}
    z^2q''_l(z)+zq'_l(z)+\left(z^2-l^2-\beta\right)q_l(z)=0
\end{equation}
Получилось уравнение Бесселя:
\begin{equation}
    q_l(z)=J_{\sqrt{l^2+\beta}}(z)\approx\frac{2}{\pi z}\cos\left(z-\frac{\pi}{2}\sqrt{l^2+\beta}-\frac{\pi}{4}\right)
\end{equation}
Сравнивая с соответсвующим коэффициентом предыдущей задачи, получим фазовый сдвиг (с учётом $\delta_l=\delta_{-l}$):
\begin{equation}
    \boxed{\delta_l=\frac{\pi}{2}(|l|-\sqrt{l^2+\beta})}
\end{equation}
Сечение рассеяния:
\begin{equation}
    \boxed{\sigma=\frac{4}{k}\sum\limits_{l=-\infty}^{\infty}\sin^2\delta_l=\frac{4}{k}\sum\limits_{l=-\infty}^{\infty}\sin^2\left(\frac{\pi}{2}(|l|-\sqrt{l^2+\beta})\right)}
\end{equation}
Рассмотрим предельные случаи:
\begin{itemize}
    \item <<Слабый>> потенциал $\beta\ll1$.\\
    Фазовые сдвиги:
    \begin{equation}
        \delta_l=\frac{\pi}{2}\left(|l|-|l|\sqrt{1+\frac{\beta}{l^2}}\right)\approx-\frac{\pi\beta}{4|l|},\quad l\neq0
    \end{equation}
    \begin{equation}
        \boxed{\begin{cases}
            \delta_l=-\frac{\pi\beta}{4|l|},\quad l\neq0\\
            \delta_0=-\frac{\pi\sqrt{\beta}}{2}
        \end{cases}}
    \end{equation}
    Сечение рассеяния:
    \begin{equation}
        \sigma\approx\frac{4}{k}\left(\sin^2\frac{\pi\sqrt{\beta}}{2}+\sum\limits_{l=-\infty}^{\infty}\sin^2\frac{\pi\beta}{4|l|}\right)\approx\frac{4}{k}\left(\frac{\pi^2\beta}{4}-\frac{\pi^4\beta^2}{48}+2\sum\limits_{l=1}^\infty\left(\frac{\pi\beta}{4l}\right)^2\right)=\frac{\pi^2\beta}{k}
    \end{equation}
    \begin{equation}
        \boxed{\sigma=\frac{\pi^2\beta}{k}+O(\beta^3)}
    \end{equation}
    \item <<Сильный>> потенциал $\beta\gg1$.\\
    Фазовые сдвиги:
    \begin{equation}
        \boxed{\delta_l=\frac{\pi}{2}(|l|-\sqrt{l^2+\beta})}
    \end{equation}
    Сечение рассеяния:
    \begin{equation}
        \sigma=\frac{4}{k}\sum\limits_{l=-\infty}^{\infty}\sin^2\left(\frac{\pi}{2}(|l|-\sqrt{l^2+\beta})\right)\approx\frac{8}{k}\int\limits_{0}^{\infty}dl\sin^2\left(\frac{\pi}{2}(l-\sqrt{l^2+\beta})\right)
    \end{equation}
    Пусть $u=\frac{\pi}{2}(l-\sqrt{l^2+\beta})$, тогда
    \begin{equation}
        \sigma=\frac{8}{k}\int\limits_{-\sqrt{\beta}}^0 du\left(\frac{1}{\pi}+\frac{\pi\beta}{4u^2}\right)\sin^2u\approx\frac{8}{k}\left(\frac{\sqrt{\beta}}{2\pi}+\frac{\pi\beta}{4}\frac{\pi}{2}+O(1)\right)=\frac{\pi^2\beta}{k}+O(\sqrt{\beta})
    \end{equation}
    \begin{equation}
        \boxed{\sigma=\frac{\pi^2\beta}{k}+O(\sqrt{\beta})}
    \end{equation}
\end{itemize}
В разных предельных случаях получился одинаковый ответ. Возможно, что при любых $\beta$ ответ
\begin{equation}
    \boxed{\sigma=\frac{\pi^2\beta}{k}}
\end{equation}
\textbf{Задача 3 (30 баллов)}\\
Найдите точное выражение для произвольного парциального сечения рассеяния на твёрдом шаре $\sigma_l$. Исследуйте полученную формулу в следующих предельных случаях:
\begin{itemize}
    \item \textbf{(15 баллов)} Оцените вклад $p$-канала для рассеяния медленных частиц;
    \item \textbf{(15 баллов)} Определите полное сечение рассеяния для быстрых частиц.
\end{itemize}
\textbf{Решение.}\\
Уравнение Шрёдингера для радиальной компоненты волновой функции:
\begin{equation}
    R''_{kl}(r)+\frac{2}{r}R'_{kl}(r)+\left(k^2-\frac{l(l+1)}{r^2}\right)R_{kl}(r)=0
\end{equation}
Введём замену $R_{kl}(r)=\frac{Q_{kl}(r)}{\sqrt{r}}$.
\begin{equation}
    R'_{kl}(r)=\frac{Q'_{kl}(r)}{\sqrt{r}}-\frac{Q_{kl}(r)}{2r^\frac{3}{2}},\quad R''_{kl}(r)=\frac{Q_{kl}''(r)}{\sqrt{r}}-\frac{Q_{kl}'(r)}{r^\frac{3}{2}}+\frac{3Q_{kl}(r)}{4r^\frac{5}{2}}
\end{equation}
\begin{equation}
    \frac{Q_{kl}''(r)}{\sqrt{r}}-\frac{Q_{kl}'(r)}{r^\frac{3}{2}}+\frac{3Q_{kl}(r)}{4r^\frac{5}{2}}+\frac{2Q'_{kl}(r)}{r^\frac{3}{2}}-\frac{Q_{kl}(r)}{r^\frac{5}{2}}+\left(k^2-\frac{l(l+1)}{r^2}\right)\frac{Q_{kl}(r)}{\sqrt{r}}=0
\end{equation}
\begin{equation}
    Q''_{kl}(r)+\frac{Q'_{kl}(r)}{r}-\frac{Q_{kl}(r)}{4r^2}+\left(k^2-\frac{l(l+1)}{r^2}\right)Q_{kl}(r)=0
\end{equation}
\begin{equation}
    Q''_{kl}(kr)+\frac{Q'_{kl}(kr)}{kr}+\left(1-\frac{(l+\frac{1}{2})^2}{(kr)^2}\right)Q_{kl}(kr)=0
\end{equation}
Получилось уравнение Бесселя с положительной энергией.
\begin{equation}
    Q_{kl}(kr)=C_1J_{l+\frac{1}{2}}(kr)+C_2Y_{l+\frac{1}{2}}(kr)
\end{equation}
Граничное условие:
\begin{equation}
    Q_{kl}(ka)=0\rightarrow\frac{C_2}{C_1}=-\frac{J_{l+\frac{1}{2}}(ka)}{Y_{l+\frac{1}{2}}(ka)}
\end{equation}
Асимптотика при $kr\gg1$:
\begin{equation}
    J_{l+\frac{1}{2}}(kr)\approx\sqrt{\frac{2}{\pi kr}}\cos\left(kr-\frac{\pi(l+\frac{1}{2})}{2}-\frac{\pi}{4}\right)=\sqrt{\frac{2}{\pi kr}}\sin\left(kr-\frac{\pi l}{2}\right)
\end{equation}
\begin{equation}
    Y_{l+\frac{1}{2}}(kr)\approx\sqrt{\frac{2}{\pi kr}}\sin\left(kr-\frac{\pi(l+\frac{1}{2})}{2}-\frac{\pi}{4}\right)=-\sqrt{\frac{2}{\pi kr}}\cos\left(kr-\frac{\pi l}{2}\right)
\end{equation}
\begin{equation}
    Q_{kl}(kr)\approx C\sqrt{\frac{2}{\pi kr}}\sin\left(kr-\frac{\pi l}{2}+\delta_l\right)
\end{equation}
\begin{equation}
    \tan\delta_l=-\frac{C_2}{C_1}=\frac{J_{l+\frac{1}{2}}(ka)}{Y_{l+\frac{1}{2}}(ka)}
\end{equation}
\begin{equation}
    \sin^2\delta_l=\frac{J^2_{l+\frac{1}{2}}(ka)}{J^2_{l+\frac{1}{2}}(ka)+Y^2_{l+\frac{1}{2}}(ka)}
\end{equation}
\begin{equation}
    \boxed{\sigma_l=\frac{4\pi}{k^2}(2l+1)\sin^2\delta_l=\frac{4\pi(2l+1)}{k^2}\frac{J^2_{l+\frac{1}{2}}(ka)}{J^2_{l+\frac{1}{2}}(ka)+Y^2_{l+\frac{1}{2}}(ka)}}
\end{equation}
Рассмотрим предельные случаи:
\begin{itemize}
    \item Медленные частицы $ka\ll1$.\\
    Асимптотика функций Бесселя и Неймана:
    \begin{equation}
        J_{l+\frac{1}{2}}(ka)\approx\frac{1}{\Gamma(l+\frac{3}{2})}\left(\frac{ka}{2}\right)^{l+\frac{1}{2}},\quad Y_{l+\frac{1}{2}}(ka)\approx-\frac{\Gamma(l+\frac{1}{2})}{\pi}\left(\frac{2}{ka}\right)^{l+\frac{1}{2}}
    \end{equation}
    Вклад $p$-канала:
    \begin{equation}
        \sigma_1=\frac{12\pi}{k^2}\frac{\frac{1}{\Gamma(\frac{5}{2})^2}\left(\frac{ka}{2}\right)^3}{\frac{1}{\Gamma(\frac{5}{2})^2}\left(\frac{ka}{2}\right)^3+\frac{\Gamma(\frac{3}{2})^2}{\pi^2}\left(\frac{2}{ka}\right)^3}\approx\frac{12\pi}{k^2}\frac{\pi^2}{\Gamma(\frac{3}{2})^2\Gamma(\frac{5}{2})^2}\left(\frac{ka}{2}\right)^6=\frac{12\pi}{k^2}\frac{\pi^2}{\frac{\pi}{4}\frac{9\pi}{16}2^6}k^6a^6
    \end{equation}
    \begin{equation}
        \boxed{\sigma_1=\frac{4\pi}{3}k^4a^6}
    \end{equation}
    \item Быстрые частицы $ka\gg1$.\\
    Граничное условие:
    \begin{equation}
        Q_{kl}(ka)\approx C\sqrt{\frac{2}{\pi kr}}\sin\left(ka-\frac{\pi l}{2}+\delta_l\right)=0\rightarrow ka-\frac{\pi l}{2}+\delta_l=0
    \end{equation}
    \begin{equation}
        \delta_l=\frac{\pi l}{2}-ka\rightarrow\sin^2\delta_l=\frac{1-\cos2\delta_l}{2}=\frac{1-(-1)^l\cos2ka}{2}
    \end{equation}
    \begin{multline}
        \sigma=\frac{4\pi}{k^2}\sum(2l+1)\sin^2\delta_l\approx\frac{4\pi}{k^2}\int\limits_0^{ka}dl(2l+1)\frac{1-\cos2ka}{2}\approx\frac{2\pi}{k^2}\int\limits_0^{ka}dl(2l+1)=\\=\frac{2\pi}{k^2}(k^2a^2+ka)\approx2\pi a^2
    \end{multline}
    \begin{equation}
        \boxed{\sigma=2\pi a^2}
    \end{equation}
\end{itemize}
\textbf{Задача 4 Рассеяние на потенциале $\sim 1/r^3$ (30 баллов).}\\
Используя фазовую теорию, найдите сечение рассеяния достаточно быстрых частиц ($kL \gg 1$) на потенциале $V(r)=\frac{L}{2m(r^2+a^2)^\frac{3}{2}}$ при дополнительных предположениях:
\begin{enumerate}
    \item \textbf{(15 баллов)} В случае $kL \ll \frac{L^2}{a^2}$. \textit{Указание}: параметрически главный вклад в сечение приходит с моментов $l \gg (kL)^{1/3}$. Покажите, что вкладом от моментов $l \ll (kL)^{1/3}$ можно пренебречь.
    \item \textbf{(15 баллов)} В случае $kL\gg\frac{L^2}{a^2}$. \textit{Указание}: вычислите вклад моментов $l \gg (kL)^{1/3}$. Сравните результат с ответом, который получается из Борновского приближения. Применимо ли оно в данном случае?
\end{enumerate}
\textbf{Решение.}
\begin{enumerate}
    \item Случай $kL\ll\frac{L^2}{a^2}$.\\
    Квазиклассическое выражение для фазовых сдвигов:
    \begin{equation}
        \delta_l=\int\limits_{r_0}^\infty dr\left(\sqrt{k^2-\frac{l(l+1)}{r^2}-2mV(r)}-k\right)-kr_0+\frac{\pi l}{2}+\frac{\pi}{4}
    \end{equation}
    Разложение:
    \begin{equation}
        \delta_l=-\int\limits_{r_0}^\infty dr\frac{mV(r)}{\sqrt{k^2-\frac{l(l+1)}{r^2}}},\quad r_0=\frac{\sqrt{l(l+1)}}{k}\approx\frac{l}{k}
    \end{equation}
    В последнем равенстве использвалось, что $l\gg1$. Это и то, что $r_0\gg a$ далее будет показано. При $r\gg a$:
    \begin{equation}
        V(r)\approx\frac{L}{2mr^3}
    \end{equation}
    \begin{equation}
        \delta_l=\int\limits_{r_0}^\infty dr\frac{L}{2r^3\sqrt{k^2-\frac{l(l+1)}{r^2}}}=\frac{L}{4l(l+1)}\int\limits_{\frac{l}{k}}^\infty\frac{d(k^2-\frac{l(l+1)}{r^2})}{\sqrt{k^2-\frac{l(l+1)}{r^2}}}=\frac{Lk}{2l(l+1)}\approx\frac{Lk}{2l^2}
    \end{equation}
    $\delta_l\sim1$ при $kL\sim l^2$. Поскольку частицы достаточно быстрые $kL\gg1$, то и $l\gg1$.
    \begin{equation}
        l^3\sim lkL\gg kL\rightarrow l\gg(kL)^\frac{1}{3}
    \end{equation}
    Покажем, что $r_0\gg a$.
    \begin{equation}
        ka^2\ll L\rightarrow (ka)^2\ll kL\sim l^2\rightarrow\frac{r_0}{a}=\frac{l}{ka}\gg1
    \end{equation}
    Сечение рассеяния:
    \begin{equation}
        \sigma=\frac{4\pi}{k^2}\sum\limits_{l=-\infty}^{\infty}(2l+1)\sin^2\delta_l=\frac{4\pi}{k^2}\int\limits_0^{\infty}dl2l\sin^2\frac{Lk}{2l^2}=\frac{\pi^2L}{4k}
    \end{equation}
    \begin{equation}
        \boxed{\sigma=\frac{\pi^2L}{4k}}
    \end{equation}
    \item Случай $kL\gg\frac{L^2}{a^2}$.\\
    Фазовые сдвиги:
    \begin{equation}
        \delta_l=\int\limits_{r_0}^\infty dr\frac{L}{2(r^2+a^2)^\frac{3}{2}\sqrt{k^2-\frac{l(l+1)}{r^2}}}=\frac{L}{2al}\int\limits_1^{\sqrt{1+(\frac{ka}{l})^2}}\frac{udu}{u^3\sqrt{1+(\frac{ka}{l})^2-u^2}},\quad u=1+\frac{a^2}{r^2}
    \end{equation}
    Сделаем замену $v=\frac{1+(\frac{ka}{l})^2}{u^2}-1$ и получим
    \begin{equation}
        \delta_l=\frac{L}{4al}\int\limits_0^{\frac{k^2a^2}{l^2}}\frac{dv}{\sqrt{v}}=\frac{kL}{2(l^2+k^2a^2)}
    \end{equation}
    В борновском приближении:
    \begin{equation}
        f(\theta)=-\frac{m}{2\pi}\tilde{V}_{\textbf{k}-\textbf{k}'}=-\frac{m}{k\sin\frac{\theta}{2}}\int\limits_0^\infty drr\sin\left(2kr\sin\frac{\theta}{2}\right)V(r)
    \end{equation}
    \begin{equation}
        f(\theta)=-\frac{L}{2k\sin\frac{\theta}{2}}\int\limits_0^\infty\frac{r\sin2kr\sin\frac{\theta}{2}}{(a^2+r^2)^\frac{3}{2}}dr\approx L\sqrt{\frac{\pi}{ka\sin\frac{\theta}{2}}}e^{-ka\sin\frac{\theta}{2}}
    \end{equation}
    Сечение рассеяния:
    \begin{equation}
        \sigma=2\pi\int\limits_0^\pi|f(\theta)|^2\sin\theta d\theta=2\pi\int\limits_0^\pi L^2\frac{\pi e^{-2ka\sin\frac{\theta}{2}}}{ka\sin\frac{\theta}{2}}\sin\theta d\theta\approx\frac{2\pi^2 L^2}{ka}\int\limits_0^\pi\frac{e^{-ka\theta}}{\frac{\theta}{2}}\theta d\theta
    \end{equation}
    \begin{equation}
        \boxed{\sigma=\frac{4\pi^2L^2}{k^2a^2}}
    \end{equation}
\end{enumerate}
\section{Открытые двухуровневые системы}
\subsection*{Задачи (100 баллов)}
\textbf{Задача 1. Флуктуационно-диссипационная теорема (15 баллов)}\\
Вне зависимости от происхождения индуцированного окружающей средой шума $\hat{\Phi}(t)$, на его корреляционные функции можно вывести соотношения самого общего вида, использующие лишь следующие общие предположения: гамильтониан $\hat{H}_e$ не зависит от времени; и матрица плотности окружающей среды имеет вид $\hat{\rho}_e =\frac{1}{Z}e^{-\beta\hat{H}_e}$.
\begin{enumerate}
    \item Используя формальное сходство между матрицей плотности и оператором эволюции в мнимом времени, покажите, что имеется связь на корреляционные функции $S_<(t) = S_>(t-i\beta)$. Что это означает для их фурье-образов, $S_<(\omega)$ и $S_>(\omega)$?
    \item Определим <<спектральную плотность>> как $J(\omega) = \frac{1}{2}(S_>(\omega)-S_<(\omega))$, и Келдышевский коррелятор как $S_K(t_1-t_2) = \left<\left\{\hat{\Phi}(t_1), \hat{\Phi}(t_2)\right\}\right>$. Как связаны $S_K(\omega)$ и $J(\omega)$?
\end{enumerate}
\textbf{Решение.}
\begin{enumerate}
    \item 
    \begin{equation}
        n_B(\omega)=\frac{1}{e^{\beta\omega}-1},\quad n_B(\omega)+1=-n_B(-\omega)
    \end{equation}
    Корреляционные функции:
    \begin{equation}\label{eq5}
        S_>(t)=\int\limits_0^\infty\frac{d\omega}{\pi}J(\omega)((n_B(\omega)+1)e^{-i\omega t}+n_B(\omega)e^{i\omega t}),\quad S_>(\omega)=2\tilde{J}(\omega)(n_B(\omega)+1)
    \end{equation}
    \begin{equation}\label{eq6}
        S_<(t)=\int\limits_0^\infty\frac{d\omega}{\pi}J(\omega)(n_B(\omega)e^{-i\omega t}+(n_B(\omega)+1)e^{i\omega t}),\quad S_<(\omega)=2\tilde{J}(\omega)n_B(\omega)
    \end{equation}
    где $\tilde{J}(\omega)=J(\omega)-J(-\omega)$.
    \begin{multline}
        S_>(t-i\beta)=\int\limits_0^\infty\frac{d\omega}{\pi}J(\omega)((n_B(\omega)+1)\exp(-i\omega t-\beta\omega)+n_B(\omega)\exp(i\omega t+\beta\omega))=\\=\int\limits_0^\infty\frac{d\omega}{\pi}J(\omega)\left(\frac{e^{\beta\omega}}{e^{\beta\omega}-1}e^{-i\omega t-\beta\omega}+\frac{1}{e^{\beta\omega}-1}e^{-i\omega t+\beta\omega}\right)=\int\limits_0^\infty\frac{d\omega}{\pi}J(\omega)\left(\frac{1}{e^{\beta\omega}-1}e^{-i\omega t}+\frac{e^{\beta\omega}}{e^{\beta\omega}-1}e^{-i\omega t}\right)=\\=\int\limits_0^\infty\frac{d\omega}{\pi}J(\omega)(n_B(\omega)e^{-i\omega t}+(n_B(\omega)+1)e^{i\omega t})=S_<(t)
    \end{multline}
    \begin{equation}
        \boxed{S_>(t-i\beta)=S_<(t)}
    \end{equation}
    Покажем, что это означает для фурье-образов:
    \begin{equation}
        S_<(\omega)=\int\limits_{-\infty}^\infty dtS_<(t)e^{i\omega t}=\int\limits_{-\infty}^\infty dtS_>(t-i\beta)e^{i\omega(t-i\beta+i\beta)}=S_>(\omega)e^{-\omega\beta}
    \end{equation}
    \begin{equation}
        \boxed{\frac{S_<(\omega)}{S_>(\omega)}=e^{-\omega\beta}}
    \end{equation}
    Это же равенство можно получить и из (\ref{eq5}) и (\ref{eq6}).
    \item
    \begin{equation}
        J(\omega) = \frac{1}{2}(S_>(\omega)-S_<(\omega))
    \end{equation}
    \begin{equation}
        S_K(t)=S_>(t)+S_<(t)
    \end{equation}
    \begin{equation}
        \frac{S_<(\omega)}{S_>(\omega)}=e^{-\omega\beta}
    \end{equation}
    \begin{equation}
        \boxed{S_K(\omega)=2J(\omega)\coth\left(\frac{\omega\beta}{2}\right)}
    \end{equation}
\end{enumerate}
\textbf{Задача 2. Релаксация (20 баллов)}\\
Полученное на семинаре время релаксации $T_1$ для спин-бозонной модели, описываемой гамильтонианом
\begin{equation}
    \hat{H}=-\frac{\Delta}{2}\hat{\sigma}_z+\sum\limits_n\omega_n\hat{a}^\dagger_n\hat{a}_n+\hat{\sigma}_x\hat{\Phi},\quad \hat{\Phi}=\sum\limits_n\lambda_n(\hat{a}_n+\hat{a}^\dagger_n)
\end{equation}
можно просто интерпретировать на языке золотого правила Ферми.
\begin{enumerate}
    \item Считая, что все осцилляторы находятся в квантовом состоянии с фиксированными числами заполнения $n_n$, вычислите частоту переходов $w_{i\rightarrow f} = w_{\uparrow\rightarrow\downarrow}$ и $w_{\downarrow\rightarrow\uparrow}$.
    \item Проведите усреднение полученных величин по Гиббсовскому ансамблю для осцилляторов $\hat{\rho}_e =\frac{1}{Z}e^{-\beta\hat{H}_e}$.
    \item Релаксационная динамика, описываемая золотым правилом Ферми, может быть сведена к классическому <<кинетическому уравнению>>:
    \begin{equation}
        \begin{cases}
            \frac{dP_\uparrow(t)}{dt}=-w_{\uparrow\rightarrow\downarrow}P_\uparrow(t)+w_{\downarrow\rightarrow\uparrow}P_\downarrow(t)\\
            \frac{dP_\downarrow(t)}{dt}=-w_{\downarrow\rightarrow\uparrow}P_\downarrow(t)+w_{\uparrow\rightarrow\downarrow}P_\uparrow(t)
        \end{cases}
    \end{equation}
    Исходя из такого уравнения, выразите время релаксации системы через величины $w_{\uparrow\rightarrow\downarrow}$ и $w_{\downarrow\rightarrow\uparrow}$ и сравните со временем релаксации, найденным на семинаре.
\end{enumerate}
%\textbf{Решение.}
%\begin{enumerate}
    %\item 
%\end{enumerate}
\textbf{Задача 3. Чистая дефазирока (50 баллов)}\\
Исследуйте эволюцию матрицы плотности двухуровневой системы, взаимодействующую с резервуаром (набором осцилляторов). Система описывается следующим гамильтонианом:
\begin{equation}
    \hat{H}=-\frac{\Delta}{2}\hat{\sigma}_z+\sum\limits_n\omega_n\hat{a}_n^\dagger\hat{a}_n+\hat{\sigma}_z\hat{\Phi},\quad\hat{\Phi}=\sum\limits_n\lambda_n(\hat{a}_n+\hat{a}^\dagger_n)
\end{equation}
\begin{enumerate}
    \item Перейдите к представлению взаимодействия. Оператор эволюции в представлении взаимодействия записывается при помощи $\mathcal{T}$-упорядоченной экспоненты:
    \begin{equation}
        \hat{S}(t, t_0) = \hat{\mathcal{T}}\left\{\exp\left(-i\int\limits_{t_0}^t\hat{V}(\tau)d\tau\right)\right\}
    \end{equation}
    Гамильтониан в различные моменты времени не коммутирует сам с собой, поэтому $\mathcal{T}$-упорядочение убрать нельзя. Однако задачу можно значительно упростить, используя тот факт, что гамильтониан в различные моменты времени коммутирует на число:
    \begin{equation}
        [\hat{V}(t_1), \hat{V}(t_2)] = i\phi(t_1, t_2)
    \end{equation}
    Используя этот факт, покажите, что знак $\mathcal{T}$-упорядочения можно снять ценой дополнительной добавки:
    \begin{equation}
        \hat{S}(t,t_0) = e^{i\Phi(t,t_0)}\exp\left(-i\int\limits_{t_0}^t\hat{V}(\tau)d\tau\right)
    \end{equation}
    \textit{Указание:} это соотношение можно вывести <<по индукции>>, рассмотрев $\hat{S}(t + \delta t, t_0) = \hat{S}(t + \delta t, t)\hat{S}(t, t_0)$ для $\delta t \rightarrow 0$, и объединив экспоненты при помощи формулы \textbf{Бейкера-Кэмбелла-Хаусдорфа}:
    \begin{equation}
        e^{\hat{A}}e^{\hat{B}}=\exp\left(\hat{A}+\hat{B}+\frac{1}{2}[\hat{A},\hat{B}]\right),\quad[\hat{A},\hat{B}]=\text{const}
    \end{equation}
    \item Пусть в начальный момент времени $t_0 = 0$ матрица плотности системы имела вид $\hat{\rho}_{tot}(t_0) = \hat{\rho}_s(t_0)\otimes\hat{\rho}_e(t_0)$, и резервуар находится в равновесии при температуре $T$: $\hat{\rho}_e(t_0) = \frac{1}{Z}e^{-\beta\hat{H}^{(e)}_0}$. Вычислите редуцированную матрицу плотности в произвольный момент времени:
    \begin{equation}
        \hat{\rho}_s(t)=\text{Tr}_e(\hat{S}(t,t_0)\hat{\rho}_s(t_0)\hat{\rho}_e\hat{S}(t_0,t))
    \end{equation}
    \textit{Указание:} для усреднения по степеням бани вам может пригодиться тождество из задачи 6.1.
    \item Поскольку величина $\hat{\sigma}_z$ в данной задаче коммутирует с гамильтонианом, никакой релаксации наблюдаться не будет. Поэтому вся динамика сведётся к следующей временной зависимости матрицы плотности:
    \begin{equation}
        \hat{\rho}_s(t)=\begin{pmatrix}
        \rho_{11} & \rho_{12}\exp(-\Gamma(t))\\
        \rho_{21}\exp(-\Gamma(t)) & \rho_{22}
        \end{pmatrix}
    \end{equation}
    Найдите выражение для \textbf{функции декогеренции} $\Gamma(t)$.
\end{enumerate}
\textbf{Решение.}
\begin{enumerate}
    \item Докажем соотношение $\hat{S}(t,t_0) = e^{i\Phi(t,t_0)}\exp\left(-i\int\limits_{t_0}^t\hat{V}(\tau)d\tau\right)$ по индукции.
    \begin{enumerate}
        \item База индукции. Пусть соотношение верно для $t=t_0+\delta t,\delta t\rightarrow0$.
        \begin{equation}
            \hat{S}(t_0+\delta t,t_0) = e^{i\Phi(t_0+\delta t,t_0)}\exp\left(-i\int\limits_{t_0}^{t_0+\delta t}\hat{V}(\tau)d\tau\right)=\exp\left(-i\int\limits_{t_0}^{t_0+\delta t}\hat{V}(\tau)d\tau\right)
        \end{equation}
        \begin{equation}
            \boxed{\Phi(t_0,t_0)=0}
        \end{equation}
        \item Пусть верно для момента $t$:
        \begin{equation}
            \hat{S}(t,t_0) = e^{i\Phi(t,t_0)}\exp\left(-i\int\limits_{t_0}^t\hat{V}(\tau)d\tau\right)
        \end{equation}
        \item Проверим, что выполняется для момента $t+\delta t$:
        \begin{multline}
            \hat{S}(t+\delta t,t_0) = e^{i\Phi(t,t_0)}\exp\left(-i\int\limits_{t_0}^{t+\delta t}\hat{V}(\tau)d\tau\right)=\\=e^{i\Phi(t,t_0)}\exp\left(-i\hat{V}(t)\delta t\right)\exp\left(-i\int\limits_{t_0}^t\hat{V}(\tau)d\tau\right)=\\=e^{i\Phi(t,t_0)}\exp\left(-i\int\limits_{t_0}^{t+\delta t}\hat{V}(\tau)d\tau-\frac{i}{2}\left[V(t)\delta t,\int\limits_{t_0}^tV(\tau)d\tau\right]\right)=\\=e^{i\Phi(t,t_0)-\frac{i\delta t}{2}\int\limits_{t_0}^t\phi(t,\tau)d\tau}\exp\left(-i\int\limits_{t_0}^{t+\delta t}\hat{V}(\tau)d\tau\right)=e^{i\Phi(t+\delta t,t_0)}\exp\left(-i\int\limits_{t_0}^{t+\delta t}\hat{V}(\tau)d\tau\right)
        \end{multline}
        \begin{equation}
            \Phi(t+\delta t,t_0)=\Phi(t,t_0)-\frac{\delta t}{2}\int\limits_{t_0}^t\phi(t,\tau)d\tau\rightarrow\frac{d\Phi(t,t_0)}{dt}=-\frac{1}{2}\int\limits_{t_0}^t\phi(t,\tau)d\tau
        \end{equation}
        \begin{equation}
            \boxed{\hat{S}(t,t_0) = e^{i\Phi(t,t_0)}\exp\left(-i\int\limits_{t_0}^t\hat{V}(\tau)d\tau\right),\quad\Phi(t,t_0)=-\frac{1}{2}\int\limits_{t_0}^t\int\limits_{t_0}^\eta\phi(\eta,\tau)d\tau d\eta}
        \end{equation}
    \end{enumerate}
    \item Из определения коммутатора:
    \begin{equation}
        \phi(t_1,t_2)=-\phi(t_2,t_1)\rightarrow\Phi(t,0)=\Phi(-t,0)
    \end{equation}
    \begin{equation}
        \hat{\rho}_s(t)=\text{Tr}_e(\hat{S}(t,t_0)\hat{\rho}_s(t_0)\hat{\rho}_e\hat{S}(t_0,t))=e^{2i\Phi(t,t_0)}\text{Tr}_e(e^{-i\int_{t_0}^t\hat{V}(\tau)d\tau}\hat{\rho}_s(t_0)\hat{\rho}_ee^{i\int_{t_0}^t\hat{V}(\tau)d\tau})
    \end{equation}
    \begin{equation}
        \hat{V}=\hat{\sigma}_z\sum\limits_n\lambda_n(a_ne^{-i\omega_nt}+a^\dagger_ne^{i\omega_nt})
    \end{equation}
    \begin{equation}
        S=-i\int\limits_{t_0}^t\hat{V}(\tau)d\tau=\hat{\sigma}_z\sum\limits_n\frac{\lambda_n}{\omega_n}(a_n(e^{-i\omega_nt_0}-e^{-i\omega_nt})+a^\dagger_n(e^{i\omega_nt}-e^{i\omega_nt_0}))
    \end{equation}
    \begin{equation}
        \hat{\rho}_s(t)=e^{2i\Phi(t,t_0)}\text{Tr}_e(e^S\hat{\rho}_s(t_0)\hat{\rho}_ee^{-S})
    \end{equation}
    \begin{equation}
        \hat{\rho}_s(t_0)=\begin{pmatrix}
        \rho_{11} & \rho_{12}\\
        \rho_{21} & \rho_{22}
        \end{pmatrix}=\hat{\rho}_d+\rho_0,\quad \hat{\rho}_d=\begin{pmatrix}
        \rho_{11} & 0\\
        0 & \rho_{22}
        \end{pmatrix},\quad\hat{\rho}_0=\begin{pmatrix}
        0 & \rho_{12}\\
        \rho_{21} & 0
        \end{pmatrix}
    \end{equation}
    \begin{equation}
        [\rho_d,\sigma_z]=0\rightarrow[\rho_d,e^S]=0
    \end{equation}
    \begin{multline}
        e^{2i\Phi(t,t_0)}\text{Tr}_e(e^S\hat{\rho}_d\hat{\rho}_ee^{-S})=e^{2i\Phi(t,t_0)}\text{Tr}_e(\hat{\rho}_de^S\hat{\rho}_ee^{-S})=\text{Tr}_e(\hat{\rho}_d\hat{S}(t,t_0)\hat{\rho}_e\hat{S}(t_0,t))=\\=\text{Tr}_e(\hat{\rho}_d\hat{\rho}_e(t))=\hat{\rho}_d
    \end{multline}
    \begin{equation}
        \hat{S}=A\hat{\sigma}_z\rightarrow e^{\hat{S}}=\sum\limits_n\frac{A^n}{n!}\hat{\sigma}^n_z=\sum\limits_k\frac{A^{2k}}{(2k)!}+\sum\limits_k\frac{A^{2k+1}}{(2k+1)!}\hat{\sigma}_z=\begin{pmatrix}
        e^A & 0\\
        0 & e^{-A}
        \end{pmatrix}
    \end{equation}
    \begin{equation*}
        e^{2i\Phi(t,t_0)}\text{Tr}_e(e^S\hat{\rho}_0(t_0)e^{-S}e^S\hat{\rho}_e(t_0)e^{-S})=\text{Tr}_e(e^S\hat{\rho}_0(t_0)e^{-S}\hat{\rho}_e(t))=\left<\begin{pmatrix}
        0 & \rho_{12}e^{2A}\\
        \rho_{21}e^{-2A} & 0
        \end{pmatrix}\right>
    \end{equation*}
    \begin{equation}
        \braket{e^{2A}}=e^{2\braket{A^2}},\quad\braket{a^\dagger_na_n}=n_B(\omega_n)=\frac{1}{e^{\beta\omega_n}-1}
    \end{equation}
    \begin{equation}
        \braket{e^{2A}}=\braket{e^{-2A}}=\exp\left(4\sum\limits_n\frac{\lambda^2_n}{\omega^2_n}2\sin^2\frac{\omega_n(t-t_0)}{2}\coth\frac{\beta\omega_n}{2}\right)
    \end{equation}
    \begin{equation*}
        \boxed{\hat{\rho}_s(t)=\begin{pmatrix}
        \rho_{11} & \rho_{12}\exp\left(4\sum\limits_n\frac{\lambda^2_n}{\omega^2_n}2\sin^2\frac{\omega(t-t_0)}{2}\coth\frac{\beta\omega_n}{2}\right)\\
        \rho_{21}\exp\left(4\sum\limits_n\frac{\lambda^2_n}{\omega^2_n}2\sin^2\frac{\omega(t-t_0)}{2}\coth\frac{\beta\omega_n}{2}\right) & \rho_{22}
        \end{pmatrix}}
    \end{equation*}
    \item
    \begin{equation}
        \hat{\rho}_s(t)=\begin{pmatrix}
        \rho_{11} & \rho_{12}\exp(-\Gamma(t))\\
        \rho_{21}\exp(-\Gamma(t)) & \rho_{22}
        \end{pmatrix}
    \end{equation}
    Сравнивая с ответом из предыдущего пункта, получим
    \begin{equation}
        \Gamma(t)=-4\sum\limits_n\frac{\lambda^2_n}{\omega^2_n}2\sin^2\frac{\omega_n(t-t_0)}{2}\coth\frac{\beta\omega_n}{2}=4\sum\limits_n\frac{\lambda^2_n}{\omega^2_n}(\cos\omega_n(t-t_0)-1)\coth\frac{\beta\omega_n}{2}
    \end{equation}
    Спектральная плотность среды:
    \begin{equation}
        J(\omega)=\pi\sum\limits_n\lambda^2_n\delta(\omega-\omega_n)
    \end{equation}
    $t_0=0$.
    \begin{equation}
        \boxed{\Gamma(t)=\frac{4}{\pi}\int\limits_0^\infty d\omega J(\omega)\frac{\cos\omega t-1}{\omega^2}\coth\frac{\omega}{2T}}
    \end{equation}
\end{enumerate}
\textbf{Задача 4 (15 баллов)}\\
Используя функцию декогеренции из предыдущей задачи:
\begin{equation}
    \Gamma(t) = \frac{4}{\pi}\int\limits_0^\infty d\omega J(\omega)\coth\frac{\omega}{2T}\frac{1-\cos\omega t}{\omega^2},
\end{equation}
исследуйте декогеренцию для омической бани, с модельной функцией $J(\omega) = \pi\alpha\omega e^{-\omega/\omega_c}$. Тут $\omega_c$ выступает в роли экспоненциальной ультрафиолетовой обрезки, и можно считать $\omega_c\gg T$.
\textit{Указание:} для вычисления удобно отделить вклад чисто квантовых флуктуаций (при $T = 0$), который зависит от обрезки $\omega_c$; а затем найти <<температурный>> вклад, для которого обрезку уже можно выбросить.\\
\textbf{Решение.}\\
Рассмотрим вклад чисто квантовых флуктуаций ($T=0$):
\begin{equation}
    \Gamma(t)\bigg|_{T=0} = \frac{4}{\pi}\int\limits_0^\infty d\omega \pi\alpha\omega e^{-\omega/\omega_c}\frac{1-\cos\omega t}{\omega^2}=2\alpha\log(1+\omega^2_ct^2)
\end{equation}
Температурный вклад ($T\neq0$). Рассмотрим 2 случая:
\begin{itemize}
    \item $Tt\gg1$. 
    \begin{equation}
        \Gamma(t)=4\alpha\int\limits_0^\infty d\omega\coth\left(\frac{\omega}{2T}\right)e^{-\frac{\omega}{\omega_c}}\frac{1-\cos\omega t}{\omega}\approx4\alpha\int\limits_0^\infty d\omega\frac{2T}{\omega}\frac{1-\cos\omega t}{\omega}=4\pi\alpha tT
    \end{equation}
    \begin{equation}
        \boxed{\Gamma(t)=4\pi\alpha tT}
    \end{equation}
    \item $Tt\ll1$.
    \begin{equation}
        \frac{1-\cos\omega t}{\omega^2}\approx\frac{t^2}{2}
    \end{equation}
    \begin{multline}
        \Gamma(t)-\Gamma(t)\bigg|_{T=0}=4\alpha\int\limits_0^\infty d\omega\left(\coth\left(\frac{\omega}{2T}\right)-1\right)e^{-\frac{\omega}{\omega_c}}\frac{1-\cos\omega t}{\omega}\approx\\\approx4\alpha\int\limits_0^\infty d\omega\left(\coth\left(\frac{\omega}{2T}\right)-1\right)e^{-\frac{\omega}{\omega_c}}\frac{\omega t^2}{2}=4\alpha t^2\left(T^2\psi'\left(\frac{T}{\omega_c}\right)-\omega^2_c\right)
    \end{multline}
    \begin{equation}
        \Gamma(t)-\Gamma(t)\bigg|_{T=0}\approx\frac{2\pi^2\alpha t^2T^2}{3}
    \end{equation}
    \begin{equation}
        \boxed{\Gamma(t)=2\alpha\log(1+\omega^2_ct^2)+\frac{2\pi^2\alpha t^2T^2}{3}}
    \end{equation}
\end{itemize}
\section{Модель Калдейры-Леггетта}
\subsection*{Задачи (100 баллов)}
\textbf{Задача 1. Модель Рубина (25 баллов)}\\
Рассмотрите тяжёлую частицу массы M, которая соединена с баней, моделируемой одномерным полубесконечным кристаллом. В гармоническом приближении, система описывается следующим гамильтонианом:
\begin{equation}
    \hat{H}=\frac{\hat{P}^2}{2M}+V(\hat{X})+\sum\limits_{n=1}^\infty\left(\frac{\hat{p}_n^2}{2m}+\frac{m\omega^2_0}{2}(\hat{u}_{n+1}-\hat{u}_n)^2\right)+\frac{m\omega^2_0}{2}(\hat{X}-\hat{u}_1)^2
\end{equation}
Вычислите явный вид спектральной плотности $J(\omega)$, а также вид <<диссипативного>> ядра $\gamma(t)$.\\
\textbf{Решение.}
\begin{equation}
    \hat{H}_\text{bath}=\sum\limits_{n=1}^\infty\left(\frac{\hat{p}_n^2}{2m}+\frac{m\omega^2_0}{2}(\hat{u}_{n+1}-\hat{u}_n)^2\right)
\end{equation}
Пусть $\hat{u}_0=0$, $\hat{u}_{-i}=-\hat{u}_i$, тогда
\begin{equation}
    \hat{u}_k=\sum\limits_{l=-\infty}^\infty\hat{u}_n\sin kn,\quad\hat{p}_k=\sum\limits_{l=-\infty}^\infty\hat{p}_n\sin kn
\end{equation}
\begin{equation}
    \hat{u}_n=\int\limits_0^\pi\frac{dk}{\pi}\hat{u}_k\sin kn,\quad\hat{p}_n=\int\limits_0^\pi\frac{dk}{\pi}\hat{p}_k\sin kn
\end{equation}
\begin{multline}
    \hat{H}_\text{bath}=\sum\limits_{n=1}^\infty\int\limits_0^\pi\int\limits_0^\pi\frac{dkdk'}{\pi^2}\left(\frac{\hat{p}_k\hat{p}_{k'}\sin kn\sin k'n}{2m}+\frac{m\omega^2_0}{2}(\hat{u}_k\hat{u}_{k'}\sin kn\sin k'n+\right.\\\left.+\hat{u}_k\hat{u}_{k'}\sin k(n+1)\sin k'(n+1)-2\hat{u}_k\hat{u}_{k'}\sin k'n\sin k(n+1))\right)
\end{multline}
\begin{equation}
    \sin kn\sin k'n=\frac{1}{2}(\cos(k-k')n-\cos(k+k')n)
\end{equation}
\begin{multline}
    \sin k'n\sin k(n+1)=\frac{1}{2}(\cos((k'-k)n-k)-\cos((k'+k)n+k))=\\=\frac{1}{2}(\cos(k'-k)n\cos k-\sin(k'-k)n\cos k-\cos((k'+k)n+k))
\end{multline}
\begin{equation}
    \frac{1}{2\pi}\sum\limits_{n=-\infty}^\infty\cos kn=\delta(k)
\end{equation}
\begin{multline}
    \hat{H}_\text{bath}=\int\limits_0^\pi\int\limits_0^\pi\frac{dkdk'}{\pi}\left(\frac{\hat{p}_k\hat{p}_{k'}\delta(k-k')}{2m}+\frac{m\omega^2_0}{2}2\hat{u}_k\hat{u}_{k'}(1-\cos k)\right)=\\=\int\limits_0^\pi\frac{dk}{\pi}\left(\frac{\hat{p}^2_k}{2m}+m\omega^2_0\hat{u}^2_k(1-\cos k)\right)
\end{multline}
\begin{equation}
    \hat{H}_\text{int}=\frac{m\omega^2_0}{2}(\hat{X}-\hat{u}_1)^2-\frac{m\omega^2_0\hat{u}^2_1}{2}=\frac{m\omega^2_0\hat{X}}{2}(\hat{X}-2\hat{u}_1)=\int\limits_0^\pi\frac{dk}{\pi}\frac{m\omega^2_0\hat{X}}{2}(\hat{X}-2\hat{u}_k\sin k)
\end{equation}
\begin{equation}
    \hat{H}=\frac{\hat{P}^2}{2M}+V(\hat{X})+\int\limits_0^\pi\frac{dk}{\pi}\left(\frac{\hat{p}^2_k}{2m}+m\omega^2_0\hat{u}^2_k(1-\cos k)+\frac{m\omega^2_0\hat{X}}{2}(\hat{X}-2\hat{u}_k\sin k)\right)
\end{equation}
\begin{equation}
    \hat{H}=\frac{\hat{P}^2}{2M}+V(\hat{X})+\frac{m\omega^2_0\hat{X}^2}{2}+\int\limits_0^\pi\frac{dk}{\pi}\left(\frac{\hat{p}^2_k}{2m}+m\omega^2_0\hat{u}^2_k(1-\cos k)-m\omega^2_0\hat{X}\hat{u}_k\sin k)\right)
\end{equation}
\begin{equation}
    \hat{H}=\frac{\hat{P}^2}{2M}+V(\hat{X})+\frac{m\omega^2_0\hat{X}^2}{2}+\int\limits_0^\pi\frac{dk}{\pi}\left(\frac{\hat{p}^2_k}{2m}+\frac{m\omega^2(k)\hat{u}^2_k}{2}-c(k)\hat{X}\hat{u}_k)\right)
\end{equation}
\begin{equation}
    \omega(k)=2\omega_0\sin\frac{k}{2},\quad c(k)=m\omega_0^2\sin k
\end{equation}
Спектральная плотность:
\begin{multline}
    J(\omega)=\int\limits_0^\pi dk\frac{c^2(k)}{2m\omega(k)}\delta(\omega-\omega(k))=\frac{m\omega^3_0}{4}\int\limits_0^\pi\frac{\sin^2k}{\sin\frac{k}{2}}\delta\left(\omega-2\omega_0\sin\frac{k}{2}\right)dk=\\=\frac{m\omega^3_0}{2}\int\limits_0^\pi dk\sin\frac{k}{2}\cos^2\frac{k}{2}\frac{\delta(k-2\arcsin\frac{\omega}{2\omega_0})}{|\omega_0\cos\frac{k}{2}|}
\end{multline}
\begin{equation}
    \boxed{J(\omega)=\frac{m\omega_0\omega}{4}\sqrt{1-\frac{\omega^2}{4\omega^2_0}}\theta\left(1-\frac{\omega}{2\omega_0}\right)}
\end{equation}
<<Диссипативное ядро>>:
\begin{equation}
    \gamma(t)=\frac{2}{\pi}\int\limits_0^\infty\frac{d\omega}{\omega}J(\omega)\cos\omega t=\frac{2}{\pi}\int\limits_0^{2\omega_0}d\omega\frac{m\omega_0}{4}\sqrt{1-\frac{\omega^2}{4\omega^2_0}}\cos\omega t
\end{equation}
\begin{equation}
    \boxed{\gamma(t)=\frac{m\omega_0}{4t}J_1(2\omega_0t)}
\end{equation}
\textbf{Задача 2. Кинетическое уравнение (35 баллов)}\\
Используя приближение Борна-Маркова для омической бани со спектральным весом $J(\omega) = \eta\omega$, выведите уравнение на преобразование Вигнера матрицы плотности $\rho_W (\textbf{x}, \textbf{p}, t)$.\\
\textbf{Задача 3. Расплывание волнового пакета (что, опять?) (40 баллов)}\\
\textbf{Упражнение. Преамбула (5 баллов)}\\
Как известно из квантовой механики, волновые пакеты имеют обыкновение расплываться. С другой стороны, как известно, электрон имеет массу $m = 9.1\cdot 10^{-28}g$, и его <<классический радиус>> $x_0\simeq 10^{-13}\text{сm}$ (определяемый как его комптоновская длина волны). Пусть в начальный момент времени электрон представлял собой ровно такой волновой пакет:
\begin{equation}
    \psi(x) = \frac{1}{(2\pi x^2_0)^{1/4}}e^{-x^2/4x^2_0}
\end{equation}
Найдите его ширину через время $t = 1\mu s$.\\
\textbf{Решение.}\\
Перейдём в импульсное представление:
\begin{equation}
    \Psi(p)=\int dxe^{-\frac{ipx}{\hbar}}\psi(x)=(8\pi x^2_0)^\frac{1}{4}e^{-\frac{p^2x^2_0}{\hbar^2}}
\end{equation}
Нестационарное уравнение Шрёдингера в импульсном представлении:
\begin{equation}
    i\hbar\frac{\partial\Psi(p,t)}{\partial t}=\bra{p}\hat{H}(t)\ket{\psi(t)}=\frac{p^2}{2m}\Psi(p,t),\quad \Psi(p,0)=\Psi(p)
\end{equation}
\begin{equation}
    \Psi(p,t)=\Psi(p)\exp\left(-\frac{ip^2}{2m\hbar}t\right)
\end{equation}
\begin{equation}
    \psi(x,t)=\int dxe^{\frac{ipx}{\hbar}}\Psi(p,t)=\left(\frac{2}{\pi}\right)^\frac{1}{4}\frac{\exp\left(-\frac{x^2}{4(x^2_0+\frac{i\hbar}{2m} t)}\right)}{\sqrt{2x_0+\frac{i\hbar t}{mx_0}}}
\end{equation}
\begin{equation}
    \text{Re}\left(\frac{1}{x^2_0+\frac{i\hbar t}{2m}}\right)=\text{Re}\left(\frac{x^2_0-\frac{i\hbar t}{2m}}{x^4_0+\frac{\hbar^2t^2}{4m^2}}\right)=\frac{1}{x^2_0+\frac{\hbar^2t^2}{4m^2x_0^2}}
\end{equation}
\begin{equation}
    \Delta x^2(t)=x^2_0+\frac{\hbar^2t^2}{4m^2x_0^2}\approx\frac{\hbar^2t^2}{4m^2x_0^2}
\end{equation}
\begin{equation}
    \boxed{\Delta x(t)=\frac{\hbar t}{2mx_0}=5.8\cdot10^6\text{cm}}
\end{equation}
\textbf{Задача (35 баллов)}\\
В предыдущем упражнении вы должны были получить ответ, который должен был вас удивить и насторожить. Рассмотрите теперь движение свободного электрона, взаимодействующего с окружающей средой в рамках модели Калдейры-Леггетта с омической баней. Получите асимптотические выражения для $\braket{\hat{x}^2(t)}$ в следующих режимах:
\begin{itemize}
    \item Классическая диффузия: $Tt\gg1$
    \item Квантовая диффузия: $Tt\ll1$
\end{itemize}
Чтобы не рассматривать <<баллистические>> эффекты, оценённые в <<преамбуле>>, можете также считать $\omega_c\gg\gamma\gg t^{-1}$.\\
\textbf{Решение.}\\
\begin{equation}
    m\Ddot{x}+\eta\dot{x}=\zeta(t)\rightarrow m\dot{v}+\eta v=\zeta(t)
\end{equation}
\begin{equation}
    v(t)=e^{-\frac{\eta t}{m}}\left(v_0+\int\limits_0^t d\tau\frac{\zeta(\tau)}{m}e^{\frac{\eta\tau}{m}}\right),\quad v(0)=0
\end{equation}
\begin{equation}
    \dot{x}^2(t)=e^{-\frac{2\eta t}{m}}\int\limits_0^td\tau_1\int\limits_0^td\tau_2\frac{e^{\frac{\eta(\tau_1+\tau_2)}{m}}}{m^2}\zeta(\tau_1)\zeta(\tau_2)
\end{equation}
\section{Функциональный интеграл}
\textbf{Упражнение (10 баллов)}\\
Восстановите координатную зависимость пропагатора квантового гармонического осциллятора $G(x, y, T)$. Исходя из полученного выражения, найдите его спектр.\\
\textbf{Решение.}\\
Лагранжиан гармонического осциллятора:
\begin{equation}
    L=\frac{m}{2}(\dot{x}^{2}-\omega^{2}x^{2})
\end{equation}
\begin{equation}
    G^R(x,y,T)=\int\limits_{x(0)=x}^{x(T)=y}\mathcal{D}[x(t)]e^{\frac{iS[x(t)]}{\hbar}},\quad\mathcal{D}[x(t)]=\left(\frac{m}{2\pi i\epsilon}\right)^{\frac{N}{2}}\prod\limits_{k=1}^{N-1}\int dx_k
\end{equation}
Сделаем подстановку в функциональный интеграл $x(t)=x_\text{cl}(t)+z(t)$, где $x_\text{cl}(t)$ -- классическая траектория, удовлетворяющая граничным условиям $x_\text{cl}(0)=x$, $x_\text{cl}(T)=y$, $z(t)$ -- произвольная функция (квантовая поправка) с граничными условиями $z(0)=0$, $z(T)=0$. Для систем, у которых действие зависит от $x$ квадратично, выполняется
\begin{equation}
    S[x(t)]=S[x_\text{cl}(t)]+S[z(t)]
\end{equation}
\begin{equation}
    S[x(t)]=\int\limits_0^Tdt\frac{m}{2}(\dot{x}(t)^{2}-\omega^{2}x(t)^{2})=\frac{m}{2}\left(x(t)\dot{x}(t)|_0^T-\int\limits_0^Tdtx(t)(\Ddot{x}(t)+\omega^2x(t))\right)
\end{equation}
Пусть $\hat{L}=-\partial^2_t-\omega^2$, тогда
\begin{equation}
    S[x(t)]=\frac{m}{2}\left(x(t)\dot{x}(t)|_0^T+\int\limits_0^Tdtx\hat{L}x\right)
\end{equation}
Поскольку на классической траектории $\hat{L}x_\text{cl}(t)=0$, то в случае $x=y=0$ выполняется $S[x_\text{cl}(t)]=0$ и
\begin{equation}
    G^R(0,0,T)=\int\limits_{x(0)=0}^{x(T)=0}\mathcal{D}[z(t)]e^{\frac{iS[z(t)]}{\hbar}}\rightarrow G^R(x,y,T)=G^R(0,0,T)e^{\frac{iS[x_\text{cl}(t)]}{\hbar}}
\end{equation}
Определим действие на классической траектории:
\begin{equation}
    \hat{L}x_\text{cl}(t)=0\rightarrow x_\text{cl}(t)=a\sin\omega t+b\cos\omega t
\end{equation}
\begin{equation}
    x_\text{cl}(0)=b=x,\quad x_\text{cl}(T)=a\sin\omega T+b\cos\omega T=y
\end{equation}
\begin{equation}
    x_\text{cl}(t)=\frac{y-x\cos\omega T}{\sin\omega T}\sin\omega t+x\cos\omega t,\quad \dot{x}_\text{cl}(t)=\frac{y-x\cos\omega T}{\sin\omega T}\omega\cos\omega t-x\omega\sin\omega t
\end{equation}
\begin{multline}
    S[x_\text{cl}(t)]=\frac{m}{2}(x_\text{cl}(T)\dot{x}_\text{cl}(T)-x_\text{cl}(0)\dot{x}_\text{cl}(0))=\frac{m}{2}\left(\frac{y-x\cos\omega T}{\sin\omega T}y\omega\cos\omega T-xy\omega\sin\omega T\right.-\\-\left.\frac{y-x\cos\omega T}{\sin\omega T}x\omega\right)=\frac{m\omega}{2}\frac{y^2\cos\omega T-xy\cos^2\omega T-xy\sin^2\omega T-yx+x^2\cos\omega T}{\sin\omega T}=\\=\frac{m\omega}{2}\frac{(x^2+y^2)\cos\omega T-2xy}{\sin\omega T}
\end{multline}
Разложим произвольную функцию $z(t)$ по базису собственных функций оператора $\hat{L}$:
\begin{equation}
    z(t)=\sum\limits_nc_nz_n(t),\quad\hat{L}z_n(t)=\lambda_nz_n(t)
\end{equation}
\begin{equation}
    z_n(t)=\sqrt{\frac{2}{T}}\sin\frac{\pi nt}{T},\quad\lambda_n=\frac{\pi^2n^2}{T^2}-\omega^2
\end{equation}
\begin{equation}
    \mathcal{D}[z(t)]=\mathcal{N}\prod\limits_{n=1}^\infty\int\limits_{-\infty}^\infty dc_n
\end{equation}
\begin{equation}
    S[z(t)]=S\left[\sum\limits_nc_nz_n(t)\right]=\frac{m}{2}\int\limits_0^Tdt\left(\sum\limits_nc_nz_n(t)\right)\hat{L}\left(\sum\limits_nc_nz_n(t)\right)
\end{equation}
В силу ортонормированности набора $z_n(t)$ действие факторизуется:
\begin{equation}
    S[z(t)]=\frac{m}{2}\int\limits_0^Tdt\left(\sum\limits_nc^2_n\lambda_nz^2_n(t)\right)=\frac{m}{2}\sum\limits_{n=1}^\infty\lambda_nc_n^2
\end{equation}
\begin{equation}
    G^R_\omega(0,0,T)=\mathcal{N}\int\limits_{-\infty}^\infty\prod\limits_{n=1}^\infty dc_n\exp\left(\frac{im}{2\hbar}\lambda_nc_n^2\right)=\mathcal{N}\prod\limits_{n=1}^\infty\sqrt{\frac{2\pi i\hbar}{m\left(\frac{\pi^2n^2}{T^2}-\omega^2\right)}}
\end{equation}
\begin{equation}
    G^R_0(0,0,T)=\mathcal{N}\prod\limits_{n=1}^\infty\sqrt{\frac{2\pi i\hbar T^2}{m\pi^2n^2}}
\end{equation}
\begin{equation}
    \frac{G^R_\omega(0,0,T)}{G^R_0(0,0,T)}=\prod\limits_{n=1}^\infty\left(1-\frac{\omega^2T^2}{\pi^2n^2}\right)^{-\frac{1}{2}}=\sqrt{\frac{\omega T}{\sin\omega T}},\quad G^R_0(0,0,T)=\sqrt{\frac{m}{2\pi i\hbar T}}
\end{equation}
\begin{equation}
    G^R_\omega(0,0,T)=\sqrt{\frac{m}{2\pi i\hbar T}}\sqrt{\frac{\omega T}{\sin\omega T}}=\sqrt{\frac{m\omega}{2\pi i\hbar\sin\omega T}}
\end{equation}
\begin{equation}
    \boxed{G_\omega^R(x,y,T)=\sqrt{\frac{m\omega}{2\pi i\hbar\sin\omega T}}\exp\left(\frac{im\omega((x^2+y^2)\cos\omega T-2xy)}{2\hbar\sin\omega T}\right)}
\end{equation}
Из знания запаздывающего пропагатора можно извлечь спектр гамильтониана.
\begin{equation}
    G_\omega^R(x,y,T)=\braket{y|e^{-\frac{i\hat{H}T}{\hbar}}|x}=\sum\limits_n\braket{y|n}e^{-\frac{iE_nT}{\hbar}}\braket{n|x}=\sum\limits_n\psi^*_n(x)\psi_n(y)e^{-\frac{iE_nT}{\hbar}}
\end{equation}
Воспользуемся разложениями:
\begin{equation}
    \frac{1}{\sqrt{2i\sin\omega T}}=e^{-\frac{i\omega T}{2}}\sum_{n=0}^{\infty}C^n_{-\frac{1}{2}}e^{-2ni\omega T},\quad -\frac{i}{2\sin\omega T}=e^{-i\omega T}\sum_{n=0}^{\infty}e^{-2ni\omega T}
\end{equation}
\begin{equation}
    i\cot\omega T=-\frac{1+e^{-2i\omega T}}{1-e^{-2i\omega T}}=-\left(1+e^{-2i\omega T}\right)\sum_{n=0}^{\infty}e^{-2ni\omega T}=-2\sum_{n=0}^{\infty}e^{-2in\omega T}
\end{equation}
\begin{multline}
    G_\omega^R(x,y,T)=\sqrt{\frac{m\omega}{2\pi\hbar}}e^{-\frac{i\omega T}{2}}\left(\sum_{n=0}^{\infty}C^n_{-\frac{1}{2}}e^{-2i\omega nT}\right)\sum_{q=0}^{\infty}\frac{1}{q!}\left(\frac{m\omega}{\hbar}\right)^{q}\left(-(x^2+y^2)\left(\sum_{j=0}^{\infty}e^{-2i\omega jT}\right)+\right.\\\left.+2xye^{-i\omega T}\sum_{j=0}^{\infty}e^{-2i\omega jT}\right)^{q}
\end{multline}
\begin{equation}
    \boxed{E_{n}=\hbar\omega\left(n+\frac{1}{2}\right)}
\end{equation}
Также получим спектр, рассмотрев следующую величину
\begin{multline*}
    \text{Tr}\;G_\omega^R(x,y,T)=\int\limits_{-\infty}^\infty dxG_\omega^R(x,x,T)=\sqrt{\frac{m\omega}{2\pi i\hbar\sin\omega T}}\int\limits_{-\infty}^\infty\exp\left(\frac{im\omega x^{2}}{\hbar\sin\omega T}\left(\cos\omega T-1\right)\right)dx=\\=\sqrt{\frac{m\omega}{2\pi i\hbar\sin\omega T}}\sqrt{\frac{\hbar\sin\omega T}{m\omega(1-\cos\omega T)}}\sqrt{\pi}e^{-i\pi/4}=\frac{1}{2i}\frac{1}{\sin\frac{\omega T}{2}}=e^{-\frac{i\omega T}{2}}\sum_{n=0}^{\infty}e^{-i\omega nT}=\sum_{n=0}^{\infty}e^{-\frac{iE_nT}{\hbar}}
\end{multline*}
\textbf{Задача 1. Пропагатор свободной частицы (15 баллов)}\\
Вычислите интеграл для запаздывающего пропагатора свободной частицы явно, используя дискретизацию по времени:
\begin{equation}
    G(x,y,t>0)=\left(\frac{m}{2\pi i\hbar\epsilon}\right)^\frac{N}{2}\int\prod\limits_{k=1}^{N-1}dx_k\exp\left(i\epsilon\sum\limits_{k=1}^N\frac{m}{2\hbar}\left(\frac{x_k-x_{k-1}}{\epsilon}\right)^2\right),\quad x_0\equiv x,\quad x_N\equiv y
\end{equation}
\textbf{Решение.}
Вычислим интеграл по индукции.
\begin{multline*}
    \sqrt{\frac{m}{2\pi\hbar i\epsilon}}\int_{-\infty}^{\infty}\exp\left(\frac{mi}{2\hbar\epsilon}((x_{2}-x_{1})^{2}+(x_{1}-x_{0})^{2})\right)dx_1=\\=\sqrt{\frac{m}{2\pi\hbar i\epsilon}}\int_{-\infty}^{\infty}\exp\left(\frac{mi}{2\pi\hbar\epsilon}\left(2\left(x_{1}-\frac{x_{0}+x_{2}}{2}\right)^{2}+\frac{(x_{2}-x_{0})^{2}}{2}\right)\right)dx_1=\frac{\exp(\frac{mi}{2\cdot2\hbar\epsilon}(x_{2}-x_{0})^{2})}{\sqrt{2}}
\end{multline*}
\begin{multline}
    \sqrt{\frac{m}{2\cdot2\pi\hbar i\epsilon}}\int_{-\infty}^{\infty}\exp\left(\frac{mi}{2\hbar\epsilon}\left((x_3-x_2)^2+\frac{(x_2-x_0)^2}{2}\right)\right)dx_2=\\=\sqrt{\frac{m}{2\cdot2\pi\hbar i\epsilon}}\int_{-\infty}^{\infty}\exp\left(\frac{mi}{2\hbar\epsilon}\left(\frac{3}{2}\left(x_{2}-\frac{x_{0}+2x_{3}}{9}\right)^{2}+\frac{(x_3-x_0)^2}{3}\right)\right)dx_2=\\=\frac{\exp(\frac{mi}{2\cdot3\hbar\epsilon}(x_3-x_{0})^{2})}{\sqrt{3}}
\end{multline}
Для $k$-го интеграла получим
\begin{multline}
    \sqrt{\frac{m}{2k\pi\hbar i\epsilon}}\int_{-\infty}^{\infty}\exp\left(\frac{mi}{2\hbar\epsilon}\left((x_{k+1}-x_k)^2+\frac{(x_k-x_0)^2}{k}\right)\right)dx_k=\\=\sqrt{\frac{m}{2k\pi\hbar i\epsilon}}\int_{-\infty}^{\infty}\exp\left(\frac{mi}{2\hbar\epsilon}\left(\frac{k+1}{k}\left(x_k-\frac{kx_{k+1}+x_0}{k+1}\right)^2+\frac{(x_{k+1}-x_0)^{2}}{k+1}\right)\right)dx_k=\\=\frac{\exp(\frac{mi}{2(k+1)\hbar\epsilon}(x_{k+1}-x_{0})^{2})}{\sqrt{k+1}}
\end{multline}
\begin{equation}
    G(x,y,t>0)=\sqrt{\frac{m}{2\pi i\hbar\epsilon N}}\exp\left(\frac{mi}{2N\hbar\epsilon}(x_N-x_0)^2\right)
\end{equation}
\begin{equation}
    x_0=x,\quad x_N=y,\quad N\epsilon=T
\end{equation}
\begin{equation}
    \boxed{G(x,y,t>0)=\sqrt{\frac{m}{2\pi i\hbar T}}\exp\left(\frac{im(y-x)^2}{2\hbar T}\right)}
\end{equation}
\textbf{Задача 2. Частица в магнитном поле (50 баллов)}\\
Вычислите пропагатор квантовой частицы, движущийся в плоскости в постоянном магнитном поле $B$, перпендикулярном этой плоскости $(\textbf{r}(t) = (x(t), y(t)))$:
\begin{equation}
    G(\textbf{R}_1,\textbf{R}_2,T)=\int\limits_{\textbf{r}(0)=\textbf{R}_1}^{\textbf{r}(T)=\textbf{R}_2}\mathcal{D}[\textbf{r}(t)]\exp\left(\frac{i}{2\hbar}\int\limits_0^T\left(m(\dot{x}^2+\dot{y}^2)+\frac{e}{c}B(x\dot{y}-y\dot{x})\right)dt\right)
\end{equation}
(сперва убедившись, что данное действие действительно описывает именно эту систему). Используя функцию Грина, найдите спектр соответствующей квантовой задачи (решение которой, напомним, даётся макроскопически вырожденными \textit{уровнями Ландау}).\\
\textit{Подсказка:} хотя этот интеграл можно вычислить и непосредственно, может оказаться удобным переход во <<вращающуюся систему отсчёта>> -- провести следующую замену в функциональном интеграле:
\begin{equation}
    \begin{cases}
        x'(t)=x(t)\cos\Omega t+y(t)\sin\Omega t,\\
        y'(t)=-x(t)\sin\Omega t+y(t)\cos\Omega t
    \end{cases}
\end{equation}
и при правильном выборе частоты $\Omega$ задача значительно упрощается. Не забудьте о модификации граничных условий из-за такой замены!\\
\textbf{Решение.}\\
Сделаем замену:
\begin{equation}
    \begin{cases}
        x'(t)=x(t)\cos\Omega t+y(t)\sin\Omega t,\\
        y'(t)=-x(t)\sin\Omega t+y(t)\cos\Omega t
    \end{cases}
\end{equation}
Обратная замена:
\begin{equation}
    \begin{cases}
        x(t)=x'(t)\cos\Omega t-y'(t)\sin\Omega t,\\
        y(t)=x'(t)\sin\Omega t+y'(t)\cos\Omega t
    \end{cases}
\end{equation}
\begin{equation}
    \begin{cases}
        \dot{x}(t)=(\dot{x}'(t)-\Omega y'[t])\cos\Omega t-(\dot{y}'(t)+\Omega x'(t))\sin\Omega t,\\
        \dot{y}(t)=(\dot{x}'(t)-\Omega y'[t])\sin\Omega t+(\dot{y}'(t)+\Omega x'(t))\cos\Omega t
    \end{cases}
\end{equation}
\begin{equation}
    \dot{x}^2(t)+\dot{y}^2(t)=\dot{x}'^2(t)+\dot{y}'^2(t)+\Omega^2(x'^2(t)+y'^2(t))+2\Omega(x'(t)\dot{y}'(t)-y'(t)\dot{x}'(t))
\end{equation}
\begin{equation}
    x(t)\dot{y}(t)-y(t)\dot{x}(t)=\Omega(x'^2(t)+y'^2(t))+x'(t)\dot{y}'(t)-\dot{x}'(t)y'(t)
\end{equation}
\begin{multline}
    m(\dot{x}^2+\dot{y}^2)+\frac{e}{c}B(x\dot{y}-y\dot{x})=m(\dot{x}'^2(t)+\dot{y}'^2(t))+\left(m\Omega^2+\frac{eB\Omega}{c}\right)(x'^2(t)+y'^2(t))+\\+\left(2m\Omega+\frac{eB}{c}\right)(x'(t)\dot{y}'(t)-\dot{x}'(t)y'(t))
\end{multline}
Пусть $\Omega=-\frac{eB}{2mc}$, тогда
\begin{multline}
    m(\dot{x}^2+\dot{y}^2)+\frac{e}{c}B(x\dot{y}-y\dot{x})=m(\dot{x}'^2(t)+\dot{y}'^2(t))+\Omega\frac{eB}{2c}(x'^2(t)+y'^2(t))=m(\dot{x}'^2(t)+\dot{y}'^2(t))-\\-\frac{e^2B^2}{4mc^2}(x'^2(t)+y'^2(t))
\end{multline}
\begin{equation}
    G(\textbf{R}_1,\textbf{R}_2,T)=\int\limits_{\textbf{r}'(0)=\textbf{R}_1}^{\textbf{r}'(T)=\textbf{R}_2}\mathcal{D}[\textbf{r}'(t)]\exp\left(\frac{i}{2\hbar}\int\limits_0^T\left(m(\dot{x}'^2+\dot{y}'^2)-\frac{e^2B^2}{4mc^2}(x'^2+y'^2)\right)dt\right)
\end{equation}
Получился пропагатор с лагранжианом:
\begin{equation}
    L=\frac{m(\dot{x}'^2+\dot{y}'^2)}{2}-\frac{e^2B^2}{8mc^2}(x'^2+y'^2)
\end{equation}
Получившийся потенциал $V(\textbf{r}\;')=\frac{e^2B^2}{8mc^2}(x'^2+y'^2)$ -- гармонический осциллятор в 2D с частотой $\omega=\frac{eB}{2mc}=\Omega$.
\begin{equation}
    G(\textbf{R}_1,\textbf{R}_2,T)=\int\limits_{\textbf{r}\;'(0)=\textbf{R}'_1}^{\textbf{r}\;'(T)=\textbf{R}'_2}\mathcal{D}[\textbf{r}\;'(t)]e^{\frac{iS[\textbf{r}'_\text{cl}(t)]}{\hbar}}
\end{equation}
\begin{equation}
    \int\limits_{\textbf{r}'(0)=\textbf{R}'_1}^{\textbf{r}'(T)=\textbf{R}'_2}\mathcal{D}[\textbf{r}'(t)]=\left(\sqrt{\frac{m\omega}{2\pi i\hbar\sin\omega T}}\right)^2=\frac{m\omega}{2\pi i\hbar\sin\omega T}
\end{equation}
\begin{equation}
    S[\textbf{r}'_\text{cl}(t)]=\frac{m\omega}{2}\frac{(r'^2(0)+r'^2(T))\cos\omega T-2 \textbf{r}'(0)\cdot\textbf{r}'(T)}{\sin\omega T}
\end{equation}
\begin{equation}
    r'^2(0)=x^2(0)+y^2(0)=r^2(0)=R_1^2,\quad r'^2(T)=x^2(T)+y^2(T)=r^2(T)=R_2^2
\end{equation}
\begin{multline}
    \textbf{r}'(0)\cdot\textbf{r}'(T)=(x(0)x(T)+y(0)y(T))\cos\Omega T+(x(0)y(T)-x(T)y(0))\sin\Omega T=\\=(R_{1x}R_{2x}+R_{1y}R_{2y})\cos\Omega T+(R_{1x}R_{2y}-R_{2x}R_{1y})\sin\Omega T
\end{multline}
\begin{equation}
    S[\textbf{r}'_\text{cl}(t)]=\frac{m\Omega}{2}((R^2_1+R^2_2)\ctg\Omega T-2((R_{1x}R_{2x}+R_{1y}R_{2y})\ctg\Omega T+(R_{1x}R_{2y}-R_{2x}R_{1y})))
\end{equation}
\begin{equation}
    S[\textbf{r}'_\text{cl}(t)]=\frac{m\Omega}{2}(((R_{1x}-R_{2x})^2+(R_{1y}-R_{2y})^2)\ctg\Omega T-2(R_{1x}R_{2y}-R_{2x}R_{1y}))
\end{equation}
\begin{equation*}
    \boxed{G(\textbf{R}_1,\textbf{R}_2,T)=\frac{m\Omega}{2\pi i\hbar\sin\Omega T}\exp\left(\frac{im\Omega}{2\hbar}(((R_{1x}-R_{2x})^2+(R_{1y}-R_{2y})^2)\ctg\Omega T-2(R_{1x}R_{2y}-R_{2x}R_{1y}))\right)}
\end{equation*}
\begin{equation*}
    \frac{1}{2i\sin\omega T}=e^{-i\omega T}\sum_{l=0}^{\infty}e^{-2i\omega lT},\quad i\cot\omega T=-\frac{1+e^{-2i\omega T}}{1-e^{-2i\omega T}}=-\left(1+e^{-2i\omega T}\right)\sum_{l=0}^{\infty}e^{-2i\omega lT}=-2\sum_{l=0}^{\infty}e^{-2i\omega lT}
\end{equation*}
Уровни энергии:
\begin{equation}
    E_{n_x,n_y}=\frac{eB\hbar}{mc}(n_x+n_y+1),\quad n_x,n_y\in\mathbb{N}_0
\end{equation}
\textbf{Задача 3. «Мацубаровский» функциональный интеграл (25 баллов)}\\
Рассмотрите квантовую частицу, описываемую гамильтонианом $\hat{H}=\frac{\hat{p}^2}{2m} + V(\hat{x})$. Если такая частица находится в равновесии при температуре $T = \beta^{-1}$, её матрица плотности имеет вид $\hat{\rho}=\frac{1}{Z}e^{-\beta\hat{H}}$, и её статсумма
\begin{equation}
    Z = \text{Tr}(e^{-\beta\hat{H}})
\end{equation}
Используя аналогию между равновесной матрицей плотности и оператором эволюции в мнимом времени $t = -i\tau\Rightarrow\hat{U}(t) = e^{-\hat{H}\tau}$, постройте представление функционального интеграла для статсуммы $Z$. Покажите, что интегрирование должно проводиться по всем траекториям в мнимом времени $x(\tau)$, которые периодичны в мнимом времени: $x(\tau+\beta)\equiv x(\tau)$. Полученное выражение при этом должно получаться формальной заменой $t = -i\tau$ в исходном выражении (такая формальная замена носит название \textit{Виковского поворота}).\\
\textbf{Решение.}\\
Запаздывающая функция Грина -- аплитуда распространения квантовомеханической частицы из точки $\textbf{x}$ в точку $\textbf{y}$ за время $\beta$:
\begin{equation}
    G^R(\textbf{x},\textbf{y},\beta)=\braket{\textbf{y}|\hat{U}(\beta)|\textbf{x}}
\end{equation}
В гамильтониане импульс и координата разделяются. Разобъём отрезок времени длины $\beta$ на $N\rightarrow\infty$ одинаковых кусочков размера $\epsilon=\frac{\beta}{N}\rightarrow0$ и представим оператор эволюции в виде $\hat{U}(\beta)=\prod\limits_{i=1}^N\hat{U}(\epsilon)$. Эволюцию на бесконечно малое время $\epsilon$ можно записать в виде $\hat{U}(\epsilon\rightarrow0)=e^{-\hat{H}\epsilon}\approx e^{-\hat{V}\epsilon}e^{-\hat{K}\epsilon}$. Получаем выражение для оператора эволюции в виде произведения $2N$ множителей (удобно каждому приписать своё время $\tau_k=k\epsilon$):
\begin{equation}
    \hat{U}(\tau)=e^{-\hat{V}\epsilon}e^{-\hat{K}\epsilon}...e^{-\hat{V}\epsilon}e^{-\hat{K}\epsilon}
\end{equation}
Вставим после каждого члена $e^{-\hat{K}\epsilon}$ (при $\tau=\tau_k$) единицу в виде разложения по координатному базису $\mathbb{I}=\int d\textbf{x}_k\ket{\textbf{x}_k}\bra{\textbf{x}_k}$, а после членов $e^{-\hat{V}\epsilon}$ -- по импульсному $\hat{I}=\int(d\textbf{p}_k)\ket{\textbf{p}_k}\bra{\textbf{p}_k}$
\begin{equation*}
    G^R(\textbf{x},\textbf{y},\beta)=\int\prod\limits_{k=1}^{N-1}d\textbf{x}_k\int\prod_{k=1}^N(d\textbf{p}_k)\braket{\textbf{y}|e^{-\hat{V}\epsilon}|\textbf{p}_N}\braket{\textbf{p}_N|e^{-i\hat{K}\epsilon}|\textbf{x}_{N-1}}...\braket{\textbf{x}_1|e^{-\hat{V}\epsilon}|\textbf{p}_1}\braket{\textbf{p}_1|e^{-i\hat{K}\epsilon}|\textbf{x}}
\end{equation*}
Поскольку операторы $e^{-\hat{V}\epsilon}$ -- собственные для координаты, а $e^{-\hat{K}\epsilon}$ -- для импульса, то блоки вычисляются:
\begin{equation}
    \braket{\textbf{x}_k|e^{-\hat{V}\epsilon}|\textbf{p}_k}\braket{\textbf{p}_k|e^{-\hat{K}\epsilon}|\textbf{x}_{k-1}}=e^{-V(\textbf{x}_k)\epsilon}e^{i\textbf{p}_k\textbf{x}_k}e^{-K(\textbf{p}_k)\epsilon}e^{-i\textbf{p}_k\textbf{x}_{k-1}}
\end{equation}
\begin{equation}
    G^R(\textbf{x},\textbf{y},\beta)=\int\prod\limits_{k=1}^{N-1}d\textbf{x}_k\int\prod\limits_{k=1}^N(d\textbf{p}_k)\exp\left(\epsilon\sum\limits_{k=1}^N\left(i\textbf{p}_k\frac{\textbf{x}_k-\textbf{x}_{k-1}}{\epsilon}-\hat{H}(\textbf{x}_k,\textbf{p}_k)\right)\right)
\end{equation}
Возьмём интеграл по импульсам. Кинетическая энергия $K(p)=\frac{p^2}{2m}$, поэтому
\begin{equation}
    \int(d\textbf{p}_k)\exp\left(\epsilon\sum\limits_{k=1}^N\left(i\textbf{p}_k\frac{\textbf{x}_k-\textbf{x}_{k-1}}{\epsilon}-\frac{\textbf{p}_k^2}{2m}\right)\right)=\left(\frac{m}{2\pi\epsilon}\right)^\frac{d}{2}\exp\left(-\epsilon\frac{m}{2}\left(\frac{\textbf{x}_k-\textbf{x}_{k-1}}{\epsilon}\right)^2\right)
\end{equation}
\begin{equation}
    G^R(\textbf{x},\textbf{y},\beta)=\left(\frac{m}{2\pi\epsilon}\right)^\frac{Nd}{2}\int\prod\limits_{k=1}^{N-1}d\textbf{x}_k\exp\left(-\epsilon\sum\limits_{k=1}^N\left(\frac{m}{2}\left(\frac{\textbf{x}_k-\textbf{x}_{k-1}}{\epsilon}\right)^2+V(\textbf{x}_k)\right)\right)
\end{equation}
В экспоненте стоит конечная аппроксимация интеграла:
\begin{equation}
    \sum\limits_{k=1}^N\epsilon\left(\frac{m}{2}\left(\frac{\textbf{x}_k-\textbf{x}_{k-1}}{\epsilon}\right)^2+V(\textbf{x}_k)\right)\approx\int\limits_0^\beta d\tau\left(\frac{m}{2}\dot{\textbf{x}}(\tau)^2+V(\textbf{x}(\tau)\right)
\end{equation}
Выберем меру интегрирования:
\begin{equation}
    \mathcal{D}[\textbf{x}(t)]=\left(\frac{m}{2\pi\epsilon}\right)^\frac{Nd}{2}\prod\limits_{k=1}^{N-1}d\textbf{x}_k
\end{equation}
\begin{equation}
    G^R(\textbf{x},\textbf{y},\beta)=\int\limits_{\textbf{x}(0)=\textbf{x}}^{\textbf{x}(\beta)=\textbf{y}}\mathcal{D}[\textbf{x}(t)]\exp\left(-\int\limits_0^\beta d\tau\left(\frac{m}{2}\dot{\textbf{x}}(\tau)^2+V(\textbf{x}(\tau)\right)\right)
\end{equation}
Матрица плотности в координатном представлении:
\begin{equation}
    \rho(\textbf{x},\textbf{y},\beta)=\braket{\textbf{y}|\hat{\rho}|\textbf{x}}=\int\limits_{\textbf{x}(0)=\textbf{x}}^{\textbf{x}(\beta)=\textbf{y}}\mathcal{D}[\textbf{x}(t)]\exp\left(-\int\limits_0^\beta d\tau\left(\frac{m}{2}\dot{\textbf{x}}(\tau)^2+V(\textbf{x}(\tau)\right)\right)
\end{equation}
Статистическая сумма:
\begin{equation*}
    \boxed{Z=\text{Tr}\rho(\textbf{x},\textbf{y},\beta)=\int d\textbf{x}\rho(\textbf{x},\textbf{x},\beta)=\int d\textbf{x}\int\limits_{\textbf{x}(0)=\textbf{x}}^{\textbf{x}(\beta)=\textbf{x}}\mathcal{D}[\textbf{x}(t)]\exp\left(-\int\limits_0^\beta d\tau\left(\frac{m}{2}\dot{\textbf{x}}(\tau)^2+V(\textbf{x}(\tau)\right)\right)}
\end{equation*}
Как видно, интегрирование идёт по периодическим траекториям $x(\tau)=x(\tau+\beta)$. Заметим, что полученное выражение получается из исходного при помощи Виковского поворота $t=-i\tau$.
\section{Инстантоны и туннелирование}
\textbf{Упражнение. Квазиклассика (30 баллов)}\\
Для двухъямного потенциала, что разбирался на семинаре, вычислите расщепление уровней в квазиклассическом приближении. Насколько квазиклассическое приближение отличается от правильного ответа, даваемого инстантонным вычислением?\\
\textit{Указание:} стандартная формула в квазиклассическом приближении гарантирует правильный численный фактор в предэкспоненте; поэтому именно с такой точностью необходимо вычислить и квазиклассическое действие. Обратите внимание, что действие необходимо считать на энергии, отличной от минимума в соответствующей яме; поправка от этого отличия существенна, поскольку она стоит в экспоненте, и имеется серия неравенств $S(0) \gg |S(E)-S(0)| \gg 1$. Несложно убедиться, что эта поправка логарифмическая; причём \textbf{вычисление необходимо проводить с точностью до числа под логарифмом} (поскольку это число в конечном итоге модифицирует численный префактор в конечном ответе!).\\
Сделать это можно следующим образом. Во-первых, технически чуть проще вычислять не само действие $S(E)$, а его производную $\partial S/\partial E$, которая логарифмически расходится при $E\rightarrow0$. Для вычисления численного фактора под логарифмом, предлагается ввести произвольный промежуточный пространственный масштаб $\Lambda$, удовлетворяющий серии неравенств $|x_1 + \eta|\ll\Lambda\ll\eta$ (где $x_1$ -- положение левой точки остановки, которая близка к минимуму $-\eta$), и разбить интеграл на два интервала $\int\limits_{x_1}^\Lambda+\int\limits_\Lambda^0$. В первом интеграле первое же неравенство позволяет использовать осцилляторное приближение для потенциала; а во втором интеграле второе неравенство позволяет использовать разложение в ряд Тейлора по энергии. После соответствующих вычислений (с необходимой точностью!), промежуточный масштаб $\Lambda$, в силу своей произвольности, из ответа должен сократиться.\\
\textbf{Решение.}\\
Гамильтониан:
\begin{equation}
    \hat{H}=-\frac{1}{2}\partial_x^2+\lambda(x^2-\eta^2)^2
\end{equation}
Действие в квазиклассическим приближении:
\begin{equation}
    S=2\sqrt{2}\int\limits_{-\eta+\delta\eta}^0\sqrt{\lambda(x^2-\eta^2)^2-E}dx,\quad E=4\lambda\eta^2\delta\eta^2
\end{equation}
Разобьём пределы интегрирования на 2:
\begin{multline}
    S=2\sqrt{2}\int\limits_{-\eta+\delta\eta}^{-\eta+\Lambda}\sqrt{\lambda(x^2-\eta^2)^2-E}dx+2\sqrt{2}\int\limits_{-\eta+\Lambda}^0\sqrt{\lambda(x^2-\eta^2)^2-E}dx=\\=2\sqrt{2}\int\limits_{\delta\eta}^{\Lambda}\sqrt{4\lambda\eta^2y^2-E}dy+2\sqrt{2}\int\limits_{-\eta+\Lambda}^0\sqrt{\lambda(x^2-\eta^2)^2-E}dx
\end{multline}
В 1 интеграле ($S_1$) работает осцилляторное приближение, поскольку $\Lambda\ll\eta$. Вычислим его, воспользовавшись уазанием:
\begin{equation*}
    \frac{\partial S_1}{\partial E}=-2\sqrt{2}\int\limits_{\delta\eta}^{\Lambda}\frac{dy}{2\sqrt{4\lambda\eta^2y^2-E}}=-\frac{\sqrt{2}}{2\sqrt{\lambda}\eta}\int\limits_1^{\frac{\Lambda}{\delta\eta}}\frac{dz}{\sqrt{z^2-1}}=-\frac{\sqrt{2}}{2\sqrt{\lambda}\eta}\log\left(\frac{\Lambda}{\delta\eta}+\sqrt{\left(\frac{\Lambda}{\delta\eta}\right)^2-1}\right)
\end{equation*}
\begin{equation}
    S_1=-\frac{\sqrt{2}}{2\sqrt{\lambda}\eta}\int\log\left(\frac{\Lambda}{\delta\eta}+\sqrt{\left(\frac{\Lambda}{\delta\eta}\right)^2-1}\right)dE,\quad dE=8\lambda\eta^2\delta\eta d(\delta\eta)
\end{equation}
\begin{multline}
    S_1=-4\sqrt{2\lambda}\eta\int\limits_0^{\delta\eta}\delta\eta\log\left(\frac{\Lambda}{\delta\eta}+\sqrt{\left(\frac{\Lambda}{\delta\eta}\right)^2-1}\right)d(\delta\eta)=-2\sqrt{2\lambda}\eta\left(\Lambda(\Lambda-\sqrt{\Lambda^2-(d\eta)^2})+\right.\\\left.+(\delta\eta)^2\log\left(\frac{\Lambda}{\delta\eta}+\sqrt{\left(\frac{\Lambda}{\delta\eta}\right)^2-1}\right)\right)\approx2\sqrt{2\lambda}\eta^2-\sqrt{2\lambda}\eta(\delta\eta)^2\left(1+2\log\left(\frac{2\Lambda}{\delta\eta}\right)\right)
\end{multline}
Во 2 интеграле ($S_2$) будет раскладывать в ряд по степеням $\frac{E}{\lambda(\eta^2-x^2)^2}$.
\begin{multline*}
    S_2=2\sqrt{2}\int\limits_{-\eta+\Lambda}^0\sqrt{\lambda(x^2-\eta^2)^2-E}dx\approx2\sqrt{2\lambda}\int\limits_{-\eta+\Lambda}^0|x^2-\eta^2|\left(1-\frac{E}{4(\eta^2-x^2)^2}\right)dx=\\=2\sqrt{2\lambda}\left(\eta^2(\Lambda-\eta)-\frac{(\Lambda-\eta)^3}{3}-\frac{E\sqrt{\lambda}}{2}\int\limits_{-\eta+\Lambda}^0\frac{dx}{\lambda(\eta^2-x^2)}\right)=\\=2\sqrt{2\lambda}\left(\frac{2}{3}\eta^3-\eta\Lambda^2+\frac{\Lambda^3}{3}-\frac{E\log\left(-1+\frac{2\eta}{\Lambda}\right)}{4\eta\lambda}\right)=2\sqrt{2\lambda}\left(\frac{2}{3}\eta^3-\eta\Lambda^2+\frac{\Lambda^3}{3}\right)-\\-\frac{E\sqrt{2}\left(\log\left(\frac{2\eta}{\Lambda}\right)-\frac{\Lambda}{2\eta}\right)}{2\eta\sqrt{\lambda}}=2\sqrt{2\lambda}\left(\frac{2\eta^3}{3}-\eta\Lambda^2+\frac{\Lambda^3}{3}-\eta(\delta\eta)^2\left(\log\left(\frac{2\eta}{\Lambda}\right)-\frac{\Lambda}{2\eta}\right)\right)
\end{multline*}
\begin{equation}
    S=S_1+S_2=\frac{2}{3}\omega\eta^2-\log\left(\frac{4\eta}{\delta\eta}\right)-\frac{1}{2},\quad\omega=8\lambda\eta^2(\delta\eta)^2
\end{equation}
Расщепление уровней:
\begin{equation}
    E=\frac{\omega}{2}\pm\frac{\omega}{2\pi}e^{-S}=\frac{\omega}{2}\pm\frac{\omega}{2\pi}e^{-S}=\frac{\omega}{2}\pm\frac{\omega}{2\pi}\exp\left(-\frac{2\omega\eta^2}{3}\right)\frac{4\eta}{\delta\eta}e^{\frac{1}{2}}
\end{equation}
\begin{equation}
    \boxed{\Delta=\frac{2\sqrt{e}}{\pi}\omega^\frac{3}{2}\eta e^{-\frac{2\omega\eta^2}{3}}}
\end{equation}
Ответ отличается от инстантонного в $\sqrt{\frac{e}{\pi}}$ раз.\\
\textbf{Задача. <<Математический маятник>> (70 баллов)}\\
Рассмотрите движение частицы массы $m = 1$ в периодическом потенциале $U(x) = \lambda(1-\cos\frac{x}{\eta})$. Вычислите одноинстантонный вклад в амплитуду перехода между двумя минимумами $0 \rightarrow 2\pi\eta$ за большое мнимое время $\beta$ -- Евклидову функцию Грина $G_E(2\pi\eta, 0, \beta)$.
\begin{enumerate}
    \item \textbf{(5 баллов)} Найдите инстантонную траекторию, соответствующую такому переходу, вычислите действие на ней.
    Вычислите явно оператор, определяющий квадратичные флуктуации в окрестности инстантона.
    \item \textbf{(10 баллов)} Исследуйте спектр полученного оператора. Вычислите фазовые сдвиги непрерывного спектра и уровни энергии связанных состояний.
    \item \textbf{(10 баллов)} Рассмотрите отнормированный на осциллятор пропагатор. Идентифицируйте нулевую моду, выполните интегрирование по ней.
    \item \textbf{(35 баллов)} Вычислите вклад в отношение определителей от дискретного и непрерывного спектра.
    \item \textbf{(10 баллов)} Просуммируйте инстантонный газ для искомой амплитуды перехода. Сравнив полученный ответ с ответом для соответствующей модели сильной связи, извлеките из полученного ответа ширину зоны непрерывного спектра соответствующей квантомеханической задачи.
\end{enumerate}
\textbf{Решение.}
\begin{enumerate}
    \item Потенциал имеет минимумы в точках $x=2\pi\eta n,n\in\mathbb{Z}$, вблизи которых он имеет осцилляторное разложение $U(x)=\frac{1}{2}\omega^2x^2$ с частотами $\omega^2=\frac{\lambda}{\eta^2}$. Евклидово действие:
    \begin{equation}
        S_E[x(\tau)]=\int\left(\frac{1}{2}(\partial_\tau x)^2+U(x(\tau))\right)d\tau
    \end{equation}
    Нужно найти экстремум действия с граничными условиями $x(\tau\rightarrow-\infty)=0$, $x(\tau\rightarrow\infty)=2\pi\eta$.
    \begin{equation}
        \frac{\delta S_E}{\delta x(\tau)}=0\rightarrow \partial^2_\tau x=U'(x(\tau))
    \end{equation}
    Закон сохранения энергии:
    \begin{equation}
        E=\frac{1}{2}(\partial_\tau x)^2-U(x(\tau))=\text{const}
    \end{equation}
    На нашей траектории $E=0$, поэтому
    \begin{equation}
        \frac{1}{2}(\partial_\tau x)^2=U(x(\tau))\rightarrow\frac{dx}{\sqrt{1-\cos\frac{x}{\eta}}}=\sqrt{2\lambda}d\tau
    \end{equation}
    \begin{equation}
        \sqrt{2}\eta\ln\left(\tan\frac{x}{4\eta}\right)=\sqrt{2\lambda}\tau\rightarrow \boxed{x_\text{cl}(\tau)=4\eta\arctan\left(e^{\omega\tau}\right)}
    \end{equation}
    Действие на инстантоне:
    \begin{equation}
        S_0=S_E[x_\text{cl}(\tau)]=\int\limits_{-\infty}^\infty(\partial_\tau x_\text{cl})^2d\tau=\int\limits_{-\infty}^\infty\frac{4\lambda}{\cosh^2\left(\frac{\sqrt{\lambda}\tau}{\eta}\right)}d\tau
    \end{equation}
    \begin{equation}
        \boxed{S_0=8\eta\sqrt{\lambda}=8\eta^2\omega}
    \end{equation}
    Квадратичное разложение вблизи потенциала:
    \begin{equation}
        S_E[x_\text{cl}(\tau)+z(\tau)]\approx S_0+\frac{1}{2}\int\limits_0^\beta d\tau((\partial_\tau z)^2+U''(x_\text{cl}(\tau))z^2(\tau))
    \end{equation}
    \begin{equation}
        U''(x_\text{cl}(\tau))=\frac{\lambda}{\eta^2}\cos\frac{x_\text{cl}}{\eta}=\omega^2\cos\left(4\arctan\left(e^{\omega\tau}\right)\right)
    \end{equation}
    В квадратичном разложении действия стоит линейный оператор:
    \begin{equation}
        \hat{L}=-\partial^2_\tau+U''(x_\text{cl}(\tau))=-\partial^2_\tau+\omega^2\cos\left(4\arctan\left(e^{\omega\tau}\right)\right)
    \end{equation}
    \begin{equation}
        \boxed{\hat{L}=-\partial^2_\tau+\omega^2\left(1-\frac{2}{\cosh^2\omega\tau}\right)}
    \end{equation}
    \item Уравнение на спектр оператора $\hat{L}$:
    \begin{equation}
        \left(-\partial^2_\tau-\frac{2\omega^2}{\cosh^2\omega\tau}\right)\psi_n(\tau)=(\epsilon_n-\omega^2)\psi_n(\tau)
    \end{equation}
    Данное уравнение -- уравнение Шрёдингера в потенциале $\frac{1}{\cosh^2x}$, которое было разобрано в семинаре 4. Воспользуемся ответами из того семинара, в которые нужно подставить следующие значения параметров:
    \begin{equation}
        m=\frac{1}{2},\quad a=\frac{1}{\omega},\quad U_0=2\omega^2
    \end{equation}
    \begin{equation}
        u=\frac{U_0}{1/2ma^2}=2,\quad s=\frac{1}{2}(\sqrt{1+4u}-1)=1
    \end{equation}
    Энергии связанных состояний:
    \begin{equation}
        E_n=-\frac{(s-n)^2}{2ma^2}=-\omega^2(s-n)^2\rightarrow E_0=-\omega^2
    \end{equation}
    \begin{equation}
        \boxed{\epsilon_0=E_0+\omega^2=0}
    \end{equation}
    Волноваяя функция нулевой моды:
    \begin{equation}
        \psi_0(\tau)=\frac{C_0}{\cosh\omega\tau}
    \end{equation}
    При целом $s=1$ рассеяние оказывается безотражательным. Непрерывный спектр характеризуется волновым вектором $k$, так что энергия $E_k=\frac{k^2}{2m}=k^2\rightarrow\epsilon_k=\omega^2+k^2$. Волновые функции:
    \begin{equation}
        \psi_k(x)\approx\begin{cases}
            \frac{\Gamma(1-ika)\Gamma(-ika)}{\Gamma(-ika-1)\Gamma(-ika+2)}e^{ikx}\equiv e^{ikx+i\delta(k)},\quad x\rightarrow-\infty\\
            e^{ikx},\quad\quad\quad\quad\quad\quad\quad\quad\quad\quad\quad\quad\quad\; x\rightarrow+\infty
        \end{cases}
    \end{equation}
    \begin{equation}
        e^{-i\delta(k)}=\frac{\Gamma(-ika-1)\Gamma(-ika+2)}{\Gamma(1-ika)\Gamma(-ika)}=\frac{1-ika}{-1-ika}=-\frac{1-ika}{1+ika}=e^{-i\pi-2i\arctan ka}
    \end{equation}
    \begin{equation}
        \boxed{\delta(k)=2\arctan ka-\pi=\begin{cases}
            \pi,\quad k=0\\
            0,\quad k\rightarrow\infty
        \end{cases}}
    \end{equation}
    Дискретизация непрерывного спектра. Волновая функция имеет вид $\psi_n(x)=A\psi_k(x)+B\psi^*_k(x)$, тогда граничные условия:
    \begin{equation}
        \begin{cases}
            A\psi_k(0)+B\psi^*_k(0)=0,\\
            A\psi_k(\beta)+B\psi^*_k(\beta)=0
        \end{cases}
    \end{equation}
    Условие совместности системы:
    \begin{equation}
        \det\begin{pmatrix}
        \psi_k(0) & \psi^*_k(0)\\
        \psi_k(\beta) & \psi^*_k(\beta)
        \end{pmatrix}=0
    \end{equation}
    Подставляем асимптотики:
    \begin{equation}
        \det\begin{pmatrix}
        1 & 1\\
        e^{ik\beta-i\delta(k)} & e^{-ik\beta+i\delta(k)}
        \end{pmatrix}=0\rightarrow k_n\beta-\delta(k_n)=\pi n
    \end{equation}
    Решим трансцендентное уравнение по теории возмущений:
    \begin{equation}
        k^{(0)}_n\approx p_n=\frac{\pi n}{\beta}\rightarrow k^{(1)}_n\approx p_n+\frac{\delta(p_n)}{\beta}
    \end{equation}
    \item Функция Грина гармонического осциллятора:
    \begin{equation}
        G_0(\beta)=\int\limits_{z(0)=0}^{z(\beta)=0}\mathcal{D}[z(\tau)]\exp\left(-\frac{1}{2}\int\limits_0^\beta((\partial_\tau z)^2+\omega^2z^2)\right)
    \end{equation}
    Отношение функций Грина:
    \begin{equation}
        \frac{G(2\pi\eta,0,\beta)}{G_0(\beta)}=e^{-S_0}\frac{\int\limits_{z(0)=0}^{z(\beta)=0}\mathcal{D}[z(\tau)]\exp\left(-\frac{1}{2}\int\limits_0^\beta d\tau((\partial_\tau z)^2+U''(x_\text{cl}(\tau))z^2(\tau))\right)}{\int\limits_{z(0)=0}^{z(\beta)=0}\mathcal{D}[z(\tau)]\exp\left(-\frac{1}{2}\int\limits_0^\beta d\tau((\partial_\tau z)^2+\omega^2z^2(\tau))\right)}
    \end{equation}
    Спектр гармонического осциллятора:
    \begin{equation}
        \epsilon_p^{(0)}=\omega^2+p^2,\quad p=p_n=\frac{\pi n}{\beta}
    \end{equation}
    Разложене в числителе $z(\tau)=\sum\limits_{n}c_n\psi_n(\tau)$ и в знаменателе $z(\tau)=\sum\limits_nc_n\psi^{(0)}_n(\tau)$.
    Интеграл по нулевой моде:
    \begin{equation}
        \int dc_0=\int d\tau_c\frac{dc_0}{d\tau_c}=\sqrt{S_0}\beta
    \end{equation}
    \begin{equation}
        \int dc_1e^{-\frac{1}{2}\epsilon^{(0)}_1c^2_1}=\sqrt{\frac{2\pi}{\epsilon^{(0)}_1}}=\sqrt{\frac{2\pi}{\omega^2+\frac{\pi^2}{\beta^2}}}\approx\frac{\sqrt{2\pi}}{\omega}
    \end{equation}
    \begin{equation}
        \boxed{\frac{G(2\pi\eta,0,\beta)}{G_0(\beta)}=e^{-S_0}\omega\beta\sqrt{\frac{S_0}{2\pi}}\left(\frac{\det'(-\partial^2_\tau+U''(x_\text{cl}(\tau)))}{\det'(-\partial^2_\tau+\omega^2)}\right)^{-\frac{1}{2}}}
    \end{equation}
    \item
    \begin{multline}
        \frac{\det'}{\det'}=\prod\limits_{n=1}^\infty\frac{\omega^2+k_n^2}{\omega^2+p^2_{n+1}}=\exp\left(\sum\limits_{n=1}^\infty\log\frac{\omega^2+k_n^2}{\omega^2+p^2_{n+1}}\right)=\exp\left(\sum\limits_{n=1}^\infty\log\left(1+\frac{k_n^2-p_{n+1}^2}{\omega^2+p^2_{n+1}}\right)\right)\approx\\\approx\exp\left(\sum\limits_{n=1}^\infty\frac{2p_{n+1}(k_n-p_{n+1})}{\omega^2+p^2_{n+1}}\right)=\exp\left(\frac{1}{\beta}\sum\limits_{n=1}^\infty\frac{2p_{n+1}(\delta(p_n)-\pi)}{\omega^2+p^2_{n+1}}\right)
    \end{multline}
    Для суммирования плавной функции заменим сумму на интеграл $\sum\limits_n=\int\limits_0^\infty\frac{\beta dp}{\pi}$:
    \begin{equation}
        \frac{\det'}{\det'}=\exp\left(\int\limits_0^\infty\frac{dp2p(\delta(p)-\pi)}{\pi(\omega^2+p^2)}\right)=\exp\left(-\frac{1}{\pi}\int\limits_0^\infty dp\frac{d\delta}{dp}\log\frac{\omega^2+p^2}{\omega^2}\right)
    \end{equation}
    \begin{equation}
        \boxed{\frac{\det'}{\det'}=\frac{1}{4}}
    \end{equation}
    Одноинстантонный ответ:
    \begin{equation}
        \boxed{\frac{G(2\pi\eta,0,\beta)}{G_0(\beta)}=e^{-S_0}\omega\beta\sqrt{\frac{S_0}{2\pi}}\sqrt{4}=\sqrt{\frac{S_0}{\pi}}\omega\beta e^{-S_0}=\sqrt{\frac{8\eta^2\omega}{\pi}}\omega\beta e^{-8\eta^2\omega}}
    \end{equation}
    \item В случае $n$ инстантонов вклад $S_n=nS_0$, полная фаза рассеяния $\delta^{(n)}(p)=n\delta(p)$. С каждым из инстантонов связана нулевая мода
    \begin{equation}
        \int\limits_{0<\tau_1<...<\tau_n<\beta}d\tau_1...d\tau_n=\frac{\beta^n}{n!}
    \end{equation}
    Для перехода $0\rightarrow 2\pi\eta$ только конфигурации с нечётным количеством интсантонов дают вклад.
    \begin{equation}
        \boxed{\frac{G(2\pi\eta,0,\beta)}{G_0(\beta)}=\sum\limits_{n=0}^\infty\frac{С^n_{2n+1}}{(2n+1)!}\left(\sqrt{\frac{8\eta^2\omega}{\pi}}\omega\beta e^{-8\eta^2\omega}\right)^{2n+1}}
    \end{equation}
\end{enumerate}
\section{Формализм Гельфанда-Яглома}
\textbf{Упражнение. «Математический маятник», продолжение (10 баллов)}\\
Вычислите функциональный детерминант, возникающий при исследовании задачи из предыдущего семинара, и сравните его с полученным непосредственным вычислением.\\
\textbf{Решение.}\\
\textbf{Задача 1. Модификации Гельфанда-Яглома (50 баллов)}
Покажите, как нужно модифицировать теорему Гельфанда-Яглома для вычисления следующих определителей типа уравнения Шрёдингера:
\begin{enumerate}
    \item\textbf{(30 баллов)} Одномерный оператор $\hat{H}=-\partial^2_x+U(x)$, но для случая периодических граничных условий $\psi(x)\equiv\psi(x + L)$. Вычислите отношение определителей $\frac{\det(-\partial^2_x+k^2_1)}{\det(-\partial^2_x+k^2_2)}$.
    \item\textbf{(10 баллов)} Двумерный оператор $\hat{H} =-\nabla^2 + U(r)$ для сферически-симметричного потенциала.
    \item\textbf{(10 баллов)} Аналогично, трёхмерный случай.
\end{enumerate}
\textbf{Решение.}
\begin{enumerate}
    \item 
    \item Двумерный оператор:
    \begin{equation}
        \hat{H}=-\nabla^2+U(r)=-\partial^2_r-\frac{1}{r}\partial_r-\frac{1}{r^2}\partial^2_\varphi+U(r)
    \end{equation}
    Уравнение на собственные значения:
    \begin{equation}
        \hat{H}\psi=\lambda\psi
    \end{equation}
    Сведём уравнение к одномерному. Разложим волновую функцию по базису $\psi(r,\varphi)=\sum\limits_{m=-\infty}^\infty f_m(r)e^{im\varphi}$ и получим
    \begin{equation}
        -\partial^2_rf_m(r)-\frac{1}{r}\partial_rf_m(r)+\left(\frac{m^2}{r^2}+U(r)-\lambda\right)f_m(r)=0
    \end{equation}
    Модификация Гельфанда-Яглома для двумерного сферически-симметричного потенциала:
    \begin{equation}
        \boxed{\frac{\det(\hat{H}_1-\lambda)}{\det(\hat{H}_2-\lambda)}=\prod\limits_{m=-\infty}^\infty\frac{f^1_m(R|\lambda)}{f^2_m(R|\lambda)}=\frac{f^1_0(R|\lambda)}{f^2_0(R|\lambda)}\prod\limits_{m=1}^\infty\left(\frac{f^1_m(R|\lambda)}{f^2_m(R|\lambda)}\right)^2}
    \end{equation}
    где $f_m^{1,2}(r|\lambda)$ -- решения задачи
    \begin{equation}
        -\partial^2_rf^{1,2}_m(r|\lambda)-\frac{1}{r}\partial_rf^{1,2}_m(r|\lambda)+\left(\frac{m^2}{r^2}+U_{1,2}(r)-\lambda\right)f^{1,2}_m(r|\lambda)=0
    \end{equation}
    с граничными условиями $f^{1,2}_m(0|\lambda)=0$, $\partial_rf^{1,2}_m(0|\lambda)=1$.
    \item Трёхмерный оператор:
    \begin{equation}
        \hat{H}=-\nabla^2+U(r)
    \end{equation}
    Уравнение на собственные значения:
    \begin{equation}
        \hat{H}\psi=\lambda\psi
    \end{equation}
    Сведём уравнение к одномерному. Разложим волновую функцию по базису $\psi(r,\theta,\varphi)=\sum\limits_{l=0}^\infty\sum\limits_{m=-l}^l R_l(r)Y_{l,m}(\theta,\varphi)$ и получим
    \begin{equation}
        -\partial^2_rR_l(r)-\frac{2}{r}\partial_rR_l+\left(\frac{l(l+1)}{r^2}+U(r)-\lambda\right)R_l(r)=0
    \end{equation}
    Модификация Гельфанда-Яглома для трёхмерного сферически-симметричного потенциала:
    \begin{equation}
        \boxed{\frac{\det(\hat{H}_1-\lambda)}{\det(\hat{H}_2-\lambda)}=\prod\limits_{l=0}^\infty\prod\limits_{m=-l}^l\frac{R^1_l(R|\lambda)}{R^2_l(R|\lambda)}=\prod\limits_{l=0}^\infty\left(\frac{R^1_l(R|\lambda)}{R^2_l(R|\lambda)}\right)^{2l+1}}
    \end{equation}
    где $R^{1,2}_l(r|\lambda)$ -- решения задачи
    \begin{equation}
        -\partial^2_rR^{1,2}_l(r)-\frac{2}{r}\partial_rR^{1,2}_l+\left(\frac{l(l+1)}{r^2}+U_{1,2}(r)-\lambda\right)R^{1,2}_l(r)=0
    \end{equation}
    с граничными условиями $R^{1,2}_l(0|\lambda)=0$, $\partial_rR^{1,2}_l(0|\lambda)=1$.
\end{enumerate}
\textbf{Задача 2. Дзета-функция Бесселя? (40 баллов)}
\begin{enumerate}
    \item Вычислите следующее бесконечное произведение, в котором $\mu^{(m)}_n$ -- $n$-тый нуль $m$-той функции Бесселя $J_m(z)$ (очевидным образом $J_{m>0}(0) = 0$; и этот тривиальный нуль в произведении не участвует):
    \begin{equation}
        P_m(R)=\prod\limits_{n=1}^\infty\frac{1+(\mu^{(m)}_n/R)^2}{(\mu^{(m)}_n/R)^2}
    \end{equation}
    Для этого найдите операторы, которые дают соответствующие собственные числа, а затем вычислите их отношения, используя теорему Гельфанда-Яглома.
    \item Пусть $\lambda_n$ -- все собственные числа эрмитового оператора $\hat{H}$. Покажите, что <<дзета-функцию>> оператора $\hat{H}$, которую мы определим следующим образом:
    \begin{equation}
        \zeta_H(s)=\sum\limits_{n=0}\frac{1}{\lambda_n^s}
    \end{equation}
    для натурального аргумента $s\in\mathbb{N}$ можно сосчитать следующим образом:
    \begin{equation}
        \zeta_H(s)=\frac{(-1)^{s-1}}{(s-1)!}\frac{\partial^s}{\partial\lambda^s}\left(\log\frac{\det(\hat{H}+\lambda)}{\det{\hat{H}}}\right)\bigg|_{\lambda=0}
    \end{equation}
    \item Докажите следующие равенства:
    \begin{equation}
        \sum\limits_{n=1}^\infty\frac{1}{(\mu^{(0)}_n)^2}=\frac{1}{4},\quad\sum\limits_{n=1}^\infty\frac{1}{(\mu^{(0)}_n)^4}=\frac{1}{32},\quad\sum\limits_{n=1}^\infty\frac{1}{(\mu^{(1)}_n)^2}=\frac{1}{8}
    \end{equation}
\end{enumerate}
\textbf{Решение.}
\begin{enumerate}
    \item Рассмотрим гамильтониан:
    \begin{equation}
        \hat{H}_m=-\partial^2_r-\frac{1}{r}\partial_r+\frac{m^2}{r^2}
    \end{equation}
    Уравнение на собственные значения:
    \begin{equation}
        \hat{H}_m\psi_m=\lambda\psi_m,\quad\psi_m(0)=\psi_m(R)=0
    \end{equation}
    \begin{equation}
        -\partial^2_r\psi_m-\frac{1}{r}\partial_r\psi_m+\frac{m^2}{r^2}\psi_m=\lambda\psi_m
    \end{equation}
    Пусть $z=\sqrt{\lambda}r$ и $f(z)=\psi(z(r))$, тогда
    \begin{equation}
        z^2\partial^2_zf_m+z\partial_zf_m+(z^2-m^2)f_m=0
    \end{equation}
    Получилась функция Бесселя (функция Неймана сингулярна в 0):
    \begin{equation}
        f_m(r|\lambda_m)=C_1J_m(z)\rightarrow\psi_m(r|\lambda_m)=C_1J_m(\sqrt{\lambda_m}r)
    \end{equation}
    \begin{equation}
        \psi_m(0|\lambda)=0,\quad\psi_m(R|\lambda)=0\rightarrow\sqrt{\lambda_n}R=\mu^{(m)}_n
    \end{equation}
    \begin{equation}
        \lambda_n=\frac{(\mu^{(m)}_n)^2}{R^2}
    \end{equation}
    \begin{equation}
        P_m(R)=\prod\limits_{n=1}^\infty\frac{1+(\mu^{(m)}_n/R)^2}{(\mu^{(m)}_n/R)^2}=\frac{\det(\hat{H}_m+1)}{\det(\hat{H}_m)}
    \end{equation}
    По теореме Гельфанда-Яглома:
    \begin{equation}
        P_m(R)=\frac{\det(\hat{H}_m+1)}{\det(\hat{H}_m)}=\frac{\psi^{(1)}_m(R|\lambda_0)}{\psi_m(R|\lambda_0)},\quad\lambda_0\rightarrow0
    \end{equation}
    где $\psi_m(r|\lambda_0)=CJ_m(\sqrt{\lambda_0}r)$, а $\psi^{(1)}_m(r|\lambda_0)$ -- решение задачи
    \begin{equation}
        -\partial^2_r\psi^{(1)}_m-\frac{1}{r}\partial_r\psi^{(1)}_m+\frac{m^2}{r^2}\psi^{(1)}_m+\psi^{(1)}_m=\lambda_0\psi^{(1)}_m
    \end{equation}
    Пусть $z=\sqrt{1-\lambda_0}r$ и $f^{(1)}(z)=\psi^{(1)}(z(r))$, тогда
    \begin{equation}
        z^2\partial^2_zf^{(1)}_m+z\partial_zf^{(1)}_m-(z^2+m^2)f^{(1)}_m=0
    \end{equation}
    Получилась функция Инфильда (функция Макдональда сингулярна в 0):
    \begin{equation}
        \psi^{(1)}_m(r|\lambda_0)=C_2I_m(\sqrt{1-\lambda_0}r)
    \end{equation}
    Асимптотики функций Бесселя и Инфильда:
    \begin{equation}
    \begin{cases}
        J_m(\sqrt{\lambda_0}r)\approx\frac{1}{m!}\left(\frac{\sqrt{\lambda_0}r}{2}\right)^m\\
        I_m(\sqrt{1-\lambda_0}r)\approx\frac{1}{m!}\left(\frac{\sqrt{1-\lambda_0}r}{2}\right)^m
    \end{cases},\quad r\rightarrow0
    \end{equation}
    Для того, чтобы $\psi^{(1)}_m(r)\sim \psi_m(r)$, выберем $\frac{C_2}{C_1}=\left(\frac{\lambda_0}{1-\lambda_0}\right)^\frac{m}{2}$.
    \begin{equation}
        P_m(R)=\lim\limits_{\lambda_0\rightarrow0}\left(\frac{\lambda_0}{1-\lambda_0}\right)^\frac{m}{2}\frac{I_m(\sqrt{1-\lambda_0}R)}{J_m(\sqrt{\lambda_0}R)}=\frac{I_m(R)2^mm!}{R^m}
    \end{equation}
    \begin{equation}
        \boxed{P_m(R)=\frac{2^mm!}{R^m}I_m(R)}
    \end{equation}
    \item
    \begin{equation}
        \frac{\det(\hat{H}+\lambda)}{\det\hat{H}}=\prod\limits_{n=0}^\infty\frac{\lambda_n+\lambda}{\lambda_n}=\prod\limits_{n=0}^\infty\left(1+\frac{\lambda}{\lambda_n}\right)
    \end{equation}
    \begin{equation}
        \log\left(\frac{\det(\hat{H}+\lambda)}{\det\hat{H}}\right)=\sum\limits_{n=0}^\infty\log\left(1+\frac{\lambda}{\lambda_n}\right)=\sum\limits_{n=0}^\infty\sum\limits_{k=1}^\infty\frac{(-1)^{k-1}}{k}\left(\frac{\lambda}{\lambda_n}\right)^k
    \end{equation}
    \begin{multline}
        \frac{\partial^s}{\partial\lambda^s}\left(\log\left(\frac{\det(\hat{H}+\lambda)}{\det{\hat{H}}}\right)\right)\bigg|_{\lambda=0}=\sum\limits_{n=0}^\infty\sum\limits_{k=s}^\infty\frac{(-1)^{k-1}}{k}\frac{\lambda^{k-s}}{\lambda^k_n}\prod\limits_{m=0}^{s-1}(k-m)\bigg|_{\lambda=0}=\\=\sum\limits_{n=0}^\infty\frac{(-1)^{s-1}}{s}\frac{1}{\lambda^s_n}s!=\sum\limits_{n=0}^\infty\frac{(-1)^{s-1}}{\lambda^s_n}(s-1)!
    \end{multline}
    \begin{equation}
        \boxed{\zeta_H(s)=\sum\limits_{n=0}\frac{1}{\lambda_n^s}=\frac{(-1)^{s-1}}{(s-1)!}\frac{\partial^s}{\partial\lambda^s}\left(\log\left(\frac{\det(\hat{H}+\lambda)}{\det{\hat{H}}}\right)\right)\bigg|_{\lambda=0}}
    \end{equation}
    \item 
    \begin{equation}
        \frac{\det(\hat{H}_m+\lambda)}{\det(\hat{H}_m)}=\prod\limits_{n=0}^\infty\left(1+\frac{\lambda}{\lambda_n}\right),\quad\lambda_n=\frac{(\mu^{(m)}_n)^2}{R^2}
    \end{equation}
    Положим $R=1$, тогда из 2 пункта:
    \begin{equation}
        \sum\limits_{n=0}\frac{1}{(\mu^{(m)}_n)^{2s}}=\frac{(-1)^{s-1}}{(s-1)!}\frac{\partial^s}{\partial\lambda^s}\left(\log\left(\frac{\det(\hat{H}_m+\lambda)}{\det{\hat{H}_m}}\right)\right)\bigg|_{\lambda=0}
    \end{equation}
    Повторяя рассуждения пункта 1, получим выражение (R=1):
    \begin{equation}
        \frac{\det(\hat{H}_m+\lambda)}{\det(\hat{H}_m)}=\frac{2^mm!}{\lambda^\frac{m}{2}}I_m(\sqrt{\lambda})
    \end{equation}
    \begin{equation}
        \sum\limits_{n=0}\frac{1}{(\mu^{(m)}_n)^{2s}}=\frac{(-1)^{s-1}}{(s-1)!}\frac{\partial^s}{\partial\lambda^s}\left(\log\left(\frac{2^mm!}{\lambda^\frac{m}{2}}I_m(\sqrt{\lambda})\right)\right)\bigg|_{\lambda=0}
    \end{equation}
    \begin{equation}
        \sum_{n=0}^{\infty}\frac{1}{\left(\mu_{n}^{(0)}\right)^{2}}=\frac{\partial}{\partial\lambda}\left(\log\left(I_0(\sqrt{\lambda})\right)\right)\bigg|_{\lambda=0}=\frac{\partial}{\partial\lambda}\left(\log\left(1+\frac{\lambda}{4}+...\right)\right)\bigg|_{\lambda=0}
    \end{equation}
    \begin{equation}
        \boxed{\sum_{n=0}^{\infty}\frac{1}{\left(\mu_{n}^{(0)}\right)^2}=\frac{1}{4}}
    \end{equation}
    \begin{equation}
        \sum_{n=0}^{\infty}\frac{1}{\left(\mu_{n}^{(0)}\right)^4}=\frac{\partial^2}{\partial\lambda^2}\left(\log\left(I_0(\sqrt{\lambda})\right)\right)\bigg|_{\lambda=0}=\frac{\partial^2}{\partial\lambda^2}\left(\log\left(1+\frac{\lambda}{4}+\frac{\lambda^2}{2^6}...\right)\right)\bigg|_{\lambda=0}
    \end{equation}
    \begin{equation}
        \boxed{\sum_{n=0}^{\infty}\frac{1}{\left(\mu_{n}^{(0)}\right)^4}=\frac{1}{32}}
    \end{equation}
    \begin{equation*}
        \sum_{n=0}^{\infty}\frac{1}{\left(\mu_{n}^{(1)}\right)^{2}}=\frac{\partial}{\partial\lambda}\left(\frac{2}{\sqrt{\lambda}}\log\left(I_1(\sqrt{\lambda})\right)\right)\bigg|_{\lambda=0}=\frac{\partial}{\partial\lambda}\left(\log\left(\frac{2}{\sqrt{\lambda}}\left(\frac{\sqrt{\lambda}}{2}+\frac{\lambda^\frac{3}{2}}{2^{4}}+...\right)\right)\right)\bigg|_{\lambda=0}
    \end{equation*}
    \begin{equation}
        \boxed{\sum_{n=0}^{\infty}\frac{1}{\left(\mu_{n}^{(1)}\right)^2}=\frac{1}{8}}
    \end{equation}
\end{enumerate}
\section{Распад метастабильного состояния}
\subsection*{Упражнения (50 баллов)}
\textbf{Упражнение 1 (30 баллов)}\\
В этом упражнении мы будем работать с потенциалом, рассматриваемом на семинаре
\begin{enumerate}
    \item \textbf{(10 баллов)} Вычислите соответствующее отношение функциональных определителей.
    \item \textbf{(20 баллов)} Найдите ширину уровней энергии для рассматриваемого на семинаре потенциале в квазиклассическом приближении. Сравните квазиклассическое приближение для основного состояния с найденным на семинаре ответом (см. комментарий к упражнению из семинара про двухъямный потенциал).
\end{enumerate}
\textbf{Упражнение 2 (20 баллов)}\\
Рассмотрите следующий интеграл, сходящийся при $\text{Re}g > 0$:
\begin{equation}
    I(g) =\int\limits_{-\infty}^\infty dx\cdot\exp\left(-\frac{1}{2}x^2-gx^4\right)
\end{equation}
\begin{enumerate}
    \item Постройте аналитическое продолжение этого интеграла на всю комплексную плоскость (включая область Reg < 0, где этот интеграл буквально расходится).
    \item Считая $g \ll 1$, найдите линии Стокса для построенного аналитического продолжения. Найдите асимпотическое поведение интеграла в различных секторах комплексной плоскости. Что можно сказать про вклады на самой линии Стокса?
\end{enumerate}
\subsection*{Задача (50 баллов)}
Рассмотрите распад метастабильного состояния частицы массы $m = 1$ в потенциале $U(x) = \lambda x^2(\eta^2-x^2)$. Рассматривая Евклидову функцию Грина $G_E(0, 0, \beta)$, найдите ширину уровня основного метастабильного состояния.
\begin{enumerate}
    \item \textbf{(20 баллов)} Найдите одноинстантонные траектории, соответствующую такому переходу, вычислите действия на них. Вычислите явно оператор, определяющий квадратичные флуктуации в окрестности инстантона.
    \item \textbf{(20 баллов)} Вычислите интегралы по нулевым модам, найдите соответствующий флуктуационный детерминант (используя метод Гельфанда-Яглома).
    \item \textbf{(10 баллов)} Просуммируйте разреженный инстантонный газ, найдите ответ.
\end{enumerate}
\end{document}

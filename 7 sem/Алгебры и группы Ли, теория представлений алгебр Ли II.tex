\documentclass[12pt]{article}

% report, book
%  Русский язык

\usepackage{hyperref,bookmark}
\usepackage[warn]{mathtext} %русский язык в формулах
\usepackage[T2A]{fontenc}			% кодировка
\usepackage[utf8]{inputenc}			% кодировка исходного текста
\usepackage[english,russian]{babel}	% локализация и переносы
\usepackage[title,toc,page,header]{appendix}
\usepackage{amsfonts}


% Математика
\usepackage{amsmath,amsfonts,amssymb,amsthm,mathtools} 
%%% Дополнительная работа с математикой
%\usepackage{amsmath,amsfonts,amssymb,amsthm,mathtools} % AMS
%\usepackage{icomma} % "Умная" запятая: $0,2$ --- число, $0, 2$ --- перечисление

\usepackage{cancel}%зачёркивание
\usepackage{braket}
%% Шрифты
\usepackage{euscript}	 % Шрифт Евклид
\usepackage{mathrsfs} % Красивый матшрифт


\usepackage[left=2cm,right=2cm,top=1cm,bottom=2cm,bindingoffset=0cm]{geometry}
\usepackage{wasysym}

%размеры
\renewcommand{\appendixtocname}{Приложения}
\renewcommand{\appendixpagename}{Приложения}
\renewcommand{\appendixname}{Приложение}
\makeatletter
\let\oriAlph\Alph
\let\orialph\alph
\renewcommand{\@resets@pp}{\par
  \@ppsavesec
  \stepcounter{@pps}
  \setcounter{subsection}{0}%
  \if@chapter@pp
    \setcounter{chapter}{0}%
    \renewcommand\@chapapp{\appendixname}%
    \renewcommand\thechapter{\@Alph\c@chapter}%
  \else
    \setcounter{subsubsection}{0}%
    \renewcommand\thesubsection{\@Alph\c@subsection}%
  \fi
  \if@pphyper
    \if@chapter@pp
      \renewcommand{\theHchapter}{\theH@pps.\oriAlph{chapter}}%
    \else
      \renewcommand{\theHsubsection}{\theH@pps.\oriAlph{subsection}}%
    \fi
    \def\Hy@chapapp{appendix}%
  \fi
  \restoreapp
}
\makeatother
\newtheorem{theorem}{Теорема}[]
\newtheorem{predl}[theorem]{Предложение}
\newtheorem{sled}[theorem]{Следствие}

\theoremstyle{definition}
\newtheorem{zad}{Задача}[section]
\newtheorem{upr}[zad]{Упражнение}
\newtheorem{defin}{Определение}[]

\title{Решение заданий\\ ОП "Квантовая теория поля, теория струн и математическая физика"\\[2cm]
Алгебры и группы Ли,\\ теория представлений алгебр Ли I\\ (Б.Л. Фейгин)}
\author{Коцевич Андрей Витальевич, Б02-920с}
\date{6 семестр, 2022}

\begin{document}
\setlength{\parindent}{0pt}
\section{Категория $\mathcal{O}$ для $\mathfrak{sl_2}$ и явное описание неразложимых представлений}
\begin{defin}
\textit{Категория $\mathcal{C}$} --- это
\begin{itemize}
    \item Класс \textit{объектов} $Ob_{\mathcal{C}}$;
    \item Для каждой пары объектов $A$, $B$ задано множество (возможно, пустое) \textit{морфизмов} (или \textit{стрелок}) $Hom_{\mathcal{C}}(A, B)$, причём каждому морфизму соответствуют единствнные $A$ и $B$;
    \item Для пары морфизмов $f \in Hom_{\mathcal{C}}(A, B)$ и $g \in Hom_{\mathcal{C}}(B, C)$ определена композиция $g \circ f \in Hom_{\mathcal{C}}(A, C)$;
    \item Для каждого объекта $A$ задан тождественный морфизм $id_A \in Hom_{\mathcal{C}}(A, A)$;
\end{itemize}
и выполняются аксиомы
\begin{itemize}
    \item Операция композиции ассоциативна: $h \circ (g \circ f) = (h \circ g) \circ f$;
    \item Тождественный морфизм действует тривиально: $f \circ id_A = id_B \circ f$ для всех $f \in Hom_{\mathcal{C}}(A, B)$.
\end{itemize}
\end{defin}

\begin{defin}
$\mathfrak{sl_2}(\mathbb{C})$ --- это алгебра Ли матриц 2 × 2 с нулевым следом.
\end{defin}

Алгебру Ли $\mathfrak{sl_2}(\mathbb{C})$ можно задать стандартными образующими $e$, $f$, $h$ со следующими коммутационными соотношениями:
\begin{enumerate}
    \item $[e, f] = h$;
    \item $[h, e] = 2e$;
    \item $[h, f] = -2f$.
\end{enumerate}

Полезно разобраться с тем, как устроены конечномерные неприводимые представления $\mathfrak{sl_2}(\mathbb{C})$. Пусть $(V, \rho)$ --- конечномерное представление алгебры Ли $\mathfrak{sl_2}(\mathbb{C})$, то есть $V$ --- конечномерное векторное пространство, а $\rho(g)$ --- линейные операторы, действующие на нём, $g \in \mathfrak{sl_2}(\mathbb{C})$.

\begin{theorem}
    Пусть $v \in V$ --- собственный вектор оператора $\rho(h)$ с собственным значением $\mu$. Тогда вектор $\rho(e) v$ --- собственный для $\rho(h)$ с собственным значением $\mu + 2$, а вектор $\rho(f) v$ --- собственный для $\rho(h)$ с собственным значением $\mu - 2$.
    \begin{proof}
        Имеем $\rho(h) \rho(e) v = \rho(e) \rho(h) v + \rho([h, e])v = \mu \rho(e) v + 2 \rho(e) v$, что и требовалось. Аналогично получается и для $f$.
    \end{proof}
\end{theorem}

\begin{theorem}
    Существует собственный для $\rho(h)$ вектор $v \in V$ такой, что $\rho(e) v = 0$. Такие вектора называются особыми.
    \begin{proof}
        Пусть $\lambda$ --- собственное значение оператора $\rho(h)$ с максимальной действительной частью (такое существует в силу конечномерности), $v_{\lambda}$ --- соответствующий собственный вектор. Тогда  $\rho(e) v_{\lambda}$ --- собственный вектор с собственным значением $\lambda + 2$, либо нулевой вектор. Так как действительная часть $\lambda + 2$ больше действительной части $\lambda$, такое возможно только если $\rho(e) v_{\lambda} = 0$.
    \end{proof}
\end{theorem}

Пусть $v_{\lambda}$ --- особый вектор. Можно построить цепочку собственных векторов для $\rho(h)$, последовательно действуя на $v_{\lambda}$ оператором $\rho(f)$, а точнее рассмотреть подпространство $V_{\lambda}$, натянутое на вектора вида $\rho(f) ^ k v_{\lambda}$.

\begin{theorem}
    $V_{\lambda}$ является подпредставлением в $V$.
    \begin{proof}
        Достаточно проверить, что $\rho(e) V_{\lambda} \subset V_{\lambda}$, $\rho(f) V_{\lambda} \subset V_{\lambda}$, $\rho(h) V_{\lambda} \subset V_{\lambda}$. Последнее очевидно, так как подпространство $V_{\lambda}$ натянуто на собственные вектора оператора $\rho(h)$. Предпоследнее также очевидно, так как оператор $\rho(f)$, действуя на базисный вектор из $V_{\lambda}$, то есть на элемент цепочки $\rho(f) ^ k v_{\lambda}$, совершает переход к следующему элементу цепочки. Осталось доказать для $\rho(e)$, что мы сделаем по индукции. А именно, докажем, что $\rho(e) \rho(f) ^ k v_{\lambda} = (k \lambda - k(k-1)) \rho(f)^{k-1} v_{\lambda} \in V_{lambda}$. База индукции: $k=0$, $\rho(e) v_{\lambda} = 0$. Пусть верно для всех $k < K$. Имеем
        $$ \rho(e) \rho(f) ^ K v_{\lambda} = (\rho(e) \rho(f)) \rho(f) ^ {K-1} v_{\lambda} = (\rho(f) \rho(e) + \rho(h)) \rho(f) ^ {K-1} v_{\lambda} = $$
        $$ = ((K-1) \lambda - (K-1)(K-2) + \lambda - 2 (K-1)) \rho(f) ^ {K-1} v_{\lambda} = (K \lambda - K (K-1)) \rho(f) ^ {K-1} v_{\lambda}. $$
    \end{proof}
\end{theorem}

\begin{theorem}
    Подпространство $V_{\lambda}$ конечномерно тогда и только тогда, когда $\lambda \in \mathbb{Z}_{\ge 0}$.
    \begin{proof}
        Если $V_{\lambda}$ конечномерно, то найдётся такое $k$, что $\rho(f) ^ k v_{\lambda} = 0$. Пусть $n$ таково, что $\rho(f) ^ n v_{\lambda} \ne 0$, $\rho(f) ^ {n+1} v_{\lambda} = 0$. Тогда
        $$ 0 = \rho(e) \rho(f) ^ {n+1} v_{\lambda} = ((n+1) \lambda - n (n+1)) \rho(f) ^ n v_{\lambda}, $$
        откуда $\lambda = n$.
    \end{proof}
\end{theorem}

Таким образом, всякое неприводимое представление изоморфно $V_n$ для некоторого $n \in \mathbb{Z}_{\ge 0}$. Так как $dim V_n = n + 1$, то неприводимое представление данной размерности единственно с точностью до изоморфизма.

Представления алгебры Ли $\mathfrak{sl_2}(\mathbb{C})$ также называют \textit{$\mathfrak{sl_2}(\mathbb{C})$-модулями}. Модуль в данном случае понимается в том смысле, что пространство представления инвариантно относительно действия алгебры $\mathfrak{sl_2}(\mathbb{C})$.

Мы доказали, что все конечномерные неприводимые представления $\mathfrak{sl_2}(\mathbb{C})$ обладают следующими свойствами:
\begin{enumerate}
    \item Оператор $\rho(h)$ действует \textit{полупросто}, то есть существует базис из собственных векторов $\rho(h)$ в пространстве представления;
    \item Оператор $\rho(e)$ действует \textit{локально нильпотентно}, то есть для любого вектора $v$ существует такое натуральное число $k$, что $\rho(e) ^ k v = 0$.
\end{enumerate}

Конечнопорождённые $\mathfrak{sl_2}(\mathbb{C})$-модули, то есть представления, пространства которых представляются в виде линейной (относительно поля $\mathbb{C}$) оболочки элементов орбит (относительно действия алгебры $\mathfrak{sl_2}(\mathbb{C})$) нескольких векторов, образуют категорию $\mathcal{O}$.

\begin{itemize}
    \item Объектами данной категории являются $\mathfrak{sl_2}(\mathbb{C})$-модули.
    \item Морфизмами являются линейные отображения между модулями, коммутирующие с действием алгебры, то есть если $(V, \rho_V)$ и $(U, \rho_U)$ --- два представления $\mathfrak{sl_2}(\mathbb{C})$, то линейное отображение $\varphi : V \rightarrow U$ между ними будет морфизмом категории $\mathcal{O}$, если $\varphi \rho_V(g) = \rho_U(g) \varphi$ для любого $g \in \mathfrak{sl_2}(\mathbb{C})$.
\end{itemize}

Примером бесконечномерного, но конечнопорождённого объекта этой категории является \textit{модуль Верма со старшим весом $\lambda$}: это векторное пространство с базисом $v_{\lambda} ^ {(k)}$, $k \in \mathbb{Z}_+$, а элементы алгебры Ли действуют следующим образом:
$$ \rho(e) v_{\lambda} ^ {(k)} = (k \lambda - k (k - 1)) v_{\lambda} ^ {(k-1)}, $$
$$ \rho(f) v_{\lambda} ^ {(k)} = v_{\lambda} ^ {(k+1)}, $$
$$ \rho(h) v_{\lambda} ^ {(k)} = (\lambda - 2k) v_{\lambda} ^ {(k)}. $$

Это такая же весовая конструкция, только без конечномерности.

В терминах категории $\mathcal{O}$, утверждение про неприводимые представления переформулируется следующим образом:
\begin{predl}
    Всякое неприводимое конечномерное представление алгебры Ли $\mathfrak{sl_2}(\mathbb{C})$ лежит в категории $\mathcal{O}$. Всякий неприводимый объект категории $\mathcal{O}$ является фактором какого-нибудь модуля Верма. Модуль Верма приводим тогда и только тогда, когда $\lambda \in \mathbb{Z}_{\ge 0}$, и в этом случае имеет единственный неприводимый фактор $V_n$.
\end{predl}
    

\section{Центр универсальной обёртывающей алгебры. Явное описание для случая $\mathfrak{g}=\mathfrak{sl}_n(\mathbb{C})$}
\begin{defin}
Пусть $A$ -- векторное пространство над полем $K$, снабжённое билинейной операцией $\cdot:A\times A\rightarrow A$ (умножением), т.е. $\forall x,y,z\in A,a,b\in K$
\begin{enumerate}
    \item $(x+y)\cdot z=x\cdot z+y\cdot z$
    \item $x\cdot(y+z)=x\cdot y+x\cdot z$
    \item $(ax)\cdot(by)=ab(x\cdot y)$
\end{enumerate}
Тогда $A$ называется \textit{алгеброй над $K$}.
\end{defin}
\begin{defin}
Алгебра называется \textit{ассоциативной}, если операция умножения в ней ассоциативна.
\end{defin}
\begin{defin}
\textit{Алгеброй Ли} называется векторное пространство с билинейной операцией (коммутатором) $[\cdot,\cdot]$, удовлетворяющей следующим аксиомам:
\begin{enumerate}
    \item (кососимметричность или антикоммутативность) $[x, y] = -[y, x]$.
    \item (тождество Якоби) $[x, [y, z]] + [y, [z, x]] + [z, [x, y]] = 0$.
\end{enumerate}
\end{defin}
\textit{Примеры:}
\begin{enumerate}
    \item Ассоциативная алгебра $A$ над полем $K$ обладает естественной структурой алгебры Ли над $K$ со следующей скобкой Ли: $[a,b] = ab - ba$, т.е. из ассоциативного произведения можно построить скобку Ли с помощью простого взятия коммутатора. Обозначим эту алгебру Ли $A_L$.
    \item Пространство $\mathbb{R}^3$ с операцией векторного произведения -- алгебра Ли над $\mathbb{R}$.
    \item Произвольное векторное пространство с тождественно нулевой операцией коммутатора. Такая алгебра Ли называется \textit{абелевой}.
\end{enumerate}
Более общий пример:
\begin{defin}
    \textit{Дифференцированием алгебры} $A$ называется линейный оператор $D:A\rightarrow A$, для которого выполнено тождество Лейбница:
    \begin{equation}
        D(a\cdot b)=D(a)\cdot b+a\cdot D(b)
    \end{equation}
\end{defin}
Все дифференцирования алгебры A образуют векторное пространство.
\begin{defin}
    Векторное подпространство $I$ в алгебре Ли $L$ называется \textit{идеалом}, если $\forall x \in I, y\in L\hookrightarrow[x,y]\in I$.
\end{defin}

Построение универсальной обёртывающей алгебры пытается процесс построения $A_L$ из $A$: для данной алгебры Ли $\mathfrak{g}$ над $K$ находят <<наиболее общую>> ассоциативную $K$-алгебру $U(\mathfrak{g}):$ алгебра Ли $U_L(\mathfrak{g})$ содержит $\mathfrak{g}$. Важное ограничение -- сохранение теории представлений: представления $\mathfrak{g}$ соотносятся точь-в-точь так же как и модули над $U(\mathfrak{g})$. В типичном контексте, где $\mathfrak{g}$ задаётся инфинитезимальными преобразованиями, элементы $U(\mathfrak{g})$ действуют как дифференциальные операторы всех порядков.
\begin{defin}
\textit{Универсальной обертывающей алгеброй} алгебры Ли $\mathfrak{g}$ называется пара $(U(\mathfrak{g}),\varepsilon)$, где $U(\mathfrak{g})$ -- ассоциативная алгебра с единицей, $\varepsilon:\mathfrak{g}\rightarrow U_L(\mathfrak{g})$ -- гомоморфизм алгебр Ли (т.е. $\varepsilon([x,y])=[\varepsilon(x),\varepsilon(y)]$), обладающий следующим \textit{универсальным свойством}: $\forall$ ассоциативной алгебры $A$ и гомоморфизма алгебр Ли $\varphi:\mathfrak{g}\rightarrow A_L$ (т.е. $\varphi([x,y])=[\varphi(x),\varphi(y)]$) $\exists !$ гомоморфизм ассоциативных алгебр $\Phi:U(\mathfrak{g}) \rightarrow A:\varphi=\Phi\circ\varepsilon$ (рис. 1).
\end{defin}
\begin{figure}
    \centering
    \includegraphics[scale=0.3]{1.jpg}
    \caption{Универсальное свойство}
    \label{fig:my_label}
\end{figure}
Из универсального свойства, в частности, следует, что любое представление алгебры Ли $\mathfrak{g}$ $\varphi:\mathfrak{g}\rightarrow \text{End}(V)$ несет структуру представления ассоциативной алгебры $U(\mathfrak{g})$, причем любой гомоморфизм представлений алгебры Ли $\mathfrak{g}$ есть также и гомоморфизм представлений $U(\mathfrak{g})$. Для этого нужно в качестве ассоциативной алгебры взять $A=\text{End}(V)$.
\begin{predl}
Универсальная обёртывающая алгебра единственна с точностью до изоморфизма.
\begin{proof}
    От противного, пусть $\exists$ две универсальные обёртывающие алгебры $(U_1(\mathfrak{g}),\varepsilon_1)$ и $(U_2(\mathfrak{g}),\varepsilon_2)$. Тогда из универсального свойства (рис. 2) следует единственность.
\end{proof}
\end{predl}
\begin{figure}
    \centering
    \includegraphics[scale=0.3]{2.jpg}
    \caption{Единственность универсальной обёртывающей}
    \label{fig:my_label}
\end{figure}
\begin{predl}
    Универсальная обёртывающая алгебра $\exists$ $\forall$ алгебры Ли $\mathfrak{g}$.
    \begin{proof}
        Зададим алгебру $U(\mathfrak{g})$ явно образующими и соотношениями. Пусть $T(\mathfrak{g})=\mathbb{C}\oplus \mathfrak{g}\oplus\mathfrak{g}\otimes \mathfrak{g}\oplus...$ -- тензорная алгебра пространства $\mathfrak{g}$ (т.е. свободная ассоциативная алгебра, порождённая пространством $\mathfrak{g}$) и пусть $J\subset T(\mathfrak{g})$ -- двусторонний идеал, порождённый элементами $x\otimes y-y\otimes x-[x,y]\;\forall x,y\in \mathfrak{g}$. Тогда ассоциативная алгебра $U(\mathfrak{g}):=T(\mathfrak{g})/J$ с тождественным отображением $\epsilon:\mathfrak{g}\rightarrow\mathfrak{g}\subset T(\mathfrak{g})$ обладает требуемым универсальным свойством. Иначе говоря, пусть $x_1,...,x_n$ -- базис в алгебре Ли $\mathfrak{g}$ и пусть $[x_i,x_j]=\sum\limits_{k=1}^nc^k_{ij}x_k$. Тогда $U(\mathfrak{g})$ -- ассоциативная алгебра с образующими $x_1,...,x_n$ и определяющими соотношениями $x_ix_j-x_jx_i=\sum\limits_{k=1}^nc^k_{ij}x_k$, причём $\epsilon(x_i)=x_i$: $U(\mathfrak{g})=\mathbb{C}\braket{x_1,...,x_n}/(x_ix_j-x_jx_i-\sum\limits_kc^k_{ij}x_k)$.
    \end{proof}
\end{predl}
\textit{Пример.} Для абелевой группы Ли $L$ с базисом $x_1,...,x_n$ универсальная обёртывающая алгебра $U(L)$ -- симметрическая алгебра $S(L):=T(L)/(x_ix_j-x_jx_i)$, т.е. алгебра многочленов $\mathbb{C}[x_1,...,x_n]$.
\begin{predl}
    Пусть $x_1,...,x_n$ -- базис алгебры Ли $\mathfrak{g}$. Элементы вида $x_1^{k_1}x_2^{k_2}...x_n^{k_n}$ образуют полную систему в универсальной обёртывающей алгебре $U(\mathfrak{g})$ (любой элемент $U(\mathfrak{g})$ может быть линейно выражен через такие упорядоченные мономы).
    \begin{proof}
        Мономы вида $x_{i_1}x_{i_2}...x_{i_N}$ образуют полную систему в универсальной обёртывающей алгебре $U(\mathfrak{g})$. Пусть $\sum\limits_{i=1}^nk_i=N$ -- степень упорядоченного монома. Докажем утверждение по индукции.
        \begin{itemize}
            \item База индукции: $N=1$, верно.
            \item Пусть верно для всех $\sum\limits_{i=1}^nk_i<N$.
            \item Докажем, что верно для $\sum\limits_{i=1}^nk_i=N$. Если в какой-то части монома $x_{i_1}x_{i_2}...x_{i_N}$ индексы расположены не по возрастанию, то их можно переставить, используя коммутационное соотношение $x_{i_1}x_{i_2}=x_{i_2}x_{i_1}+[x_{i_1},x_{i_2}]$. При этом получится упорядоченный моном и мономы меньшего размера, для которых утверждение доказано.
        \end{itemize}
    \end{proof}
\end{predl}
\begin{theorem}[Пуанкаре-Биркгофа-Витта]
Если $x_1,...,x_n$ -- базис в алгебре Ли $\mathfrak{g}$, то мономы $x_1^{k_1}x_2^{k_2}...x_n^{k_n}$ образуют базис в пространстве $U(\mathfrak{g})$.\\
Эквивалентная формулировка. Пусть $i:S(\mathfrak{g})\rightarrow T(\mathfrak{g})$ -- вложение (симметризация):
\begin{equation}
    i(v_1\cdot v_2\cdot...\cdot v_n)=\frac{1}{n!}\sum\limits_{\sigma\in S_n}v_{\sigma(1)}\otimes v_{\sigma(2)}\otimes...\otimes v_{\sigma(n)}
\end{equation}
Пусть $\tau:T(\mathfrak{g})\rightarrow U(\mathfrak{g})$ -- отображение факторизации. Тогда $\sigma=\tau\circ i$ -- отображение симметризации -- является изоморфизмом векторных пространств и представлений алгебры Ли $\mathfrak{g}$ (см. рис. 3). На самом деле это изоморфизм $\mathfrak{g}$-модулей относительно присоединённого действия.
\end{theorem}
\begin{defin}
    Центром алгебры $A$ называется подпространство $Z(A)\subset A$, состоящее из элементов $x \in A:[x,y]=0\;\forall y\in A$.
\end{defin}
 Центр алгебры Ли $\mathfrak{Z}(L)\subset L$ является идеалом в $L$.
\begin{defin}
    \textit{Инварианты относительно действия группы Ли $G$} -- элементы алгебры, которые переходят в себя при действии алгебры Ли $G$ сопряжениями.
\end{defin}
\begin{defin}
    \textit{Инварианты относительно действия алгебры Ли $\mathfrak{g}$} -- элементы алгебры, которые переходят в 0 при действии алгебры Ли $\mathfrak{g}$.
\end{defin}
Если элемент является инвариантом относительно действия группы Ли $G$, то он является инвариантом относительно действия соответствующей алгебры Ли $\mathfrak{g}$, и наоборот.\\
Перейдём к рассмотрению центра универсальной обёртывающей алгебры:
\begin{equation}
    ZU(\mathfrak{g})=\{a\in U(\mathfrak{g})|\forall x\in\mathfrak{g}\hookrightarrow[x,a]=\text{ad}_x(a)=0\}
\end{equation}
$ZU(\mathfrak{g})=U(\mathfrak{g})^{\mathfrak{g}}$ -- элементы алгебры $U(\mathfrak{g})$, инвариантные оносительно действия алгебры Ли $\mathfrak{g}$.\\
По теореме Пуанкаре-Биркгофа-Витта (ПБВ) отображение симметризации $\sigma:S(\mathfrak{g})\rightarrow U(\mathfrak{g})$ -- изоморфизм векторных пространств. $\sigma$ -- композиция 2 отображений, уважающих действие алгебры Ли дифференцированиями. Т.е. $\sigma$ -- гомоморфизм относительно действия $\text{ad}_x,x\in\mathfrak{g}$. Значит, $\sigma:S(\mathfrak{g})^\mathfrak{g}\rightarrow U(\mathfrak{g})^\mathfrak{g}$ -- изоморфизм векторных пространств (но не гомоморфизм алгебр!) и, как векторное пространство, $ZU(\mathfrak{g})$ -- пространство $\mathfrak{g}$-инвариантов в $S(\mathfrak{g})$.\\
\textit{Пример.} $\mathfrak{g}=\mathfrak{sl}_2(\mathbb{C})$. Найдём её центр универсальной обёртывающей $U(\mathfrak{sl}_2(\mathbb{C}))$.
\begin{equation}
    S(\mathfrak{sl}_2(\mathbb{C}))^{\mathfrak{sl}_2(\mathbb{C})}=\mathbb{C}[e,f,h]^{\mathfrak{sl}_2(\mathbb{C})}=\mathbb{C}[e,f,h]^{SL_2(\mathbb{C})}
\end{equation}
\begin{equation}
    e=\begin{pmatrix}
        0 & 1\\
        0 & 0
    \end{pmatrix},\quad f=\begin{pmatrix}
        0 & 0\\
        1 & 0
    \end{pmatrix},\quad h=\begin{pmatrix}
        1 & 0\\
        0 & -1
    \end{pmatrix}
\end{equation}
Многочлены от $e,f,h$ -- функции на $\mathfrak{sl}_2(\mathbb{C})$, т.е. элементы $\mathfrak{sl}^*_2(\mathbb{C})$.
\begin{equation}
    \mathfrak{sl}^*_2(\mathbb{C})\simeq\mathfrak{sl}_2(\mathbb{C}),\quad x\rightarrow \text{Tr}(x\cdot)
\end{equation}
Т.е. нужно найти $\mathbb{C}[\mathfrak{sl}_2(\mathbb{C})]^{SL_2(\mathbb{C})}$ -- инвариантные многочлены от $A\in\mathfrak{sl}_2(\mathbb{C})$ относительно сопряжений матрицами из $SL_2(\mathbb{C})$. Такие многочлены определяются своими значениями на диагональных матрицах, поскольку  пространство диагонализуемых матриц плотно в $\mathfrak{sl}_2(\mathbb{C})$. Т.е. любой инвариантный многочлен -- симметрический многочлен от собственных значений матрицы. По основной теореме о симметрических многочленах любой симмметрический многочлен -- многочлен от элементарных симметрических -- коэффициентов характеристического многочлена. В нашем случае единственным нетривиальным коэффициентом характеристического многочлена является определитель.
\begin{equation}
    \det\begin{pmatrix}
    a & b\\
    c & -a
    \end{pmatrix}=-a^2-bc
\end{equation}
Определитель записан через элементы $\mathfrak{sl}^*_2(\mathbb{C})$. Выразим его через элементы $\mathfrak{sl}^*_2(\mathbb{C})$.
\begin{equation}
    \text{Tr}e\begin{pmatrix}
        a & b\\
        c & -a
    \end{pmatrix}=c,\quad\text{Tr}f\begin{pmatrix}
        a & b\\
        c & -a
    \end{pmatrix}=b,\quad \text{Tr}h\begin{pmatrix}
        a & b\\
        c & -a
    \end{pmatrix}=2a
\end{equation}
\begin{equation}
    e\rightarrow c,\quad f\rightarrow b,\quad h\rightarrow 2a
\end{equation}
\begin{equation}
    \det=\frac{h^2}{4}-fe
\end{equation}
\begin{equation}
    \sigma(\det)=\tau\circ i\left(-\frac{h^2}{4}-fe\right)=\tau\left(-\frac{h\otimes h}{4}-\frac{f\otimes e}{2}-\frac{e\otimes f}{2}\right)=-\frac{h^2}{4}-\frac{1}{2}(fe+ef)
\end{equation}
Квадратичный оператор Казимира:
\begin{equation}
    \boxed{C=\frac{h^2}{2}+fe+ef=\frac{h^2}{2}+h+2fe=\frac{h^2}{2}-h+2ef}
\end{equation}
\begin{predl}
    Центр универсальной обёртывающей $ZU(\mathfrak{sl}_2(\mathbb{C}))$ порождается квадратичным Казимиром $C$: $ZU(\mathfrak{sl}_2(\mathbb{C}))=\mathbb{C}[C]$.
    \begin{proof}
        Размеры центра универсальной обёртывающей и инвариантов симметрической алгебры имеют одинаковый размер, а инварианты симметрической алгебры -- многочлены от Казимира.
    \end{proof}
\end{predl}
Аналогично, для $\mathfrak{g}=\mathfrak{sl}_n(\mathbb{C})$ будет $n-1$ Казимиров (квадратичный, кубический, ...). Для примера посчитаем кубический Казимир $\mathfrak{sl}_3(\mathbb{C})$.
\begin{predl}[лемма Шура]
    Пусть $\varphi:V_1\rightarrow V_2$ -- гомоморфизм неприводимых представлений алгебры Ли $L$. Тогда или $\varphi=0$, или $\varphi$ -- изоморфизм.
    \begin{proof}
        $\text{Ker}\;\varphi$ -- подпредставление в $V_1$, а $\text{Im}\varphi$ -- подпредставление в $V_2$. Поскольку представления неприводимы, имеем либо $\text{Ker}\;\varphi=V_1$ (и тогда $\varphi=0$), либо $\text{Ker}\;\varphi=0$ (и тогда $\text{Im}\;\varphi=V_2$, следовательно $\varphi$ -- изоморфизм).
    \end{proof}
\end{predl}
Согласно лемме Шура, центральный элемент универсальной обертывающей алгебры в каждом неприводимом представлении действует скалярным оператором.
\begin{predl}
    В неприводимом представлении $V_n$ имеем $\rho(C)=\frac{n^2}{2}+n$.
    \begin{proof}
        Проверим это на старшем векторе $v_n\in V_n$.
        \begin{equation}
            \rho(C)v_n=\rho\left(\frac{1}{2}h^2+ef+fe\right)v_n=\rho\left(\frac{1}{2}h^2+h\right)v_n=\left(\frac{n^2}{2}+n\right)v_n
        \end{equation}
    \end{proof}
\end{predl}
Этот скаляр инвариантен относительно отражений $n\rightarrow-n-2$.\\
Таким образом, в различных неприводимых представлениях оператор Казимира действует различными скалярами.\\
В общем случае $\mathfrak{g}=\mathfrak{sl}_n$ скаляр будет инвариантен относительно сдвинутого действия группы Вейля:
\begin{equation}
    \lambda\rightarrow w(\lambda+\rho)-\rho
\end{equation}
\end{document}

%https://math.hse.ru/bac3-12-lie, https://math.hse.ru/lie_2015
